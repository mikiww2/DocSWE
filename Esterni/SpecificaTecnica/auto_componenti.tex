\subsection{Quizzipedia::Client}
Racchiude tutte le componenti necessarie per il front-end del prodotto. Visualizza i dati dell'utente e invia richieste al server.
\begin{figure}[H]
\centering
\noindent\makebox[\textwidth]{\includegraphics[width=\textwidth]{Img/quizzipedia-client.pdf}}
\caption[Schema Componente Client]{Schema Componente Quizzipedia::Client}
\end{figure}
\subsection{Quizzipedia::Client::ModelClient}
Rappresenta il modello dei dati che verranno utilizzati dal sistema lato client.
\begin{figure}[H]
\centering
\noindent\makebox[\textwidth]{\includegraphics[width=\textwidth]{Img/quizzipedia-client-modelclient.pdf}}
\caption[Schema Componente Quizzipedia::Client::ModelClient]{Schema Componente Quizzipedia::Client::ModelClient}
\end{figure}
\subsubsection{Componenti contenute}
\begin{itemize}
\item Quizzipedia::Client::ModelClient::Organization
\item Quizzipedia::Client::ModelClient::Requests
\item Quizzipedia::Client::ModelClient::Services
\item Quizzipedia::Client::ModelClient::Statistics
\item Quizzipedia::Client::ModelClient::Users
\end{itemize}
\subsubsection{Interazioni con altre componenti}
\paragraph{Entranti}
\begin{itemize}
\item usata da Quizzipedia::Client::ControllerClient per Rappresenta il modello dei dati che verranno utilizzati dal sistema lato client
\end{itemize}
\subsection{Quizzipedia::Client::ModelClient::Organization}
La componente gestisce le classi e gli enti, ovvero il sistema in base a cui sono organizzati gli utenti nel sistema.
\begin{figure}[H]
\centering
\noindent\makebox[\textwidth]{\includegraphics[width=\textwidth]{Img/quizzipedia-client-modelclient-organization.pdf}}
\caption[Schema Componente Quizzipedia::Client::ModelClient::Organization]{Schema Componente Quizzipedia::Client::ModelClient::Organization}
\end{figure}
\subsubsection{Interazioni con altre componenti}
\paragraph{Entranti}
\begin{itemize}
\item usata da Quizzipedia::Client::ModelClient::Users per La componente gestisce le classi e gli enti, ovvero il sistema in base a cui sono organizzati gli utenti nel sistema
\end{itemize}
\subsubsection{Classe Class}
Contiene informazioni relative alla struttura delle classi.
\begin{figure}[H]
\centering
\noindent\makebox[\textwidth]{\includegraphics[width=\textwidth]{Img/quizzipedia-client-modelclient-organization-class.pdf}}
\caption[Schema Classe Class]{Schema Classe Quizzipedia::Client::ModelClient::Organization::Class}
\end{figure}
\paragraph{Relazioni con altre classi}
\subparagraph{Entranti}
\begin{itemize}
\item usata da Quizzipedia::Client::ModelClient::Services::Quiz per La componente gestisce le classi e gli enti, ovvero il sistema in base a cui sono organizzati gli utenti nel sistema
\item usata da Quizzipedia::Client::ModelClient::Users::Director per La componente gestisce le classi e gli enti, ovvero il sistema in base a cui sono organizzati gli utenti nel sistema
\item usata da Quizzipedia::Client::ModelClient::Users::User per La componente gestisce le classi e gli enti, ovvero il sistema in base a cui sono organizzati gli utenti nel sistema
\end{itemize}
\subsubsection{Classe Institution}
La classe contiene le informazioni relative alla struttura dell'ente.
\begin{figure}[H]
\centering
\noindent\makebox[\textwidth]{\includegraphics[width=\textwidth]{Img/quizzipedia-client-modelclient-organization-institution.pdf}}
\caption[Schema Classe Institution]{Schema Classe Quizzipedia::Client::ModelClient::Organization::Institution}
\end{figure}
\subsection{Quizzipedia::Client::ModelClient::Requests}
Questo package contiene le classi necessarie a gestire le richieste di ruolo e di classe degli utenti.
\begin{figure}[H]
\centering
\noindent\makebox[\textwidth]{\includegraphics[width=\textwidth]{Img/quizzipedia-client-modelclient-requests.pdf}}
\caption[Schema Componente Quizzipedia::Client::ModelClient::Requests]{Schema Componente Quizzipedia::Client::ModelClient::Requests}
\end{figure}
\subsubsection{Interazioni con altre componenti}
\paragraph{Entranti}
\begin{itemize}
\item usata da Quizzipedia::Client::ModelClient::Users per Questo package contiene le classi necessarie a gestire le richieste di ruolo e di classe degli utenti
\end{itemize}
\subsubsection{Classe ClassList}
Questa classe gestisce le richieste da parte di docenti o studenti per l'assegnazione a una specifica classe.
\begin{figure}[H]
\centering
\noindent\makebox[\textwidth]{\includegraphics[width=\textwidth]{Img/quizzipedia-client-modelclient-requests-classlist.pdf}}
\caption[Schema Classe ClassList]{Schema Classe Quizzipedia::Client::ModelClient::Requests::ClassList}
\end{figure}
\paragraph{Relazioni con altre classi}
\subparagraph{Entranti}
\begin{itemize}
\item usata da Quizzipedia::Client::ControllerClient::CtrlRequests::CtrlRequestClass per Questo package contiene le classi necessarie a gestire le richieste di ruolo e di classe degli utenti
\item usata da Quizzipedia::Client::ModelClient::Users::Student per Questo package contiene le classi necessarie a gestire le richieste di ruolo e di classe degli utenti
\item usata da Quizzipedia::Client::ModelClient::Users::Teacher per Questo package contiene le classi necessarie a gestire le richieste di ruolo e di classe degli utenti
\end{itemize}
\subparagraph{Uscenti}
\begin{itemize}
\item usa Quizzipedia::Client::ModelClient::Requests::Request per Questo package contiene le classi necessarie a gestire le richieste di ruolo e di classe degli utenti
\end{itemize}
\subsubsection{Classe Request}
La classe memorizza l'utente che invia la richiesta di inserimento in una classe e la classe per cui ha fatto richiesta.
\begin{figure}[H]
\centering
\noindent\makebox[\textwidth]{\includegraphics[width=\textwidth]{Img/quizzipedia-client-modelclient-requests-request.pdf}}
\caption[Schema Classe Request]{Schema Classe Quizzipedia::Client::ModelClient::Requests::Request}
\end{figure}
\paragraph{Relazioni con altre classi}
\subparagraph{Entranti}
\begin{itemize}
\item usata da Quizzipedia::Client::ModelClient::Requests::ClassList per Questo package contiene le classi necessarie a gestire le richieste di ruolo e di classe degli utenti
\end{itemize}
\subsubsection{Classe RoleList}
Gli utenti senza ruolo inviano le proprie richieste per l'assegnazione al ruolo di studente o docente al responsabile di un ente. Questa classe gestisce tali richieste.
\begin{figure}[H]
\centering
\noindent\makebox[\textwidth]{\includegraphics[width=\textwidth]{Img/quizzipedia-client-modelclient-requests-rolelist.pdf}}
\caption[Schema Classe RoleList]{Schema Classe Quizzipedia::Client::ModelClient::Requests::RoleList}
\end{figure}
\paragraph{Relazioni con altre classi}
\subparagraph{Entranti}
\begin{itemize}
\item usata da Quizzipedia::Client::ControllerClient::CtrlRequests::CtrlRequestRole per Questo package contiene le classi necessarie a gestire le richieste di ruolo e di classe degli utenti
\item usata da Quizzipedia::Client::ModelClient::Users::Director per Questo package contiene le classi necessarie a gestire le richieste di ruolo e di classe degli utenti
\item usata da Quizzipedia::Client::ModelClient::Users::NoRole per Questo package contiene le classi necessarie a gestire le richieste di ruolo e di classe degli utenti
\end{itemize}
\subparagraph{Uscenti}
\begin{itemize}
\item usa Quizzipedia::Client::ModelClient::Users::NoRole per Questo package contiene le classi necessarie a gestire le richieste di ruolo e di classe degli utenti
\end{itemize}
\subsection{Quizzipedia::Client::ModelClient::Services}
Il package racchiude i modelli necessari alla creazione di domande e quiz, i servizi principali offerti dal nostro prodotto.
\begin{figure}[H]
\centering
\noindent\makebox[\textwidth]{\includegraphics[width=\textwidth]{Img/quizzipedia-client-modelclient-services.pdf}}
\caption[Schema Componente Quizzipedia::Client::ModelClient::Services]{Schema Componente Quizzipedia::Client::ModelClient::Services}
\end{figure}
\subsubsection{Componenti contenute}
\begin{itemize}
\item Quizzipedia::Client::ModelClient::Services::Questions
\end{itemize}
\subsubsection{Interazioni con altre componenti}
\paragraph{Entranti}
\begin{itemize}
\item usata da Quizzipedia::Client::ModelClient::Users per Il package racchiude i modelli necessari alla creazione di domande e quiz, i servizi principali offerti dal nostro prodotto
\end{itemize}
\subsubsection{Classe Info}
Riassume le informazioni principali su quiz e domande, necessarie per una presentazione sintetica e puntuale all'utente. È poi possibile risalire alla domanda o al quiz completi.
\begin{figure}[H]
\centering
\noindent\makebox[\textwidth]{\includegraphics[width=\textwidth]{Img/quizzipedia-client-modelclient-services-info.pdf}}
\caption[Schema Classe Info]{Schema Classe Quizzipedia::Client::ModelClient::Services::Info}
\end{figure}
\paragraph{Relazioni con altre classi}
\subparagraph{Entranti}
\begin{itemize}
\item usata da Quizzipedia::Client::ModelClient::Users::User per Il package racchiude i modelli necessari alla creazione di domande e quiz, i servizi principali offerti dal nostro prodotto
\end{itemize}
\subsubsection{Classe Quiz}
Include la struttura del quiz.
\begin{figure}[H]
\centering
\noindent\makebox[\textwidth]{\includegraphics[width=\textwidth]{Img/quizzipedia-client-modelclient-services-quiz.pdf}}
\caption[Schema Classe Quiz]{Schema Classe Quizzipedia::Client::ModelClient::Services::Quiz}
\end{figure}
\paragraph{Relazioni con altre classi}
\subparagraph{Entranti}
\begin{itemize}
\item usata da Quizzipedia::Client::ControllerClient::CtrlServices::CtrlQuiz per Il package racchiude i modelli necessari alla creazione di domande e quiz, i servizi principali offerti dal nostro prodotto
\item usata da Quizzipedia::Client::ModelClient::Users::Teacher per Il package racchiude i modelli necessari alla creazione di domande e quiz, i servizi principali offerti dal nostro prodotto
\end{itemize}
\subparagraph{Uscenti}
\begin{itemize}
\item usa Quizzipedia::Client::ModelClient::Organization::Class per Il package racchiude i modelli necessari alla creazione di domande e quiz, i servizi principali offerti dal nostro prodotto
\item usa Quizzipedia::Client::ModelClient::Services::Questions::GenericQuestion per Il package racchiude i modelli necessari alla creazione di domande e quiz, i servizi principali offerti dal nostro prodotto
\end{itemize}
\subsubsection{Classe Topics}
Modella la struttura necessaria a memorizzare la lista di argomenti. A ogni domanda e a ogni quiz verranno poi associati i relativi argomenti .
\begin{figure}[H]
\centering
\noindent\makebox[\textwidth]{\includegraphics[width=\textwidth]{Img/quizzipedia-client-modelclient-services-topics.pdf}}
\caption[Schema Classe Topics]{Schema Classe Quizzipedia::Client::ModelClient::Services::Topics}
\end{figure}
\paragraph{Relazioni con altre classi}
\subparagraph{Entranti}
\begin{itemize}
\item usata da Quizzipedia::Client::ControllerClient::CtrlServices::CtrlTopics per Il package racchiude i modelli necessari alla creazione di domande e quiz, i servizi principali offerti dal nostro prodotto
\item usata da Quizzipedia::Client::ModelClient::Users::Director per Il package racchiude i modelli necessari alla creazione di domande e quiz, i servizi principali offerti dal nostro prodotto
\end{itemize}
\subsection{Quizzipedia::Client::ModelClient::Services::Questions}
Descrive il modo in cui sono strutturati i vari tipi di domande che l'utente può incontrare durante la creazione o la compilazione di quiz.
\begin{figure}[H]
\centering
\noindent\makebox[\textwidth]{\includegraphics[width=\textwidth]{Img/quizzipedia-client-modelclient-services-questions.pdf}}
\caption[Schema Componente Quizzipedia::Client::ModelClient::Services::Questions]{Schema Componente Quizzipedia::Client::ModelClient::Services::Questions}
\end{figure}
\subsubsection{Classe Cell}
La classe descrive ogni singola riga (quindi ogni opzione) della colonna della domanda a collegamento.
\begin{figure}[H]
\centering
\noindent\makebox[\textwidth]{\includegraphics[width=\textwidth]{Img/quizzipedia-client-modelclient-services-questions-cell.pdf}}
\caption[Schema Classe Cell]{Schema Classe Quizzipedia::Client::ModelClient::Services::Questions::Cell}
\end{figure}
\paragraph{Relazioni con altre classi}
\subparagraph{Entranti}
\begin{itemize}
\item usata da Quizzipedia::Client::ModelClient::Services::Questions::Column per Descrive il modo in cui sono strutturati i vari tipi di domande che l'utente può incontrare durante la creazione o la compilazione di quiz
\end{itemize}
\subsubsection{Classe Column}
La classe descrive le colonne della domanda a collegamenti.
\begin{figure}[H]
\centering
\noindent\makebox[\textwidth]{\includegraphics[width=\textwidth]{Img/quizzipedia-client-modelclient-services-questions-column.pdf}}
\caption[Schema Classe Column]{Schema Classe Quizzipedia::Client::ModelClient::Services::Questions::Column}
\end{figure}
\paragraph{Relazioni con altre classi}
\subparagraph{Entranti}
\begin{itemize}
\item usata da Quizzipedia::Client::ModelClient::Services::Questions::MatchingQ per Descrive il modo in cui sono strutturati i vari tipi di domande che l'utente può incontrare durante la creazione o la compilazione di quiz
\end{itemize}
\subparagraph{Uscenti}
\begin{itemize}
\item usa Quizzipedia::Client::ModelClient::Services::Questions::Cell per Descrive il modo in cui sono strutturati i vari tipi di domande che l'utente può incontrare durante la creazione o la compilazione di quiz
\end{itemize}
\subsubsection{Classe CompletionQ}
Descrive le domande a completamento. Il docente fornirà un testo incompleto e una lista di possibili completamenti; lo studente dovrà inserire le parole adeguate nella giusta posizione.
\begin{figure}[H]
\centering
\noindent\makebox[\textwidth]{\includegraphics[width=\textwidth]{Img/quizzipedia-client-modelclient-services-questions-completionq.pdf}}
\caption[Schema Classe CompletionQ]{Schema Classe Quizzipedia::Client::ModelClient::Services::Questions::CompletionQ}
\end{figure}
\paragraph{Relazioni con altre classi}
\subparagraph{Uscenti}
\begin{itemize}
\item usa Quizzipedia::Client::ModelClient::Services::Questions::GenericQuestion per Descrive il modo in cui sono strutturati i vari tipi di domande che l'utente può incontrare durante la creazione o la compilazione di quiz
\end{itemize}
\subsubsection{Classe GenericQuestion}
Descrive le parti comuni a tutti i tipi di domanda presenti nel sistema.
\begin{figure}[H]
\centering
\noindent\makebox[\textwidth]{\includegraphics[width=\textwidth]{Img/quizzipedia-client-modelclient-services-questions-genericquestion.pdf}}
\caption[Schema Classe GenericQuestion]{Schema Classe Quizzipedia::Client::ModelClient::Services::Questions::GenericQuestion}
\end{figure}
\paragraph{Relazioni con altre classi}
\subparagraph{Entranti}
\begin{itemize}
\item usata da Quizzipedia::Client::ControllerClient::CtrlServices::CtrlQuestion per Descrive il modo in cui sono strutturati i vari tipi di domande che l'utente può incontrare durante la creazione o la compilazione di quiz
\item usata da Quizzipedia::Client::ModelClient::Services::Questions::CompletionQ per Descrive il modo in cui sono strutturati i vari tipi di domande che l'utente può incontrare durante la creazione o la compilazione di quiz
\item usata da Quizzipedia::Client::ModelClient::Services::Questions::MatchingQ per Descrive il modo in cui sono strutturati i vari tipi di domande che l'utente può incontrare durante la creazione o la compilazione di quiz
\item usata da Quizzipedia::Client::ModelClient::Services::Questions::MultipleChoiceQ per Descrive il modo in cui sono strutturati i vari tipi di domande che l'utente può incontrare durante la creazione o la compilazione di quiz
\item usata da Quizzipedia::Client::ModelClient::Services::Questions::ShortAnswerQ per Descrive il modo in cui sono strutturati i vari tipi di domande che l'utente può incontrare durante la creazione o la compilazione di quiz
\item usata da Quizzipedia::Client::ModelClient::Services::Questions::TrueFalseQ per Descrive il modo in cui sono strutturati i vari tipi di domande che l'utente può incontrare durante la creazione o la compilazione di quiz
\item usata da Quizzipedia::Client::ModelClient::Services::Quiz per Descrive il modo in cui sono strutturati i vari tipi di domande che l'utente può incontrare durante la creazione o la compilazione di quiz
\item usata da Quizzipedia::Client::ModelClient::Users::Teacher per Descrive il modo in cui sono strutturati i vari tipi di domande che l'utente può incontrare durante la creazione o la compilazione di quiz
\end{itemize}
\subsubsection{Classe MatchingQ}
La struttura descrive le domande a collegamento. L'utente dovrà formare la risposta collegando le entrate da un numero variabile di colonne .
\begin{figure}[H]
\centering
\noindent\makebox[\textwidth]{\includegraphics[width=\textwidth]{Img/quizzipedia-client-modelclient-services-questions-matchingq.pdf}}
\caption[Schema Classe MatchingQ]{Schema Classe Quizzipedia::Client::ModelClient::Services::Questions::MatchingQ}
\end{figure}
\paragraph{Relazioni con altre classi}
\subparagraph{Uscenti}
\begin{itemize}
\item usa Quizzipedia::Client::ModelClient::Services::Questions::Column per Descrive il modo in cui sono strutturati i vari tipi di domande che l'utente può incontrare durante la creazione o la compilazione di quiz
\item usa Quizzipedia::Client::ModelClient::Services::Questions::GenericQuestion per Descrive il modo in cui sono strutturati i vari tipi di domande che l'utente può incontrare durante la creazione o la compilazione di quiz
\end{itemize}
\subsubsection{Classe MultipleChoiceQ}
La struttura descrive le domande a scelta multipla; viene presentata una lista di opzioni tra cui scegliere quelle corrette.
\begin{figure}[H]
\centering
\noindent\makebox[\textwidth]{\includegraphics[width=\textwidth]{Img/quizzipedia-client-modelclient-services-questions-multiplechoiceq.pdf}}
\caption[Schema Classe MultipleChoiceQ]{Schema Classe Quizzipedia::Client::ModelClient::Services::Questions::MultipleChoiceQ}
\end{figure}
\paragraph{Relazioni con altre classi}
\subparagraph{Uscenti}
\begin{itemize}
\item usa Quizzipedia::Client::ModelClient::Services::Questions::GenericQuestion per Descrive il modo in cui sono strutturati i vari tipi di domande che l'utente può incontrare durante la creazione o la compilazione di quiz
\end{itemize}
\subsubsection{Classe ShortAnswerQ}
La struttura descrive le domande aperte, ovvero quelle la cui risposta consiste in un termine o una frase specifici.
\begin{figure}[H]
\centering
\noindent\makebox[\textwidth]{\includegraphics[width=\textwidth]{Img/quizzipedia-client-modelclient-services-questions-shortanswerq.pdf}}
\caption[Schema Classe ShortAnswerQ]{Schema Classe Quizzipedia::Client::ModelClient::Services::Questions::ShortAnswerQ}
\end{figure}
\paragraph{Relazioni con altre classi}
\subparagraph{Uscenti}
\begin{itemize}
\item usa Quizzipedia::Client::ModelClient::Services::Questions::GenericQuestion per Descrive il modo in cui sono strutturati i vari tipi di domande che l'utente può incontrare durante la creazione o la compilazione di quiz
\end{itemize}
\subsubsection{Classe TrueFalseQ}
Viene descritta la struttura delle domande che prevedono di decidere la veridicità di un'affermazione.
\begin{figure}[H]
\centering
\noindent\makebox[\textwidth]{\includegraphics[width=\textwidth]{Img/quizzipedia-client-modelclient-services-questions-truefalseq.pdf}}
\caption[Schema Classe TrueFalseQ]{Schema Classe Quizzipedia::Client::ModelClient::Services::Questions::TrueFalseQ}
\end{figure}
\paragraph{Relazioni con altre classi}
\subparagraph{Uscenti}
\begin{itemize}
\item usa Quizzipedia::Client::ModelClient::Services::Questions::GenericQuestion per Descrive il modo in cui sono strutturati i vari tipi di domande che l'utente può incontrare durante la creazione o la compilazione di quiz
\end{itemize}
\subsection{Quizzipedia::Client::ModelClient::Statistics}
Qui sono raccolte le classi con il compito di reperire informazioni sulle statistiche dal server e presentarle all'utente finale. Sono disponibili statistiche per le domande, per i quiz e per gli studenti di ogni classe.
\begin{figure}[H]
\centering
\noindent\makebox[\textwidth]{\includegraphics[width=\textwidth]{Img/quizzipedia-client-modelclient-statistics.pdf}}
\caption[Schema Componente Quizzipedia::Client::ModelClient::Statistics]{Schema Componente Quizzipedia::Client::ModelClient::Statistics}
\end{figure}
\subsubsection{Interazioni con altre componenti}
\paragraph{Entranti}
\begin{itemize}
\item usata da Quizzipedia::Client::ModelClient::Users per Qui sono raccolte le classi con il compito di reperire informazioni sulle statistiche dal server e presentarle all'utente finale. Sono disponibili statistiche per le domande, per i quiz e per gli studenti di ogni classe
\end{itemize}
\subsubsection{Classe ClassQuiz}
La classe raccoglie le statistiche riguardanti gli studenti di una classe relativamente a un quiz assegnato.
\begin{figure}[H]
\centering
\noindent\makebox[\textwidth]{\includegraphics[width=\textwidth]{Img/quizzipedia-client-modelclient-statistics-classquiz.pdf}}
\caption[Schema Classe ClassQuiz]{Schema Classe Quizzipedia::Client::ModelClient::Statistics::ClassQuiz}
\end{figure}
\paragraph{Relazioni con altre classi}
\subparagraph{Entranti}
\begin{itemize}
\item usata da Quizzipedia::Client::ControllerClient::CtrlStatistics::CtrlStats per Qui sono raccolte le classi con il compito di reperire informazioni sulle statistiche dal server e presentarle all'utente finale. Sono disponibili statistiche per le domande, per i quiz e per gli studenti di ogni classe
\item usata da Quizzipedia::Client::ModelClient::Users::Teacher per Qui sono raccolte le classi con il compito di reperire informazioni sulle statistiche dal server e presentarle all'utente finale. Sono disponibili statistiche per le domande, per i quiz e per gli studenti di ogni classe
\end{itemize}
\subparagraph{Uscenti}
\begin{itemize}
\item usa Quizzipedia::Client::ModelClient::Statistics::StudentStatistics per Qui sono raccolte le classi con il compito di reperire informazioni sulle statistiche dal server e presentarle all'utente finale. Sono disponibili statistiche per le domande, per i quiz e per gli studenti di ogni classe
\end{itemize}
\subsubsection{Classe GenericQuestions}
Le statistiche relative a più domande vengono raccolte e organizzate per permetterne la visualizzazione agli utenti.
\begin{figure}[H]
\centering
\noindent\makebox[\textwidth]{\includegraphics[width=\textwidth]{Img/quizzipedia-client-modelclient-statistics-genericquestions.pdf}}
\caption[Schema Classe GenericQuestions]{Schema Classe Quizzipedia::Client::ModelClient::Statistics::GenericQuestions}
\end{figure}
\paragraph{Relazioni con altre classi}
\subparagraph{Entranti}
\begin{itemize}
\item usata da Quizzipedia::Client::ControllerClient::CtrlStatistics::CtrlStats per Qui sono raccolte le classi con il compito di reperire informazioni sulle statistiche dal server e presentarle all'utente finale. Sono disponibili statistiche per le domande, per i quiz e per gli studenti di ogni classe
\item usata da Quizzipedia::Client::ModelClient::Users::Teacher per Qui sono raccolte le classi con il compito di reperire informazioni sulle statistiche dal server e presentarle all'utente finale. Sono disponibili statistiche per le domande, per i quiz e per gli studenti di ogni classe
\end{itemize}
\subparagraph{Uscenti}
\begin{itemize}
\item usa Quizzipedia::Client::ModelClient::Statistics::QuestionStatistics per Qui sono raccolte le classi con il compito di reperire informazioni sulle statistiche dal server e presentarle all'utente finale. Sono disponibili statistiche per le domande, per i quiz e per gli studenti di ogni classe
\end{itemize}
\subsubsection{Classe GenericQuiz}
Le statistiche relative a più quiz vengono raccolte e organizzate per permetterne la visualizzazione agli utenti.
\begin{figure}[H]
\centering
\noindent\makebox[\textwidth]{\includegraphics[width=\textwidth]{Img/quizzipedia-client-modelclient-statistics-genericquiz.pdf}}
\caption[Schema Classe GenericQuiz]{Schema Classe Quizzipedia::Client::ModelClient::Statistics::GenericQuiz}
\end{figure}
\paragraph{Relazioni con altre classi}
\subparagraph{Entranti}
\begin{itemize}
\item usata da Quizzipedia::Client::ControllerClient::CtrlStatistics::CtrlStats per Qui sono raccolte le classi con il compito di reperire informazioni sulle statistiche dal server e presentarle all'utente finale. Sono disponibili statistiche per le domande, per i quiz e per gli studenti di ogni classe
\item usata da Quizzipedia::Client::ModelClient::Users::Teacher per Qui sono raccolte le classi con il compito di reperire informazioni sulle statistiche dal server e presentarle all'utente finale. Sono disponibili statistiche per le domande, per i quiz e per gli studenti di ogni classe
\end{itemize}
\subparagraph{Uscenti}
\begin{itemize}
\item usa Quizzipedia::Client::ModelClient::Statistics::QuizStatistics per Qui sono raccolte le classi con il compito di reperire informazioni sulle statistiche dal server e presentarle all'utente finale. Sono disponibili statistiche per le domande, per i quiz e per gli studenti di ogni classe
\end{itemize}
\subsubsection{Classe QuestionStatistics}
La classe raccoglie le statistiche principali riguardanti una singola domanda. Da qui è poi possibile risalire alla domanda.
\begin{figure}[H]
\centering
\noindent\makebox[\textwidth]{\includegraphics[width=\textwidth]{Img/quizzipedia-client-modelclient-statistics-questionstatistics.pdf}}
\caption[Schema Classe QuestionStatistics]{Schema Classe Quizzipedia::Client::ModelClient::Statistics::QuestionStatistics}
\end{figure}
\paragraph{Relazioni con altre classi}
\subparagraph{Entranti}
\begin{itemize}
\item usata da Quizzipedia::Client::ModelClient::Statistics::GenericQuestions per Qui sono raccolte le classi con il compito di reperire informazioni sulle statistiche dal server e presentarle all'utente finale. Sono disponibili statistiche per le domande, per i quiz e per gli studenti di ogni classe
\end{itemize}
\subsubsection{Classe QuizStatistics}
La classe raccoglie le statistiche principali riguardanti un singolo quiz. Da qui è poi possibile ottenere il quiz in questione.
\begin{figure}[H]
\centering
\noindent\makebox[\textwidth]{\includegraphics[width=\textwidth]{Img/quizzipedia-client-modelclient-statistics-quizstatistics.pdf}}
\caption[Schema Classe QuizStatistics]{Schema Classe Quizzipedia::Client::ModelClient::Statistics::QuizStatistics}
\end{figure}
\paragraph{Relazioni con altre classi}
\subparagraph{Entranti}
\begin{itemize}
\item usata da Quizzipedia::Client::ModelClient::Statistics::GenericQuiz per Qui sono raccolte le classi con il compito di reperire informazioni sulle statistiche dal server e presentarle all'utente finale. Sono disponibili statistiche per le domande, per i quiz e per gli studenti di ogni classe
\end{itemize}
\subsubsection{Classe StudentStatistics}
Qui è memorizzata la struttura che permette di associare a un utente le statistiche riguardanti un quiz (voto, superamento).
\begin{figure}[H]
\centering
\noindent\makebox[\textwidth]{\includegraphics[width=\textwidth]{Img/quizzipedia-client-modelclient-statistics-studentstatistics.pdf}}
\caption[Schema Classe StudentStatistics]{Schema Classe Quizzipedia::Client::ModelClient::Statistics::StudentStatistics}
\end{figure}
\paragraph{Relazioni con altre classi}
\subparagraph{Entranti}
\begin{itemize}
\item usata da Quizzipedia::Client::ModelClient::Statistics::ClassQuiz per Qui sono raccolte le classi con il compito di reperire informazioni sulle statistiche dal server e presentarle all'utente finale. Sono disponibili statistiche per le domande, per i quiz e per gli studenti di ogni classe
\end{itemize}
\subsection{Quizzipedia::Client::ModelClient::Users}
Raccoglie le classi necessarie a descrivere le diverse tipologie di utente che possono accedere al sistema.
\begin{figure}[H]
\centering
\noindent\makebox[\textwidth]{\includegraphics[width=\textwidth]{Img/quizzipedia-client-modelclient-users.pdf}}
\caption[Schema Componente Quizzipedia::Client::ModelClient::Users]{Schema Componente Quizzipedia::Client::ModelClient::Users}
\end{figure}
\subsubsection{Interazioni con altre componenti}
\paragraph{Uscenti}
\begin{itemize}
\item usa Quizzipedia::Client::ModelClient::Organization per Raccoglie le classi necessarie a descrivere le diverse tipologie di utente che possono accedere al sistema
\item usa Quizzipedia::Client::ModelClient::Requests per Raccoglie le classi necessarie a descrivere le diverse tipologie di utente che possono accedere al sistema
\item usa Quizzipedia::Client::ModelClient::Services per Raccoglie le classi necessarie a descrivere le diverse tipologie di utente che possono accedere al sistema
\item usa Quizzipedia::Client::ModelClient::Statistics per Raccoglie le classi necessarie a descrivere le diverse tipologie di utente che possono accedere al sistema
\end{itemize}
\subsubsection{Classe AuthenticationData}
Questa classe gestisce le informazioni di autenticazione comuni a tutti gli utenti.
\begin{figure}[H]
\centering
\noindent\makebox[\textwidth]{\includegraphics[width=\textwidth]{Img/quizzipedia-client-modelclient-users-authenticationdata.pdf}}
\caption[Schema Classe AuthenticationData]{Schema Classe Quizzipedia::Client::ModelClient::Users::AuthenticationData}
\end{figure}
\paragraph{Relazioni con altre classi}
\subparagraph{Entranti}
\begin{itemize}
\item usata da Quizzipedia::Client::ModelClient::Users::User per Raccoglie le classi necessarie a descrivere le diverse tipologie di utente che possono accedere al sistema
\end{itemize}
\subsubsection{Classe Director}
Rappresenta un responsabile, ovvero colui che gestisce docenti e studenti per ogni ente del sistema.
\begin{figure}[H]
\centering
\noindent\makebox[\textwidth]{\includegraphics[width=\textwidth]{Img/quizzipedia-client-modelclient-users-director.pdf}}
\caption[Schema Classe Director]{Schema Classe Quizzipedia::Client::ModelClient::Users::Director}
\end{figure}
\paragraph{Relazioni con altre classi}
\subparagraph{Uscenti}
\begin{itemize}
\item usa Quizzipedia::Client::ModelClient::Organization::Class per Raccoglie le classi necessarie a descrivere le diverse tipologie di utente che possono accedere al sistema
\item usa Quizzipedia::Client::ModelClient::Requests::RoleList per Raccoglie le classi necessarie a descrivere le diverse tipologie di utente che possono accedere al sistema
\item usa Quizzipedia::Client::ModelClient::Services::Topics per Raccoglie le classi necessarie a descrivere le diverse tipologie di utente che possono accedere al sistema
\item usa Quizzipedia::Client::ModelClient::Users::User per Raccoglie le classi necessarie a descrivere le diverse tipologie di utente che possono accedere al sistema
\end{itemize}
\subsubsection{Classe NoRole}
Rappresenta gli utenti senza ruolo del sistema; coloro che si sono registrati e autenticati ma non hanno ancora fatto richiesta per l'assegnazione ad alcun ruolo.
\begin{figure}[H]
\centering
\noindent\makebox[\textwidth]{\includegraphics[width=\textwidth]{Img/quizzipedia-client-modelclient-users-norole.pdf}}
\caption[Schema Classe NoRole]{Schema Classe Quizzipedia::Client::ModelClient::Users::NoRole}
\end{figure}
\paragraph{Relazioni con altre classi}
\subparagraph{Entranti}
\begin{itemize}
\item usata da Quizzipedia::Client::ModelClient::Requests::RoleList per Raccoglie le classi necessarie a descrivere le diverse tipologie di utente che possono accedere al sistema
\end{itemize}
\subparagraph{Uscenti}
\begin{itemize}
\item usa Quizzipedia::Client::ModelClient::Requests::RoleList per Raccoglie le classi necessarie a descrivere le diverse tipologie di utente che possono accedere al sistema
\item usa Quizzipedia::Client::ModelClient::Users::User per Raccoglie le classi necessarie a descrivere le diverse tipologie di utente che possono accedere al sistema
\end{itemize}
\subsubsection{Classe Student}
Rappresenta uno studente del sistema e implementa le sue funzioni specifiche oltre a quelle ereditate da utente.
\begin{figure}[H]
\centering
\noindent\makebox[\textwidth]{\includegraphics[width=\textwidth]{Img/quizzipedia-client-modelclient-users-student.pdf}}
\caption[Schema Classe Student]{Schema Classe Quizzipedia::Client::ModelClient::Users::Student}
\end{figure}
\paragraph{Relazioni con altre classi}
\subparagraph{Uscenti}
\begin{itemize}
\item usa Quizzipedia::Client::ModelClient::Requests::ClassList per Raccoglie le classi necessarie a descrivere le diverse tipologie di utente che possono accedere al sistema
\item usa Quizzipedia::Client::ModelClient::Users::User per Raccoglie le classi necessarie a descrivere le diverse tipologie di utente che possono accedere al sistema
\end{itemize}
\subsubsection{Classe Teacher}
Rappresenta un docente del sistema e ne implementa le funzionalità specifiche in aggiunta a quelle comuni a tutti gli utenti.
\begin{figure}[H]
\centering
\noindent\makebox[\textwidth]{\includegraphics[width=\textwidth]{Img/quizzipedia-client-modelclient-users-teacher.pdf}}
\caption[Schema Classe Teacher]{Schema Classe Quizzipedia::Client::ModelClient::Users::Teacher}
\end{figure}
\paragraph{Relazioni con altre classi}
\subparagraph{Uscenti}
\begin{itemize}
\item usa Quizzipedia::Client::ModelClient::Requests::ClassList per Raccoglie le classi necessarie a descrivere le diverse tipologie di utente che possono accedere al sistema
\item usa Quizzipedia::Client::ModelClient::Services::Questions::GenericQuestion per Raccoglie le classi necessarie a descrivere le diverse tipologie di utente che possono accedere al sistema
\item usa Quizzipedia::Client::ModelClient::Services::Quiz per Raccoglie le classi necessarie a descrivere le diverse tipologie di utente che possono accedere al sistema
\item usa Quizzipedia::Client::ModelClient::Statistics::ClassQuiz per Raccoglie le classi necessarie a descrivere le diverse tipologie di utente che possono accedere al sistema
\item usa Quizzipedia::Client::ModelClient::Statistics::GenericQuestions per Raccoglie le classi necessarie a descrivere le diverse tipologie di utente che possono accedere al sistema
\item usa Quizzipedia::Client::ModelClient::Statistics::GenericQuiz per Raccoglie le classi necessarie a descrivere le diverse tipologie di utente che possono accedere al sistema
\item usa Quizzipedia::Client::ModelClient::Users::User per Raccoglie le classi necessarie a descrivere le diverse tipologie di utente che possono accedere al sistema
\end{itemize}
\subsubsection{Classe User}
Questa è una classe astratta e raccoglie le funzionalità comuni a tutti gli utenti.
\begin{figure}[H]
\centering
\noindent\makebox[\textwidth]{\includegraphics[width=\textwidth]{Img/quizzipedia-client-modelclient-users-user.pdf}}
\caption[Schema Classe User]{Schema Classe Quizzipedia::Client::ModelClient::Users::User}
\end{figure}
\paragraph{Relazioni con altre classi}
\subparagraph{Entranti}
\begin{itemize}
\item usata da Quizzipedia::Client::ControllerClient::CtrlUsers::CtrlData per Raccoglie le classi necessarie a descrivere le diverse tipologie di utente che possono accedere al sistema
\item usata da Quizzipedia::Client::ControllerClient::CtrlUsers::CtrlUserManager per Raccoglie le classi necessarie a descrivere le diverse tipologie di utente che possono accedere al sistema
\item usata da Quizzipedia::Client::ModelClient::Users::Director per Raccoglie le classi necessarie a descrivere le diverse tipologie di utente che possono accedere al sistema
\item usata da Quizzipedia::Client::ModelClient::Users::NoRole per Raccoglie le classi necessarie a descrivere le diverse tipologie di utente che possono accedere al sistema
\item usata da Quizzipedia::Client::ModelClient::Users::Student per Raccoglie le classi necessarie a descrivere le diverse tipologie di utente che possono accedere al sistema
\item usata da Quizzipedia::Client::ModelClient::Users::Teacher per Raccoglie le classi necessarie a descrivere le diverse tipologie di utente che possono accedere al sistema
\end{itemize}
\subparagraph{Uscenti}
\begin{itemize}
\item usa Quizzipedia::Client::ModelClient::Organization::Class per Raccoglie le classi necessarie a descrivere le diverse tipologie di utente che possono accedere al sistema
\item usa Quizzipedia::Client::ModelClient::Services::Info per Raccoglie le classi necessarie a descrivere le diverse tipologie di utente che possono accedere al sistema
\item usa Quizzipedia::Client::ModelClient::Users::AuthenticationData per Raccoglie le classi necessarie a descrivere le diverse tipologie di utente che possono accedere al sistema
\end{itemize}
\subsection{Quizzipedia::Client::ViewClient}
Racchiude tutte le componenti necessarie per presentare il prodotto all'utente.
\begin{figure}[H]
\centering
\noindent\makebox[\textwidth]{\includegraphics[width=\textwidth]{Img/quizzipedia-client-viewclient.pdf}}
\caption[Schema Componente Quizzipedia::Client::ViewClient]{Schema Componente Quizzipedia::Client::ViewClient}
\end{figure}
\subsubsection{Componenti contenute}
\begin{itemize}
\item Quizzipedia::Client::ViewClient::ViewErrors
\item Quizzipedia::Client::ViewClient::ViewOrgManager
\item Quizzipedia::Client::ViewClient::ViewQuestionManager
\item Quizzipedia::Client::ViewClient::ViewQuizManager
\item Quizzipedia::Client::ViewClient::ViewQuizSolver
\item Quizzipedia::Client::ViewClient::ViewRequests
\item Quizzipedia::Client::ViewClient::ViewSearch
\item Quizzipedia::Client::ViewClient::ViewStatistics
\item Quizzipedia::Client::ViewClient::ViewTopicManager
\item Quizzipedia::Client::ViewClient::ViewUsers
\end{itemize}
\subsubsection{Interazioni con altre componenti}
\paragraph{Uscenti}
\begin{itemize}
\item usa Quizzipedia::Client::ControllerClient per Racchiude tutte le componenti necessarie per presentare il prodotto all'utente
\end{itemize}
\subsection{Quizzipedia::Client::ViewClient::ViewErrors}
Il package raccoglie tutte le finestre di errore che il sistema può, eventualmente, visualizzare.
\begin{figure}[H]
\centering
\noindent\makebox[\textwidth]{\includegraphics[width=\textwidth]{Img/quizzipedia-client-viewclient.pdf}}
\caption[Schema Componente Quizzipedia::Client::ViewClient::ViewErrors]{Schema Componente Quizzipedia::Client::ViewClient::ViewErrors}
\end{figure}
\subsubsection{Classe ViewAuthenticationErr}
La classe si occupa della visualizzazione del messaggio di errore in fase di autenticazione.
\begin{figure}[H]
\centering
\noindent\makebox[\textwidth]{\includegraphics[width=\textwidth]{Img/quizzipedia-client-viewclient-viewerrors-viewauthenticationerr.pdf}}
\caption[Schema Classe ViewAuthenticationErr]{Schema Classe Quizzipedia::Client::ViewClient::ViewErrors::ViewAuthenticationErr}
\end{figure}
\subsubsection{Classe ViewModifyPwErr}
La classe si occupa della visualizzazione del messaggio di errore in fase di modifica della password.
\begin{figure}[H]
\centering
\noindent\makebox[\textwidth]{\includegraphics[width=\textwidth]{Img/quizzipedia-client-viewclient-viewerrors-viewmodifypwerr.pdf}}
\caption[Schema Classe ViewModifyPwErr]{Schema Classe Quizzipedia::Client::ViewClient::ViewErrors::ViewModifyPwErr}
\end{figure}
\subsubsection{Classe ViewModifyQuizErr}
La classe si occupa della visualizzazione del messaggio di errore in fase di modifica di un quiz.
\begin{figure}[H]
\centering
\noindent\makebox[\textwidth]{\includegraphics[width=\textwidth]{Img/quizzipedia-client-viewclient-viewerrors-viewmodifyquizerr.pdf}}
\caption[Schema Classe ViewModifyQuizErr]{Schema Classe Quizzipedia::Client::ViewClient::ViewErrors::ViewModifyQuizErr}
\end{figure}
\subsubsection{Classe ViewRecoveryErr}
La classe si occupa della visualizzazione del messaggio di errore in fase di recupero password.
\begin{figure}[H]
\centering
\noindent\makebox[\textwidth]{\includegraphics[width=\textwidth]{Img/quizzipedia-client-viewclient-viewerrors-viewrecoveryerr.pdf}}
\caption[Schema Classe ViewRecoveryErr]{Schema Classe Quizzipedia::Client::ViewClient::ViewErrors::ViewRecoveryErr}
\end{figure}
\subsubsection{Classe ViewRegistrationErr}
La classe si occupa della visualizzazione del messaggio di errore in fase di registrazione.
\begin{figure}[H]
\centering
\noindent\makebox[\textwidth]{\includegraphics[width=\textwidth]{Img/quizzipedia-client-viewclient-viewerrors-viewregistrationerr.pdf}}
\caption[Schema Classe ViewRegistrationErr]{Schema Classe Quizzipedia::Client::ViewClient::ViewErrors::ViewRegistrationErr}
\end{figure}
\subsection{Quizzipedia::Client::ViewClient::ViewOrgManager}
Qui sono raccolte le classi responsabili della presentazione delle pagine da cui sarà possibile gestire le classi e gli enti.
\begin{figure}[H]
\centering
\noindent\makebox[\textwidth]{\includegraphics[width=\textwidth]{Img/quizzipedia-client-viewclient-vieworgmanager.pdf}}
\caption[Schema Componente Quizzipedia::Client::ViewClient::ViewOrgManager]{Schema Componente Quizzipedia::Client::ViewClient::ViewOrgManager}
\end{figure}
\subsubsection{Interazioni con altre componenti}
\paragraph{Uscenti}
\begin{itemize}
\item usa Quizzipedia::Client::ControllerClient::CtrlOrganization per Qui sono raccolte le classi responsabili della presentazione delle pagine da cui sarà possibile gestire le classi e gli enti
\end{itemize}
\subsubsection{Classe ViewCreateClass}
Classe responsabile della creazione della pagina da cui sarà possibile creare una nuova classe.
\begin{figure}[H]
\centering
\noindent\makebox[\textwidth]{\includegraphics[width=\textwidth]{Img/quizzipedia-client-viewclient-vieworgmanager-viewcreateclass.pdf}}
\caption[Schema Classe ViewCreateClass]{Schema Classe Quizzipedia::Client::ViewClient::ViewOrgManager::ViewCreateClass}
\end{figure}
\paragraph{Relazioni con altre classi}
\subparagraph{Uscenti}
\begin{itemize}
\item usa Quizzipedia::Client::ControllerClient::CtrlOrganization::CtrlClass per Qui sono raccolte le classi responsabili della presentazione delle pagine da cui sarà possibile gestire le classi e gli enti
\end{itemize}
\subsubsection{Classe ViewModifyOrg}
Presenta all'utente la pagina da cui sarà possibile modificare le informazioni su una classe o su un ente esistente.
\begin{figure}[H]
\centering
\noindent\makebox[\textwidth]{\includegraphics[width=\textwidth]{Img/quizzipedia-client-viewclient-vieworgmanager-viewmodifyorg.pdf}}
\caption[Schema Classe ViewModifyOrg]{Schema Classe Quizzipedia::Client::ViewClient::ViewOrgManager::ViewModifyOrg}
\end{figure}
\paragraph{Relazioni con altre classi}
\subparagraph{Uscenti}
\begin{itemize}
\item usa Quizzipedia::Client::ControllerClient::CtrlOrganization::CtrlInstitution per Qui sono raccolte le classi responsabili della presentazione delle pagine da cui sarà possibile gestire le classi e gli enti
\end{itemize}
\subsubsection{Classe ViewUsersClassList}
La classe si occupa di presentare una lista degli utenti iscritti alla classe e altre informazioni aggiuntive.
\begin{figure}[H]
\centering
\noindent\makebox[\textwidth]{\includegraphics[width=\textwidth]{Img/quizzipedia-client-viewclient-vieworgmanager-viewusersclasslist.pdf}}
\caption[Schema Classe ViewUsersClassList]{Schema Classe Quizzipedia::Client::ViewClient::ViewOrgManager::ViewUsersClassList}
\end{figure}
\paragraph{Relazioni con altre classi}
\subparagraph{Uscenti}
\begin{itemize}
\item usa Quizzipedia::Client::ControllerClient::CtrlOrganization::CtrlClass per Qui sono raccolte le classi responsabili della presentazione delle pagine da cui sarà possibile gestire le classi e gli enti
\end{itemize}
\subsection{Quizzipedia::Client::ViewClient::ViewQuestionManager}
Qui sono raccolte le classi responsabili della presentazione delle pagine da cui sarà possibile gestire le domande.
\begin{figure}[H]
\centering
\noindent\makebox[\textwidth]{\includegraphics[width=\textwidth]{Img/quizzipedia-client-viewclient-viewquestionmanager.pdf}}
\caption[Schema Componente Quizzipedia::Client::ViewClient::ViewQuestionManager]{Schema Componente Quizzipedia::Client::ViewClient::ViewQuestionManager}
\end{figure}
\subsubsection{Interazioni con altre componenti}
\paragraph{Uscenti}
\begin{itemize}
\item usa Quizzipedia::Client::ControllerClient::CtrlServices per Qui sono raccolte le classi responsabili della presentazione delle pagine da cui sarà possibile gestire le domande
\end{itemize}
\subsubsection{Classe ViewCreateQuestion}
Presenta la pagina da cui sarà possibile creare una nuova domanda.
\begin{figure}[H]
\centering
\noindent\makebox[\textwidth]{\includegraphics[width=\textwidth]{Img/quizzipedia-client-viewclient-viewquestionmanager-viewcreatequestion.pdf}}
\caption[Schema Classe ViewCreateQuestion]{Schema Classe Quizzipedia::Client::ViewClient::ViewQuestionManager::ViewCreateQuestion}
\end{figure}
\paragraph{Relazioni con altre classi}
\subparagraph{Uscenti}
\begin{itemize}
\item usa Quizzipedia::Client::ControllerClient::CtrlServices::CtrlQuestion per Qui sono raccolte le classi responsabili della presentazione delle pagine da cui sarà possibile gestire le domande
\end{itemize}
\subsubsection{Classe ViewModifyQuestion}
Gestisce la visualizzazione della pagina da cui è possibile modificare una domanda esistente.
\begin{figure}[H]
\centering
\noindent\makebox[\textwidth]{\includegraphics[width=\textwidth]{Img/quizzipedia-client-viewclient-viewquestionmanager-viewmodifyquestion.pdf}}
\caption[Schema Classe ViewModifyQuestion]{Schema Classe Quizzipedia::Client::ViewClient::ViewQuestionManager::ViewModifyQuestion}
\end{figure}
\paragraph{Relazioni con altre classi}
\subparagraph{Uscenti}
\begin{itemize}
\item usa Quizzipedia::Client::ControllerClient::CtrlServices::CtrlQuestion per Qui sono raccolte le classi responsabili della presentazione delle pagine da cui sarà possibile gestire le domande
\end{itemize}
\subsubsection{Classe ViewQuestionList}
Presenta all'utente il pannello da cui sarà possibile visualizzare una lista di informazioni riassuntive sulle domande e compiere alcune operazioni su di esse.
\begin{figure}[H]
\centering
\noindent\makebox[\textwidth]{\includegraphics[width=\textwidth]{Img/quizzipedia-client-viewclient-viewquestionmanager-viewquestionlist.pdf}}
\caption[Schema Classe ViewQuestionList]{Schema Classe Quizzipedia::Client::ViewClient::ViewQuestionManager::ViewQuestionList}
\end{figure}
\paragraph{Relazioni con altre classi}
\subparagraph{Uscenti}
\begin{itemize}
\item usa Quizzipedia::Client::ControllerClient::CtrlServices::CtrlQuestion per Qui sono raccolte le classi responsabili della presentazione delle pagine da cui sarà possibile gestire le domande
\end{itemize}
\subsection{Quizzipedia::Client::ViewClient::ViewQuizManager}
Qui sono raccolte le classi responsabili della presentazione delle pagine da cui sarà possibile gestire i quiz.
\begin{figure}[H]
\centering
\noindent\makebox[\textwidth]{\includegraphics[width=\textwidth]{Img/quizzipedia-client-viewclient-viewquizmanager.pdf}}
\caption[Schema Componente Quizzipedia::Client::ViewClient::ViewQuizManager]{Schema Componente Quizzipedia::Client::ViewClient::ViewQuizManager}
\end{figure}
\subsubsection{Interazioni con altre componenti}
\paragraph{Uscenti}
\begin{itemize}
\item usa Quizzipedia::Client::ControllerClient::CtrlServices per Qui sono raccolte le classi responsabili della presentazione delle pagine da cui sarà possibile gestire i quiz
\end{itemize}
\subsubsection{Classe ViewCreateQuiz}
Presenta la pagina da cui sarà possibile creare un nuovo quiz.
\begin{figure}[H]
\centering
\noindent\makebox[\textwidth]{\includegraphics[width=\textwidth]{Img/quizzipedia-client-viewclient-viewquizmanager-viewcreatequiz.pdf}}
\caption[Schema Classe ViewCreateQuiz]{Schema Classe Quizzipedia::Client::ViewClient::ViewQuizManager::ViewCreateQuiz}
\end{figure}
\paragraph{Relazioni con altre classi}
\subparagraph{Uscenti}
\begin{itemize}
\item usa Quizzipedia::Client::ControllerClient::CtrlServices::CtrlQuiz per Qui sono raccolte le classi responsabili della presentazione delle pagine da cui sarà possibile gestire i quiz
\end{itemize}
\subsubsection{Classe ViewModifyQuiz}
Presenta all'utente una pagina da cui è possibile modificare un quiz esistente.
\begin{figure}[H]
\centering
\noindent\makebox[\textwidth]{\includegraphics[width=\textwidth]{Img/quizzipedia-client-viewclient-viewquizmanager-viewmodifyquiz.pdf}}
\caption[Schema Classe ViewModifyQuiz]{Schema Classe Quizzipedia::Client::ViewClient::ViewQuizManager::ViewModifyQuiz}
\end{figure}
\paragraph{Relazioni con altre classi}
\subparagraph{Uscenti}
\begin{itemize}
\item usa Quizzipedia::Client::ControllerClient::CtrlServices::CtrlQuiz per Qui sono raccolte le classi responsabili della presentazione delle pagine da cui sarà possibile gestire i quiz
\end{itemize}
\subsubsection{Classe ViewQuizList}
Carica una pagina contenente una lista con informazioni riassuntive sui quiz e un pannello da cui sarà possibile svolgere delle operazioni sugli stessi.
\begin{figure}[H]
\centering
\noindent\makebox[\textwidth]{\includegraphics[width=\textwidth]{Img/quizzipedia-client-viewclient-viewquizmanager-viewquizlist.pdf}}
\caption[Schema Classe ViewQuizList]{Schema Classe Quizzipedia::Client::ViewClient::ViewQuizManager::ViewQuizList}
\end{figure}
\paragraph{Relazioni con altre classi}
\subparagraph{Uscenti}
\begin{itemize}
\item usa Quizzipedia::Client::ControllerClient::CtrlServices::CtrlQuiz per Qui sono raccolte le classi responsabili della presentazione delle pagine da cui sarà possibile gestire i quiz
\end{itemize}
\subsection{Quizzipedia::Client::ViewClient::ViewQuizSolver}
Il package raccoglie le classi necessarie alla visualizzazione delle pagine da cui sarà possibile svolgere quiz.
\begin{figure}[H]
\centering
\noindent\makebox[\textwidth]{\includegraphics[width=\textwidth]{Img/quizzipedia-client-viewclient-viewquizsolver.pdf}}
\caption[Schema Componente Quizzipedia::Client::ViewClient::ViewQuizSolver]{Schema Componente Quizzipedia::Client::ViewClient::ViewQuizSolver}
\end{figure}
\subsubsection{Componenti contenute}
\begin{itemize}
\item Quizzipedia::Client::ViewClient::ViewQuizSolver::ViewQuestionSolver
\end{itemize}
\subsubsection{Interazioni con altre componenti}
\paragraph{Uscenti}
\begin{itemize}
\item usa Quizzipedia::Client::ControllerClient::CtrlServices per Il package raccoglie le classi necessarie alla visualizzazione delle pagine da cui sarà possibile svolgere quiz
\end{itemize}
\subsubsection{Classe ViewResults}
La classe ha il compito di costruire la pagina da cui sarà possibile vedere l'esito di un quiz.
\begin{figure}[H]
\centering
\noindent\makebox[\textwidth]{\includegraphics[width=\textwidth]{Img/quizzipedia-client-viewclient-viewquizsolver-viewresults.pdf}}
\caption[Schema Classe ViewResults]{Schema Classe Quizzipedia::Client::ViewClient::ViewQuizSolver::ViewResults}
\end{figure}
\paragraph{Relazioni con altre classi}
\subparagraph{Uscenti}
\begin{itemize}
\item usa Quizzipedia::Client::ControllerClient::CtrlServices::CtrlQuiz per Il package raccoglie le classi necessarie alla visualizzazione delle pagine da cui sarà possibile svolgere quiz
\end{itemize}
\subsection{Quizzipedia::Client::ViewClient::ViewQuizSolver::ViewQuestionSolver}
Il package raccoglie le classi necessarie alla visualizzazione delle pagine da cui sarà possibile rispondere alle singole domande.
\begin{figure}[H]
\centering
\noindent\makebox[\textwidth]{\includegraphics[width=\textwidth]{Img/quizzipedia-client-viewclient-viewquizsolver-viewquestionsolver.pdf}}
\caption[Schema Componente Quizzipedia::Client::ViewClient::ViewQuizSolver::ViewQuestionSolver]{Schema Componente Quizzipedia::Client::ViewClient::ViewQuizSolver::ViewQuestionSolver}
\end{figure}
\subsubsection{Classe ViewCompletionQ}
Presenta all'utente la pagina da cui sarà possibile rispondere a una domanda a completamento.
\begin{figure}[H]
\centering
\noindent\makebox[\textwidth]{\includegraphics[width=\textwidth]{Img/quizzipedia-client-viewclient-viewquizsolver-viewquestionsolver-viewcompletionq.pdf}}
\caption[Schema Classe ViewCompletionQ]{Schema Classe Quizzipedia::Client::ViewClient::ViewQuizSolver::ViewQuestionSolver::ViewCompletionQ}
\end{figure}
\paragraph{Relazioni con altre classi}
\subparagraph{Uscenti}
\begin{itemize}
\item usa Quizzipedia::Client::ControllerClient::CtrlServices::CtrlQuestion per Il package raccoglie le classi necessarie alla visualizzazione delle pagine da cui sarà possibile rispondere alle singole domande
\end{itemize}
\subsubsection{Classe ViewMatchingQ}
Presenta all'utente la pagina da cui sarà possibile rispondere a una domanda a collegamenti.
\begin{figure}[H]
\centering
\noindent\makebox[\textwidth]{\includegraphics[width=\textwidth]{Img/quizzipedia-client-viewclient-viewquizsolver-viewquestionsolver-viewmatchingq.pdf}}
\caption[Schema Classe ViewMatchingQ]{Schema Classe Quizzipedia::Client::ViewClient::ViewQuizSolver::ViewQuestionSolver::ViewMatchingQ}
\end{figure}
\paragraph{Relazioni con altre classi}
\subparagraph{Uscenti}
\begin{itemize}
\item usa Quizzipedia::Client::ControllerClient::CtrlServices::CtrlQuestion per Il package raccoglie le classi necessarie alla visualizzazione delle pagine da cui sarà possibile rispondere alle singole domande
\end{itemize}
\subsubsection{Classe ViewMultipleChoiceQ}
Presenta all'utente la pagina da cui sarà possibile rispondere a una domanda a risposta multipla.
\begin{figure}[H]
\centering
\noindent\makebox[\textwidth]{\includegraphics[width=\textwidth]{Img/quizzipedia-client-viewclient-viewquizsolver-viewquestionsolver-viewmultiplechoiceq.pdf}}
\caption[Schema Classe ViewMultipleChoiceQ]{Schema Classe Quizzipedia::Client::ViewClient::ViewQuizSolver::ViewQuestionSolver::ViewMultipleChoiceQ}
\end{figure}
\paragraph{Relazioni con altre classi}
\subparagraph{Uscenti}
\begin{itemize}
\item usa Quizzipedia::Client::ControllerClient::CtrlServices::CtrlQuestion per Il package raccoglie le classi necessarie alla visualizzazione delle pagine da cui sarà possibile rispondere alle singole domande
\end{itemize}
\subsubsection{Classe ViewShortAnswerQ}
Presenta all'utente la pagina da cui sarà possibile rispondere a una domanda a risposta aperta.
\begin{figure}[H]
\centering
\noindent\makebox[\textwidth]{\includegraphics[width=\textwidth]{Img/quizzipedia-client-viewclient-viewquizsolver-viewquestionsolver-viewshortanswerq.pdf}}
\caption[Schema Classe ViewShortAnswerQ]{Schema Classe Quizzipedia::Client::ViewClient::ViewQuizSolver::ViewQuestionSolver::ViewShortAnswerQ}
\end{figure}
\paragraph{Relazioni con altre classi}
\subparagraph{Uscenti}
\begin{itemize}
\item usa Quizzipedia::Client::ControllerClient::CtrlServices::CtrlQuestion per Il package raccoglie le classi necessarie alla visualizzazione delle pagine da cui sarà possibile rispondere alle singole domande
\end{itemize}
\subsubsection{Classe ViewTrueFalseQ}
Presenta all'utente la pagina da cui sarà possibile rispondere a una domanda di tipo vero/falso.
\begin{figure}[H]
\centering
\noindent\makebox[\textwidth]{\includegraphics[width=\textwidth]{Img/quizzipedia-client-viewclient-viewquizsolver-viewquestionsolver-viewtruefalseq.pdf}}
\caption[Schema Classe ViewTrueFalseQ]{Schema Classe Quizzipedia::Client::ViewClient::ViewQuizSolver::ViewQuestionSolver::ViewTrueFalseQ}
\end{figure}
\paragraph{Relazioni con altre classi}
\subparagraph{Uscenti}
\begin{itemize}
\item usa Quizzipedia::Client::ControllerClient::CtrlServices::CtrlQuestion per Il package raccoglie le classi necessarie alla visualizzazione delle pagine da cui sarà possibile rispondere alle singole domande
\end{itemize}
\subsection{Quizzipedia::Client::ViewClient::ViewRequests}
Qui sono raccolte le pagine che permettono all'utente di gestire le richieste di ruolo e classe.
\begin{figure}[H]
\centering
\noindent\makebox[\textwidth]{\includegraphics[width=\textwidth]{Img/quizzipedia-client-viewclient-viewrequests.pdf}}
\caption[Schema Componente Quizzipedia::Client::ViewClient::ViewRequests]{Schema Componente Quizzipedia::Client::ViewClient::ViewRequests}
\end{figure}
\subsubsection{Interazioni con altre componenti}
\paragraph{Uscenti}
\begin{itemize}
\item usa Quizzipedia::Client::ControllerClient::CtrlRequests per Qui sono raccolte le pagine che permettono all'utente di gestire le richieste di ruolo e classe
\end{itemize}
\subsubsection{Classe RequestClass}
Costruisce la pagina da cui l'utente potrà richiedere di entrare in una classe.
\begin{figure}[H]
\centering
\noindent\makebox[\textwidth]{\includegraphics[width=\textwidth]{Img/quizzipedia-client-viewclient-viewrequests-requestclass.pdf}}
\caption[Schema Classe RequestClass]{Schema Classe Quizzipedia::Client::ViewClient::ViewRequests::RequestClass}
\end{figure}
\paragraph{Relazioni con altre classi}
\subparagraph{Uscenti}
\begin{itemize}
\item usa Quizzipedia::Client::ControllerClient::CtrlRequests::CtrlRequestClass per Qui sono raccolte le pagine che permettono all'utente di gestire le richieste di ruolo e classe
\end{itemize}
\subsubsection{Classe RequestRole}
Costruisce la pagina da cui l'utente potrà richiedere un ruolo (studente o docente).
\begin{figure}[H]
\centering
\noindent\makebox[\textwidth]{\includegraphics[width=\textwidth]{Img/quizzipedia-client-viewclient-viewrequests-requestrole.pdf}}
\caption[Schema Classe RequestRole]{Schema Classe Quizzipedia::Client::ViewClient::ViewRequests::RequestRole}
\end{figure}
\paragraph{Relazioni con altre classi}
\subparagraph{Uscenti}
\begin{itemize}
\item usa Quizzipedia::Client::ControllerClient::CtrlRequests::CtrlRequestRole per Qui sono raccolte le pagine che permettono all'utente di gestire le richieste di ruolo e classe
\end{itemize}
\subsubsection{Classe ViewClassList}
Classe responsabile della visualizzazione del pannello da cui sarà possibile gestire le richieste di inserimento in una classe.
\begin{figure}[H]
\centering
\noindent\makebox[\textwidth]{\includegraphics[width=\textwidth]{Img/quizzipedia-client-viewclient-viewrequests-viewclasslist.pdf}}
\caption[Schema Classe ViewClassList]{Schema Classe Quizzipedia::Client::ViewClient::ViewRequests::ViewClassList}
\end{figure}
\paragraph{Relazioni con altre classi}
\subparagraph{Uscenti}
\begin{itemize}
\item usa Quizzipedia::Client::ControllerClient::CtrlRequests::CtrlRequestClass per Qui sono raccolte le pagine che permettono all'utente di gestire le richieste di ruolo e classe
\end{itemize}
\subsubsection{Classe ViewRolesList}
Classe responsabile della visualizzazione del pannello da cui sarà possibile gestire le richieste di assegnazione di ruolo.
\begin{figure}[H]
\centering
\noindent\makebox[\textwidth]{\includegraphics[width=\textwidth]{Img/quizzipedia-client-viewclient-viewrequests-viewroleslist.pdf}}
\caption[Schema Classe ViewRolesList]{Schema Classe Quizzipedia::Client::ViewClient::ViewRequests::ViewRolesList}
\end{figure}
\paragraph{Relazioni con altre classi}
\subparagraph{Uscenti}
\begin{itemize}
\item usa Quizzipedia::Client::ControllerClient::CtrlRequests::CtrlRequestRole per Qui sono raccolte le pagine che permettono all'utente di gestire le richieste di ruolo e classe
\end{itemize}
\subsection{Quizzipedia::Client::ViewClient::ViewSearch}
Il package contiene le classi responsabili della creazione delle pagine da cui sarà possibile ricercare domande, quiz e classi .
\begin{figure}[H]
\centering
\noindent\makebox[\textwidth]{\includegraphics[width=\textwidth]{Img/quizzipedia-client-viewclient-viewsearch.pdf}}
\caption[Schema Componente Quizzipedia::Client::ViewClient::ViewSearch]{Schema Componente Quizzipedia::Client::ViewClient::ViewSearch}
\end{figure}
\subsubsection{Interazioni con altre componenti}
\paragraph{Uscenti}
\begin{itemize}
\item usa Quizzipedia::Client::ControllerClient::CtrlOrganization per Il package contiene le classi responsabili della creazione delle pagine da cui sarà possibile ricercare domande, quiz e classi 
\item usa Quizzipedia::Client::ControllerClient::CtrlServices per Il package contiene le classi responsabili della creazione delle pagine da cui sarà possibile ricercare domande, quiz e classi 
\end{itemize}
\subsubsection{Classe ViewSearchClass}
La classe carica la pagina da cui sarà possibile ricercare classi all'interno di un ente.
\begin{figure}[H]
\centering
\noindent\makebox[\textwidth]{\includegraphics[width=\textwidth]{Img/quizzipedia-client-viewclient-viewsearch-viewsearchclass.pdf}}
\caption[Schema Classe ViewSearchClass]{Schema Classe Quizzipedia::Client::ViewClient::ViewSearch::ViewSearchClass}
\end{figure}
\paragraph{Relazioni con altre classi}
\subparagraph{Uscenti}
\begin{itemize}
\item usa Quizzipedia::Client::ControllerClient::CtrlOrganization::CtrlInstitution per Il package contiene le classi responsabili della creazione delle pagine da cui sarà possibile ricercare domande, quiz e classi 
\end{itemize}
\subsubsection{Classe ViewSearchQuestion}
Classe che ha il compito di caricare la pagina da cui sarà possibile effettuare la ricerca di domande.
\begin{figure}[H]
\centering
\noindent\makebox[\textwidth]{\includegraphics[width=\textwidth]{Img/quizzipedia-client-viewclient-viewsearch-viewsearchquestion.pdf}}
\caption[Schema Classe ViewSearchQuestion]{Schema Classe Quizzipedia::Client::ViewClient::ViewSearch::ViewSearchQuestion}
\end{figure}
\paragraph{Relazioni con altre classi}
\subparagraph{Uscenti}
\begin{itemize}
\item usa Quizzipedia::Client::ControllerClient::CtrlServices::CtrlSearch per Il package contiene le classi responsabili della creazione delle pagine da cui sarà possibile ricercare domande, quiz e classi 
\end{itemize}
\subsubsection{Classe ViewSearchQuiz}
Raccoglie i metodi necessari alla creazione della pagina da cui sarà possibile cercare un quiz.
\begin{figure}[H]
\centering
\noindent\makebox[\textwidth]{\includegraphics[width=\textwidth]{Img/quizzipedia-client-viewclient-viewsearch-viewsearchquiz.pdf}}
\caption[Schema Classe ViewSearchQuiz]{Schema Classe Quizzipedia::Client::ViewClient::ViewSearch::ViewSearchQuiz}
\end{figure}
\paragraph{Relazioni con altre classi}
\subparagraph{Uscenti}
\begin{itemize}
\item usa Quizzipedia::Client::ControllerClient::CtrlServices::CtrlSearch per Il package contiene le classi responsabili della creazione delle pagine da cui sarà possibile ricercare domande, quiz e classi 
\end{itemize}
\subsection{Quizzipedia::Client::ViewClient::ViewStatistics}
Package che gestisce le pagine in cui verranno visualizzate le statistiche.
\begin{figure}[H]
\centering
\noindent\makebox[\textwidth]{\includegraphics[width=\textwidth]{Img/quizzipedia-client-viewclient-viewstatistics.pdf}}
\caption[Schema Componente Quizzipedia::Client::ViewClient::ViewStatistics]{Schema Componente Quizzipedia::Client::ViewClient::ViewStatistics}
\end{figure}
\subsubsection{Interazioni con altre componenti}
\paragraph{Uscenti}
\begin{itemize}
\item usa Quizzipedia::Client::ControllerClient::CtrlStatistics per Package che gestisce le pagine in cui verranno visualizzate le statistiche
\end{itemize}
\subsubsection{Classe ViewClassStats}
Vengono rappresentate le statistiche relative a una singola classe in relazione a un particolare quiz.
\begin{figure}[H]
\centering
\noindent\makebox[\textwidth]{\includegraphics[width=\textwidth]{Img/quizzipedia-client-viewclient-viewstatistics-viewclassstats.pdf}}
\caption[Schema Classe ViewClassStats]{Schema Classe Quizzipedia::Client::ViewClient::ViewStatistics::ViewClassStats}
\end{figure}
\paragraph{Relazioni con altre classi}
\subparagraph{Uscenti}
\begin{itemize}
\item usa Quizzipedia::Client::ControllerClient::CtrlStatistics::CtrlStats per Package che gestisce le pagine in cui verranno visualizzate le statistiche
\end{itemize}
\subsubsection{Classe ViewQuestionStats}
Vengono rappresentate le statistiche generali relative alle domande.
\begin{figure}[H]
\centering
\noindent\makebox[\textwidth]{\includegraphics[width=\textwidth]{Img/quizzipedia-client-viewclient-viewstatistics-viewquestionstats.pdf}}
\caption[Schema Classe ViewQuestionStats]{Schema Classe Quizzipedia::Client::ViewClient::ViewStatistics::ViewQuestionStats}
\end{figure}
\paragraph{Relazioni con altre classi}
\subparagraph{Uscenti}
\begin{itemize}
\item usa Quizzipedia::Client::ControllerClient::CtrlStatistics::CtrlStats per Package che gestisce le pagine in cui verranno visualizzate le statistiche
\end{itemize}
\subsubsection{Classe ViewQuizStats}
Vengono rappresentate le statistiche generali riguardanti i quiz.
\begin{figure}[H]
\centering
\noindent\makebox[\textwidth]{\includegraphics[width=\textwidth]{Img/quizzipedia-client-viewclient-viewstatistics-viewquizstats.pdf}}
\caption[Schema Classe ViewQuizStats]{Schema Classe Quizzipedia::Client::ViewClient::ViewStatistics::ViewQuizStats}
\end{figure}
\paragraph{Relazioni con altre classi}
\subparagraph{Uscenti}
\begin{itemize}
\item usa Quizzipedia::Client::ControllerClient::CtrlStatistics::CtrlStats per Package che gestisce le pagine in cui verranno visualizzate le statistiche
\end{itemize}
\subsection{Quizzipedia::Client::ViewClient::ViewTopicManager}
Qui sono raccolte le classi responsabili della presentazione delle pagine da cui sarà possibile gestire gli argomenti di domande e quiz.
\begin{figure}[H]
\centering
\noindent\makebox[\textwidth]{\includegraphics[width=\textwidth]{Img/quizzipedia-client-viewclient-viewtopicmanager.pdf}}
\caption[Schema Componente Quizzipedia::Client::ViewClient::ViewTopicManager]{Schema Componente Quizzipedia::Client::ViewClient::ViewTopicManager}
\end{figure}
\subsubsection{Interazioni con altre componenti}
\paragraph{Uscenti}
\begin{itemize}
\item usa Quizzipedia::Client::ControllerClient::CtrlServices per Qui sono raccolte le classi responsabili della presentazione delle pagine da cui sarà possibile gestire gli argomenti di domande e quiz
\end{itemize}
\subsubsection{Classe ViewCreateTopic}
La classe caricherà la pagina da cui sarà possibile creare un nuovo argomento.
\begin{figure}[H]
\centering
\noindent\makebox[\textwidth]{\includegraphics[width=\textwidth]{Img/quizzipedia-client-viewclient-viewtopicmanager-viewcreatetopic.pdf}}
\caption[Schema Classe ViewCreateTopic]{Schema Classe Quizzipedia::Client::ViewClient::ViewTopicManager::ViewCreateTopic}
\end{figure}
\paragraph{Relazioni con altre classi}
\subparagraph{Uscenti}
\begin{itemize}
\item usa Quizzipedia::Client::ControllerClient::CtrlServices::CtrlTopics per Qui sono raccolte le classi responsabili della presentazione delle pagine da cui sarà possibile gestire gli argomenti di domande e quiz
\end{itemize}
\subsubsection{Classe ViewTopicList}
La classe è responsabile della creazione della pagina da cui sarà possibile vedere la lista degli argomenti disponibili.
\begin{figure}[H]
\centering
\noindent\makebox[\textwidth]{\includegraphics[width=\textwidth]{Img/quizzipedia-client-viewclient-viewtopicmanager-viewtopiclist.pdf}}
\caption[Schema Classe ViewTopicList]{Schema Classe Quizzipedia::Client::ViewClient::ViewTopicManager::ViewTopicList}
\end{figure}
\paragraph{Relazioni con altre classi}
\subparagraph{Uscenti}
\begin{itemize}
\item usa Quizzipedia::Client::ControllerClient::CtrlServices::CtrlTopics per Qui sono raccolte le classi responsabili della presentazione delle pagine da cui sarà possibile gestire gli argomenti di domande e quiz
\end{itemize}
\subsection{Quizzipedia::Client::ViewClient::ViewUsers}
Raccoglie le classi necessarie a presentare all'utente le pagine da cui visualizzare le informazioni che lo riguardano e compiere le funzioni principali.
\begin{figure}[H]
\centering
\noindent\makebox[\textwidth]{\includegraphics[width=\textwidth]{Img/quizzipedia-client-viewclient-viewusers.pdf}}
\caption[Schema Componente Quizzipedia::Client::ViewClient::ViewUsers]{Schema Componente Quizzipedia::Client::ViewClient::ViewUsers}
\end{figure}
\subsubsection{Interazioni con altre componenti}
\paragraph{Uscenti}
\begin{itemize}
\item usa Quizzipedia::Client::ControllerClient::CtrlUsers per Raccoglie le classi necessarie a presentare all'utente le pagine da cui visualizzare le informazioni che lo riguardano e compiere le funzioni principali
\end{itemize}
\subsubsection{Classe Login}
Presenta la pagina necessaria per effettuare il login nel sistema.
\begin{figure}[H]
\centering
\noindent\makebox[\textwidth]{\includegraphics[width=\textwidth]{Img/quizzipedia-client-viewclient-viewusers-login.pdf}}
\caption[Schema Classe Login]{Schema Classe Quizzipedia::Client::ViewClient::ViewUsers::Login}
\end{figure}
\paragraph{Relazioni con altre classi}
\subparagraph{Uscenti}
\begin{itemize}
\item usa Quizzipedia::Client::ControllerClient::CtrlUsers::CtrlData per Raccoglie le classi necessarie a presentare all'utente le pagine da cui visualizzare le informazioni che lo riguardano e compiere le funzioni principali
\end{itemize}
\subsubsection{Classe Logout}
Presenta la pagina necessaria per effettuare il logout dal sistema.
\begin{figure}[H]
\centering
\noindent\makebox[\textwidth]{\includegraphics[width=\textwidth]{Img/quizzipedia-client-viewclient-viewusers-logout.pdf}}
\caption[Schema Classe Logout]{Schema Classe Quizzipedia::Client::ViewClient::ViewUsers::Logout}
\end{figure}
\paragraph{Relazioni con altre classi}
\subparagraph{Uscenti}
\begin{itemize}
\item usa Quizzipedia::Client::ControllerClient::CtrlUsers::CtrlData per Raccoglie le classi necessarie a presentare all'utente le pagine da cui visualizzare le informazioni che lo riguardano e compiere le funzioni principali
\end{itemize}
\subsubsection{Classe Menu}
Presenta all'utente il menù da cui potrà svolgere le proprie funzioni principali.
\begin{figure}[H]
\centering
\noindent\makebox[\textwidth]{\includegraphics[width=\textwidth]{Img/quizzipedia-client-viewclient-viewusers-menu.pdf}}
\caption[Schema Classe Menu]{Schema Classe Quizzipedia::Client::ViewClient::ViewUsers::Menu}
\end{figure}
\paragraph{Relazioni con altre classi}
\subparagraph{Uscenti}
\begin{itemize}
\item usa Quizzipedia::Client::ControllerClient::CtrlUsers::CtrlUserManager per Raccoglie le classi necessarie a presentare all'utente le pagine da cui visualizzare le informazioni che lo riguardano e compiere le funzioni principali
\end{itemize}
\subsubsection{Classe PersonalData}
La classe presenta all'utente la pagina da cui prendere visione delle proprie informazioni personali.
\begin{figure}[H]
\centering
\noindent\makebox[\textwidth]{\includegraphics[width=\textwidth]{Img/quizzipedia-client-viewclient-viewusers-personaldata.pdf}}
\caption[Schema Classe PersonalData]{Schema Classe Quizzipedia::Client::ViewClient::ViewUsers::PersonalData}
\end{figure}
\paragraph{Relazioni con altre classi}
\subparagraph{Uscenti}
\begin{itemize}
\item usa Quizzipedia::Client::ControllerClient::CtrlUsers::CtrlUserManager per Raccoglie le classi necessarie a presentare all'utente le pagine da cui visualizzare le informazioni che lo riguardano e compiere le funzioni principali
\end{itemize}
\subsubsection{Classe RecoveryPw}
Da questa pagina sarà possibile inserire i dati per il recupero della password.
\begin{figure}[H]
\centering
\noindent\makebox[\textwidth]{\includegraphics[width=\textwidth]{Img/quizzipedia-client-viewclient-viewusers-recoverypw.pdf}}
\caption[Schema Classe RecoveryPw]{Schema Classe Quizzipedia::Client::ViewClient::ViewUsers::RecoveryPw}
\end{figure}
\paragraph{Relazioni con altre classi}
\subparagraph{Uscenti}
\begin{itemize}
\item usa Quizzipedia::Client::ControllerClient::CtrlUsers::CtrlData per Raccoglie le classi necessarie a presentare all'utente le pagine da cui visualizzare le informazioni che lo riguardano e compiere le funzioni principali
\end{itemize}
\subsubsection{Classe Registration}
Presenta la pagina da cui effettuare la  registrazione al sistema.
\begin{figure}[H]
\centering
\noindent\makebox[\textwidth]{\includegraphics[width=\textwidth]{Img/quizzipedia-client-viewclient-viewusers-registration.pdf}}
\caption[Schema Classe Registration]{Schema Classe Quizzipedia::Client::ViewClient::ViewUsers::Registration}
\end{figure}
\paragraph{Relazioni con altre classi}
\subparagraph{Uscenti}
\begin{itemize}
\item usa Quizzipedia::Client::ControllerClient::CtrlUsers::CtrlData per Raccoglie le classi necessarie a presentare all'utente le pagine da cui visualizzare le informazioni che lo riguardano e compiere le funzioni principali
\end{itemize}
\subsubsection{Classe ViewModifyUser}
Da qui, grazie ai metodi della classe, l'utente potrà modificare le proprie informazioni personali.
\begin{figure}[H]
\centering
\noindent\makebox[\textwidth]{\includegraphics[width=\textwidth]{Img/quizzipedia-client-viewclient-viewusers-viewmodifyuser.pdf}}
\caption[Schema Classe ViewModifyUser]{Schema Classe Quizzipedia::Client::ViewClient::ViewUsers::ViewModifyUser}
\end{figure}
\paragraph{Relazioni con altre classi}
\subparagraph{Uscenti}
\begin{itemize}
\item usa Quizzipedia::Client::ControllerClient::CtrlUsers::CtrlUserManager per Raccoglie le classi necessarie a presentare all'utente le pagine da cui visualizzare le informazioni che lo riguardano e compiere le funzioni principali
\end{itemize}
\subsubsection{Classe ViewUsersList}
Presenta un pannello da cui è possibile visualizzare una lista di utenti e compiere operazioni su di loro.
\begin{figure}[H]
\centering
\noindent\makebox[\textwidth]{\includegraphics[width=\textwidth]{Img/quizzipedia-client-viewclient-viewusers-viewuserslist.pdf}}
\caption[Schema Classe ViewUsersList]{Schema Classe Quizzipedia::Client::ViewClient::ViewUsers::ViewUsersList}
\end{figure}
\paragraph{Relazioni con altre classi}
\subparagraph{Uscenti}
\begin{itemize}
\item usa Quizzipedia::Client::ControllerClient::CtrlUsers::CtrlData per Raccoglie le classi necessarie a presentare all'utente le pagine da cui visualizzare le informazioni che lo riguardano e compiere le funzioni principali
\end{itemize}
\subsection{Quizzipedia::Client::ControllerClient}
Raccoglie le classi responsabili della comunicazione tra il model e la view.
\begin{figure}[H]
\centering
\noindent\makebox[\textwidth]{\includegraphics[width=\textwidth]{Img/quizzipedia-client-controllerclient.pdf}}
\caption[Schema Componente Quizzipedia::Client::ControllerClient]{Schema Componente Quizzipedia::Client::ControllerClient}
\end{figure}
\subsubsection{Componenti contenute}
\begin{itemize}
\item Quizzipedia::Client::ControllerClient::CtrlOrganization
\item Quizzipedia::Client::ControllerClient::CtrlRequests
\item Quizzipedia::Client::ControllerClient::CtrlServices
\item Quizzipedia::Client::ControllerClient::CtrlStatistics
\item Quizzipedia::Client::ControllerClient::CtrlUsers
\end{itemize}
\subsubsection{Interazioni con altre componenti}
\paragraph{Entranti}
\begin{itemize}
\item usata da Quizzipedia::Client::ViewClient per Raccoglie le classi responsabili della comunicazione tra il model e la view
\end{itemize}
\paragraph{Uscenti}
\begin{itemize}
\item usa Quizzipedia::Client::ModelClient per Raccoglie le classi responsabili della comunicazione tra il model e la view
\end{itemize}
\subsection{Quizzipedia::Client::ControllerClient::CtrlOrganization}
Raccoglie le classi che si occupano delle comunicazioni necessarie per la creazione di enti e classi.
\begin{figure}[H]
\centering
\noindent\makebox[\textwidth]{\includegraphics[width=\textwidth]{Img/quizzipedia-client-controllerclient-ctrlorganization.pdf}}
\caption[Schema Componente Quizzipedia::Client::ControllerClient::CtrlOrganization]{Schema Componente Quizzipedia::Client::ControllerClient::CtrlOrganization}
\end{figure}
\subsubsection{Interazioni con altre componenti}
\paragraph{Entranti}
\begin{itemize}
\item usata da Quizzipedia::Client::ViewClient::ViewOrgManager per Raccoglie le classi che si occupano delle comunicazioni necessarie per la creazione di enti e classi
\item usata da Quizzipedia::Client::ViewClient::ViewSearch per Raccoglie le classi che si occupano delle comunicazioni necessarie per la creazione di enti e classi
\end{itemize}
\subsubsection{Classe CtrlClass}
Vi sono presenti metodi necessari alla creazione delle classi.
\begin{figure}[H]
\centering
\noindent\makebox[\textwidth]{\includegraphics[width=\textwidth]{Img/quizzipedia-client-controllerclient-ctrlorganization-ctrlclass.pdf}}
\caption[Schema Classe CtrlClass]{Schema Classe Quizzipedia::Client::ControllerClient::CtrlOrganization::CtrlClass}
\end{figure}
\paragraph{Relazioni con altre classi}
\subparagraph{Entranti}
\begin{itemize}
\item usata da Quizzipedia::Client::ControllerClient::CtrlOrganization::CtrlInstitution per Raccoglie le classi che si occupano delle comunicazioni necessarie per la creazione di enti e classi
\item usata da Quizzipedia::Client::ViewClient::ViewOrgManager::ViewCreateClass per Raccoglie le classi che si occupano delle comunicazioni necessarie per la creazione di enti e classi
\item usata da Quizzipedia::Client::ViewClient::ViewOrgManager::ViewUsersClassList per Raccoglie le classi che si occupano delle comunicazioni necessarie per la creazione di enti e classi
\end{itemize}
\subsubsection{Classe CtrlInstitution}
Vi sono presenti metodi necessari alla creazione degli enti.
\begin{figure}[H]
\centering
\noindent\makebox[\textwidth]{\includegraphics[width=\textwidth]{Img/quizzipedia-client-controllerclient-ctrlorganization-ctrlinstitution.pdf}}
\caption[Schema Classe CtrlInstitution]{Schema Classe Quizzipedia::Client::ControllerClient::CtrlOrganization::CtrlInstitution}
\end{figure}
\paragraph{Relazioni con altre classi}
\subparagraph{Entranti}
\begin{itemize}
\item usata da Quizzipedia::Client::ViewClient::ViewOrgManager::ViewModifyOrg per Raccoglie le classi che si occupano delle comunicazioni necessarie per la creazione di enti e classi
\item usata da Quizzipedia::Client::ViewClient::ViewSearch::ViewSearchClass per Raccoglie le classi che si occupano delle comunicazioni necessarie per la creazione di enti e classi
\end{itemize}
\subparagraph{Uscenti}
\begin{itemize}
\item usa Quizzipedia::Client::ControllerClient::CtrlOrganization::CtrlClass per Raccoglie le classi che si occupano delle comunicazioni necessarie per la creazione di enti e classi
\end{itemize}
\subsection{Quizzipedia::Client::ControllerClient::CtrlRequests}
Questo package contiene classi necessarie alla gestione delle richieste di ruolo o classe.
\begin{figure}[H]
\centering
\noindent\makebox[\textwidth]{\includegraphics[width=\textwidth]{Img/quizzipedia-client-controllerclient-ctrlrequests.pdf}}
\caption[Schema Componente Quizzipedia::Client::ControllerClient::CtrlRequests]{Schema Componente Quizzipedia::Client::ControllerClient::CtrlRequests}
\end{figure}
\subsubsection{Interazioni con altre componenti}
\paragraph{Entranti}
\begin{itemize}
\item usata da Quizzipedia::Client::ViewClient::ViewRequests per Questo package contiene classi necessarie alla gestione delle richieste di ruolo o classe
\end{itemize}
\subsubsection{Classe CtrlRequestClass}
Si occupa delle comunicazioni necessarie per la gestione delle richieste di inserimento in una classe.
\begin{figure}[H]
\centering
\noindent\makebox[\textwidth]{\includegraphics[width=\textwidth]{Img/quizzipedia-client-controllerclient-ctrlrequests-ctrlrequestclass.pdf}}
\caption[Schema Classe CtrlRequestClass]{Schema Classe Quizzipedia::Client::ControllerClient::CtrlRequests::CtrlRequestClass}
\end{figure}
\paragraph{Relazioni con altre classi}
\subparagraph{Entranti}
\begin{itemize}
\item usata da Quizzipedia::Client::ViewClient::ViewRequests::RequestClass per Questo package contiene classi necessarie alla gestione delle richieste di ruolo o classe
\item usata da Quizzipedia::Client::ViewClient::ViewRequests::ViewClassList per Questo package contiene classi necessarie alla gestione delle richieste di ruolo o classe
\end{itemize}
\subparagraph{Uscenti}
\begin{itemize}
\item usa Quizzipedia::Client::ModelClient::Requests::ClassList per Questo package contiene classi necessarie alla gestione delle richieste di ruolo o classe
\end{itemize}
\subsubsection{Classe CtrlRequestRole}
Si occupa delle comunicazioni necessarie per la gestione delle richieste di ruolo.
\begin{figure}[H]
\centering
\noindent\makebox[\textwidth]{\includegraphics[width=\textwidth]{Img/quizzipedia-client-controllerclient-ctrlrequests-ctrlrequestrole.pdf}}
\caption[Schema Classe CtrlRequestRole]{Schema Classe Quizzipedia::Client::ControllerClient::CtrlRequests::CtrlRequestRole}
\end{figure}
\paragraph{Relazioni con altre classi}
\subparagraph{Entranti}
\begin{itemize}
\item usata da Quizzipedia::Client::ViewClient::ViewRequests::RequestRole per Questo package contiene classi necessarie alla gestione delle richieste di ruolo o classe
\item usata da Quizzipedia::Client::ViewClient::ViewRequests::ViewRolesList per Questo package contiene classi necessarie alla gestione delle richieste di ruolo o classe
\end{itemize}
\subparagraph{Uscenti}
\begin{itemize}
\item usa Quizzipedia::Client::ModelClient::Requests::RoleList per Questo package contiene classi necessarie alla gestione delle richieste di ruolo o classe
\end{itemize}
\subsection{Quizzipedia::Client::ControllerClient::CtrlServices}
Raccoglie gli elementi necessari alla creazione, svolgimento e ricerca di quiz e domande.
\begin{figure}[H]
\centering
\noindent\makebox[\textwidth]{\includegraphics[width=\textwidth]{Img/quizzipedia-client-controllerclient-ctrlservices.pdf}}
\caption[Schema Componente Quizzipedia::Client::ControllerClient::CtrlServices]{Schema Componente Quizzipedia::Client::ControllerClient::CtrlServices}
\end{figure}
\subsubsection{Interazioni con altre componenti}
\paragraph{Entranti}
\begin{itemize}
\item usata da Quizzipedia::Client::ViewClient::ViewQuestionManager per Raccoglie gli elementi necessari alla creazione, svolgimento e ricerca di quiz e domande
\item usata da Quizzipedia::Client::ViewClient::ViewQuizManager per Raccoglie gli elementi necessari alla creazione, svolgimento e ricerca di quiz e domande
\item usata da Quizzipedia::Client::ViewClient::ViewQuizSolver per Raccoglie gli elementi necessari alla creazione, svolgimento e ricerca di quiz e domande
\item usata da Quizzipedia::Client::ViewClient::ViewSearch per Raccoglie gli elementi necessari alla creazione, svolgimento e ricerca di quiz e domande
\item usata da Quizzipedia::Client::ViewClient::ViewTopicManager per Raccoglie gli elementi necessari alla creazione, svolgimento e ricerca di quiz e domande
\end{itemize}
\subsubsection{Classe CtrlQuestion}
Fornisce i metodi necessari per la comunicazione tra view e model durante la creazione, modifica e svolgimento di una domanda.
\begin{figure}[H]
\centering
\noindent\makebox[\textwidth]{\includegraphics[width=\textwidth]{Img/quizzipedia-client-controllerclient-ctrlservices-ctrlquestion.pdf}}
\caption[Schema Classe CtrlQuestion]{Schema Classe Quizzipedia::Client::ControllerClient::CtrlServices::CtrlQuestion}
\end{figure}
\paragraph{Relazioni con altre classi}
\subparagraph{Entranti}
\begin{itemize}
\item usata da Quizzipedia::Client::ViewClient::ViewQuestionManager::ViewCreateQuestion per Raccoglie gli elementi necessari alla creazione, svolgimento e ricerca di quiz e domande
\item usata da Quizzipedia::Client::ViewClient::ViewQuestionManager::ViewModifyQuestion per Raccoglie gli elementi necessari alla creazione, svolgimento e ricerca di quiz e domande
\item usata da Quizzipedia::Client::ViewClient::ViewQuestionManager::ViewQuestionList per Raccoglie gli elementi necessari alla creazione, svolgimento e ricerca di quiz e domande
\item usata da Quizzipedia::Client::ViewClient::ViewQuizSolver::ViewQuestionSolver::ViewCompletionQ per Raccoglie gli elementi necessari alla creazione, svolgimento e ricerca di quiz e domande
\item usata da Quizzipedia::Client::ViewClient::ViewQuizSolver::ViewQuestionSolver::ViewMatchingQ per Raccoglie gli elementi necessari alla creazione, svolgimento e ricerca di quiz e domande
\item usata da Quizzipedia::Client::ViewClient::ViewQuizSolver::ViewQuestionSolver::ViewMultipleChoiceQ per Raccoglie gli elementi necessari alla creazione, svolgimento e ricerca di quiz e domande
\item usata da Quizzipedia::Client::ViewClient::ViewQuizSolver::ViewQuestionSolver::ViewShortAnswerQ per Raccoglie gli elementi necessari alla creazione, svolgimento e ricerca di quiz e domande
\item usata da Quizzipedia::Client::ViewClient::ViewQuizSolver::ViewQuestionSolver::ViewTrueFalseQ per Raccoglie gli elementi necessari alla creazione, svolgimento e ricerca di quiz e domande
\end{itemize}
\subparagraph{Uscenti}
\begin{itemize}
\item usa Quizzipedia::Client::ModelClient::Services::Questions::GenericQuestion per Raccoglie gli elementi necessari alla creazione, svolgimento e ricerca di quiz e domande
\end{itemize}
\subsubsection{Classe CtrlQuiz}
Raccoglie i metodi necessari alle comunicazioni tra view e model richieste per la gestione e lo svolgimento dei quiz.
\begin{figure}[H]
\centering
\noindent\makebox[\textwidth]{\includegraphics[width=\textwidth]{Img/quizzipedia-client-controllerclient-ctrlservices-ctrlquiz.pdf}}
\caption[Schema Classe CtrlQuiz]{Schema Classe Quizzipedia::Client::ControllerClient::CtrlServices::CtrlQuiz}
\end{figure}
\paragraph{Relazioni con altre classi}
\subparagraph{Entranti}
\begin{itemize}
\item usata da Quizzipedia::Client::ViewClient::ViewQuizManager::ViewCreateQuiz per Raccoglie gli elementi necessari alla creazione, svolgimento e ricerca di quiz e domande
\item usata da Quizzipedia::Client::ViewClient::ViewQuizManager::ViewModifyQuiz per Raccoglie gli elementi necessari alla creazione, svolgimento e ricerca di quiz e domande
\item usata da Quizzipedia::Client::ViewClient::ViewQuizManager::ViewQuizList per Raccoglie gli elementi necessari alla creazione, svolgimento e ricerca di quiz e domande
\item usata da Quizzipedia::Client::ViewClient::ViewQuizSolver::ViewResults per Raccoglie gli elementi necessari alla creazione, svolgimento e ricerca di quiz e domande
\end{itemize}
\subparagraph{Uscenti}
\begin{itemize}
\item usa Quizzipedia::Client::ModelClient::Services::Quiz per Raccoglie gli elementi necessari alla creazione, svolgimento e ricerca di quiz e domande
\end{itemize}
\subsubsection{Classe CtrlSearch}
La classe contiene i metodi necessari alla comunicazione tra model e view nell'effettuare la ricerca di quiz o domande.
\begin{figure}[H]
\centering
\noindent\makebox[\textwidth]{\includegraphics[width=\textwidth]{Img/quizzipedia-client-controllerclient-ctrlservices-ctrlsearch.pdf}}
\caption[Schema Classe CtrlSearch]{Schema Classe Quizzipedia::Client::ControllerClient::CtrlServices::CtrlSearch}
\end{figure}
\paragraph{Relazioni con altre classi}
\subparagraph{Entranti}
\begin{itemize}
\item usata da Quizzipedia::Client::ViewClient::ViewSearch::ViewSearchQuestion per Raccoglie gli elementi necessari alla creazione, svolgimento e ricerca di quiz e domande
\item usata da Quizzipedia::Client::ViewClient::ViewSearch::ViewSearchQuiz per Raccoglie gli elementi necessari alla creazione, svolgimento e ricerca di quiz e domande
\end{itemize}
\subparagraph{Uscenti}
\begin{itemize}
\item usa Quizzipedia::Server::RoutingManager::SearchRouter per Raccoglie gli elementi necessari alla creazione, svolgimento e ricerca di quiz e domande
\end{itemize}
\subsubsection{Classe CtrlTopics}
Fornisce i metodi necessari alle comunicazioni richieste per la gestione degli argomenti.
\begin{figure}[H]
\centering
\noindent\makebox[\textwidth]{\includegraphics[width=\textwidth]{Img/quizzipedia-client-controllerclient-ctrlservices-ctrltopics.pdf}}
\caption[Schema Classe CtrlTopics]{Schema Classe Quizzipedia::Client::ControllerClient::CtrlServices::CtrlTopics}
\end{figure}
\paragraph{Relazioni con altre classi}
\subparagraph{Entranti}
\begin{itemize}
\item usata da Quizzipedia::Client::ViewClient::ViewTopicManager::ViewCreateTopic per Raccoglie gli elementi necessari alla creazione, svolgimento e ricerca di quiz e domande
\item usata da Quizzipedia::Client::ViewClient::ViewTopicManager::ViewTopicList per Raccoglie gli elementi necessari alla creazione, svolgimento e ricerca di quiz e domande
\end{itemize}
\subparagraph{Uscenti}
\begin{itemize}
\item usa Quizzipedia::Client::ModelClient::Services::Topics per Raccoglie gli elementi necessari alla creazione, svolgimento e ricerca di quiz e domande
\end{itemize}
\subsection{Quizzipedia::Client::ControllerClient::CtrlStatistics}
Raccoglie le classi necessarie a recuperare le statistiche da presentare all'utente.
\begin{figure}[H]
\centering
\noindent\makebox[\textwidth]{\includegraphics[width=\textwidth]{Img/quizzipedia-client-controllerclient-ctrlstatistics.pdf}}
\caption[Schema Componente Quizzipedia::Client::ControllerClient::CtrlStatistics]{Schema Componente Quizzipedia::Client::ControllerClient::CtrlStatistics}
\end{figure}
\subsubsection{Interazioni con altre componenti}
\paragraph{Entranti}
\begin{itemize}
\item usata da Quizzipedia::Client::ViewClient::ViewStatistics per Raccoglie le classi necessarie a recuperare le statistiche da presentare all'utente
\end{itemize}
\subsubsection{Classe CtrlStats}
Classe necessaria al caricamento delle statistiche relative ai quiz, alle domande e alle classi.
\begin{figure}[H]
\centering
\noindent\makebox[\textwidth]{\includegraphics[width=\textwidth]{Img/quizzipedia-client-controllerclient-ctrlstatistics-ctrlstats.pdf}}
\caption[Schema Classe CtrlStats]{Schema Classe Quizzipedia::Client::ControllerClient::CtrlStatistics::CtrlStats}
\end{figure}
\paragraph{Relazioni con altre classi}
\subparagraph{Entranti}
\begin{itemize}
\item usata da Quizzipedia::Client::ViewClient::ViewStatistics::ViewClassStats per Raccoglie le classi necessarie a recuperare le statistiche da presentare all'utente
\item usata da Quizzipedia::Client::ViewClient::ViewStatistics::ViewQuestionStats per Raccoglie le classi necessarie a recuperare le statistiche da presentare all'utente
\item usata da Quizzipedia::Client::ViewClient::ViewStatistics::ViewQuizStats per Raccoglie le classi necessarie a recuperare le statistiche da presentare all'utente
\end{itemize}
\subparagraph{Uscenti}
\begin{itemize}
\item usa Quizzipedia::Client::ModelClient::Statistics::ClassQuiz per Raccoglie le classi necessarie a recuperare le statistiche da presentare all'utente
\item usa Quizzipedia::Client::ModelClient::Statistics::GenericQuestions per Raccoglie le classi necessarie a recuperare le statistiche da presentare all'utente
\item usa Quizzipedia::Client::ModelClient::Statistics::GenericQuiz per Raccoglie le classi necessarie a recuperare le statistiche da presentare all'utente
\end{itemize}
\subsection{Quizzipedia::Client::ControllerClient::CtrlUsers}
Il package raccoglie le classi che permettono la comunicazione per quanto riguarda funzioni e dati dell'utente.
\begin{figure}[H]
\centering
\noindent\makebox[\textwidth]{\includegraphics[width=\textwidth]{Img/quizzipedia-client-controllerclient-ctrlusers.pdf}}
\caption[Schema Componente Quizzipedia::Client::ControllerClient::CtrlUsers]{Schema Componente Quizzipedia::Client::ControllerClient::CtrlUsers}
\end{figure}
\subsubsection{Interazioni con altre componenti}
\paragraph{Entranti}
\begin{itemize}
\item usata da Quizzipedia::Client::ViewClient::ViewUsers per Il package raccoglie le classi che permettono la comunicazione per quanto riguarda funzioni e dati dell'utente
\end{itemize}
\subsubsection{Classe CtrlData}
La classe raccoglie i metodi per gestire le informazioni personali di tutti gli utenti autenticati.
\begin{figure}[H]
\centering
\noindent\makebox[\textwidth]{\includegraphics[width=\textwidth]{Img/quizzipedia-client-controllerclient-ctrlusers-ctrldata.pdf}}
\caption[Schema Classe CtrlData]{Schema Classe Quizzipedia::Client::ControllerClient::CtrlUsers::CtrlData}
\end{figure}
\paragraph{Relazioni con altre classi}
\subparagraph{Entranti}
\begin{itemize}
\item usata da Quizzipedia::Client::ViewClient::ViewUsers::Login per Il package raccoglie le classi che permettono la comunicazione per quanto riguarda funzioni e dati dell'utente
\item usata da Quizzipedia::Client::ViewClient::ViewUsers::Logout per Il package raccoglie le classi che permettono la comunicazione per quanto riguarda funzioni e dati dell'utente
\item usata da Quizzipedia::Client::ViewClient::ViewUsers::RecoveryPw per Il package raccoglie le classi che permettono la comunicazione per quanto riguarda funzioni e dati dell'utente
\item usata da Quizzipedia::Client::ViewClient::ViewUsers::Registration per Il package raccoglie le classi che permettono la comunicazione per quanto riguarda funzioni e dati dell'utente
\item usata da Quizzipedia::Client::ViewClient::ViewUsers::ViewUsersList per Il package raccoglie le classi che permettono la comunicazione per quanto riguarda funzioni e dati dell'utente
\end{itemize}
\subparagraph{Uscenti}
\begin{itemize}
\item usa Quizzipedia::Client::ModelClient::Users::User per Il package raccoglie le classi che permettono la comunicazione per quanto riguarda funzioni e dati dell'utente
\item usa Quizzipedia::Server::RoutingManager::AuthenticationRouter per Il package raccoglie le classi che permettono la comunicazione per quanto riguarda funzioni e dati dell'utente
\end{itemize}
\subsubsection{Classe CtrlUserManager}
La classe permette agli utenti di gestire il proprio profilo personale.
\begin{figure}[H]
\centering
\noindent\makebox[\textwidth]{\includegraphics[width=\textwidth]{Img/quizzipedia-client-controllerclient-ctrlusers-ctrlusermanager.pdf}}
\caption[Schema Classe CtrlUserManager]{Schema Classe Quizzipedia::Client::ControllerClient::CtrlUsers::CtrlUserManager}
\end{figure}
\paragraph{Relazioni con altre classi}
\subparagraph{Entranti}
\begin{itemize}
\item usata da Quizzipedia::Client::ViewClient::ViewUsers::Menu per Il package raccoglie le classi che permettono la comunicazione per quanto riguarda funzioni e dati dell'utente
\item usata da Quizzipedia::Client::ViewClient::ViewUsers::PersonalData per Il package raccoglie le classi che permettono la comunicazione per quanto riguarda funzioni e dati dell'utente
\item usata da Quizzipedia::Client::ViewClient::ViewUsers::ViewModifyUser per Il package raccoglie le classi che permettono la comunicazione per quanto riguarda funzioni e dati dell'utente
\end{itemize}
\subparagraph{Uscenti}
\begin{itemize}
\item usa Quizzipedia::Client::ModelClient::Users::User per Il package raccoglie le classi che permettono la comunicazione per quanto riguarda funzioni e dati dell'utente
\end{itemize}
\subsection{Quizzipedia::Server}
Racchiude tutte le componenti necessarie per il back-end del prodotto. Contiene anche le componenti che si occupano del QML.
\begin{figure}[H]
\centering
\noindent\makebox[\textwidth]{\includegraphics[width=\textwidth]{Img/quizzipedia-server.pdf}}
\caption[Schema Componente Server]{Schema Componente Quizzipedia::Server}
\end{figure}
\subsection{Quizzipedia::Server::ModelServer}
Rappresenta il modello dei dati che verranno utilizzati dal sistema lato server.
\begin{figure}[H]
\centering
\noindent\makebox[\textwidth]{\includegraphics[width=\textwidth]{Img/quizzipedia-server-modelserver.pdf}}
\caption[Schema Componente Quizzipedia::Server::ModelServer]{Schema Componente Quizzipedia::Server::ModelServer}
\end{figure}
\subsubsection{Componenti contenute}
\begin{itemize}
\item Quizzipedia::Server::ModelServer::Services
\end{itemize}
\subsubsection{Interazioni con altre componenti}
\paragraph{Entranti}
\begin{itemize}
\item usata da Quizzipedia::Server::ControllerServer per Rappresenta il modello dei dati che verranno utilizzati dal sistema lato server
\end{itemize}
\subsection{Quizzipedia::Server::ModelServer::Services}
Il package racchiude i modelli necessari alla creazione di domande e quiz, i servizi principali offerti dal nostro prodotto.
\begin{figure}[H]
\centering
\noindent\makebox[\textwidth]{\includegraphics[width=\textwidth]{Img/quizzipedia-server-modelserver-services.pdf}}
\caption[Schema Componente Quizzipedia::Server::ModelServer::Services]{Schema Componente Quizzipedia::Server::ModelServer::Services}
\end{figure}
\subsubsection{Componenti contenute}
\begin{itemize}
\item Quizzipedia::Server::ModelServer::Services::Questions
\end{itemize}
\subsubsection{Classe Info}
Riassume le informazioni principali su quiz e domande, necessarie per una presentazione sintetica e puntuale all'utente. È poi possibile risalire alla domanda o al quiz completi.
\begin{figure}[H]
\centering
\noindent\makebox[\textwidth]{\includegraphics[width=\textwidth]{Img/quizzipedia-server-modelserver-services-info.pdf}}
\caption[Schema Classe Info]{Schema Classe Quizzipedia::Server::ModelServer::Services::Info}
\end{figure}
\subsubsection{Classe Quiz}
Include la struttura del quiz.
\begin{figure}[H]
\centering
\noindent\makebox[\textwidth]{\includegraphics[width=\textwidth]{Img/quizzipedia-server-modelserver-services-quiz.pdf}}
\caption[Schema Classe Quiz]{Schema Classe Quizzipedia::Server::ModelServer::Services::Quiz}
\end{figure}
\subsubsection{Classe Topics}
Modella la struttura necessaria a memorizzare la lista di argomenti. A ogni domanda e a ogni quiz verranno poi associati i relativi argomenti.
\begin{figure}[H]
\centering
\noindent\makebox[\textwidth]{\includegraphics[width=\textwidth]{Img/quizzipedia-server-modelserver-services-topics.pdf}}
\caption[Schema Classe Topics]{Schema Classe Quizzipedia::Server::ModelServer::Services::Topics}
\end{figure}
\subsection{Quizzipedia::Server::ModelServer::Services::Questions}
Descrive il modo in cui sono strutturati i vari tipi di domande che l'utente può incontrare durante la creazione o la compilazione di quiz.
\begin{figure}[H]
\centering
\noindent\makebox[\textwidth]{\includegraphics[width=\textwidth]{Img/quizzipedia-server-modelserver-services-questions.pdf}}
\caption[Schema Componente Quizzipedia::Server::ModelServer::Services::Questions]{Schema Componente Quizzipedia::Server::ModelServer::Services::Questions}
\end{figure}
\subsubsection{Classe Cell}
La classe descrive ogni singola riga (quindi ogni opzione) della colonna della domanda a collegamento.
\begin{figure}[H]
\centering
\noindent\makebox[\textwidth]{\includegraphics[width=\textwidth]{Img/quizzipedia-server-modelserver-services-questions-cell.pdf}}
\caption[Schema Classe Cell]{Schema Classe Quizzipedia::Server::ModelServer::Services::Questions::Cell}
\end{figure}
\subsubsection{Classe Column}
La classe descrive le colonne della domanda a collegamenti.
\begin{figure}[H]
\centering
\noindent\makebox[\textwidth]{\includegraphics[width=\textwidth]{Img/quizzipedia-server-modelserver-services-questions-column.pdf}}
\caption[Schema Classe Column]{Schema Classe Quizzipedia::Server::ModelServer::Services::Questions::Column}
\end{figure}
\subsubsection{Classe CompletationQ}
Descrive le domande a completamento. Il docente fornirà un testo incompleto e una lista di possibili completamenti; lo studente dovrà inserire le parole adeguate nella giusta posizione.
\begin{figure}[H]
\centering
\noindent\makebox[\textwidth]{\includegraphics[width=\textwidth]{Img/quizzipedia-server-modelserver-services-questions-completationq.pdf}}
\caption[Schema Classe CompletationQ]{Schema Classe Quizzipedia::Server::ModelServer::Services::Questions::CompletationQ}
\end{figure}
\subsubsection{Classe GenericQuestion}
Descrive le parti comuni a tutti i tipi di domanda presenti nel sistema.
\begin{figure}[H]
\centering
\noindent\makebox[\textwidth]{\includegraphics[width=\textwidth]{Img/quizzipedia-server-modelserver-services-questions-genericquestion.pdf}}
\caption[Schema Classe GenericQuestion]{Schema Classe Quizzipedia::Server::ModelServer::Services::Questions::GenericQuestion}
\end{figure}
\subsubsection{Classe MatchingQ}
La struttura descrive le domande a collegamento. L'utente dovrà formare la risposta collegando le entrate da un numero variabile di colonne.
\begin{figure}[H]
\centering
\noindent\makebox[\textwidth]{\includegraphics[width=\textwidth]{Img/quizzipedia-server-modelserver-services-questions-matchingq.pdf}}
\caption[Schema Classe MatchingQ]{Schema Classe Quizzipedia::Server::ModelServer::Services::Questions::MatchingQ}
\end{figure}
\subsubsection{Classe MultipleChoiceQ}
La struttura descrive le domande a scelta multipla; viene presentata una lista di opzioni tra cui scegliere quelle corrette.
\begin{figure}[H]
\centering
\noindent\makebox[\textwidth]{\includegraphics[width=\textwidth]{Img/quizzipedia-server-modelserver-services-questions-multiplechoiceq.pdf}}
\caption[Schema Classe MultipleChoiceQ]{Schema Classe Quizzipedia::Server::ModelServer::Services::Questions::MultipleChoiceQ}
\end{figure}
\subsubsection{Classe ShortAnswerQ}
La struttura descrive le domande aperte, ovvero quelle la cui risposta consiste in un termine o una frase specifici.
\begin{figure}[H]
\centering
\noindent\makebox[\textwidth]{\includegraphics[width=\textwidth]{Img/quizzipedia-server-modelserver-services-questions-shortanswerq.pdf}}
\caption[Schema Classe ShortAnswerQ]{Schema Classe Quizzipedia::Server::ModelServer::Services::Questions::ShortAnswerQ}
\end{figure}
\subsubsection{Classe TrueFalseQ}
Viene descritta la struttura delle domande che prevedono di decidere la veridicità di un'affermazione.
\begin{figure}[H]
\centering
\noindent\makebox[\textwidth]{\includegraphics[width=\textwidth]{Img/quizzipedia-server-modelserver-services-questions-truefalseq.pdf}}
\caption[Schema Classe TrueFalseQ]{Schema Classe Quizzipedia::Server::ModelServer::Services::Questions::TrueFalseQ}
\end{figure}
\subsection{Quizzipedia::Server::ControllerServer}
Questo package contiene tutti i servizi che ricalcano il pattern architetturale DAO in modo da isolare l'accesso al database relazionale. Avviene sempre un controllo dell'utente che genera una determinata richiesta al server affinchè sia abilitato per farla.
\begin{figure}[H]
\centering
\noindent\makebox[\textwidth]{\includegraphics[width=\textwidth]{Img/quizzipedia-server-controllerserver.pdf}}
\caption[Schema Componente Quizzipedia::Server::ControllerServer]{Schema Componente Quizzipedia::Server::ControllerServer}
\end{figure}
\subsubsection{Componenti contenute}
\begin{itemize}
\item Quizzipedia::Server::ControllerServer::AuthenticationManager
\item Quizzipedia::Server::ControllerServer::ClassManager
\item Quizzipedia::Server::ControllerServer::CompanyManager
\item Quizzipedia::Server::ControllerServer::ProfileManager
\item Quizzipedia::Server::ControllerServer::QuestionsManager
\item Quizzipedia::Server::ControllerServer::QuizManager
\item Quizzipedia::Server::ControllerServer::RequestsManager
\item Quizzipedia::Server::ControllerServer::SearchManager
\item Quizzipedia::Server::ControllerServer::StatisticsManager
\item Quizzipedia::Server::ControllerServer::TopicManager
\end{itemize}
\subsubsection{Interazioni con altre componenti}
\paragraph{Entranti}
\begin{itemize}
\item usata da Quizzipedia::Server::RoutingManager per Questo package contiene tutti i servizi che ricalcano il pattern architetturale DAO in modo da isolare l'accesso al database relazionale. Avviene sempre un controllo dell'utente che genera una determinata richiesta al server affinchè sia abilitato per farla
\end{itemize}
\paragraph{Uscenti}
\begin{itemize}
\item usa Quizzipedia::Server::ModelServer per Questo package contiene tutti i servizi che ricalcano il pattern architetturale DAO in modo da isolare l'accesso al database relazionale. Avviene sempre un controllo dell'utente che genera una determinata richiesta al server affinchè sia abilitato per farla
\end{itemize}
\subsection{Quizzipedia::Server::ControllerServer::AuthenticationManager}
Package che permette di gestire le funzioni base per una corretta autenticazione al sistema.
\begin{figure}[H]
\centering
\noindent\makebox[\textwidth]{\includegraphics[width=\textwidth]{Img/quizzipedia-server-controllerserver-authenticationmanager.pdf}}
\caption[Schema Componente Quizzipedia::Server::ControllerServer::AuthenticationManager]{Schema Componente Quizzipedia::Server::ControllerServer::AuthenticationManager}
\end{figure}
\subsubsection{Classe LoggerIn}
Permette l'autenticazione nel sistema da parte di utenti preventivamente registrati.
\begin{figure}[H]
\centering
\noindent\makebox[\textwidth]{\includegraphics[width=\textwidth]{Img/quizzipedia-server-controllerserver-authenticationmanager-loggerin.pdf}}
\caption[Schema Classe LoggerIn]{Schema Classe Quizzipedia::Server::ControllerServer::AuthenticationManager::LoggerIn}
\end{figure}
\paragraph{Relazioni con altre classi}
\subparagraph{Entranti}
\begin{itemize}
\item usata da Quizzipedia::Server::RoutingManager::AuthenticationRouter per Package che permette di gestire le funzioni base per una corretta autenticazione al sistema
\end{itemize}
\subsubsection{Classe LoggerOut}
Permette l'uscita dal sistema ad utenti autenticati.
\begin{figure}[H]
\centering
\noindent\makebox[\textwidth]{\includegraphics[width=\textwidth]{Img/quizzipedia-server-controllerserver-authenticationmanager-loggerout.pdf}}
\caption[Schema Classe LoggerOut]{Schema Classe Quizzipedia::Server::ControllerServer::AuthenticationManager::LoggerOut}
\end{figure}
\paragraph{Relazioni con altre classi}
\subparagraph{Entranti}
\begin{itemize}
\item usata da Quizzipedia::Server::RoutingManager::AuthenticationRouter per Package che permette di gestire le funzioni base per una corretta autenticazione al sistema
\end{itemize}
\subsubsection{Classe PasswordRecover}
Permette il recupero della password da parte di un utente in caso di smarrimento o dimenticanza.
\begin{figure}[H]
\centering
\noindent\makebox[\textwidth]{\includegraphics[width=\textwidth]{Img/quizzipedia-server-controllerserver-authenticationmanager-passwordrecover.pdf}}
\caption[Schema Classe PasswordRecover]{Schema Classe Quizzipedia::Server::ControllerServer::AuthenticationManager::PasswordRecover}
\end{figure}
\paragraph{Relazioni con altre classi}
\subparagraph{Entranti}
\begin{itemize}
\item usata da Quizzipedia::Server::RoutingManager::AuthenticationRouter per Package che permette di gestire le funzioni base per una corretta autenticazione al sistema
\end{itemize}
\subparagraph{Uscenti}
\begin{itemize}
\item usa Quizzipedia::Server::ControllerServer::AuthenticationManager::SessionController per Package che permette di gestire le funzioni base per una corretta autenticazione al sistema
\end{itemize}
\subsubsection{Classe Register}
Permette la registrazione di un utente nel sistema.
\begin{figure}[H]
\centering
\noindent\makebox[\textwidth]{\includegraphics[width=\textwidth]{Img/quizzipedia-server-controllerserver-authenticationmanager-register.pdf}}
\caption[Schema Classe Register]{Schema Classe Quizzipedia::Server::ControllerServer::AuthenticationManager::Register}
\end{figure}
\paragraph{Relazioni con altre classi}
\subparagraph{Entranti}
\begin{itemize}
\item usata da Quizzipedia::Server::RoutingManager::AuthenticationRouter per Package che permette di gestire le funzioni base per una corretta autenticazione al sistema
\end{itemize}
\subsubsection{Classe SessionController}
Effettua il controllo sull'utente per verificare che egli sia in possesso dell'autorizzazione necessaria per compiere determinate richieste alla base di dati.
\begin{figure}[H]
\centering
\noindent\makebox[\textwidth]{\includegraphics[width=\textwidth]{Img/quizzipedia-server-controllerserver-authenticationmanager-sessioncontroller.pdf}}
\caption[Schema Classe SessionController]{Schema Classe Quizzipedia::Server::ControllerServer::AuthenticationManager::SessionController}
\end{figure}
\paragraph{Relazioni con altre classi}
\subparagraph{Entranti}
\begin{itemize}
\item usata da Quizzipedia::Server::ControllerServer::AuthenticationManager::PasswordRecover per Package che permette di gestire le funzioni base per una corretta autenticazione al sistema
\end{itemize}
\subsection{Quizzipedia::Server::ControllerServer::ClassManager}
Package che racchiude tutte le funzionalità adibite al salvataggio e alla visualizzazione delle informazioni riguardanti le classi di un ente.
\begin{figure}[H]
\centering
\noindent\makebox[\textwidth]{\includegraphics[width=\textwidth]{Img/quizzipedia-server-controllerserver-classmanager.pdf}}
\caption[Schema Componente Quizzipedia::Server::ControllerServer::ClassManager]{Schema Componente Quizzipedia::Server::ControllerServer::ClassManager}
\end{figure}
\subsubsection{Classe ClassAdder}
Permette la creazione di una nuova classe.
\begin{figure}[H]
\centering
\noindent\makebox[\textwidth]{\includegraphics[width=\textwidth]{Img/quizzipedia-server-controllerserver-classmanager-classadder.pdf}}
\caption[Schema Classe ClassAdder]{Schema Classe Quizzipedia::Server::ControllerServer::ClassManager::ClassAdder}
\end{figure}
\paragraph{Relazioni con altre classi}
\subparagraph{Entranti}
\begin{itemize}
\item usata da Quizzipedia::Server::RoutingManager::ClassRouter per Package che racchiude tutte le funzionalità adibite al salvataggio e alla visualizzazione delle informazioni riguardanti le classi di un ente
\end{itemize}
\subparagraph{Uscenti}
\begin{itemize}
\item usa Quizzipedia::Server::ControllerServer::ClassManager::SessionController per Package che racchiude tutte le funzionalità adibite al salvataggio e alla visualizzazione delle informazioni riguardanti le classi di un ente
\end{itemize}
\subsubsection{Classe ClassDeleter}
Permette la rimozione delle classi dal sistema.
\begin{figure}[H]
\centering
\noindent\makebox[\textwidth]{\includegraphics[width=\textwidth]{Img/quizzipedia-server-controllerserver-classmanager-classdeleter.pdf}}
\caption[Schema Classe ClassDeleter]{Schema Classe Quizzipedia::Server::ControllerServer::ClassManager::ClassDeleter}
\end{figure}
\paragraph{Relazioni con altre classi}
\subparagraph{Entranti}
\begin{itemize}
\item usata da Quizzipedia::Server::RoutingManager::ClassRouter per Package che racchiude tutte le funzionalità adibite al salvataggio e alla visualizzazione delle informazioni riguardanti le classi di un ente
\end{itemize}
\subparagraph{Uscenti}
\begin{itemize}
\item usa Quizzipedia::Server::ControllerServer::ClassManager::SessionController per Package che racchiude tutte le funzionalità adibite al salvataggio e alla visualizzazione delle informazioni riguardanti le classi di un ente
\end{itemize}
\subsubsection{Classe ClassUpdater}
Permette la modifica delle informazioni di base di una determinata classe.
\begin{figure}[H]
\centering
\noindent\makebox[\textwidth]{\includegraphics[width=\textwidth]{Img/quizzipedia-server-controllerserver-classmanager-classupdater.pdf}}
\caption[Schema Classe ClassUpdater]{Schema Classe Quizzipedia::Server::ControllerServer::ClassManager::ClassUpdater}
\end{figure}
\paragraph{Relazioni con altre classi}
\subparagraph{Entranti}
\begin{itemize}
\item usata da Quizzipedia::Server::RoutingManager::ClassRouter per Package che racchiude tutte le funzionalità adibite al salvataggio e alla visualizzazione delle informazioni riguardanti le classi di un ente
\end{itemize}
\subparagraph{Uscenti}
\begin{itemize}
\item usa Quizzipedia::Server::ControllerServer::ClassManager::SessionController per Package che racchiude tutte le funzionalità adibite al salvataggio e alla visualizzazione delle informazioni riguardanti le classi di un ente
\end{itemize}
\subsubsection{Classe FromClassRemover}
Permette la rimozione di un utente da una determinata classe.
\begin{figure}[H]
\centering
\noindent\makebox[\textwidth]{\includegraphics[width=\textwidth]{Img/quizzipedia-server-controllerserver-classmanager-fromclassremover.pdf}}
\caption[Schema Classe FromClassRemover]{Schema Classe Quizzipedia::Server::ControllerServer::ClassManager::FromClassRemover}
\end{figure}
\paragraph{Relazioni con altre classi}
\subparagraph{Entranti}
\begin{itemize}
\item usata da Quizzipedia::Server::RoutingManager::ClassRouter per Package che racchiude tutte le funzionalità adibite al salvataggio e alla visualizzazione delle informazioni riguardanti le classi di un ente
\end{itemize}
\subparagraph{Uscenti}
\begin{itemize}
\item usa Quizzipedia::Server::ControllerServer::ClassManager::SessionController per Package che racchiude tutte le funzionalità adibite al salvataggio e alla visualizzazione delle informazioni riguardanti le classi di un ente
\end{itemize}
\subsubsection{Classe InClassAdder}
Permette l'inserimento di un utente in una specifica classe.
\begin{figure}[H]
\centering
\noindent\makebox[\textwidth]{\includegraphics[width=\textwidth]{Img/quizzipedia-server-controllerserver-classmanager-inclassadder.pdf}}
\caption[Schema Classe InClassAdder]{Schema Classe Quizzipedia::Server::ControllerServer::ClassManager::InClassAdder}
\end{figure}
\paragraph{Relazioni con altre classi}
\subparagraph{Entranti}
\begin{itemize}
\item usata da Quizzipedia::Server::RoutingManager::ClassRouter per Package che racchiude tutte le funzionalità adibite al salvataggio e alla visualizzazione delle informazioni riguardanti le classi di un ente
\end{itemize}
\subparagraph{Uscenti}
\begin{itemize}
\item usa Quizzipedia::Server::ControllerServer::ClassManager::SessionController per Package che racchiude tutte le funzionalità adibite al salvataggio e alla visualizzazione delle informazioni riguardanti le classi di un ente
\end{itemize}
\subsubsection{Classe SessionController}
Effettua il controllo sull'utente per verificare che egli sia in possesso dell'autorizzazione necessaria per compiere determinate richieste alla base di dati.
\begin{figure}[H]
\centering
\noindent\makebox[\textwidth]{\includegraphics[width=\textwidth]{Img/quizzipedia-server-controllerserver-classmanager-sessioncontroller.pdf}}
\caption[Schema Classe SessionController]{Schema Classe Quizzipedia::Server::ControllerServer::ClassManager::SessionController}
\end{figure}
\paragraph{Relazioni con altre classi}
\subparagraph{Entranti}
\begin{itemize}
\item usata da Quizzipedia::Server::ControllerServer::ClassManager::ClassAdder per Package che racchiude tutte le funzionalità adibite al salvataggio e alla visualizzazione delle informazioni riguardanti le classi di un ente
\item usata da Quizzipedia::Server::ControllerServer::ClassManager::ClassDeleter per Package che racchiude tutte le funzionalità adibite al salvataggio e alla visualizzazione delle informazioni riguardanti le classi di un ente
\item usata da Quizzipedia::Server::ControllerServer::ClassManager::ClassUpdater per Package che racchiude tutte le funzionalità adibite al salvataggio e alla visualizzazione delle informazioni riguardanti le classi di un ente
\item usata da Quizzipedia::Server::ControllerServer::ClassManager::FromClassRemover per Package che racchiude tutte le funzionalità adibite al salvataggio e alla visualizzazione delle informazioni riguardanti le classi di un ente
\item usata da Quizzipedia::Server::ControllerServer::ClassManager::InClassAdder per Package che racchiude tutte le funzionalità adibite al salvataggio e alla visualizzazione delle informazioni riguardanti le classi di un ente
\item usata da Quizzipedia::Server::ControllerServer::ClassManager::StudentsClassFetcher per Package che racchiude tutte le funzionalità adibite al salvataggio e alla visualizzazione delle informazioni riguardanti le classi di un ente
\item usata da Quizzipedia::Server::ControllerServer::ClassManager::TeachersClassFetcher per Package che racchiude tutte le funzionalità adibite al salvataggio e alla visualizzazione delle informazioni riguardanti le classi di un ente
\end{itemize}
\subsubsection{Classe StudentsClassFetcher}
Recupera la lista degli studenti appartenenti ad una determinata classe.
\begin{figure}[H]
\centering
\noindent\makebox[\textwidth]{\includegraphics[width=\textwidth]{Img/quizzipedia-server-controllerserver-classmanager-studentsclassfetcher.pdf}}
\caption[Schema Classe StudentsClassFetcher]{Schema Classe Quizzipedia::Server::ControllerServer::ClassManager::StudentsClassFetcher}
\end{figure}
\paragraph{Relazioni con altre classi}
\subparagraph{Entranti}
\begin{itemize}
\item usata da Quizzipedia::Server::RoutingManager::ClassRouter per Package che racchiude tutte le funzionalità adibite al salvataggio e alla visualizzazione delle informazioni riguardanti le classi di un ente
\end{itemize}
\subparagraph{Uscenti}
\begin{itemize}
\item usa Quizzipedia::Server::ControllerServer::ClassManager::SessionController per Package che racchiude tutte le funzionalità adibite al salvataggio e alla visualizzazione delle informazioni riguardanti le classi di un ente
\end{itemize}
\subsubsection{Classe TeachersClassFetcher}
Recupera la lista degli insegnanti relativi ad una determinata classe.
\begin{figure}[H]
\centering
\noindent\makebox[\textwidth]{\includegraphics[width=\textwidth]{Img/quizzipedia-server-controllerserver-classmanager-teachersclassfetcher.pdf}}
\caption[Schema Classe TeachersClassFetcher]{Schema Classe Quizzipedia::Server::ControllerServer::ClassManager::TeachersClassFetcher}
\end{figure}
\paragraph{Relazioni con altre classi}
\subparagraph{Entranti}
\begin{itemize}
\item usata da Quizzipedia::Server::RoutingManager::ClassRouter per Package che racchiude tutte le funzionalità adibite al salvataggio e alla visualizzazione delle informazioni riguardanti le classi di un ente
\end{itemize}
\subparagraph{Uscenti}
\begin{itemize}
\item usa Quizzipedia::Server::ControllerServer::ClassManager::SessionController per Package che racchiude tutte le funzionalità adibite al salvataggio e alla visualizzazione delle informazioni riguardanti le classi di un ente
\end{itemize}
\subsection{Quizzipedia::Server::ControllerServer::CompanyManager}
Permette la gestione dell'ente da parte del responsabile.
\begin{figure}[H]
\centering
\noindent\makebox[\textwidth]{\includegraphics[width=\textwidth]{Img/quizzipedia-server-controllerserver-companymanager.pdf}}
\caption[Schema Componente Quizzipedia::Server::ControllerServer::CompanyManager]{Schema Componente Quizzipedia::Server::ControllerServer::CompanyManager}
\end{figure}
\subsubsection{Classe CompanyUpdater}
Permette la modifica dell'ente da parte del responsabile.
\begin{figure}[H]
\centering
\noindent\makebox[\textwidth]{\includegraphics[width=\textwidth]{Img/quizzipedia-server-controllerserver-companymanager-companyupdater.pdf}}
\caption[Schema Classe CompanyUpdater]{Schema Classe Quizzipedia::Server::ControllerServer::CompanyManager::CompanyUpdater}
\end{figure}
\paragraph{Relazioni con altre classi}
\subparagraph{Entranti}
\begin{itemize}
\item usata da Quizzipedia::Server::RoutingManager::CompanyRouter per Permette la gestione dell'ente da parte del responsabile
\end{itemize}
\subparagraph{Uscenti}
\begin{itemize}
\item usa Quizzipedia::Server::ControllerServer::CompanyManager::SessionController per Permette la gestione dell'ente da parte del responsabile
\end{itemize}
\subsubsection{Classe SessionController}
Effettua il controllo sull'utente per verificare che egli sia in possesso dell'autorizzazione necessaria per compiere determinate richieste alla base di dati.
\begin{figure}[H]
\centering
\noindent\makebox[\textwidth]{\includegraphics[width=\textwidth]{Img/quizzipedia-server-controllerserver-companymanager-sessioncontroller.pdf}}
\caption[Schema Classe SessionController]{Schema Classe Quizzipedia::Server::ControllerServer::CompanyManager::SessionController}
\end{figure}
\paragraph{Relazioni con altre classi}
\subparagraph{Entranti}
\begin{itemize}
\item usata da Quizzipedia::Server::ControllerServer::CompanyManager::CompanyUpdater per Permette la gestione dell'ente da parte del responsabile
\end{itemize}
\subsection{Quizzipedia::Server::ControllerServer::ProfileManager}
Package che racchiude tutte le funzionalità adibite al salvataggio e alla visualizzazione delle informazioni personali da parte di un utente autenticato.
\begin{figure}[H]
\centering
\noindent\makebox[\textwidth]{\includegraphics[width=\textwidth]{Img/quizzipedia-server-controllerserver-profilemanager.pdf}}
\caption[Schema Componente Quizzipedia::Server::ControllerServer::ProfileManager]{Schema Componente Quizzipedia::Server::ControllerServer::ProfileManager}
\end{figure}
\subsubsection{Classe AccountDeleter}
Permette la rimozione di un account dal sistema.
\begin{figure}[H]
\centering
\noindent\makebox[\textwidth]{\includegraphics[width=\textwidth]{Img/quizzipedia-server-controllerserver-profilemanager-accountdeleter.pdf}}
\caption[Schema Classe AccountDeleter]{Schema Classe Quizzipedia::Server::ControllerServer::ProfileManager::AccountDeleter}
\end{figure}
\paragraph{Relazioni con altre classi}
\subparagraph{Entranti}
\begin{itemize}
\item usata da Quizzipedia::Server::RoutingManager::ProfileRouter per Package che racchiude tutte le funzionalità adibite al salvataggio e alla visualizzazione delle informazioni personali da parte di un utente autenticato
\end{itemize}
\subparagraph{Uscenti}
\begin{itemize}
\item usa Quizzipedia::Server::ControllerServer::ProfileManager::SessionController per Package che racchiude tutte le funzionalità adibite al salvataggio e alla visualizzazione delle informazioni personali da parte di un utente autenticato
\end{itemize}
\subsubsection{Classe PasswordSetter}
Permette ad un utente di impostare una nuova password relativa al proprio account.
\begin{figure}[H]
\centering
\noindent\makebox[\textwidth]{\includegraphics[width=\textwidth]{Img/quizzipedia-server-controllerserver-profilemanager-passwordsetter.pdf}}
\caption[Schema Classe PasswordSetter]{Schema Classe Quizzipedia::Server::ControllerServer::ProfileManager::PasswordSetter}
\end{figure}
\paragraph{Relazioni con altre classi}
\subparagraph{Entranti}
\begin{itemize}
\item usata da Quizzipedia::Server::RoutingManager::ProfileRouter per Package che racchiude tutte le funzionalità adibite al salvataggio e alla visualizzazione delle informazioni personali da parte di un utente autenticato
\end{itemize}
\subparagraph{Uscenti}
\begin{itemize}
\item usa Quizzipedia::Server::ControllerServer::ProfileManager::SessionController per Package che racchiude tutte le funzionalità adibite al salvataggio e alla visualizzazione delle informazioni personali da parte di un utente autenticato
\end{itemize}
\subsubsection{Classe PersonalDataFetcher}
Ritorna tutte le informazioni personali riferite all'utente che ne effettua la richiesta.
\begin{figure}[H]
\centering
\noindent\makebox[\textwidth]{\includegraphics[width=\textwidth]{Img/quizzipedia-server-controllerserver-profilemanager-personaldatafetcher.pdf}}
\caption[Schema Classe PersonalDataFetcher]{Schema Classe Quizzipedia::Server::ControllerServer::ProfileManager::PersonalDataFetcher}
\end{figure}
\paragraph{Relazioni con altre classi}
\subparagraph{Entranti}
\begin{itemize}
\item usata da Quizzipedia::Server::RoutingManager::ProfileRouter per Package che racchiude tutte le funzionalità adibite al salvataggio e alla visualizzazione delle informazioni personali da parte di un utente autenticato
\end{itemize}
\subparagraph{Uscenti}
\begin{itemize}
\item usa Quizzipedia::Server::ControllerServer::ProfileManager::SessionController per Package che racchiude tutte le funzionalità adibite al salvataggio e alla visualizzazione delle informazioni personali da parte di un utente autenticato
\end{itemize}
\subsubsection{Classe PersonalDataSetter}
Permette ad un utente la modifica delle informazioni personali.
\begin{figure}[H]
\centering
\noindent\makebox[\textwidth]{\includegraphics[width=\textwidth]{Img/quizzipedia-server-controllerserver-profilemanager-personaldatasetter.pdf}}
\caption[Schema Classe PersonalDataSetter]{Schema Classe Quizzipedia::Server::ControllerServer::ProfileManager::PersonalDataSetter}
\end{figure}
\paragraph{Relazioni con altre classi}
\subparagraph{Entranti}
\begin{itemize}
\item usata da Quizzipedia::Server::RoutingManager::ProfileRouter per Package che racchiude tutte le funzionalità adibite al salvataggio e alla visualizzazione delle informazioni personali da parte di un utente autenticato
\end{itemize}
\subparagraph{Uscenti}
\begin{itemize}
\item usa Quizzipedia::Server::ControllerServer::ProfileManager::SessionController per Package che racchiude tutte le funzionalità adibite al salvataggio e alla visualizzazione delle informazioni personali da parte di un utente autenticato
\end{itemize}
\subsubsection{Classe PersonalQuizFetcher}
Ritorna una lista contenente tutti i quiz che un utente autenticato ha svolto fino a quel momento.
\begin{figure}[H]
\centering
\noindent\makebox[\textwidth]{\includegraphics[width=\textwidth]{Img/quizzipedia-server-controllerserver-profilemanager-personalquizfetcher.pdf}}
\caption[Schema Classe PersonalQuizFetcher]{Schema Classe Quizzipedia::Server::ControllerServer::ProfileManager::PersonalQuizFetcher}
\end{figure}
\paragraph{Relazioni con altre classi}
\subparagraph{Entranti}
\begin{itemize}
\item usata da Quizzipedia::Server::RoutingManager::ProfileRouter per Package che racchiude tutte le funzionalità adibite al salvataggio e alla visualizzazione delle informazioni personali da parte di un utente autenticato
\end{itemize}
\subparagraph{Uscenti}
\begin{itemize}
\item usa Quizzipedia::Server::ControllerServer::ProfileManager::SessionController per Package che racchiude tutte le funzionalità adibite al salvataggio e alla visualizzazione delle informazioni personali da parte di un utente autenticato
\end{itemize}
\subsubsection{Classe SessionController}
Effettua il controllo sull'utente per verificare che egli sia in possesso dell'autorizzazione necessaria per compiere determinate richieste alla base di dati.
\begin{figure}[H]
\centering
\noindent\makebox[\textwidth]{\includegraphics[width=\textwidth]{Img/quizzipedia-server-controllerserver-profilemanager-sessioncontroller.pdf}}
\caption[Schema Classe SessionController]{Schema Classe Quizzipedia::Server::ControllerServer::ProfileManager::SessionController}
\end{figure}
\paragraph{Relazioni con altre classi}
\subparagraph{Entranti}
\begin{itemize}
\item usata da Quizzipedia::Server::ControllerServer::ProfileManager::AccountDeleter per Package che racchiude tutte le funzionalità adibite al salvataggio e alla visualizzazione delle informazioni personali da parte di un utente autenticato
\item usata da Quizzipedia::Server::ControllerServer::ProfileManager::PasswordSetter per Package che racchiude tutte le funzionalità adibite al salvataggio e alla visualizzazione delle informazioni personali da parte di un utente autenticato
\item usata da Quizzipedia::Server::ControllerServer::ProfileManager::PersonalDataFetcher per Package che racchiude tutte le funzionalità adibite al salvataggio e alla visualizzazione delle informazioni personali da parte di un utente autenticato
\item usata da Quizzipedia::Server::ControllerServer::ProfileManager::PersonalDataSetter per Package che racchiude tutte le funzionalità adibite al salvataggio e alla visualizzazione delle informazioni personali da parte di un utente autenticato
\item usata da Quizzipedia::Server::ControllerServer::ProfileManager::PersonalQuizFetcher per Package che racchiude tutte le funzionalità adibite al salvataggio e alla visualizzazione delle informazioni personali da parte di un utente autenticato
\end{itemize}
\subsection{Quizzipedia::Server::ControllerServer::QuestionsManager}
Pacchetto relativo alla gestione delle domande.
\begin{figure}[H]
\centering
\noindent\makebox[\textwidth]{\includegraphics[width=\textwidth]{Img/quizzipedia-server-controllerserver-questionsmanager.pdf}}
\caption[Schema Componente Quizzipedia::Server::ControllerServer::QuestionsManager]{Schema Componente Quizzipedia::Server::ControllerServer::QuestionsManager}
\end{figure}
\subsubsection{Classe QuestionCreator}
Permette il salvataggio nella base di dati di una nuova domanda.
\begin{figure}[H]
\centering
\noindent\makebox[\textwidth]{\includegraphics[width=\textwidth]{Img/quizzipedia-server-controllerserver-questionsmanager-questioncreator.pdf}}
\caption[Schema Classe QuestionCreator]{Schema Classe Quizzipedia::Server::ControllerServer::QuestionsManager::QuestionCreator}
\end{figure}
\paragraph{Relazioni con altre classi}
\subparagraph{Entranti}
\begin{itemize}
\item usata da Quizzipedia::Server::RoutingManager::QuestionRouter per Pacchetto relativo alla gestione delle domande
\end{itemize}
\subparagraph{Uscenti}
\begin{itemize}
\item usa Quizzipedia::Server::ControllerServer::QuestionsManager::SessionController per Pacchetto relativo alla gestione delle domande
\end{itemize}
\subsubsection{Classe QuestionEraser}
Permette la cancellazione di una domanda dalla base di dati.
\begin{figure}[H]
\centering
\noindent\makebox[\textwidth]{\includegraphics[width=\textwidth]{Img/quizzipedia-server-controllerserver-questionsmanager-questioneraser.pdf}}
\caption[Schema Classe QuestionEraser]{Schema Classe Quizzipedia::Server::ControllerServer::QuestionsManager::QuestionEraser}
\end{figure}
\paragraph{Relazioni con altre classi}
\subparagraph{Entranti}
\begin{itemize}
\item usata da Quizzipedia::Server::RoutingManager::QuestionRouter per Pacchetto relativo alla gestione delle domande
\end{itemize}
\subparagraph{Uscenti}
\begin{itemize}
\item usa Quizzipedia::Server::ControllerServer::QuestionsManager::SessionController per Pacchetto relativo alla gestione delle domande
\end{itemize}
\subsubsection{Classe QuestionUpdater}
Permette la modifica di una domanda già esistente.
\begin{figure}[H]
\centering
\noindent\makebox[\textwidth]{\includegraphics[width=\textwidth]{Img/quizzipedia-server-controllerserver-questionsmanager-questionupdater.pdf}}
\caption[Schema Classe QuestionUpdater]{Schema Classe Quizzipedia::Server::ControllerServer::QuestionsManager::QuestionUpdater}
\end{figure}
\paragraph{Relazioni con altre classi}
\subparagraph{Entranti}
\begin{itemize}
\item usata da Quizzipedia::Server::RoutingManager::QuestionRouter per Pacchetto relativo alla gestione delle domande
\end{itemize}
\subparagraph{Uscenti}
\begin{itemize}
\item usa Quizzipedia::Server::ControllerServer::QuestionsManager::SessionController per Pacchetto relativo alla gestione delle domande
\end{itemize}
\subsubsection{Classe SessionController}
Effettua il controllo sull'utente per verificare che egli sia in possesso dell'autorizzazione necessaria per compiere determinate richieste alla base di dati.
\begin{figure}[H]
\centering
\noindent\makebox[\textwidth]{\includegraphics[width=\textwidth]{Img/quizzipedia-server-controllerserver-questionsmanager-sessioncontroller.pdf}}
\caption[Schema Classe SessionController]{Schema Classe Quizzipedia::Server::ControllerServer::QuestionsManager::SessionController}
\end{figure}
\paragraph{Relazioni con altre classi}
\subparagraph{Entranti}
\begin{itemize}
\item usata da Quizzipedia::Server::ControllerServer::QuestionsManager::QuestionCreator per Pacchetto relativo alla gestione delle domande
\item usata da Quizzipedia::Server::ControllerServer::QuestionsManager::QuestionEraser per Pacchetto relativo alla gestione delle domande
\item usata da Quizzipedia::Server::ControllerServer::QuestionsManager::QuestionUpdater per Pacchetto relativo alla gestione delle domande
\end{itemize}
\subsection{Quizzipedia::Server::ControllerServer::QuizManager}
Package che racchiude tutte le funzionalità adibite alla creazione, modifica e al recupero di quiz per lo svolgimento da parte di un utente.
\begin{figure}[H]
\centering
\noindent\makebox[\textwidth]{\includegraphics[width=\textwidth]{Img/quizzipedia-server-controllerserver-quizmanager.pdf}}
\caption[Schema Componente Quizzipedia::Server::ControllerServer::QuizManager]{Schema Componente Quizzipedia::Server::ControllerServer::QuizManager}
\end{figure}
\subsubsection{Componenti contenute}
\begin{itemize}
\item Quizzipedia::Server::ControllerServer::QuizManager::QMLAgent
\end{itemize}
\subsubsection{Classe QuizCreator}
Permette il salvataggio nella base di dati di un nuovo quiz.
\begin{figure}[H]
\centering
\noindent\makebox[\textwidth]{\includegraphics[width=\textwidth]{Img/quizzipedia-server-controllerserver-quizmanager-quizcreator.pdf}}
\caption[Schema Classe QuizCreator]{Schema Classe Quizzipedia::Server::ControllerServer::QuizManager::QuizCreator}
\end{figure}
\paragraph{Relazioni con altre classi}
\subparagraph{Entranti}
\begin{itemize}
\item usata da Quizzipedia::Server::RoutingManager::QuizRouter per Package che racchiude tutte le funzionalità adibite alla creazione, modifica e al recupero di quiz per lo svolgimento da parte di un utente
\end{itemize}
\subparagraph{Uscenti}
\begin{itemize}
\item usa Quizzipedia::Server::ControllerServer::QuizManager::QMLAgent::QMLGenerator per Package che racchiude tutte le funzionalità adibite alla creazione, modifica e al recupero di quiz per lo svolgimento da parte di un utente
\item usa Quizzipedia::Server::ControllerServer::QuizManager::SessionController per Package che racchiude tutte le funzionalità adibite alla creazione, modifica e al recupero di quiz per lo svolgimento da parte di un utente
\end{itemize}
\subsubsection{Classe QuizEraser}
Permette la cancellazione di un quiz dalla base di dati.
\begin{figure}[H]
\centering
\noindent\makebox[\textwidth]{\includegraphics[width=\textwidth]{Img/quizzipedia-server-controllerserver-quizmanager-quizeraser.pdf}}
\caption[Schema Classe QuizEraser]{Schema Classe Quizzipedia::Server::ControllerServer::QuizManager::QuizEraser}
\end{figure}
\paragraph{Relazioni con altre classi}
\subparagraph{Entranti}
\begin{itemize}
\item usata da Quizzipedia::Server::RoutingManager::QuizRouter per Package che racchiude tutte le funzionalità adibite alla creazione, modifica e al recupero di quiz per lo svolgimento da parte di un utente
\end{itemize}
\subparagraph{Uscenti}
\begin{itemize}
\item usa Quizzipedia::Server::ControllerServer::QuizManager::SessionController per Package che racchiude tutte le funzionalità adibite alla creazione, modifica e al recupero di quiz per lo svolgimento da parte di un utente
\end{itemize}
\subsubsection{Classe QuizFetcher}
Ritorna un determinato quiz pronto per essere svolto da un utente .
\begin{figure}[H]
\centering
\noindent\makebox[\textwidth]{\includegraphics[width=\textwidth]{Img/quizzipedia-server-controllerserver-quizmanager-quizfetcher.pdf}}
\caption[Schema Classe QuizFetcher]{Schema Classe Quizzipedia::Server::ControllerServer::QuizManager::QuizFetcher}
\end{figure}
\paragraph{Relazioni con altre classi}
\subparagraph{Entranti}
\begin{itemize}
\item usata da Quizzipedia::Server::RoutingManager::QuizRouter per Package che racchiude tutte le funzionalità adibite alla creazione, modifica e al recupero di quiz per lo svolgimento da parte di un utente
\end{itemize}
\subparagraph{Uscenti}
\begin{itemize}
\item usa Quizzipedia::Server::ControllerServer::QuizManager::QMLAgent::QMLParser per Package che racchiude tutte le funzionalità adibite alla creazione, modifica e al recupero di quiz per lo svolgimento da parte di un utente
\item usa Quizzipedia::Server::ControllerServer::QuizManager::SessionController per Package che racchiude tutte le funzionalità adibite alla creazione, modifica e al recupero di quiz per lo svolgimento da parte di un utente
\end{itemize}
\subsubsection{Classe QuizUpdater}
Permette la modifica di un quiz già presente nella base di dati.
\begin{figure}[H]
\centering
\noindent\makebox[\textwidth]{\includegraphics[width=\textwidth]{Img/quizzipedia-server-controllerserver-quizmanager-quizupdater.pdf}}
\caption[Schema Classe QuizUpdater]{Schema Classe Quizzipedia::Server::ControllerServer::QuizManager::QuizUpdater}
\end{figure}
\paragraph{Relazioni con altre classi}
\subparagraph{Entranti}
\begin{itemize}
\item usata da Quizzipedia::Server::RoutingManager::QuizRouter per Package che racchiude tutte le funzionalità adibite alla creazione, modifica e al recupero di quiz per lo svolgimento da parte di un utente
\end{itemize}
\subparagraph{Uscenti}
\begin{itemize}
\item usa Quizzipedia::Server::ControllerServer::QuizManager::QMLAgent::QMLGenerator per Package che racchiude tutte le funzionalità adibite alla creazione, modifica e al recupero di quiz per lo svolgimento da parte di un utente
\item usa Quizzipedia::Server::ControllerServer::QuizManager::SessionController per Package che racchiude tutte le funzionalità adibite alla creazione, modifica e al recupero di quiz per lo svolgimento da parte di un utente
\end{itemize}
\subsubsection{Classe ResultsUpdater}
Aggiorna i risultati dei quiz ad ogni svolgimento degli stessi da parte di un utente.
\begin{figure}[H]
\centering
\noindent\makebox[\textwidth]{\includegraphics[width=\textwidth]{Img/quizzipedia-server-controllerserver-quizmanager-resultsupdater.pdf}}
\caption[Schema Classe ResultsUpdater]{Schema Classe Quizzipedia::Server::ControllerServer::QuizManager::ResultsUpdater}
\end{figure}
\paragraph{Relazioni con altre classi}
\subparagraph{Entranti}
\begin{itemize}
\item usata da Quizzipedia::Server::RoutingManager::QuizRouter per Package che racchiude tutte le funzionalità adibite alla creazione, modifica e al recupero di quiz per lo svolgimento da parte di un utente
\end{itemize}
\subparagraph{Uscenti}
\begin{itemize}
\item usa Quizzipedia::Server::ControllerServer::QuizManager::SessionController per Package che racchiude tutte le funzionalità adibite alla creazione, modifica e al recupero di quiz per lo svolgimento da parte di un utente
\end{itemize}
\subsubsection{Classe SessionController}
Effettua il controllo sull'utente per verificare che egli sia in possesso dell'autorizzazione necessaria per compiere determinate richieste alla base di dati.
\begin{figure}[H]
\centering
\noindent\makebox[\textwidth]{\includegraphics[width=\textwidth]{Img/quizzipedia-server-controllerserver-quizmanager-sessioncontroller.pdf}}
\caption[Schema Classe SessionController]{Schema Classe Quizzipedia::Server::ControllerServer::QuizManager::SessionController}
\end{figure}
\paragraph{Relazioni con altre classi}
\subparagraph{Entranti}
\begin{itemize}
\item usata da Quizzipedia::Server::ControllerServer::QuizManager::QuizCreator per Package che racchiude tutte le funzionalità adibite alla creazione, modifica e al recupero di quiz per lo svolgimento da parte di un utente
\item usata da Quizzipedia::Server::ControllerServer::QuizManager::QuizEraser per Package che racchiude tutte le funzionalità adibite alla creazione, modifica e al recupero di quiz per lo svolgimento da parte di un utente
\item usata da Quizzipedia::Server::ControllerServer::QuizManager::QuizFetcher per Package che racchiude tutte le funzionalità adibite alla creazione, modifica e al recupero di quiz per lo svolgimento da parte di un utente
\item usata da Quizzipedia::Server::ControllerServer::QuizManager::QuizUpdater per Package che racchiude tutte le funzionalità adibite alla creazione, modifica e al recupero di quiz per lo svolgimento da parte di un utente
\item usata da Quizzipedia::Server::ControllerServer::QuizManager::ResultsUpdater per Package che racchiude tutte le funzionalità adibite alla creazione, modifica e al recupero di quiz per lo svolgimento da parte di un utente
\item usata da Quizzipedia::Server::ControllerServer::QuizManager::StatisticsUpdater per Package che racchiude tutte le funzionalità adibite alla creazione, modifica e al recupero di quiz per lo svolgimento da parte di un utente
\end{itemize}
\subsubsection{Classe StatisticsUpdater}
Aggiorna le statistiche relative ad un quiz ogni volta che viene svolto da parte degli utenti.
\begin{figure}[H]
\centering
\noindent\makebox[\textwidth]{\includegraphics[width=\textwidth]{Img/quizzipedia-server-controllerserver-quizmanager-statisticsupdater.pdf}}
\caption[Schema Classe StatisticsUpdater]{Schema Classe Quizzipedia::Server::ControllerServer::QuizManager::StatisticsUpdater}
\end{figure}
\paragraph{Relazioni con altre classi}
\subparagraph{Entranti}
\begin{itemize}
\item usata da Quizzipedia::Server::RoutingManager::QuizRouter per Package che racchiude tutte le funzionalità adibite alla creazione, modifica e al recupero di quiz per lo svolgimento da parte di un utente
\end{itemize}
\subparagraph{Uscenti}
\begin{itemize}
\item usa Quizzipedia::Server::ControllerServer::QuizManager::SessionController per Package che racchiude tutte le funzionalità adibite alla creazione, modifica e al recupero di quiz per lo svolgimento da parte di un utente
\end{itemize}
\subsection{Quizzipedia::Server::ControllerServer::QuizManager::QMLAgent}
Questo package racchiude i moduli necessari alla traduzione, da QML ad un formato comprensibile dal sistema, delle informazioni estratte dal database per la generazione delle pagine HTML relative ad un quiz e viceversa.
\begin{figure}[H]
\centering
\noindent\makebox[\textwidth]{\includegraphics[width=\textwidth]{Img/quizzipedia-server-controllerserver-quizmanager-qmlagent.pdf}}
\caption[Schema Componente Quizzipedia::Server::ControllerServer::QuizManager::QMLAgent]{Schema Componente Quizzipedia::Server::ControllerServer::QuizManager::QMLAgent}
\end{figure}
\subsubsection{Classe QMLGenerator}
Permette la traduzione in formato QML di un quiz nel caso si voglia procedere al salvataggio dello stesso all'interno del database.
\begin{figure}[H]
\centering
\noindent\makebox[\textwidth]{\includegraphics[width=\textwidth]{Img/quizzipedia-server-controllerserver-quizmanager-qmlagent-qmlgenerator.pdf}}
\caption[Schema Classe QMLGenerator]{Schema Classe Quizzipedia::Server::ControllerServer::QuizManager::QMLAgent::QMLGenerator}
\end{figure}
\paragraph{Relazioni con altre classi}
\subparagraph{Entranti}
\begin{itemize}
\item usata da Quizzipedia::Server::ControllerServer::QuizManager::QuizCreator per Questo package racchiude i moduli necessari alla traduzione, da QML ad un formato comprensibile dal sistema, delle informazioni estratte dal database per la generazione delle pagine HTML relative ad un quiz e viceversa
\item usata da Quizzipedia::Server::ControllerServer::QuizManager::QuizUpdater per Questo package racchiude i moduli necessari alla traduzione, da QML ad un formato comprensibile dal sistema, delle informazioni estratte dal database per la generazione delle pagine HTML relative ad un quiz e viceversa
\end{itemize}
\subsubsection{Classe QMLParser}
Permette la traduzione di un quiz dal formato QML ad uno comprensibile dal sistema per la generazione delle relative pagine HTML.
\begin{figure}[H]
\centering
\noindent\makebox[\textwidth]{\includegraphics[width=\textwidth]{Img/quizzipedia-server-controllerserver-quizmanager-qmlagent-qmlparser.pdf}}
\caption[Schema Classe QMLParser]{Schema Classe Quizzipedia::Server::ControllerServer::QuizManager::QMLAgent::QMLParser}
\end{figure}
\paragraph{Relazioni con altre classi}
\subparagraph{Entranti}
\begin{itemize}
\item usata da Quizzipedia::Server::ControllerServer::QuizManager::QuizFetcher per Questo package racchiude i moduli necessari alla traduzione, da QML ad un formato comprensibile dal sistema, delle informazioni estratte dal database per la generazione delle pagine HTML relative ad un quiz e viceversa
\end{itemize}
\subsection{Quizzipedia::Server::ControllerServer::RequestsManager}
Package che si occupa di memorizzare richieste da parte degli utenti, mostrarle al responsabile e permettergli di accettarle o meno.
\begin{figure}[H]
\centering
\noindent\makebox[\textwidth]{\includegraphics[width=\textwidth]{Img/quizzipedia-server-controllerserver-requestsmanager.pdf}}
\caption[Schema Componente Quizzipedia::Server::ControllerServer::RequestsManager]{Schema Componente Quizzipedia::Server::ControllerServer::RequestsManager}
\end{figure}
\subsubsection{Classe ClassRequestsAdder}
Permette la memorizzazione delle richieste di creazione di una classe.
\begin{figure}[H]
\centering
\noindent\makebox[\textwidth]{\includegraphics[width=\textwidth]{Img/quizzipedia-server-controllerserver-requestsmanager-classrequestsadder.pdf}}
\caption[Schema Classe ClassRequestsAdder]{Schema Classe Quizzipedia::Server::ControllerServer::RequestsManager::ClassRequestsAdder}
\end{figure}
\paragraph{Relazioni con altre classi}
\subparagraph{Entranti}
\begin{itemize}
\item usata da Quizzipedia::Server::RoutingManager::RequestsRouter per Package che si occupa di memorizzare richieste da parte degli utenti, mostrarle al responsabile e permettergli di accettarle o meno
\end{itemize}
\subparagraph{Uscenti}
\begin{itemize}
\item usa Quizzipedia::Server::ControllerServer::RequestsManager::SessionController per Package che si occupa di memorizzare richieste da parte degli utenti, mostrarle al responsabile e permettergli di accettarle o meno
\end{itemize}
\subsubsection{Classe InsertClassRequestsAdder}
Permette la memorizzazione nella base di dati di tutte le richieste, fatte da parte degli utenti, di essere inseriti in una determinata classe.
\begin{figure}[H]
\centering
\noindent\makebox[\textwidth]{\includegraphics[width=\textwidth]{Img/quizzipedia-server-controllerserver-requestsmanager-insertclassrequestsadder.pdf}}
\caption[Schema Classe InsertClassRequestsAdder]{Schema Classe Quizzipedia::Server::ControllerServer::RequestsManager::InsertClassRequestsAdder}
\end{figure}
\paragraph{Relazioni con altre classi}
\subparagraph{Entranti}
\begin{itemize}
\item usata da Quizzipedia::Server::RoutingManager::RequestsRouter per Package che si occupa di memorizzare richieste da parte degli utenti, mostrarle al responsabile e permettergli di accettarle o meno
\end{itemize}
\subparagraph{Uscenti}
\begin{itemize}
\item usa Quizzipedia::Server::ControllerServer::RequestsManager::SessionController per Package che si occupa di memorizzare richieste da parte degli utenti, mostrarle al responsabile e permettergli di accettarle o meno
\end{itemize}
\subsubsection{Classe RequestsFetcher}
Permette la visualizzazione di tutte le richieste da parte degli utenti.
\begin{figure}[H]
\centering
\noindent\makebox[\textwidth]{\includegraphics[width=\textwidth]{Img/quizzipedia-server-controllerserver-requestsmanager-requestsfetcher.pdf}}
\caption[Schema Classe RequestsFetcher]{Schema Classe Quizzipedia::Server::ControllerServer::RequestsManager::RequestsFetcher}
\end{figure}
\paragraph{Relazioni con altre classi}
\subparagraph{Entranti}
\begin{itemize}
\item usata da Quizzipedia::Server::RoutingManager::RequestsRouter per Package che si occupa di memorizzare richieste da parte degli utenti, mostrarle al responsabile e permettergli di accettarle o meno
\end{itemize}
\subparagraph{Uscenti}
\begin{itemize}
\item usa Quizzipedia::Server::ControllerServer::RequestsManager::SessionController per Package che si occupa di memorizzare richieste da parte degli utenti, mostrarle al responsabile e permettergli di accettarle o meno
\end{itemize}
\subsubsection{Classe RoleAccepter}
Permette l'accettazione o la negazione dell'assegnazione di un ruolo ad un utente dopo che ne ha fatto richiesta.
\begin{figure}[H]
\centering
\noindent\makebox[\textwidth]{\includegraphics[width=\textwidth]{Img/quizzipedia-server-controllerserver-requestsmanager-roleaccepter.pdf}}
\caption[Schema Classe RoleAccepter]{Schema Classe Quizzipedia::Server::ControllerServer::RequestsManager::RoleAccepter}
\end{figure}
\paragraph{Relazioni con altre classi}
\subparagraph{Entranti}
\begin{itemize}
\item usata da Quizzipedia::Server::RoutingManager::RequestsRouter per Package che si occupa di memorizzare richieste da parte degli utenti, mostrarle al responsabile e permettergli di accettarle o meno
\end{itemize}
\subparagraph{Uscenti}
\begin{itemize}
\item usa Quizzipedia::Server::ControllerServer::RequestsManager::SessionController per Package che si occupa di memorizzare richieste da parte degli utenti, mostrarle al responsabile e permettergli di accettarle o meno
\end{itemize}
\subsubsection{Classe RoleRequestAdder}
Memorizza richieste da parte degli utenti di assumere un determinato ruolo all'interno del sistema.
\begin{figure}[H]
\centering
\noindent\makebox[\textwidth]{\includegraphics[width=\textwidth]{Img/quizzipedia-server-controllerserver-requestsmanager-rolerequestadder.pdf}}
\caption[Schema Classe RoleRequestAdder]{Schema Classe Quizzipedia::Server::ControllerServer::RequestsManager::RoleRequestAdder}
\end{figure}
\paragraph{Relazioni con altre classi}
\subparagraph{Entranti}
\begin{itemize}
\item usata da Quizzipedia::Server::RoutingManager::RequestsRouter per Package che si occupa di memorizzare richieste da parte degli utenti, mostrarle al responsabile e permettergli di accettarle o meno
\end{itemize}
\subparagraph{Uscenti}
\begin{itemize}
\item usa Quizzipedia::Server::ControllerServer::RequestsManager::SessionController per Package che si occupa di memorizzare richieste da parte degli utenti, mostrarle al responsabile e permettergli di accettarle o meno
\end{itemize}
\subsubsection{Classe SessionController}
Effettua il controllo sull'utente per verificare che egli sia in possesso dell'autorizzazione necessaria per compiere determinate richieste alla base di dati.
\begin{figure}[H]
\centering
\noindent\makebox[\textwidth]{\includegraphics[width=\textwidth]{Img/quizzipedia-server-controllerserver-requestsmanager-sessioncontroller.pdf}}
\caption[Schema Classe SessionController]{Schema Classe Quizzipedia::Server::ControllerServer::RequestsManager::SessionController}
\end{figure}
\paragraph{Relazioni con altre classi}
\subparagraph{Entranti}
\begin{itemize}
\item usata da Quizzipedia::Server::ControllerServer::RequestsManager::ClassRequestsAdder per Package che si occupa di memorizzare richieste da parte degli utenti, mostrarle al responsabile e permettergli di accettarle o meno
\item usata da Quizzipedia::Server::ControllerServer::RequestsManager::InsertClassRequestsAdder per Package che si occupa di memorizzare richieste da parte degli utenti, mostrarle al responsabile e permettergli di accettarle o meno
\item usata da Quizzipedia::Server::ControllerServer::RequestsManager::RequestsFetcher per Package che si occupa di memorizzare richieste da parte degli utenti, mostrarle al responsabile e permettergli di accettarle o meno
\item usata da Quizzipedia::Server::ControllerServer::RequestsManager::RoleAccepter per Package che si occupa di memorizzare richieste da parte degli utenti, mostrarle al responsabile e permettergli di accettarle o meno
\item usata da Quizzipedia::Server::ControllerServer::RequestsManager::RoleRequestAdder per Package che si occupa di memorizzare richieste da parte degli utenti, mostrarle al responsabile e permettergli di accettarle o meno
\end{itemize}
\subsection{Quizzipedia::Server::ControllerServer::SearchManager}
Questo package permette di effettuare una ricerca nel database di quiz o domande richiesti dall'utente e ritornare una lista che corrisponde ai parametri desiderati.
\begin{figure}[H]
\centering
\noindent\makebox[\textwidth]{\includegraphics[width=\textwidth]{Img/quizzipedia-server-controllerserver-searchmanager.pdf}}
\caption[Schema Componente Quizzipedia::Server::ControllerServer::SearchManager]{Schema Componente Quizzipedia::Server::ControllerServer::SearchManager}
\end{figure}
\subsubsection{Classe QuestionsSearcher}
Ritorna una lista di domande che corrispondono ai parametri di ricerca impostati dall'utente.
\begin{figure}[H]
\centering
\noindent\makebox[\textwidth]{\includegraphics[width=\textwidth]{Img/quizzipedia-server-controllerserver-searchmanager-questionssearcher.pdf}}
\caption[Schema Classe QuestionsSearcher]{Schema Classe Quizzipedia::Server::ControllerServer::SearchManager::QuestionsSearcher}
\end{figure}
\paragraph{Relazioni con altre classi}
\subparagraph{Entranti}
\begin{itemize}
\item usata da Quizzipedia::Server::RoutingManager::SearchRouter per Questo package permette di effettuare una ricerca nel database di quiz o domande richiesti dall'utente e ritornare una lista che corrisponde ai parametri desiderati
\end{itemize}
\subparagraph{Uscenti}
\begin{itemize}
\item usa Quizzipedia::Server::ControllerServer::SearchManager::SessionController per Questo package permette di effettuare una ricerca nel database di quiz o domande richiesti dall'utente e ritornare una lista che corrisponde ai parametri desiderati
\end{itemize}
\subsubsection{Classe QuizSearcher}
Ritorna una lista di quiz che corrispondono ai parametri di ricerca impostati dall'utente.
\begin{figure}[H]
\centering
\noindent\makebox[\textwidth]{\includegraphics[width=\textwidth]{Img/quizzipedia-server-controllerserver-searchmanager-quizsearcher.pdf}}
\caption[Schema Classe QuizSearcher]{Schema Classe Quizzipedia::Server::ControllerServer::SearchManager::QuizSearcher}
\end{figure}
\paragraph{Relazioni con altre classi}
\subparagraph{Entranti}
\begin{itemize}
\item usata da Quizzipedia::Server::RoutingManager::SearchRouter per Questo package permette di effettuare una ricerca nel database di quiz o domande richiesti dall'utente e ritornare una lista che corrisponde ai parametri desiderati
\end{itemize}
\subparagraph{Uscenti}
\begin{itemize}
\item usa Quizzipedia::Server::ControllerServer::SearchManager::SessionController per Questo package permette di effettuare una ricerca nel database di quiz o domande richiesti dall'utente e ritornare una lista che corrisponde ai parametri desiderati
\end{itemize}
\subsubsection{Classe SessionController}
Effettua il controllo sull'utente per verificare che egli sia in possesso dell'autorizzazione necessaria per compiere determinate richieste alla base di dati.
\begin{figure}[H]
\centering
\noindent\makebox[\textwidth]{\includegraphics[width=\textwidth]{Img/quizzipedia-server-controllerserver-searchmanager-sessioncontroller.pdf}}
\caption[Schema Classe SessionController]{Schema Classe Quizzipedia::Server::ControllerServer::SearchManager::SessionController}
\end{figure}
\paragraph{Relazioni con altre classi}
\subparagraph{Entranti}
\begin{itemize}
\item usata da Quizzipedia::Server::ControllerServer::SearchManager::QuestionsSearcher per Questo package permette di effettuare una ricerca nel database di quiz o domande richiesti dall'utente e ritornare una lista che corrisponde ai parametri desiderati
\item usata da Quizzipedia::Server::ControllerServer::SearchManager::QuizSearcher per Questo package permette di effettuare una ricerca nel database di quiz o domande richiesti dall'utente e ritornare una lista che corrisponde ai parametri desiderati
\end{itemize}
\subsection{Quizzipedia::Server::ControllerServer::StatisticsManager}
Questo package ha il compito di recuperare tutte le informazioni sottoforma di statistiche relative ad un quiz, una domanda in particolare o ad un utente.
\begin{figure}[H]
\centering
\noindent\makebox[\textwidth]{\includegraphics[width=\textwidth]{Img/quizzipedia-server-controllerserver-statisticsmanager.pdf}}
\caption[Schema Componente Quizzipedia::Server::ControllerServer::StatisticsManager]{Schema Componente Quizzipedia::Server::ControllerServer::StatisticsManager}
\end{figure}
\subsubsection{Classe PersonalStatisticsFetcher}
Ritorna le statistiche personali.
\begin{figure}[H]
\centering
\noindent\makebox[\textwidth]{\includegraphics[width=\textwidth]{Img/quizzipedia-server-controllerserver-statisticsmanager-personalstatisticsfetcher.pdf}}
\caption[Schema Classe PersonalStatisticsFetcher]{Schema Classe Quizzipedia::Server::ControllerServer::StatisticsManager::PersonalStatisticsFetcher}
\end{figure}
\paragraph{Relazioni con altre classi}
\subparagraph{Entranti}
\begin{itemize}
\item usata da Quizzipedia::Server::RoutingManager::StatisticsRouter per Questo package ha il compito di recuperare tutte le informazioni sottoforma di statistiche relative ad un quiz, una domanda in particolare o ad un utente
\end{itemize}
\subparagraph{Uscenti}
\begin{itemize}
\item usa Quizzipedia::Server::ControllerServer::StatisticsManager::SessionController per Questo package ha il compito di recuperare tutte le informazioni sottoforma di statistiche relative ad un quiz, una domanda in particolare o ad un utente
\end{itemize}
\subsubsection{Classe QuestionStatisticsFetcher}
Ritorna le statistiche riferite ad una domanda.
\begin{figure}[H]
\centering
\noindent\makebox[\textwidth]{\includegraphics[width=\textwidth]{Img/quizzipedia-server-controllerserver-statisticsmanager-questionstatisticsfetcher.pdf}}
\caption[Schema Classe QuestionStatisticsFetcher]{Schema Classe Quizzipedia::Server::ControllerServer::StatisticsManager::QuestionStatisticsFetcher}
\end{figure}
\paragraph{Relazioni con altre classi}
\subparagraph{Entranti}
\begin{itemize}
\item usata da Quizzipedia::Server::RoutingManager::StatisticsRouter per Questo package ha il compito di recuperare tutte le informazioni sottoforma di statistiche relative ad un quiz, una domanda in particolare o ad un utente
\end{itemize}
\subparagraph{Uscenti}
\begin{itemize}
\item usa Quizzipedia::Server::ControllerServer::StatisticsManager::SessionController per Questo package ha il compito di recuperare tutte le informazioni sottoforma di statistiche relative ad un quiz, una domanda in particolare o ad un utente
\end{itemize}
\subsubsection{Classe QuizStatisticsFetcher}
Ritorna le statistiche riferite ad un quiz.
\begin{figure}[H]
\centering
\noindent\makebox[\textwidth]{\includegraphics[width=\textwidth]{Img/quizzipedia-server-controllerserver-statisticsmanager-quizstatisticsfetcher.pdf}}
\caption[Schema Classe QuizStatisticsFetcher]{Schema Classe Quizzipedia::Server::ControllerServer::StatisticsManager::QuizStatisticsFetcher}
\end{figure}
\paragraph{Relazioni con altre classi}
\subparagraph{Entranti}
\begin{itemize}
\item usata da Quizzipedia::Server::RoutingManager::StatisticsRouter per Questo package ha il compito di recuperare tutte le informazioni sottoforma di statistiche relative ad un quiz, una domanda in particolare o ad un utente
\end{itemize}
\subparagraph{Uscenti}
\begin{itemize}
\item usa Quizzipedia::Server::ControllerServer::StatisticsManager::SessionController per Questo package ha il compito di recuperare tutte le informazioni sottoforma di statistiche relative ad un quiz, una domanda in particolare o ad un utente
\end{itemize}
\subsubsection{Classe SessionController}
Effettua il controllo sull'utente per verificare che egli sia in possesso dell'autorizzazione necessaria per compiere determinate richieste alla base di dati.
\begin{figure}[H]
\centering
\noindent\makebox[\textwidth]{\includegraphics[width=\textwidth]{Img/quizzipedia-server-controllerserver-statisticsmanager-sessioncontroller.pdf}}
\caption[Schema Classe SessionController]{Schema Classe Quizzipedia::Server::ControllerServer::StatisticsManager::SessionController}
\end{figure}
\paragraph{Relazioni con altre classi}
\subparagraph{Entranti}
\begin{itemize}
\item usata da Quizzipedia::Server::ControllerServer::StatisticsManager::PersonalStatisticsFetcher per Questo package ha il compito di recuperare tutte le informazioni sottoforma di statistiche relative ad un quiz, una domanda in particolare o ad un utente
\item usata da Quizzipedia::Server::ControllerServer::StatisticsManager::QuestionStatisticsFetcher per Questo package ha il compito di recuperare tutte le informazioni sottoforma di statistiche relative ad un quiz, una domanda in particolare o ad un utente
\item usata da Quizzipedia::Server::ControllerServer::StatisticsManager::QuizStatisticsFetcher per Questo package ha il compito di recuperare tutte le informazioni sottoforma di statistiche relative ad un quiz, una domanda in particolare o ad un utente
\item usata da Quizzipedia::Server::ControllerServer::StatisticsManager::StudentStatisticsFetcher per Questo package ha il compito di recuperare tutte le informazioni sottoforma di statistiche relative ad un quiz, una domanda in particolare o ad un utente
\item usata da Quizzipedia::Server::ControllerServer::StatisticsManager::TeacherStatisticsFetcher per Questo package ha il compito di recuperare tutte le informazioni sottoforma di statistiche relative ad un quiz, una domanda in particolare o ad un utente
\end{itemize}
\subsubsection{Classe StudentStatisticsFetcher}
Ritorna le statistiche riferite ad uno studente.
\begin{figure}[H]
\centering
\noindent\makebox[\textwidth]{\includegraphics[width=\textwidth]{Img/quizzipedia-server-controllerserver-statisticsmanager-studentstatisticsfetcher.pdf}}
\caption[Schema Classe StudentStatisticsFetcher]{Schema Classe Quizzipedia::Server::ControllerServer::StatisticsManager::StudentStatisticsFetcher}
\end{figure}
\paragraph{Relazioni con altre classi}
\subparagraph{Entranti}
\begin{itemize}
\item usata da Quizzipedia::Server::RoutingManager::StatisticsRouter per Questo package ha il compito di recuperare tutte le informazioni sottoforma di statistiche relative ad un quiz, una domanda in particolare o ad un utente
\end{itemize}
\subparagraph{Uscenti}
\begin{itemize}
\item usa Quizzipedia::Server::ControllerServer::StatisticsManager::SessionController per Questo package ha il compito di recuperare tutte le informazioni sottoforma di statistiche relative ad un quiz, una domanda in particolare o ad un utente
\end{itemize}
\subsubsection{Classe TeacherStatisticsFetcher}
Ritorna le statistiche riferite ad un docente.
\begin{figure}[H]
\centering
\noindent\makebox[\textwidth]{\includegraphics[width=\textwidth]{Img/quizzipedia-server-controllerserver-statisticsmanager-teacherstatisticsfetcher.pdf}}
\caption[Schema Classe TeacherStatisticsFetcher]{Schema Classe Quizzipedia::Server::ControllerServer::StatisticsManager::TeacherStatisticsFetcher}
\end{figure}
\paragraph{Relazioni con altre classi}
\subparagraph{Entranti}
\begin{itemize}
\item usata da Quizzipedia::Server::RoutingManager::StatisticsRouter per Questo package ha il compito di recuperare tutte le informazioni sottoforma di statistiche relative ad un quiz, una domanda in particolare o ad un utente
\end{itemize}
\subparagraph{Uscenti}
\begin{itemize}
\item usa Quizzipedia::Server::ControllerServer::StatisticsManager::SessionController per Questo package ha il compito di recuperare tutte le informazioni sottoforma di statistiche relative ad un quiz, una domanda in particolare o ad un utente
\end{itemize}
\subsection{Quizzipedia::Server::ControllerServer::TopicManager}
Package che permette la creazione di un nuovo argomento o l'eliminazione di uno già esistente.
\begin{figure}[H]
\centering
\noindent\makebox[\textwidth]{\includegraphics[width=\textwidth]{Img/quizzipedia-server-controllerserver-topicmanager.pdf}}
\caption[Schema Componente Quizzipedia::Server::ControllerServer::TopicManager]{Schema Componente Quizzipedia::Server::ControllerServer::TopicManager}
\end{figure}
\subsubsection{Classe SessionController}
Effettua il controllo sull'utente per verificare che egli sia in possesso dell'autorizzazione necessaria per compiere determinate richieste alla base di dati.
\begin{figure}[H]
\centering
\noindent\makebox[\textwidth]{\includegraphics[width=\textwidth]{Img/quizzipedia-server-controllerserver-topicmanager-sessioncontroller.pdf}}
\caption[Schema Classe SessionController]{Schema Classe Quizzipedia::Server::ControllerServer::TopicManager::SessionController}
\end{figure}
\paragraph{Relazioni con altre classi}
\subparagraph{Entranti}
\begin{itemize}
\item usata da Quizzipedia::Server::ControllerServer::TopicManager::TopicCreator per Package che permette la creazione di un nuovo argomento o l'eliminazione di uno già esistente
\item usata da Quizzipedia::Server::ControllerServer::TopicManager::TopicEraser per Package che permette la creazione di un nuovo argomento o l'eliminazione di uno già esistente
\end{itemize}
\subsubsection{Classe TopicCreator}
Permette la creazione di un nuovo argomento.
\begin{figure}[H]
\centering
\noindent\makebox[\textwidth]{\includegraphics[width=\textwidth]{Img/quizzipedia-server-controllerserver-topicmanager-topiccreator.pdf}}
\caption[Schema Classe TopicCreator]{Schema Classe Quizzipedia::Server::ControllerServer::TopicManager::TopicCreator}
\end{figure}
\paragraph{Relazioni con altre classi}
\subparagraph{Entranti}
\begin{itemize}
\item usata da Quizzipedia::Server::RoutingManager::TopicRouter per Package che permette la creazione di un nuovo argomento o l'eliminazione di uno già esistente
\end{itemize}
\subparagraph{Uscenti}
\begin{itemize}
\item usa Quizzipedia::Server::ControllerServer::TopicManager::SessionController per Package che permette la creazione di un nuovo argomento o l'eliminazione di uno già esistente
\end{itemize}
\subsubsection{Classe TopicEraser}
Permette l'eliminazione di un argomento.
\begin{figure}[H]
\centering
\noindent\makebox[\textwidth]{\includegraphics[width=\textwidth]{Img/quizzipedia-server-controllerserver-topicmanager-topiceraser.pdf}}
\caption[Schema Classe TopicEraser]{Schema Classe Quizzipedia::Server::ControllerServer::TopicManager::TopicEraser}
\end{figure}
\paragraph{Relazioni con altre classi}
\subparagraph{Entranti}
\begin{itemize}
\item usata da Quizzipedia::Server::RoutingManager::TopicRouter per Package che permette la creazione di un nuovo argomento o l'eliminazione di uno già esistente
\end{itemize}
\subparagraph{Uscenti}
\begin{itemize}
\item usa Quizzipedia::Server::ControllerServer::TopicManager::SessionController per Package che permette la creazione di un nuovo argomento o l'eliminazione di uno già esistente
\end{itemize}
\subsection{Quizzipedia::Server::RoutingManager}
Questo pacchetto costituisce lo strato superiore a ControllerServer e contiene tutte le API necessarie per comunicare con il client tramite socket. Esso ha il compito di indirizzare le richieste ai vari services in base alla richiesta da parte dell'utente.
\begin{figure}[H]
\centering
\noindent\makebox[\textwidth]{\includegraphics[width=\textwidth]{Img/quizzipedia-server-routingmanager.pdf}}
\caption[Schema Componente Quizzipedia::Server::RoutingManager]{Schema Componente Quizzipedia::Server::RoutingManager}
\end{figure}
\subsubsection{Interazioni con altre componenti}
\paragraph{Uscenti}
\begin{itemize}
\item usa Quizzipedia::Server::ControllerServer per Questo pacchetto costituisce lo strato superiore a ControllerServer e contiene tutte le API necessarie per comunicare con il client tramite socket. Esso ha il compito di indirizzare le richieste ai vari services in base alla richiesta da parte dell'utente
\end{itemize}
\subsubsection{Classe AbstractRouter}
Classe astratta supertipo delle altre classi del pacchetto.
\begin{figure}[H]
\centering
\noindent\makebox[\textwidth]{\includegraphics[width=\textwidth]{Img/quizzipedia-server-routingmanager-abstractrouter.pdf}}
\caption[Schema Classe AbstractRouter]{Schema Classe Quizzipedia::Server::RoutingManager::AbstractRouter}
\end{figure}
\subsubsection{Classe AuthenticationRouter}
Invocato dal client per interagire con AuthenticationManager.
\begin{figure}[H]
\centering
\noindent\makebox[\textwidth]{\includegraphics[width=\textwidth]{Img/quizzipedia-server-routingmanager-authenticationrouter.pdf}}
\caption[Schema Classe AuthenticationRouter]{Schema Classe Quizzipedia::Server::RoutingManager::AuthenticationRouter}
\end{figure}
\paragraph{Relazioni con altre classi}
\subparagraph{Entranti}
\begin{itemize}
\item usata da Quizzipedia::Client::ControllerClient::CtrlUsers::CtrlData per Questo pacchetto costituisce lo strato superiore a ControllerServer e contiene tutte le API necessarie per comunicare con il client tramite socket. Esso ha il compito di indirizzare le richieste ai vari services in base alla richiesta da parte dell'utente
\end{itemize}
\subparagraph{Uscenti}
\begin{itemize}
\item usa Quizzipedia::Server::ControllerServer::AuthenticationManager::LoggerIn per Questo pacchetto costituisce lo strato superiore a ControllerServer e contiene tutte le API necessarie per comunicare con il client tramite socket. Esso ha il compito di indirizzare le richieste ai vari services in base alla richiesta da parte dell'utente
\item usa Quizzipedia::Server::ControllerServer::AuthenticationManager::LoggerOut per Questo pacchetto costituisce lo strato superiore a ControllerServer e contiene tutte le API necessarie per comunicare con il client tramite socket. Esso ha il compito di indirizzare le richieste ai vari services in base alla richiesta da parte dell'utente
\item usa Quizzipedia::Server::ControllerServer::AuthenticationManager::PasswordRecover per Questo pacchetto costituisce lo strato superiore a ControllerServer e contiene tutte le API necessarie per comunicare con il client tramite socket. Esso ha il compito di indirizzare le richieste ai vari services in base alla richiesta da parte dell'utente
\item usa Quizzipedia::Server::ControllerServer::AuthenticationManager::Register per Questo pacchetto costituisce lo strato superiore a ControllerServer e contiene tutte le API necessarie per comunicare con il client tramite socket. Esso ha il compito di indirizzare le richieste ai vari services in base alla richiesta da parte dell'utente
\end{itemize}
\subsubsection{Classe ClassRouter}
Invocato dal client per interagire con ClassManager.
\begin{figure}[H]
\centering
\noindent\makebox[\textwidth]{\includegraphics[width=\textwidth]{Img/quizzipedia-server-routingmanager-classrouter.pdf}}
\caption[Schema Classe ClassRouter]{Schema Classe Quizzipedia::Server::RoutingManager::ClassRouter}
\end{figure}
\paragraph{Relazioni con altre classi}
\subparagraph{Uscenti}
\begin{itemize}
\item usa Quizzipedia::Server::ControllerServer::ClassManager::ClassAdder per Questo pacchetto costituisce lo strato superiore a ControllerServer e contiene tutte le API necessarie per comunicare con il client tramite socket. Esso ha il compito di indirizzare le richieste ai vari services in base alla richiesta da parte dell'utente
\item usa Quizzipedia::Server::ControllerServer::ClassManager::ClassDeleter per Questo pacchetto costituisce lo strato superiore a ControllerServer e contiene tutte le API necessarie per comunicare con il client tramite socket. Esso ha il compito di indirizzare le richieste ai vari services in base alla richiesta da parte dell'utente
\item usa Quizzipedia::Server::ControllerServer::ClassManager::ClassUpdater per Questo pacchetto costituisce lo strato superiore a ControllerServer e contiene tutte le API necessarie per comunicare con il client tramite socket. Esso ha il compito di indirizzare le richieste ai vari services in base alla richiesta da parte dell'utente
\item usa Quizzipedia::Server::ControllerServer::ClassManager::FromClassRemover per Questo pacchetto costituisce lo strato superiore a ControllerServer e contiene tutte le API necessarie per comunicare con il client tramite socket. Esso ha il compito di indirizzare le richieste ai vari services in base alla richiesta da parte dell'utente
\item usa Quizzipedia::Server::ControllerServer::ClassManager::InClassAdder per Questo pacchetto costituisce lo strato superiore a ControllerServer e contiene tutte le API necessarie per comunicare con il client tramite socket. Esso ha il compito di indirizzare le richieste ai vari services in base alla richiesta da parte dell'utente
\item usa Quizzipedia::Server::ControllerServer::ClassManager::StudentsClassFetcher per Questo pacchetto costituisce lo strato superiore a ControllerServer e contiene tutte le API necessarie per comunicare con il client tramite socket. Esso ha il compito di indirizzare le richieste ai vari services in base alla richiesta da parte dell'utente
\item usa Quizzipedia::Server::ControllerServer::ClassManager::TeachersClassFetcher per Questo pacchetto costituisce lo strato superiore a ControllerServer e contiene tutte le API necessarie per comunicare con il client tramite socket. Esso ha il compito di indirizzare le richieste ai vari services in base alla richiesta da parte dell'utente
\end{itemize}
\subsubsection{Classe CompanyRouter}
Invocato dal client per interagire con CompanyManager.
\begin{figure}[H]
\centering
\noindent\makebox[\textwidth]{\includegraphics[width=\textwidth]{Img/quizzipedia-server-routingmanager-companyrouter.pdf}}
\caption[Schema Classe CompanyRouter]{Schema Classe Quizzipedia::Server::RoutingManager::CompanyRouter}
\end{figure}
\paragraph{Relazioni con altre classi}
\subparagraph{Uscenti}
\begin{itemize}
\item usa Quizzipedia::Server::ControllerServer::CompanyManager::CompanyUpdater per Questo pacchetto costituisce lo strato superiore a ControllerServer e contiene tutte le API necessarie per comunicare con il client tramite socket. Esso ha il compito di indirizzare le richieste ai vari services in base alla richiesta da parte dell'utente
\end{itemize}
\subsubsection{Classe ProfileRouter}
Invocato dal client per interagire con ProfileManager.
\begin{figure}[H]
\centering
\noindent\makebox[\textwidth]{\includegraphics[width=\textwidth]{Img/quizzipedia-server-routingmanager-profilerouter.pdf}}
\caption[Schema Classe ProfileRouter]{Schema Classe Quizzipedia::Server::RoutingManager::ProfileRouter}
\end{figure}
\paragraph{Relazioni con altre classi}
\subparagraph{Uscenti}
\begin{itemize}
\item usa Quizzipedia::Server::ControllerServer::ProfileManager::AccountDeleter per Questo pacchetto costituisce lo strato superiore a ControllerServer e contiene tutte le API necessarie per comunicare con il client tramite socket. Esso ha il compito di indirizzare le richieste ai vari services in base alla richiesta da parte dell'utente
\item usa Quizzipedia::Server::ControllerServer::ProfileManager::PasswordSetter per Questo pacchetto costituisce lo strato superiore a ControllerServer e contiene tutte le API necessarie per comunicare con il client tramite socket. Esso ha il compito di indirizzare le richieste ai vari services in base alla richiesta da parte dell'utente
\item usa Quizzipedia::Server::ControllerServer::ProfileManager::PersonalDataFetcher per Questo pacchetto costituisce lo strato superiore a ControllerServer e contiene tutte le API necessarie per comunicare con il client tramite socket. Esso ha il compito di indirizzare le richieste ai vari services in base alla richiesta da parte dell'utente
\item usa Quizzipedia::Server::ControllerServer::ProfileManager::PersonalDataSetter per Questo pacchetto costituisce lo strato superiore a ControllerServer e contiene tutte le API necessarie per comunicare con il client tramite socket. Esso ha il compito di indirizzare le richieste ai vari services in base alla richiesta da parte dell'utente
\item usa Quizzipedia::Server::ControllerServer::ProfileManager::PersonalQuizFetcher per Questo pacchetto costituisce lo strato superiore a ControllerServer e contiene tutte le API necessarie per comunicare con il client tramite socket. Esso ha il compito di indirizzare le richieste ai vari services in base alla richiesta da parte dell'utente
\end{itemize}
\subsubsection{Classe QuestionRouter}
Invocato dal client per interagire con QuestionsManager.
\begin{figure}[H]
\centering
\noindent\makebox[\textwidth]{\includegraphics[width=\textwidth]{Img/quizzipedia-server-routingmanager-questionrouter.pdf}}
\caption[Schema Classe QuestionRouter]{Schema Classe Quizzipedia::Server::RoutingManager::QuestionRouter}
\end{figure}
\paragraph{Relazioni con altre classi}
\subparagraph{Uscenti}
\begin{itemize}
\item usa Quizzipedia::Server::ControllerServer::QuestionsManager::QuestionCreator per Questo pacchetto costituisce lo strato superiore a ControllerServer e contiene tutte le API necessarie per comunicare con il client tramite socket. Esso ha il compito di indirizzare le richieste ai vari services in base alla richiesta da parte dell'utente
\item usa Quizzipedia::Server::ControllerServer::QuestionsManager::QuestionEraser per Questo pacchetto costituisce lo strato superiore a ControllerServer e contiene tutte le API necessarie per comunicare con il client tramite socket. Esso ha il compito di indirizzare le richieste ai vari services in base alla richiesta da parte dell'utente
\item usa Quizzipedia::Server::ControllerServer::QuestionsManager::QuestionUpdater per Questo pacchetto costituisce lo strato superiore a ControllerServer e contiene tutte le API necessarie per comunicare con il client tramite socket. Esso ha il compito di indirizzare le richieste ai vari services in base alla richiesta da parte dell'utente
\end{itemize}
\subsubsection{Classe QuizRouter}
Invocato dal client per interagire con QuizManager.
\begin{figure}[H]
\centering
\noindent\makebox[\textwidth]{\includegraphics[width=\textwidth]{Img/quizzipedia-server-routingmanager-quizrouter.pdf}}
\caption[Schema Classe QuizRouter]{Schema Classe Quizzipedia::Server::RoutingManager::QuizRouter}
\end{figure}
\paragraph{Relazioni con altre classi}
\subparagraph{Uscenti}
\begin{itemize}
\item usa Quizzipedia::Server::ControllerServer::QuizManager::QuizCreator per Questo pacchetto costituisce lo strato superiore a ControllerServer e contiene tutte le API necessarie per comunicare con il client tramite socket. Esso ha il compito di indirizzare le richieste ai vari services in base alla richiesta da parte dell'utente
\item usa Quizzipedia::Server::ControllerServer::QuizManager::QuizEraser per Questo pacchetto costituisce lo strato superiore a ControllerServer e contiene tutte le API necessarie per comunicare con il client tramite socket. Esso ha il compito di indirizzare le richieste ai vari services in base alla richiesta da parte dell'utente
\item usa Quizzipedia::Server::ControllerServer::QuizManager::QuizFetcher per Questo pacchetto costituisce lo strato superiore a ControllerServer e contiene tutte le API necessarie per comunicare con il client tramite socket. Esso ha il compito di indirizzare le richieste ai vari services in base alla richiesta da parte dell'utente
\item usa Quizzipedia::Server::ControllerServer::QuizManager::QuizUpdater per Questo pacchetto costituisce lo strato superiore a ControllerServer e contiene tutte le API necessarie per comunicare con il client tramite socket. Esso ha il compito di indirizzare le richieste ai vari services in base alla richiesta da parte dell'utente
\item usa Quizzipedia::Server::ControllerServer::QuizManager::ResultsUpdater per Questo pacchetto costituisce lo strato superiore a ControllerServer e contiene tutte le API necessarie per comunicare con il client tramite socket. Esso ha il compito di indirizzare le richieste ai vari services in base alla richiesta da parte dell'utente
\item usa Quizzipedia::Server::ControllerServer::QuizManager::StatisticsUpdater per Questo pacchetto costituisce lo strato superiore a ControllerServer e contiene tutte le API necessarie per comunicare con il client tramite socket. Esso ha il compito di indirizzare le richieste ai vari services in base alla richiesta da parte dell'utente
\end{itemize}
\subsubsection{Classe RequestsRouter}
Invocato dal client per interagire con RequestsManager.
\begin{figure}[H]
\centering
\noindent\makebox[\textwidth]{\includegraphics[width=\textwidth]{Img/quizzipedia-server-routingmanager-requestsrouter.pdf}}
\caption[Schema Classe RequestsRouter]{Schema Classe Quizzipedia::Server::RoutingManager::RequestsRouter}
\end{figure}
\paragraph{Relazioni con altre classi}
\subparagraph{Uscenti}
\begin{itemize}
\item usa Quizzipedia::Server::ControllerServer::RequestsManager::ClassRequestsAdder per Questo pacchetto costituisce lo strato superiore a ControllerServer e contiene tutte le API necessarie per comunicare con il client tramite socket. Esso ha il compito di indirizzare le richieste ai vari services in base alla richiesta da parte dell'utente
\item usa Quizzipedia::Server::ControllerServer::RequestsManager::InsertClassRequestsAdder per Questo pacchetto costituisce lo strato superiore a ControllerServer e contiene tutte le API necessarie per comunicare con il client tramite socket. Esso ha il compito di indirizzare le richieste ai vari services in base alla richiesta da parte dell'utente
\item usa Quizzipedia::Server::ControllerServer::RequestsManager::RequestsFetcher per Questo pacchetto costituisce lo strato superiore a ControllerServer e contiene tutte le API necessarie per comunicare con il client tramite socket. Esso ha il compito di indirizzare le richieste ai vari services in base alla richiesta da parte dell'utente
\item usa Quizzipedia::Server::ControllerServer::RequestsManager::RoleAccepter per Questo pacchetto costituisce lo strato superiore a ControllerServer e contiene tutte le API necessarie per comunicare con il client tramite socket. Esso ha il compito di indirizzare le richieste ai vari services in base alla richiesta da parte dell'utente
\item usa Quizzipedia::Server::ControllerServer::RequestsManager::RoleRequestAdder per Questo pacchetto costituisce lo strato superiore a ControllerServer e contiene tutte le API necessarie per comunicare con il client tramite socket. Esso ha il compito di indirizzare le richieste ai vari services in base alla richiesta da parte dell'utente
\end{itemize}
\subsubsection{Classe SearchRouter}
Invocato dal client per interagire con SearchManager.
\begin{figure}[H]
\centering
\noindent\makebox[\textwidth]{\includegraphics[width=\textwidth]{Img/quizzipedia-server-routingmanager-searchrouter.pdf}}
\caption[Schema Classe SearchRouter]{Schema Classe Quizzipedia::Server::RoutingManager::SearchRouter}
\end{figure}
\paragraph{Relazioni con altre classi}
\subparagraph{Entranti}
\begin{itemize}
\item usata da Quizzipedia::Client::ControllerClient::CtrlServices::CtrlSearch per Questo pacchetto costituisce lo strato superiore a ControllerServer e contiene tutte le API necessarie per comunicare con il client tramite socket. Esso ha il compito di indirizzare le richieste ai vari services in base alla richiesta da parte dell'utente
\end{itemize}
\subparagraph{Uscenti}
\begin{itemize}
\item usa Quizzipedia::Server::ControllerServer::SearchManager::QuestionsSearcher per Questo pacchetto costituisce lo strato superiore a ControllerServer e contiene tutte le API necessarie per comunicare con il client tramite socket. Esso ha il compito di indirizzare le richieste ai vari services in base alla richiesta da parte dell'utente
\item usa Quizzipedia::Server::ControllerServer::SearchManager::QuizSearcher per Questo pacchetto costituisce lo strato superiore a ControllerServer e contiene tutte le API necessarie per comunicare con il client tramite socket. Esso ha il compito di indirizzare le richieste ai vari services in base alla richiesta da parte dell'utente
\end{itemize}
\subsubsection{Classe StatisticsRouter}
Invocato dal client per interagire con StatisticsManager.
\begin{figure}[H]
\centering
\noindent\makebox[\textwidth]{\includegraphics[width=\textwidth]{Img/quizzipedia-server-routingmanager-statisticsrouter.pdf}}
\caption[Schema Classe StatisticsRouter]{Schema Classe Quizzipedia::Server::RoutingManager::StatisticsRouter}
\end{figure}
\paragraph{Relazioni con altre classi}
\subparagraph{Uscenti}
\begin{itemize}
\item usa Quizzipedia::Server::ControllerServer::StatisticsManager::PersonalStatisticsFetcher per Questo pacchetto costituisce lo strato superiore a ControllerServer e contiene tutte le API necessarie per comunicare con il client tramite socket. Esso ha il compito di indirizzare le richieste ai vari services in base alla richiesta da parte dell'utente
\item usa Quizzipedia::Server::ControllerServer::StatisticsManager::QuestionStatisticsFetcher per Questo pacchetto costituisce lo strato superiore a ControllerServer e contiene tutte le API necessarie per comunicare con il client tramite socket. Esso ha il compito di indirizzare le richieste ai vari services in base alla richiesta da parte dell'utente
\item usa Quizzipedia::Server::ControllerServer::StatisticsManager::QuizStatisticsFetcher per Questo pacchetto costituisce lo strato superiore a ControllerServer e contiene tutte le API necessarie per comunicare con il client tramite socket. Esso ha il compito di indirizzare le richieste ai vari services in base alla richiesta da parte dell'utente
\item usa Quizzipedia::Server::ControllerServer::StatisticsManager::StudentStatisticsFetcher per Questo pacchetto costituisce lo strato superiore a ControllerServer e contiene tutte le API necessarie per comunicare con il client tramite socket. Esso ha il compito di indirizzare le richieste ai vari services in base alla richiesta da parte dell'utente
\item usa Quizzipedia::Server::ControllerServer::StatisticsManager::TeacherStatisticsFetcher per Questo pacchetto costituisce lo strato superiore a ControllerServer e contiene tutte le API necessarie per comunicare con il client tramite socket. Esso ha il compito di indirizzare le richieste ai vari services in base alla richiesta da parte dell'utente
\end{itemize}
\subsubsection{Classe TopicRouter}
Invocato dal client per interagire con TopicManager.
\begin{figure}[H]
\centering
\noindent\makebox[\textwidth]{\includegraphics[width=\textwidth]{Img/quizzipedia-server-routingmanager-topicrouter.pdf}}
\caption[Schema Classe TopicRouter]{Schema Classe Quizzipedia::Server::RoutingManager::TopicRouter}
\end{figure}
\paragraph{Relazioni con altre classi}
\subparagraph{Uscenti}
\begin{itemize}
\item usa Quizzipedia::Server::ControllerServer::TopicManager::TopicCreator per Questo pacchetto costituisce lo strato superiore a ControllerServer e contiene tutte le API necessarie per comunicare con il client tramite socket. Esso ha il compito di indirizzare le richieste ai vari services in base alla richiesta da parte dell'utente
\item usa Quizzipedia::Server::ControllerServer::TopicManager::TopicEraser per Questo pacchetto costituisce lo strato superiore a ControllerServer e contiene tutte le API necessarie per comunicare con il client tramite socket. Esso ha il compito di indirizzare le richieste ai vari services in base alla richiesta da parte dell'utente
\end{itemize}
