\subsection{\pkg{Quizzipedia:: Client}}
Racchiude tutte le componenti necessarie per il front-end del prodotto. Visualizza i dati dell'utente e invia richieste al server che dovrà gestirle e reindirizzare una risposta al client.
\begin{figure}[H]
\centering
\noindent\makebox[\textwidth]{\includegraphics[width=\textwidth]{Img/quizzipedia-client.pdf}}
\caption[Schema Componente Client]{Schema Componente Quizzipedia:: Client}
\end{figure}
\subsection{\pkg{Quizzipedia:: Client:: ModelClient}}
Rappresenta il modello dei dati che verranno utilizzati dal sistema lato client. Viene utilizzato tale model per facilitare il recupero di alcune informazioni che altrimenti dovrebbero esser recuperate dal server ogni volta che viene svolta una richiesta dall'utente.
Attraverso l'uso di Angular.js il controller svolge automaticamente le modifiche richieste dalla view nel model in modo tale da tenerlo sempre aggiornato.
\begin{figure}[H]
\centering
\noindent\makebox[\textwidth]{\includegraphics[width=\textwidth]{Img/quizzipedia-client-modelclient.pdf}}
\caption[Schema Componente ModelClient]{Schema Componente Quizzipedia:: Client:: ModelClient}
\end{figure}
\subsubsection{Componenti contenute}
\begin{itemize}
\item \pkg{Quizzipedia::Client::ModelClient::Organizations}
\item \pkg{Quizzipedia::Client::ModelClient::Requests}
\item \pkg{Quizzipedia::Client::ModelClient::Services}
\item \pkg{Quizzipedia::Client::ModelClient::Statistics}
\item \pkg{Quizzipedia::Client::ModelClient::Users}
\end{itemize}
\subsubsection{Interazioni con altre componenti}
\begin{itemize}
\item \bold{Entranti}
\begin{itemize}
\item usata da \pkg{Quizzipedia::Client::ViewModelClient} per avere accesso alla struttura degli oggetti che manipola
\end{itemize}
\end{itemize}
\subsection{\pkg{Quizzipedia:: Client:: ModelClient:: Organizations}}
La componente gestisce le classi e gli enti, ovvero il sistema in base a cui sono organizzati gli utenti nel sistema.
\begin{figure}[H]
\centering
\noindent\makebox[\textwidth]{\includegraphics[width=\textwidth]{Img/quizzipedia-client-modelclient-organizations.pdf}}
\caption[Schema Componente Organizations]{Schema Componente Quizzipedia:: Client:: ModelClient:: Organizations}
\end{figure}
\subsubsection{Interazioni con altre componenti}
\begin{itemize}
\item \bold{Uscenti}
\begin{itemize}
\item usa \pkg{Quizzipedia::Client::ModelClient::Requests} per permettere agli utenti di enti e classi di effettuare richieste di ruolo o entrata in una classe
\item usa \pkg{Quizzipedia::Client::ModelClient::Services} per poter avere una lista di argomenti associata ad ogni ente
\item usa \pkg{Quizzipedia::Client::ModelClient::Users} per inserire utenti all'interno di classi e enti
\end{itemize}
\end{itemize}
\subsubsection{Classe \cls{Class}}
Rappresenta una classe all'interno di un ente. Memorizza le informazioni che definiscono ogni classe,
informazioni che saranno utilizzate per la visualizzazione e per la gestione della classe stessa.
\begin{figure}[H]
\centering
\noindent\makebox[\textwidth]{\includegraphics[width=\textwidth]{Img/quizzipedia-client-modelclient-organizations-class.pdf}}
\caption[Schema Classe Class]{Schema Classe Quizzipedia:: Client:: ModelClient:: Organizations:: Class}
\end{figure}
\paragraph{Relazioni con altre classi}
\subparagraph{Entranti}
\begin{itemize}
\item usata da \cls{Quizzipedia::Client::ModelClient::Organizations::Institution} per memorizzare la lista
delle classi presenti in un ente
\item usata da \cls{Quizzipedia::Client::ModelClient::Requests::RequestClass} per identificare la classe
in cui l'utente vuole essere inserito
\item usata da \cls{Quizzipedia::Client::ModelClient::Services::Quiz} per tenere traccia delle classi per
cui il quiz è stato creato. Se un quiz è assegnato a delle classi specifiche, allora è privato e
accessibile dai soli studenti delle classi
\item usata da \cls{Quizzipedia::Client::ViewModelClient::CtrlOrganization::CtrlInstitution} per avere
accesso alla struttura della classe e potere quindi svolgere correttamente operazioni su di
essa
\item usata da \cls{Quizzipedia::Client::ViewModelClient::CtrlServices::CtrlQuiz} per fornire al docente la lista delle classi a cui poter assegnare il quiz
\item usata da \cls{Quizzipedia::Client::ViewModelClient::CtrlServices::CtrlQuizSelection} per ottenere le classi a cui l'utente è iscritto
\item usata da \cls{Quizzipedia::Client::ViewModelClient::CtrlStatistics::CtrlStudentsQuizStats} per fornire all'utente la lista di classi tra cui scegliere
\end{itemize}
\subsubsection{Classe \cls{Institution}}
Tale classe rappresenta un ente. Contiene le informazioni relative alla struttura dell'ente che saranno visualizzate dall'utente e gestiste dal controller, come ad esempio la lista delle classi presenti nell'ente, la lista degli studenti e degli insegnati all'interno dell'ente.
\begin{figure}[H]
\centering
\noindent\makebox[\textwidth]{\includegraphics[width=\textwidth]{Img/quizzipedia-client-modelclient-organizations-institution.pdf}}
\caption[Schema Classe Institution]{Schema Classe Quizzipedia:: Client:: ModelClient:: Organizations:: Institution}
\end{figure}
\paragraph{Relazioni con altre classi}
\subparagraph{Entranti}
\begin{itemize}
\item usata da \cls{Quizzipedia::Client::ViewModelClient::CtrlOrganization::CtrlInstitution} per avere accesso alla struttura dell'istituto e caricare correttamente l'istituto corrente
\end{itemize}
\subparagraph{Uscenti}
\begin{itemize}
\item usa \cls{Quizzipedia::Client::ModelClient::Organizations::Class} per memorizzare la lista
delle classi presenti in un ente
\item usa \cls{Quizzipedia::Client::ModelClient::Requests::ClassList} per gestire la lista delle richieste di accesso a una delle proprie classi
\item usa \cls{Quizzipedia::Client::ModelClient::Requests::RoleList} per gestire la lista delle richieste di assegnazione di ruolo degli utenti
\item usa \cls{Quizzipedia::Client::ModelClient::Services::Topics} per avere un elenco degli argomenti possibili all'interno dell'ente
\item usa \cls{Quizzipedia::Client::ModelClient::Users::Director} per identificare il responsabile dell'ente
\end{itemize}
\subsection{\pkg{Quizzipedia:: Client:: ModelClient:: Requests}}
Questo componente contiene le classi necessarie a gestire le richieste di ruolo e di classe degli utenti autenticati.
\begin{figure}[H]
\centering
\noindent\makebox[\textwidth]{\includegraphics[width=\textwidth]{Img/quizzipedia-client-modelclient-requests.pdf}}
\caption[Schema Componente Requests]{Schema Componente Quizzipedia:: Client:: ModelClient:: Requests}
\end{figure}
\subsubsection{Interazioni con altre componenti}
\begin{itemize}
\item \bold{Entranti}
\begin{itemize}
\item usata da \pkg{Quizzipedia::Client::ModelClient::Organizations} per permettere agli utenti di enti e classi di effettuare richieste di ruolo o entrata in una classe
\end{itemize}
\end{itemize}
\subsubsection{Classe \cls{ClassList}}
Questa classe gestisce le richieste da parte di docenti o studenti per l'assegnazione a una specifica classe. Contiene i metodi da cui è possibile accettare o rifiutare la richiesta dell'utente.
\begin{figure}[H]
\centering
\noindent\makebox[\textwidth]{\includegraphics[width=\textwidth]{Img/quizzipedia-client-modelclient-requests-classlist.pdf}}
\caption[Schema Classe ClassList]{Schema Classe Quizzipedia:: Client:: ModelClient:: Requests:: ClassList}
\end{figure}
\paragraph{Relazioni con altre classi}
\subparagraph{Entranti}
\begin{itemize}
\item usata da \cls{Quizzipedia::Client::ModelClient::Organizations::Institution} per gestire la lista delle richieste di accesso a una delle proprie classi
\item usata da \cls{Quizzipedia::Client::ViewModelClient::CtrlRequests::CtrlRequestClass} per avere accesso alla lista di richieste e poterle gestire correttamente
\end{itemize}
\subparagraph{Uscenti}
\begin{itemize}
\item usa \cls{Quizzipedia::Client::ModelClient::Requests::RequestClass} per creare una lista in cui compaiano gli utenti e le classi in cui desiderano entrare
\end{itemize}
\subsubsection{Classe \cls{RequestClass}}
La classe memorizza l'utente che invia la richiesta di inserimento in una classe e la classe per cui la richiesta è stata effettuata.
\begin{figure}[H]
\centering
\noindent\makebox[\textwidth]{\includegraphics[width=\textwidth]{Img/quizzipedia-client-modelclient-requests-requestclass.pdf}}
\caption[Schema Classe RequestClass]{Schema Classe Quizzipedia:: Client:: ModelClient:: Requests:: RequestClass}
\end{figure}
\paragraph{Relazioni con altre classi}
\subparagraph{Entranti}
\begin{itemize}
\item usata da \cls{Quizzipedia::Client::ModelClient::Requests::ClassList} per creare una lista in cui compaiano gli utenti e le classi in cui desiderano entrare
\end{itemize}
\subparagraph{Uscenti}
\begin{itemize}
\item usa \cls{Quizzipedia::Client::ModelClient::Organizations::Class} per identificare la classe
in cui l'utente vuole essere inserito
\end{itemize}
\subsubsection{Classe \cls{RequestRole}}
La classe memorizza l'utente che invia una richiesta di ruolo e il ruolo che vuole ricoprire.
\begin{figure}[H]
\centering
\noindent\makebox[\textwidth]{\includegraphics[width=\textwidth]{Img/quizzipedia-client-modelclient-requests-requestrole.pdf}}
\caption[Schema Classe RequestRole]{Schema Classe Quizzipedia:: Client:: ModelClient:: Requests:: RequestRole}
\end{figure}
\paragraph{Relazioni con altre classi}
\subparagraph{Entranti}
\begin{itemize}
\item usata da \cls{Quizzipedia::Client::ModelClient::Requests::RoleList} per memorizzare una lista delle richieste di ruolo degli utenti di un istituto
\end{itemize}
\subsubsection{Classe \cls{RoleList}}
Gli utenti senza ruolo inviano le proprie richieste per l'assegnazione al ruolo di studente o docente al responsabile di un ente. Questa classe gestisce tali richieste.
\begin{figure}[H]
\centering
\noindent\makebox[\textwidth]{\includegraphics[width=\textwidth]{Img/quizzipedia-client-modelclient-requests-rolelist.pdf}}
\caption[Schema Classe RoleList]{Schema Classe Quizzipedia:: Client:: ModelClient:: Requests:: RoleList}
\end{figure}
\paragraph{Relazioni con altre classi}
\subparagraph{Entranti}
\begin{itemize}
\item usata da \cls{Quizzipedia::Client::ModelClient::Organizations::Institution} per gestire la lista delle richieste di assegnazione di ruolo degli utenti
\item usata da \cls{Quizzipedia::Client::ViewModelClient::CtrlRequests::CtrlRequestRole} per gestire le richieste di ruolo pendenti
\end{itemize}
\subparagraph{Uscenti}
\begin{itemize}
\item usa \cls{Quizzipedia::Client::ModelClient::Requests::RequestRole} per memorizzare una lista delle richieste di ruolo degli utenti di un istituto
\end{itemize}
\subsection{\pkg{Quizzipedia:: Client:: ModelClient:: Services}}
Il componente racchiude i modelli necessari alla creazione di domande e quiz, i servizi principali offerti dal nostro prodotto.
\begin{figure}[H]
\centering
\noindent\makebox[\textwidth]{\includegraphics[width=\textwidth]{Img/quizzipedia-client-modelclient-services.pdf}}
\caption[Schema Componente Services]{Schema Componente Quizzipedia:: Client:: ModelClient:: Services}
\end{figure}
\subsubsection{Componenti contenute}
\begin{itemize}
\item \pkg{Quizzipedia::Client::ModelClient::Services::Answers}
\item \pkg{Quizzipedia::Client::ModelClient::Services::Questions}
\end{itemize}
\subsubsection{Interazioni con altre componenti}
\begin{itemize}
\item \bold{Entranti}
\begin{itemize}
\item usata da \pkg{Quizzipedia::Client::ModelClient::Organizations} per poter avere una lista di argomenti associata ad ogni ente
\item usata da \pkg{Quizzipedia::Client::ModelClient::Users} per permettere agli utenti di tenere traccia dello storico dei quiz svolti
\end{itemize}
\item \bold{Uscenti}
\begin{itemize}
\item usa \pkg{Quizzipedia::Client::ModelClient::Statistics} per poter memorizzare le statistiche di quiz e risposte
\end{itemize}
\end{itemize}
\subsubsection{Classe \cls{Quiz}}
Include la struttura del quiz.
\begin{figure}[H]
\centering
\noindent\makebox[\textwidth]{\includegraphics[width=\textwidth]{Img/quizzipedia-client-modelclient-services-quiz.pdf}}
\caption[Schema Classe Quiz]{Schema Classe Quizzipedia:: Client:: ModelClient:: Services:: Quiz}
\end{figure}
\paragraph{Relazioni con altre classi}
\subparagraph{Entranti}
\begin{itemize}
\item usata da \cls{Quizzipedia::Client::ModelClient::Services::Answers::AnswerQuiz} per individuare il quiz a cui associare le risposte date dall'utente
\item usata da \cls{Quizzipedia::Client::ModelClient::Users::Teacher} per tenere traccia dei quiz da lui creati
\item usata da \cls{Quizzipedia::Client::ViewModelClient::CtrlServices::CtrlQuiz} per avere un riferimento alla struttura dei quiz durante la creazione
\item usata da \cls{Quizzipedia::Client::ViewModelClient::CtrlServices::CtrlQuizManager} per dare al docente lista di quiz di cui è proprietario con possibilità di modifica e rimozione
\item usata da \cls{Quizzipedia::Client::ViewModelClient::CtrlServices::CtrlQuizSelection} per avere un riferimento al quiz che l'utente intende risolvere
\item usata da \cls{Quizzipedia::Client::ViewModelClient::CtrlServices::CtrlSearchQuiz} per poter restituire all'utente un quiz che rispetti i criteri specificati
\item usata da \cls{Quizzipedia::Client::ViewModelClient::CtrlStatistics::CtrlStudentsQuizStats} per selezionare il quiz di cui si vogliono avere le statistiche
\end{itemize}
\subparagraph{Uscenti}
\begin{itemize}
\item usa \cls{Quizzipedia::Client::ModelClient::Organizations::Class} per tenere traccia delle classi per
cui il quiz è stato creato. Se un quiz è assegnato a delle classi specifiche, allora è privato e
accessibile dai soli studenti delle classi
\item usa \cls{Quizzipedia::Client::ModelClient::Services::Questions::GenericQuestion} per memorizzare la lista delle domande che compongono un quiz
\item usa \cls{Quizzipedia::Client::ModelClient::Statistics::QuizStatistics} per memorizzare le statistiche generali riguardo al quiz
\item usa \cls{Quizzipedia::Client::ModelClient::Statistics::StudentsStatisticsQuiz} per memorizzare le statistiche relative agli studenti che hanno svolto il quiz
\end{itemize}
\subsubsection{Classe \cls{Topics}}
Modella la struttura necessaria a memorizzare la lista di argomenti. A ogni domanda e a ogni quiz verranno poi associati i relativi argomenti .
\begin{figure}[H]
\centering
\noindent\makebox[\textwidth]{\includegraphics[width=\textwidth]{Img/quizzipedia-client-modelclient-services-topics.pdf}}
\caption[Schema Classe Topics]{Schema Classe Quizzipedia:: Client:: ModelClient:: Services:: Topics}
\end{figure}
\paragraph{Relazioni con altre classi}
\subparagraph{Entranti}
\begin{itemize}
\item usata da \cls{Quizzipedia::Client::ModelClient::Organizations::Institution} per avere un elenco degli argomenti possibili all'interno dell'ente
\item usata da \cls{Quizzipedia::Client::ViewModelClient::CtrlServices::CtrlQuestions::CtrlQuestion} per fornire all'utente la lista di argomenti tra cui scegliere quello per la propria domanda
\item usata da \cls{Quizzipedia::Client::ViewModelClient::CtrlServices::CtrlQuiz} per fornire al docente una lista di possibili argomenti tra cui scegliere
\item usata da \cls{Quizzipedia::Client::ViewModelClient::CtrlServices::CtrlSearchQuestion} per fornire all'utente la lista di argomenti tra cui scegliere quello desiderato
\item usata da \cls{Quizzipedia::Client::ViewModelClient::CtrlServices::CtrlSearchQuiz} per fornire all'utente la lista degli argomenti disponibili tra cui scegliere
\item usata da \cls{Quizzipedia::Client::ViewModelClient::CtrlServices::CtrlTopics} per gestire l'inserimento e la rimozione di argomenti
\end{itemize}
\subsection{\pkg{Quizzipedia:: Client:: ModelClient:: Services:: Answers}}
Raccoglie tutte le componenti necessarie alla gestione della verifica dei quiz e delle domande svolte.
\begin{figure}[H]
\centering
\noindent\makebox[\textwidth]{\includegraphics[width=\textwidth]{Img/quizzipedia-client-modelclient-services-answers.pdf}}
\caption[Schema Componente Answers]{Schema Componente Quizzipedia:: Client:: ModelClient:: Services:: Answers}
\end{figure}
\subsubsection{Interazioni con altre componenti}
\begin{itemize}
\item \bold{Entranti}
\begin{itemize}
\item usata da \pkg{Quizzipedia::Client::ViewModelClient::CtrlServices} per avere accesso alla struttura e poter ricavare le risposte degli utenti a quiz e domande
\end{itemize}
\item \bold{Uscenti}
\begin{itemize}
\item usa \pkg{Quizzipedia::Client::ModelClient::Services::Questions} per associare alle risposte le domande e i quiz a cui si riferiscono
\end{itemize}
\end{itemize}
\subsubsection{Classe \cls{AnswerColumn}}
Classe utilizzata da AnswerMatchingQ per salvare le risposte della domanda MatchingQ.
\begin{figure}[H]
\centering
\noindent\makebox[\textwidth]{\includegraphics[width=\textwidth]{Img/quizzipedia-client-modelclient-services-answers-answercolumn.pdf}}
\caption[Schema Classe AnswerColumn]{Schema Classe Quizzipedia:: Client:: ModelClient:: Services:: Answers:: AnswerColumn}
\end{figure}
\paragraph{Relazioni con altre classi}
\subparagraph{Entranti}
\begin{itemize}
\item usata da \cls{Quizzipedia::Client::ModelClient::Services::Answers::AnswerMatchingQ} per memorizzare le risposte date dall'utente a una domanda a collegamenti
\end{itemize}
\subsubsection{Classe \cls{AnswerCompletionQ}}
Classe che si occupa di memorizzare le informazioni di un domanda CompletionQ risolta.
\begin{figure}[H]
\centering
\noindent\makebox[\textwidth]{\includegraphics[width=\textwidth]{Img/quizzipedia-client-modelclient-services-answers-answercompletionq.pdf}}
\caption[Schema Classe AnswerCompletionQ]{Schema Classe Quizzipedia:: Client:: ModelClient:: Services:: Answers:: AnswerCompletionQ}
\end{figure}
\paragraph{Relazioni con altre classi}
\subparagraph{Entranti}
\begin{itemize}
\item usata da \cls{Quizzipedia::Client::ModelClient::Services::Questions::CompletionQ} per poter impostare la risposta data dall'utente
\end{itemize}
\subparagraph{Uscenti}
\begin{itemize}
\item usa \cls{Quizzipedia::Client::ModelClient::Services::Answers::AnswerQuestion} per concretizzarla, ereditandone così i campi dati e i metodi concreti. Inoltre, la classe concretizza il metodo check e memorizza correttamente la risposta data in caso di risposta a domanda a completamento
\end{itemize}
\subsubsection{Classe \cls{AnswerMatchingQ}}
Classe che si occupa di memorizzare le informazioni di un domanda MatchingQ risolta.
\begin{figure}[H]
\centering
\noindent\makebox[\textwidth]{\includegraphics[width=\textwidth]{Img/quizzipedia-client-modelclient-services-answers-answermatchingq.pdf}}
\caption[Schema Classe AnswerMatchingQ]{Schema Classe Quizzipedia:: Client:: ModelClient:: Services:: Answers:: AnswerMatchingQ}
\end{figure}
\paragraph{Relazioni con altre classi}
\subparagraph{Entranti}
\begin{itemize}
\item usata da \cls{Quizzipedia::Client::ModelClient::Services::Questions::MatchingQ} per poter impostare la risposta data dall'utente
\end{itemize}
\subparagraph{Uscenti}
\begin{itemize}
\item usa \cls{Quizzipedia::Client::ModelClient::Services::Answers::AnswerColumn} per memorizzare le risposte date dall'utente a una domanda a collegamenti
\item usa \cls{Quizzipedia::Client::ModelClient::Services::Answers::AnswerQuestion} per concretizzarla, ereditandone così i campi dati e i metodi concreti. Inoltre, la classe concretizza il metodo check e memorizza correttamente la risposta data in caso di risposta a domanda a collegamento
\end{itemize}
\subsubsection{Classe \cls{AnswerMultipleChoiceQ}}
Classe che si occupa di memorizzare le informazioni di un domanda MultipleChoiceQ risolta.
\begin{figure}[H]
\centering
\noindent\makebox[\textwidth]{\includegraphics[width=\textwidth]{Img/quizzipedia-client-modelclient-services-answers-answermultiplechoiceq.pdf}}
\caption[Schema Classe AnswerMultipleChoiceQ]{Schema Classe Quizzipedia:: Client:: ModelClient:: Services:: Answers:: AnswerMultipleChoiceQ}
\end{figure}
\paragraph{Relazioni con altre classi}
\subparagraph{Entranti}
\begin{itemize}
\item usata da \cls{Quizzipedia::Client::ModelClient::Services::Questions::MultipleChoiceQ} per poter impostare la risposta data dall'utente
\end{itemize}
\subparagraph{Uscenti}
\begin{itemize}
\item usa \cls{Quizzipedia::Client::ModelClient::Services::Answers::AnswerQuestion} per concretizzarla, ereditandone così i campi dati e i metodi concreti. Inoltre, la classe concretizza il metodo check e memorizza correttamente la risposta data in caso di risposta a domanda a scelta multipla
\end{itemize}
\subsubsection{Classe \cls{AnswerQuestion}}
Classe che si occupa di memorizzare le informazioni di un domanda risolta.
\begin{figure}[H]
\centering
\noindent\makebox[\textwidth]{\includegraphics[width=\textwidth]{Img/quizzipedia-client-modelclient-services-answers-answerquestion.pdf}}
\caption[Schema Classe AnswerQuestion]{Schema Classe Quizzipedia:: Client:: ModelClient:: Services:: Answers:: AnswerQuestion}
\end{figure}
\paragraph{Relazioni con altre classi}
\subparagraph{Entranti}
\begin{itemize}
\item usata da \cls{Quizzipedia::Client::ModelClient::Services::Answers::AnswerCompletionQ} per concretizzarla, ereditandone così i campi dati e i metodi concreti. Inoltre, la classe concretizza il metodo check e memorizza correttamente la risposta data in caso di risposta a domanda a completamento
\item usata da \cls{Quizzipedia::Client::ModelClient::Services::Answers::AnswerMatchingQ} per concretizzarla, ereditandone così i campi dati e i metodi concreti. Inoltre, la classe concretizza il metodo check e memorizza correttamente la risposta data in caso di risposta a domanda a collegamento
\item usata da \cls{Quizzipedia::Client::ModelClient::Services::Answers::AnswerMultipleChoiceQ} per concretizzarla, ereditandone così i campi dati e i metodi concreti. Inoltre, la classe concretizza il metodo check e memorizza correttamente la risposta data in caso di risposta a domanda a scelta multipla
\item usata da \cls{Quizzipedia::Client::ModelClient::Services::Answers::AnswerQuiz} per memorizzare e verificare le risposte date dagli utenti alle domande che compongono il quiz
\item usata da \cls{Quizzipedia::Client::ModelClient::Services::Answers::AnswerShortAnswerQ} per concretizzarla, ereditandone così i campi dati e i metodi concreti. Inoltre, la classe concretizza il metodo check e memorizza correttamente la risposta data in caso di risposta a domanda aperta
\item usata da \cls{Quizzipedia::Client::ModelClient::Services::Answers::AnswerTrueFalseQ} per concretizzarla, ereditandone così i campi dati e i metodi concreti. Inoltre, la classe concretizza il metodo check e memorizza correttamente la risposta data in caso di risposta a domanda di tipo vero/falso
\item usata da \cls{Quizzipedia::Client::ViewModelClient::CtrlServices::CtrlQuizExec} per memorizzare le risposte date dagli utenti alle singole domande
\end{itemize}
\subparagraph{Uscenti}
\begin{itemize}
\item usa \cls{Quizzipedia::Client::ModelClient::Services::Questions::GenericQuestion} per tenere traccia della domanda a cui corrisponde la risposta in questione
\item usa \cls{Quizzipedia::Client::ModelClient::Users::User} per tenere traccia dell'utente che sta svolgendo la domanda
\end{itemize}
\subsubsection{Classe \cls{AnswerQuiz}}
Classe che si occupa di memorizzare le informazioni di un quiz completato e gestisce la verifica del suo superamento.
\begin{figure}[H]
\centering
\noindent\makebox[\textwidth]{\includegraphics[width=\textwidth]{Img/quizzipedia-client-modelclient-services-answers-answerquiz.pdf}}
\caption[Schema Classe AnswerQuiz]{Schema Classe Quizzipedia:: Client:: ModelClient:: Services:: Answers:: AnswerQuiz}
\end{figure}
\paragraph{Relazioni con altre classi}
\subparagraph{Entranti}
\begin{itemize}
\item usata da \cls{Quizzipedia::Client::ModelClient::Users::NoRole} per tenere traccia dei quiz che ha già svolto e del loro esito
\item usata da \cls{Quizzipedia::Client::ModelClient::Users::Student} per tenere traccia dei quiz che ha già svolto e del loro esito
\item usata da \cls{Quizzipedia::Client::ViewModelClient::CtrlServices::CtrlQuizExec} per memorizzare la soluzione del quiz fornita dall'utente
\end{itemize}
\subparagraph{Uscenti}
\begin{itemize}
\item usa \cls{Quizzipedia::Client::ModelClient::Services::Answers::AnswerQuestion} per memorizzare e verificare le risposte date dagli utenti alle domande che compongono il quiz
\item usa \cls{Quizzipedia::Client::ModelClient::Services::Quiz} per individuare il quiz a cui associare le risposte date dall'utente
\end{itemize}
\subsubsection{Classe \cls{AnswerShortAnswerQ}}
Classe che si occupa di memorizzare le informazioni di un domanda ShortAnswerQ risolta.
\begin{figure}[H]
\centering
\noindent\makebox[\textwidth]{\includegraphics[width=\textwidth]{Img/quizzipedia-client-modelclient-services-answers-answershortanswerq.pdf}}
\caption[Schema Classe AnswerShortAnswerQ]{Schema Classe Quizzipedia:: Client:: ModelClient:: Services:: Answers:: AnswerShortAnswerQ}
\end{figure}
\paragraph{Relazioni con altre classi}
\subparagraph{Entranti}
\begin{itemize}
\item usata da \cls{Quizzipedia::Client::ModelClient::Services::Questions::ShortAnswerQ} per poter impostare la risposta data dall'utente
\end{itemize}
\subparagraph{Uscenti}
\begin{itemize}
\item usa \cls{Quizzipedia::Client::ModelClient::Services::Answers::AnswerQuestion} per concretizzarla, ereditandone così i campi dati e i metodi concreti. Inoltre, la classe concretizza il metodo check e memorizza correttamente la risposta data in caso di risposta a domanda aperta
\end{itemize}
\subsubsection{Classe \cls{AnswerTrueFalseQ}}
Classe che si occupa di memorizzare le informazioni di un domanda TrueFalseQ risolta.
\begin{figure}[H]
\centering
\noindent\makebox[\textwidth]{\includegraphics[width=\textwidth]{Img/quizzipedia-client-modelclient-services-answers-answertruefalseq.pdf}}
\caption[Schema Classe AnswerTrueFalseQ]{Schema Classe Quizzipedia:: Client:: ModelClient:: Services:: Answers:: AnswerTrueFalseQ}
\end{figure}
\paragraph{Relazioni con altre classi}
\subparagraph{Entranti}
\begin{itemize}
\item usata da \cls{Quizzipedia::Client::ModelClient::Services::Questions::TrueFalseQ} per poter impostare la risposta data dall'utente
\end{itemize}
\subparagraph{Uscenti}
\begin{itemize}
\item usa \cls{Quizzipedia::Client::ModelClient::Services::Answers::AnswerQuestion} per concretizzarla, ereditandone così i campi dati e i metodi concreti. Inoltre, la classe concretizza il metodo check e memorizza correttamente la risposta data in caso di risposta a domanda di tipo vero/falso
\end{itemize}
\subsection{\pkg{Quizzipedia:: Client:: ModelClient:: Services:: Questions}}
Descrive il modo in cui sono strutturati i vari tipi di domande che l'utente può incontrare durante la creazione o la compilazione di quiz.
\begin{figure}[H]
\centering
\noindent\makebox[\textwidth]{\includegraphics[width=\textwidth]{Img/quizzipedia-client-modelclient-services-questions.pdf}}
\caption[Schema Componente Questions]{Schema Componente Quizzipedia:: Client:: ModelClient:: Services:: Questions}
\end{figure}
\subsubsection{Interazioni con altre componenti}
\begin{itemize}
\item \bold{Entranti}
\begin{itemize}
\item usata da \pkg{Quizzipedia::Client::ModelClient::Services::Answers} per associare alle risposte le domande e i quiz a cui si riferiscono
\item usata da \pkg{Quizzipedia::Client::ViewModelClient::CtrlServices} per avere accesso alla corretta struttura di domande e quiz
\end{itemize}
\end{itemize}
\subsubsection{Classe \cls{Cell}}
La classe descrive ogni singola riga (quindi ogni opzione) della colonna della domanda a collegamento.
\begin{figure}[H]
\centering
\noindent\makebox[\textwidth]{\includegraphics[width=\textwidth]{Img/quizzipedia-client-modelclient-services-questions-cell.pdf}}
\caption[Schema Classe Cell]{Schema Classe Quizzipedia:: Client:: ModelClient:: Services:: Questions:: Cell}
\end{figure}
\paragraph{Relazioni con altre classi}
\subparagraph{Entranti}
\begin{itemize}
\item usata da \cls{Quizzipedia::Client::ModelClient::Services::Questions::Column} per implementare correttamente le celle delle colonne richieste per la domanda a collegamento
\end{itemize}
\subsubsection{Classe \cls{Column}}
La classe descrive le colonne della domanda a collegamenti.
\begin{figure}[H]
\centering
\noindent\makebox[\textwidth]{\includegraphics[width=\textwidth]{Img/quizzipedia-client-modelclient-services-questions-column.pdf}}
\caption[Schema Classe Column]{Schema Classe Quizzipedia:: Client:: ModelClient:: Services:: Questions:: Column}
\end{figure}
\paragraph{Relazioni con altre classi}
\subparagraph{Entranti}
\begin{itemize}
\item usata da \cls{Quizzipedia::Client::ModelClient::Services::Questions::MatchingQ} per implementare correttamente le colonne richieste dalla domanda a collegamenti
\end{itemize}
\subparagraph{Uscenti}
\begin{itemize}
\item usa \cls{Quizzipedia::Client::ModelClient::Services::Questions::Cell} per implementare correttamente le celle delle colonne richieste per la domanda a collegamento
\end{itemize}
\subsubsection{Classe \cls{CompletionQ}}
Descrive le domande a completamento. Il docente fornirà un testo incompleto e una lista di possibili completamenti; lo studente dovrà inserire le parole adeguate nella giusta posizione.
\begin{figure}[H]
\centering
\noindent\makebox[\textwidth]{\includegraphics[width=\textwidth]{Img/quizzipedia-client-modelclient-services-questions-completionq.pdf}}
\caption[Schema Classe CompletionQ]{Schema Classe Quizzipedia:: Client:: ModelClient:: Services:: Questions:: CompletionQ}
\end{figure}
\paragraph{Relazioni con altre classi}
\subparagraph{Uscenti}
\begin{itemize}
\item usa \cls{Quizzipedia::Client::ModelClient::Services::Answers::AnswerCompletionQ} per poter impostare la risposta data dall'utente
\item usa \cls{Quizzipedia::Client::ModelClient::Services::Questions::GenericQuestion} per per concretizzarla, ereditandone così i campi dati e i metodi concreti
\end{itemize}
\subsubsection{Classe \cls{GenericQuestion}}
Descrive le parti comuni a tutti i tipi di domanda presenti nel sistema.
\begin{figure}[H]
\centering
\noindent\makebox[\textwidth]{\includegraphics[width=\textwidth]{Img/quizzipedia-client-modelclient-services-questions-genericquestion.pdf}}
\caption[Schema Classe GenericQuestion]{Schema Classe Quizzipedia:: Client:: ModelClient:: Services:: Questions:: GenericQuestion}
\end{figure}
\paragraph{Relazioni con altre classi}
\subparagraph{Entranti}
\begin{itemize}
\item usata da \cls{Quizzipedia::Client::ModelClient::Services::Answers::AnswerQuestion} per tenere traccia della domanda a cui corrisponde la risposta in questione
\item usata da \cls{Quizzipedia::Client::ModelClient::Services::Questions::CompletionQ} per per concretizzarla, ereditandone così i campi dati e i metodi concreti
\item usata da \cls{Quizzipedia::Client::ModelClient::Services::Questions::MatchingQ} per per concretizzarla, ereditandone così i campi dati e i metodi concreti
\item usata da \cls{Quizzipedia::Client::ModelClient::Services::Questions::MultipleChoiceQ} per per concretizzarla, ereditandone così i campi dati e i metodi concreti
\item usata da \cls{Quizzipedia::Client::ModelClient::Services::Questions::ShortAnswerQ} per per concretizzarla, ereditandone così i campi dati e i metodi concreti
\item usata da \cls{Quizzipedia::Client::ModelClient::Services::Questions::TrueFalseQ} per per concretizzarla, ereditandone così i campi dati e i metodi concreti
\item usata da \cls{Quizzipedia::Client::ModelClient::Services::Quiz} per memorizzare la lista delle domande che compongono un quiz
\item usata da \cls{Quizzipedia::Client::ViewModelClient::CtrlServices::CtrlQuestions::CtrlQuestion} per avere la struttura della domanda durante la sua creazione
\item usata da \cls{Quizzipedia::Client::ViewModelClient::CtrlServices::CtrlQuestions::CtrlQuestionManager} per poter modificare o cancellare le domande
\item usata da \cls{Quizzipedia::Client::ViewModelClient::CtrlServices::CtrlQuiz} per poter aggiungere domande all'interno del quiz
\item usata da \cls{Quizzipedia::Client::ViewModelClient::CtrlServices::CtrlQuizExec} per gestire le singole domande contenute nel quiz
\item usata da \cls{Quizzipedia::Client::ViewModelClient::CtrlServices::CtrlSearchQuestion} per comprendere e utilizzare correttamente la struttura della domanda
\item usata da \cls{Quizzipedia::Client::ViewModelClient::CtrlStatistics::CtrlTeachersStats} per poter manipolare le domande create dal docente
\end{itemize}
\subparagraph{Uscenti}
\begin{itemize}
\item usa \cls{Quizzipedia::Client::ModelClient::Statistics::QuestionStatistics} per ottenere le proprie statistiche
\end{itemize}
\subsubsection{Classe \cls{MatchingQ}}
La struttura descrive le domande a collegamento. L'utente dovrà formare la risposta collegando le entrate da un numero variabile di colonne .
\begin{figure}[H]
\centering
\noindent\makebox[\textwidth]{\includegraphics[width=\textwidth]{Img/quizzipedia-client-modelclient-services-questions-matchingq.pdf}}
\caption[Schema Classe MatchingQ]{Schema Classe Quizzipedia:: Client:: ModelClient:: Services:: Questions:: MatchingQ}
\end{figure}
\paragraph{Relazioni con altre classi}
\subparagraph{Uscenti}
\begin{itemize}
\item usa \cls{Quizzipedia::Client::ModelClient::Services::Answers::AnswerMatchingQ} per poter impostare la risposta data dall'utente
\item usa \cls{Quizzipedia::Client::ModelClient::Services::Questions::Column} per implementare correttamente le colonne richieste dalla domanda a collegamenti
\item usa \cls{Quizzipedia::Client::ModelClient::Services::Questions::GenericQuestion} per per concretizzarla, ereditandone così i campi dati e i metodi concreti
\end{itemize}
\subsubsection{Classe \cls{MultipleChoiceQ}}
La struttura descrive le domande a scelta multipla; viene presentata una lista di opzioni tra cui scegliere quelle corrette.
\begin{figure}[H]
\centering
\noindent\makebox[\textwidth]{\includegraphics[width=\textwidth]{Img/quizzipedia-client-modelclient-services-questions-multiplechoiceq.pdf}}
\caption[Schema Classe MultipleChoiceQ]{Schema Classe Quizzipedia:: Client:: ModelClient:: Services:: Questions:: MultipleChoiceQ}
\end{figure}
\paragraph{Relazioni con altre classi}
\subparagraph{Uscenti}
\begin{itemize}
\item usa \cls{Quizzipedia::Client::ModelClient::Services::Answers::AnswerMultipleChoiceQ} per poter impostare la risposta data dall'utente
\item usa \cls{Quizzipedia::Client::ModelClient::Services::Questions::GenericQuestion} per per concretizzarla, ereditandone così i campi dati e i metodi concreti
\end{itemize}
\subsubsection{Classe \cls{ShortAnswerQ}}
La struttura descrive le domande aperte, ovvero quelle la cui risposta consiste in un termine o una frase specifici.
\begin{figure}[H]
\centering
\noindent\makebox[\textwidth]{\includegraphics[width=\textwidth]{Img/quizzipedia-client-modelclient-services-questions-shortanswerq.pdf}}
\caption[Schema Classe ShortAnswerQ]{Schema Classe Quizzipedia:: Client:: ModelClient:: Services:: Questions:: ShortAnswerQ}
\end{figure}
\paragraph{Relazioni con altre classi}
\subparagraph{Uscenti}
\begin{itemize}
\item usa \cls{Quizzipedia::Client::ModelClient::Services::Answers::AnswerShortAnswerQ} per poter impostare la risposta data dall'utente
\item usa \cls{Quizzipedia::Client::ModelClient::Services::Questions::GenericQuestion} per per concretizzarla, ereditandone così i campi dati e i metodi concreti
\end{itemize}
\subsubsection{Classe \cls{TrueFalseQ}}
Viene descritta la struttura delle domande che prevedono di decidere la veridicità di un'affermazione.
\begin{figure}[H]
\centering
\noindent\makebox[\textwidth]{\includegraphics[width=\textwidth]{Img/quizzipedia-client-modelclient-services-questions-truefalseq.pdf}}
\caption[Schema Classe TrueFalseQ]{Schema Classe Quizzipedia:: Client:: ModelClient:: Services:: Questions:: TrueFalseQ}
\end{figure}
\paragraph{Relazioni con altre classi}
\subparagraph{Uscenti}
\begin{itemize}
\item usa \cls{Quizzipedia::Client::ModelClient::Services::Answers::AnswerTrueFalseQ} per poter impostare la risposta data dall'utente
\item usa \cls{Quizzipedia::Client::ModelClient::Services::Questions::GenericQuestion} per per concretizzarla, ereditandone così i campi dati e i metodi concreti
\end{itemize}
\subsection{\pkg{Quizzipedia:: Client:: ModelClient:: Statistics}}
Qui sono raccolte le classi con il compito di reperire informazioni sulle statistiche dal server e presentarle all'utente finale. Sono disponibili statistiche per le domande, per i quiz e per gli studenti di ogni classe.
\begin{figure}[H]
\centering
\noindent\makebox[\textwidth]{\includegraphics[width=\textwidth]{Img/quizzipedia-client-modelclient-statistics.pdf}}
\caption[Schema Componente Statistics]{Schema Componente Quizzipedia:: Client:: ModelClient:: Statistics}
\end{figure}
\subsubsection{Interazioni con altre componenti}
\begin{itemize}
\item \bold{Entranti}
\begin{itemize}
\item usata da \pkg{Quizzipedia::Client::ModelClient::Services} per poter memorizzare le statistiche di quiz e risposte
\end{itemize}
\item \bold{Uscenti}
\begin{itemize}
\item usa \pkg{Quizzipedia::Client::ModelClient::Users} per associare le statistiche agli utenti a cui si riferiscono
\end{itemize}
\end{itemize}
\subsubsection{Classe \cls{QuestionStatistics}}
La classe raccoglie le statistiche principali riguardanti una singola domanda..
\begin{figure}[H]
\centering
\noindent\makebox[\textwidth]{\includegraphics[width=\textwidth]{Img/quizzipedia-client-modelclient-statistics-questionstatistics.pdf}}
\caption[Schema Classe QuestionStatistics]{Schema Classe Quizzipedia:: Client:: ModelClient:: Statistics:: QuestionStatistics}
\end{figure}
\paragraph{Relazioni con altre classi}
\subparagraph{Entranti}
\begin{itemize}
\item usata da \cls{Quizzipedia::Client::ModelClient::Services::Questions::GenericQuestion} per ottenere le proprie statistiche
\item usata da \cls{Quizzipedia::Client::ViewModelClient::CtrlStatistics::CtrlQuestionStatistics} per ottenere le statistiche sulle domande richieste dall'utente
\end{itemize}
\subsubsection{Classe \cls{QuizStatistics}}
La classe raccoglie le statistiche principali riguardanti un singolo quiz.
\begin{figure}[H]
\centering
\noindent\makebox[\textwidth]{\includegraphics[width=\textwidth]{Img/quizzipedia-client-modelclient-statistics-quizstatistics.pdf}}
\caption[Schema Classe QuizStatistics]{Schema Classe Quizzipedia:: Client:: ModelClient:: Statistics:: QuizStatistics}
\end{figure}
\paragraph{Relazioni con altre classi}
\subparagraph{Entranti}
\begin{itemize}
\item usata da \cls{Quizzipedia::Client::ModelClient::Services::Quiz} per memorizzare le statistiche generali riguardo al quiz
\item usata da \cls{Quizzipedia::Client::ViewModelClient::CtrlStatistics::CtrlQuizStatistics} per ottenere le statistiche generali richieste sui quiz richieste dall'utente
\end{itemize}
\subsubsection{Classe \cls{StudentsStatisticsQuiz}}
Classe usata per visualizzare gli studenti che hanno svolto un quiz specifico e i loro voti per il quiz stesso.
\begin{figure}[H]
\centering
\noindent\makebox[\textwidth]{\includegraphics[width=\textwidth]{Img/quizzipedia-client-modelclient-statistics-studentsstatisticsquiz.pdf}}
\caption[Schema Classe StudentsStatisticsQuiz]{Schema Classe Quizzipedia:: Client:: ModelClient:: Statistics:: StudentsStatisticsQuiz}
\end{figure}
\paragraph{Relazioni con altre classi}
\subparagraph{Entranti}
\begin{itemize}
\item usata da \cls{Quizzipedia::Client::ModelClient::Services::Quiz} per memorizzare le statistiche relative agli studenti che hanno svolto il quiz
\item usata da \cls{Quizzipedia::Client::ViewModelClient::CtrlStatistics::CtrlStudentsQuizStats} per presentare le statistiche degli studenti relativamente a un quiz
\end{itemize}
\subparagraph{Uscenti}
\begin{itemize}
\item usa \cls{Quizzipedia::Client::ModelClient::Users::Student} per tenere traccia degli utenti a cui le statistiche si riferiscono
\end{itemize}
\subsection{\pkg{Quizzipedia:: Client:: ModelClient:: Users}}
Raccoglie le classi necessarie a descrivere le diverse tipologie di utente che possono accedere al sistema.
\begin{figure}[H]
\centering
\noindent\makebox[\textwidth]{\includegraphics[width=\textwidth]{Img/quizzipedia-client-modelclient-users.pdf}}
\caption[Schema Componente Users]{Schema Componente Quizzipedia:: Client:: ModelClient:: Users}
\end{figure}
\subsubsection{Interazioni con altre componenti}
\begin{itemize}
\item \bold{Entranti}
\begin{itemize}
\item usata da \pkg{Quizzipedia::Client::ModelClient::Organizations} per inserire utenti all'interno di classi e enti
\item usata da \pkg{Quizzipedia::Client::ModelClient::Statistics} per associare le statistiche agli utenti a cui si riferiscono
\item usata da \pkg{Quizzipedia::Client::ViewModelClient::CtrlUsers} per avere accesso alla struttura dei vari tipi di utente. In questo modo potrà compiere operazioni di modifica e manutenzione
\end{itemize}
\item \bold{Uscenti}
\begin{itemize}
\item usa \pkg{Quizzipedia::Client::ModelClient::Services} per permettere agli utenti di tenere traccia dello storico dei quiz svolti
\end{itemize}
\end{itemize}
\subsubsection{Classe \cls{AuthenticationData}}
Questa classe gestisce le informazioni di autenticazione comuni a tutti gli utenti.
\begin{figure}[H]
\centering
\noindent\makebox[\textwidth]{\includegraphics[width=\textwidth]{Img/quizzipedia-client-modelclient-users-authenticationdata.pdf}}
\caption[Schema Classe AuthenticationData]{Schema Classe Quizzipedia:: Client:: ModelClient:: Users:: AuthenticationData}
\end{figure}
\paragraph{Relazioni con altre classi}
\subparagraph{Entranti}
\begin{itemize}
\item usata da \cls{Quizzipedia::Client::ModelClient::Users::User} per memorizzare le informazioni di autenticazione comuni a tutti gli utenti registrati
\end{itemize}
\subsubsection{Classe \cls{Director}}
Rappresenta un responsabile, ovvero colui che gestisce docenti e studenti per ogni ente del sistema.
\begin{figure}[H]
\centering
\noindent\makebox[\textwidth]{\includegraphics[width=\textwidth]{Img/quizzipedia-client-modelclient-users-director.pdf}}
\caption[Schema Classe Director]{Schema Classe Quizzipedia:: Client:: ModelClient:: Users:: Director}
\end{figure}
\paragraph{Relazioni con altre classi}
\subparagraph{Entranti}
\begin{itemize}
\item usata da \cls{Quizzipedia::Client::ModelClient::Organizations::Institution} per identificare il responsabile dell'ente
\end{itemize}
\subparagraph{Uscenti}
\begin{itemize}
\item usa \cls{Quizzipedia::Client::ModelClient::Users::User} per ereditarne i campi dato e i metodi comuni a tutti gli utenti
\end{itemize}
\subsubsection{Classe \cls{NoRole}}
Rappresenta gli utenti senza ruolo del sistema; coloro che si sono registrati e autenticati ma non hanno ancora fatto richiesta per l'assegnazione ad alcun ruolo.
\begin{figure}[H]
\centering
\noindent\makebox[\textwidth]{\includegraphics[width=\textwidth]{Img/quizzipedia-client-modelclient-users-norole.pdf}}
\caption[Schema Classe NoRole]{Schema Classe Quizzipedia:: Client:: ModelClient:: Users:: NoRole}
\end{figure}
\paragraph{Relazioni con altre classi}
\subparagraph{Uscenti}
\begin{itemize}
\item usa \cls{Quizzipedia::Client::ModelClient::Services::Answers::AnswerQuiz} per tenere traccia dei quiz che ha già svolto e del loro esito
\item usa \cls{Quizzipedia::Client::ModelClient::Users::User} per ereditarne i campi dato e i metodi comuni a tutti gli utenti
\end{itemize}
\subsubsection{Classe \cls{Student}}
Rappresenta uno studente del sistema e implementa le sue funzioni specifiche oltre a quelle ereditate da utente.
\begin{figure}[H]
\centering
\noindent\makebox[\textwidth]{\includegraphics[width=\textwidth]{Img/quizzipedia-client-modelclient-users-student.pdf}}
\caption[Schema Classe Student]{Schema Classe Quizzipedia:: Client:: ModelClient:: Users:: Student}
\end{figure}
\paragraph{Relazioni con altre classi}
\subparagraph{Entranti}
\begin{itemize}
\item usata da \cls{Quizzipedia::Client::ModelClient::Statistics::StudentsStatisticsQuiz} per tenere traccia degli utenti a cui le statistiche si riferiscono
\end{itemize}
\subparagraph{Uscenti}
\begin{itemize}
\item usa \cls{Quizzipedia::Client::ModelClient::Services::Answers::AnswerQuiz} per tenere traccia dei quiz che ha già svolto e del loro esito
\item usa \cls{Quizzipedia::Client::ModelClient::Users::User} per ereditarne i campi dato e i metodi comuni a tutti gli utenti
\end{itemize}
\subsubsection{Classe \cls{Teacher}}
Rappresenta un docente del sistema e ne implementa le funzionalità specifiche in aggiunta a quelle comuni a tutti gli utenti.
\begin{figure}[H]
\centering
\noindent\makebox[\textwidth]{\includegraphics[width=\textwidth]{Img/quizzipedia-client-modelclient-users-teacher.pdf}}
\caption[Schema Classe Teacher]{Schema Classe Quizzipedia:: Client:: ModelClient:: Users:: Teacher}
\end{figure}
\paragraph{Relazioni con altre classi}
\subparagraph{Entranti}
\begin{itemize}
\item usata da \cls{Quizzipedia::Client::ViewModelClient::CtrlServices::CtrlQuestions::CtrlQuestionManager} per identificare il docente in modo da fornirgli la lista con le sole domande da lui create
\item usata da \cls{Quizzipedia::Client::ViewModelClient::CtrlServices::CtrlQuiz} per riconoscere l'utente come docente e memorizzarlo come creatore del quiz
\item usata da \cls{Quizzipedia::Client::ViewModelClient::CtrlServices::CtrlQuizManager} per riconoscere il docente e fornirgli una lista dei quiz da lui creati
\item usata da \cls{Quizzipedia::Client::ViewModelClient::CtrlServices::CtrlSearchQuestion} per identificare il docente che sta effettuando la ricerca
\item usata da \cls{Quizzipedia::Client::ViewModelClient::CtrlStatistics::CtrlTeachersStats} per identificare il docente di cui si vuole avere le statistiche
\end{itemize}
\subparagraph{Uscenti}
\begin{itemize}
\item usa \cls{Quizzipedia::Client::ModelClient::Services::Quiz} per tenere traccia dei quiz da lui creati
\item usa \cls{Quizzipedia::Client::ModelClient::Users::User} per ereditarne i campi dato e i metodi comuni a tutti gli utenti
\end{itemize}
\subsubsection{Classe \cls{User}}
Questa è una classe astratta e raccoglie le funzionalità comuni a tutti gli utenti.
\begin{figure}[H]
\centering
\noindent\makebox[\textwidth]{\includegraphics[width=\textwidth]{Img/quizzipedia-client-modelclient-users-user.pdf}}
\caption[Schema Classe User]{Schema Classe Quizzipedia:: Client:: ModelClient:: Users:: User}
\end{figure}
\paragraph{Relazioni con altre classi}
\subparagraph{Entranti}
\begin{itemize}
\item usata da \cls{Quizzipedia::Client::ModelClient::Services::Answers::AnswerQuestion} per tenere traccia dell'utente che sta svolgendo la domanda
\item usata da \cls{Quizzipedia::Client::ModelClient::Users::Director} per ereditarne i campi dato e i metodi comuni a tutti gli utenti
\item usata da \cls{Quizzipedia::Client::ModelClient::Users::NoRole} per ereditarne i campi dato e i metodi comuni a tutti gli utenti
\item usata da \cls{Quizzipedia::Client::ModelClient::Users::Student} per ereditarne i campi dato e i metodi comuni a tutti gli utenti
\item usata da \cls{Quizzipedia::Client::ModelClient::Users::Teacher} per ereditarne i campi dato e i metodi comuni a tutti gli utenti
\item usata da \cls{Quizzipedia::Client::ViewModelClient::CtrlServices::CtrlQuizSelection} per tenere traccia dell'utente che sta svolgendo le operazioni
\item usata da \cls{Quizzipedia::Client::ViewModelClient::CtrlServices::CtrlSearchQuiz} per identificare l'utente che sta effettuando la ricerca
\item usata da \cls{Quizzipedia::Client::ViewModelClient::CtrlUsers::CtrlHeader} per comprendere la tipologia di utente che visita la pagina e identificare quindi lo header corretto
\item usata da \cls{Quizzipedia::Client::ViewModelClient::CtrlUsers::CtrlUserManager} per caricare un utente del tipo corretto. A seconda del tipo di utente ne verrà creato uno appropriato tra studente, docente, responsabile o utente senza ruolo
\end{itemize}
\subparagraph{Uscenti}
\begin{itemize}
\item usa \cls{Quizzipedia::Client::ModelClient::Users::AuthenticationData} per memorizzare le informazioni di autenticazione comuni a tutti gli utenti registrati
\end{itemize}
\subsection{\pkg{Quizzipedia:: Client:: ViewClient}}
Racchiude tutte le componenti necessarie per presentare il prodotto all'utente.
Grazie all'uso di Angular.js ogni modifica svolta dall'utente si ripercuoterà automaticamente a sulla componente model, attraverso il controller, per mantenere sempre le informazioni consistenti e aggiornate.
Con l'uso invece di Bootstrap e Fabric.js a livello di grafica web risulta semplificata l'impaginazione delle singole pagine web dedicate all'utente e alla gestione di allegati non testuali all'interno di domande.
\begin{figure}[H]
\centering
\noindent\makebox[\textwidth]{\includegraphics[width=\textwidth]{Img/quizzipedia-client-viewclient.pdf}}
\caption[Schema Componente ViewClient]{Schema Componente Quizzipedia:: Client:: ViewClient}
\end{figure}
\subsubsection{Componenti contenute}
\begin{itemize}
\item \pkg{Quizzipedia::Client::ViewClient::Shared}
\item \pkg{Quizzipedia::Client::ViewClient::ViewOrgManager}
\item \pkg{Quizzipedia::Client::ViewClient::ViewQuestionManager}
\item \pkg{Quizzipedia::Client::ViewClient::ViewQuizManager}
\item \pkg{Quizzipedia::Client::ViewClient::ViewQuizSolver}
\item \pkg{Quizzipedia::Client::ViewClient::ViewRequests}
\item \pkg{Quizzipedia::Client::ViewClient::ViewSearch}
\item \pkg{Quizzipedia::Client::ViewClient::ViewStatistics}
\item \pkg{Quizzipedia::Client::ViewClient::ViewTopicManager}
\item \pkg{Quizzipedia::Client::ViewClient::ViewUsers}
\end{itemize}
\subsubsection{Interazioni con altre componenti}
\begin{itemize}
\item \bold{Uscenti}
\begin{itemize}
\item usa \pkg{Quizzipedia::Client::ViewModelClient} per aggiungi
\end{itemize}
\end{itemize}
\subsection{\pkg{Quizzipedia:: Client:: ViewClient:: Shared}}
Contiene componenti condivise da più pagine del sistema, che vengono quindi qui raccolte perché siano accessibili a tutti.
\begin{figure}[H]
\centering
\noindent\makebox[\textwidth]{\includegraphics[width=\textwidth]{Img/quizzipedia-client-viewclient-shared.pdf}}
\caption[Schema Componente Shared]{Schema Componente Quizzipedia:: Client:: ViewClient:: Shared}
\end{figure}
\subsubsection{Componenti contenute}
\begin{itemize}
\item \pkg{Quizzipedia::Client::ViewClient::Shared::Header}
\item \pkg{Quizzipedia::Client::ViewClient::Shared::Homepage}
\end{itemize}
\subsubsection{Classe \cls{CollapsableTips}}
Contiene il codice necessario a creare il dato elemento grafico.
\begin{figure}[H]
\centering
\noindent\makebox[\textwidth]{\includegraphics[width=\textwidth]{Img/quizzipedia-client-viewclient-shared-collapsabletips.pdf}}
\caption[Schema Classe CollapsableTips]{Schema Classe Quizzipedia:: Client:: ViewClient:: Shared:: CollapsableTips}
\end{figure}
\subsubsection{Classe \cls{Footer}}
Contiene il codice necessario a creare il footer, che è costante su tutte le pagine del sito.
\begin{figure}[H]
\centering
\noindent\makebox[\textwidth]{\includegraphics[width=\textwidth]{Img/quizzipedia-client-viewclient-shared-footer.pdf}}
\caption[Schema Classe Footer]{Schema Classe Quizzipedia:: Client:: ViewClient:: Shared:: Footer}
\end{figure}
\subsubsection{Classe \cls{Pagination}}
Contiene il codice necessario a gestire il sistema di paginazione.
\begin{figure}[H]
\centering
\noindent\makebox[\textwidth]{\includegraphics[width=\textwidth]{Img/quizzipedia-client-viewclient-shared-pagination.pdf}}
\caption[Schema Classe Pagination]{Schema Classe Quizzipedia:: Client:: ViewClient:: Shared:: Pagination}
\end{figure}
\subsubsection{Classe \cls{ProgressBar}}
Contiene il codice necessario a creare la barra di progresso.
\begin{figure}[H]
\centering
\noindent\makebox[\textwidth]{\includegraphics[width=\textwidth]{Img/quizzipedia-client-viewclient-shared-progressbar.pdf}}
\caption[Schema Classe ProgressBar]{Schema Classe Quizzipedia:: Client:: ViewClient:: Shared:: ProgressBar}
\end{figure}
\subsection{\pkg{Quizzipedia:: Client:: ViewClient:: Shared:: Header}}
Contiene i template per caricare header e breadcrumb delle pagine a seconda della posizione e del ruolo dell'utente.
\begin{figure}[H]
\centering
\noindent\makebox[\textwidth]{\includegraphics[width=\textwidth]{Img/quizzipedia-client-viewclient-shared-header.pdf}}
\caption[Schema Componente Header]{Schema Componente Quizzipedia:: Client:: ViewClient:: Shared:: Header}
\end{figure}
\subsubsection{Classe \cls{Breadcrumbs}}
Contiene il codice necessario a creare il breadcrumb.
\begin{figure}[H]
\centering
\noindent\makebox[\textwidth]{\includegraphics[width=\textwidth]{Img/quizzipedia-client-viewclient-shared-header-breadcrumbs.pdf}}
\caption[Schema Classe Breadcrumbs]{Schema Classe Quizzipedia:: Client:: ViewClient:: Shared:: Header:: Breadcrumbs}
\end{figure}
\paragraph{Relazioni con altre classi}
\subparagraph{Entranti}
\begin{itemize}
\item usata da \cls{Quizzipedia::Client::ViewClient::Shared::Header::HeaderLogged} per tenere traccia del percorso compiuto dall'utente durante la navigazione
\item usata da \cls{Quizzipedia::Client::ViewClient::Shared::Header::HeaderNotLogged} per tenere traccia del percorso compiuto dall'utente durante la navigazione
\end{itemize}
\subsubsection{Classe \cls{HeaderLogged}}
Carica lo header che verrà visualizzato dall'utente registrato.
\begin{figure}[H]
\centering
\noindent\makebox[\textwidth]{\includegraphics[width=\textwidth]{Img/quizzipedia-client-viewclient-shared-header-headerlogged.pdf}}
\caption[Schema Classe HeaderLogged]{Schema Classe Quizzipedia:: Client:: ViewClient:: Shared:: Header:: HeaderLogged}
\end{figure}
\paragraph{Relazioni con altre classi}
\subparagraph{Entranti}
\begin{itemize}
\item usata da \cls{Quizzipedia::Client::ViewClient::Shared::Header::HeaderLoggedDirector} per aggiungere allo header visualizzato dal responsabile la parte comune a tutti gli utenti registrati
\item usata da \cls{Quizzipedia::Client::ViewClient::Shared::Header::HeaderLoggedStudent} per aggiungere allo header visualizzato dallo studente la parte comune a tutti gli utenti registrati
\item usata da \cls{Quizzipedia::Client::ViewClient::Shared::Homepage::HomepageTeacher} per aggiungere allo header visualizzato dal docente la parte comune a tutti gli utenti registrati
\end{itemize}
\subparagraph{Uscenti}
\begin{itemize}
\item usa \cls{Quizzipedia::Client::ViewClient::Shared::Header::Breadcrumbs} per tenere traccia del percorso compiuto dall'utente durante la navigazione
\item usa \cls{Quizzipedia::Client::ViewModelClient::CtrlUsers::CtrlHeader} per ricevere dati su che tipologia di utente è collegata al sito e caricare, quindi, lo header corretto
\end{itemize}
\subsubsection{Classe \cls{HeaderLoggedDirector}}
Carica lo header che verrà visualizzato dal responsabile.
\begin{figure}[H]
\centering
\noindent\makebox[\textwidth]{\includegraphics[width=\textwidth]{Img/quizzipedia-client-viewclient-shared-header-headerloggeddirector.pdf}}
\caption[Schema Classe HeaderLoggedDirector]{Schema Classe Quizzipedia:: Client:: ViewClient:: Shared:: Header:: HeaderLoggedDirector}
\end{figure}
\paragraph{Relazioni con altre classi}
\subparagraph{Uscenti}
\begin{itemize}
\item usa \cls{Quizzipedia::Client::ViewClient::Shared::Header::HeaderLogged} per aggiungere allo header visualizzato dal responsabile la parte comune a tutti gli utenti registrati
\end{itemize}
\subsubsection{Classe \cls{HeaderLoggedStudent}}
Carica lo header che verrà visualizzato dallo studente.
\begin{figure}[H]
\centering
\noindent\makebox[\textwidth]{\includegraphics[width=\textwidth]{Img/quizzipedia-client-viewclient-shared-header-headerloggedstudent.pdf}}
\caption[Schema Classe HeaderLoggedStudent]{Schema Classe Quizzipedia:: Client:: ViewClient:: Shared:: Header:: HeaderLoggedStudent}
\end{figure}
\paragraph{Relazioni con altre classi}
\subparagraph{Uscenti}
\begin{itemize}
\item usa \cls{Quizzipedia::Client::ViewClient::Shared::Header::HeaderLogged} per aggiungere allo header visualizzato dallo studente la parte comune a tutti gli utenti registrati
\end{itemize}
\subsubsection{Classe \cls{HeaderLoggedTeacher}}
Carica lo header che verrà visualizzato dall'insegnante.
\begin{figure}[H]
\centering
\noindent\makebox[\textwidth]{\includegraphics[width=\textwidth]{Img/quizzipedia-client-viewclient-shared-header-headerloggedteacher.pdf}}
\caption[Schema Classe HeaderLoggedTeacher]{Schema Classe Quizzipedia:: Client:: ViewClient:: Shared:: Header:: HeaderLoggedTeacher}
\end{figure}
\subsubsection{Classe \cls{HeaderNotLogged}}
Carica lo header che verrà visualizzato da un visitatore non registrato.
\begin{figure}[H]
\centering
\noindent\makebox[\textwidth]{\includegraphics[width=\textwidth]{Img/quizzipedia-client-viewclient-shared-header-headernotlogged.pdf}}
\caption[Schema Classe HeaderNotLogged]{Schema Classe Quizzipedia:: Client:: ViewClient:: Shared:: Header:: HeaderNotLogged}
\end{figure}
\paragraph{Relazioni con altre classi}
\subparagraph{Uscenti}
\begin{itemize}
\item usa \cls{Quizzipedia::Client::ViewClient::Shared::Header::Breadcrumbs} per tenere traccia del percorso compiuto dall'utente durante la navigazione
\item usa \cls{Quizzipedia::Client::ViewModelClient::CtrlUsers::CtrlHeader} per ricevere informazioni sul tipo di utente connesso e caricare lo header correttamente
\end{itemize}
\subsection{\pkg{Quizzipedia:: Client:: ViewClient:: Shared:: Homepage}}
Contiene le pagine necessarie alla creazione della homepage e delle parti che possono variare a seconda del tipo di utente che la visita.
\begin{figure}[H]
\centering
\noindent\makebox[\textwidth]{\includegraphics[width=\textwidth]{Img/quizzipedia-client-viewclient-shared-homepage.pdf}}
\caption[Schema Componente Homepage]{Schema Componente Quizzipedia:: Client:: ViewClient:: Shared:: Homepage}
\end{figure}
\subsubsection{Classe \cls{HomepageDirector}}
Carica la homepage che verrà visualizzata dal responsabile.
\begin{figure}[H]
\centering
\noindent\makebox[\textwidth]{\includegraphics[width=\textwidth]{Img/quizzipedia-client-viewclient-shared-homepage-homepagedirector.pdf}}
\caption[Schema Classe HomepageDirector]{Schema Classe Quizzipedia:: Client:: ViewClient:: Shared:: Homepage:: HomepageDirector}
\end{figure}
\subsubsection{Classe \cls{HomepageGeneric}}
Carica la homepage che verrà visualizzata da un visitatore generico.
\begin{figure}[H]
\centering
\noindent\makebox[\textwidth]{\includegraphics[width=\textwidth]{Img/quizzipedia-client-viewclient-shared-homepage-homepagegeneric.pdf}}
\caption[Schema Classe HomepageGeneric]{Schema Classe Quizzipedia:: Client:: ViewClient:: Shared:: Homepage:: HomepageGeneric}
\end{figure}
\subsubsection{Classe \cls{HomepageStudent}}
Carica la homepage che verrà visualizzata dallo studente.
\begin{figure}[H]
\centering
\noindent\makebox[\textwidth]{\includegraphics[width=\textwidth]{Img/quizzipedia-client-viewclient-shared-homepage-homepagestudent.pdf}}
\caption[Schema Classe HomepageStudent]{Schema Classe Quizzipedia:: Client:: ViewClient:: Shared:: Homepage:: HomepageStudent}
\end{figure}
\subsubsection{Classe \cls{HomepageTeacher}}
Carica la homepage che verrà visualizzata dal docente.
\begin{figure}[H]
\centering
\noindent\makebox[\textwidth]{\includegraphics[width=\textwidth]{Img/quizzipedia-client-viewclient-shared-homepage-homepageteacher.pdf}}
\caption[Schema Classe HomepageTeacher]{Schema Classe Quizzipedia:: Client:: ViewClient:: Shared:: Homepage:: HomepageTeacher}
\end{figure}
\paragraph{Relazioni con altre classi}
\subparagraph{Uscenti}
\begin{itemize}
\item usa \cls{Quizzipedia::Client::ViewClient::Shared::Header::HeaderLogged} per aggiungere allo header visualizzato dal docente la parte comune a tutti gli utenti registrati
\end{itemize}
\subsection{\pkg{Quizzipedia:: Client:: ViewClient:: ViewOrgManager}}
Qui sono raccolte le classi responsabili della presentazione delle pagine da cui sarà possibile gestire le classi e gli enti presenti in Quizzipedia.
\begin{figure}[H]
\centering
\noindent\makebox[\textwidth]{\includegraphics[width=\textwidth]{Img/quizzipedia-client-viewclient-vieworgmanager.pdf}}
\caption[Schema Componente ViewOrgManager]{Schema Componente Quizzipedia:: Client:: ViewClient:: ViewOrgManager}
\end{figure}
\subsubsection{Interazioni con altre componenti}
\begin{itemize}
\item \bold{Uscenti}
\begin{itemize}
\item usa \pkg{Quizzipedia::Client::ViewModelClient::CtrlOrganization} per aggiungi
\end{itemize}
\end{itemize}
\subsubsection{Classe \cls{ViewClassMembers}}
La classe si occupa di presentare una lista degli utenti iscritti alla classe e altre informazioni aggiuntive.
\begin{figure}[H]
\centering
\noindent\makebox[\textwidth]{\includegraphics[width=\textwidth]{Img/quizzipedia-client-viewclient-vieworgmanager-viewclassmembers.pdf}}
\caption[Schema Classe ViewClassMembers]{Schema Classe Quizzipedia:: Client:: ViewClient:: ViewOrgManager:: ViewClassMembers}
\end{figure}
\subsubsection{Classe \cls{ViewModifyOrg}}
Presenta all'utente la pagina da cui sarà possibile modificare le informazioni su una classe o su un ente esistente.
\begin{figure}[H]
\centering
\noindent\makebox[\textwidth]{\includegraphics[width=\textwidth]{Img/quizzipedia-client-viewclient-vieworgmanager-viewmodifyorg.pdf}}
\caption[Schema Classe ViewModifyOrg]{Schema Classe Quizzipedia:: Client:: ViewClient:: ViewOrgManager:: ViewModifyOrg}
\end{figure}
\paragraph{Relazioni con altre classi}
\subparagraph{Uscenti}
\begin{itemize}
\item usa \cls{Quizzipedia::Client::ViewModelClient::CtrlOrganization::CtrlInstitution} per passare al controller i dati immessi dall'utente durante le operazioni di modifica, creazione o rimozione di una classe di un istituto
\end{itemize}
\subsection{\pkg{Quizzipedia:: Client:: ViewClient:: ViewQuestionManager}}
Qui sono raccolte le classi responsabili della presentazione delle pagine da cui sarà possibile gestire le domande.
\begin{figure}[H]
\centering
\noindent\makebox[\textwidth]{\includegraphics[width=\textwidth]{Img/quizzipedia-client-viewclient-viewquestionmanager.pdf}}
\caption[Schema Componente ViewQuestionManager]{Schema Componente Quizzipedia:: Client:: ViewClient:: ViewQuestionManager}
\end{figure}
\subsubsection{Interazioni con altre componenti}
\begin{itemize}
\item \bold{Uscenti}
\begin{itemize}
\item usa \pkg{Quizzipedia::Client::ViewModelClient::CtrlServices} per aggiungi
\end{itemize}
\end{itemize}
\subsubsection{Classe \cls{CreateCompletionQ}}
Presenta la pagina da cui sarà possibile creare una nuova domanda a completamento.
\begin{figure}[H]
\centering
\noindent\makebox[\textwidth]{\includegraphics[width=\textwidth]{Img/quizzipedia-client-viewclient-viewquestionmanager-createcompletionq.pdf}}
\caption[Schema Classe CreateCompletionQ]{Schema Classe Quizzipedia:: Client:: ViewClient:: ViewQuestionManager:: CreateCompletionQ}
\end{figure}
\subsubsection{Classe \cls{CreateGenericQ}}
Presenta la pagina da cui è possibile creare una nuova domanda, inserendo le informazioni comuni a tutte le domande. Poi si potrà selezionare il tipo di domanda desiderato.
\begin{figure}[H]
\centering
\noindent\makebox[\textwidth]{\includegraphics[width=\textwidth]{Img/quizzipedia-client-viewclient-viewquestionmanager-creategenericq.pdf}}
\caption[Schema Classe CreateGenericQ]{Schema Classe Quizzipedia:: Client:: ViewClient:: ViewQuestionManager:: CreateGenericQ}
\end{figure}
\paragraph{Relazioni con altre classi}
\subparagraph{Uscenti}
\begin{itemize}
\item usa \cls{Quizzipedia::Client::ViewModelClient::CtrlServices::CtrlQuestions::CtrlQuestion} per passare al controller i dati immessi dall'utente durante la creazione della domanda
\end{itemize}
\subsubsection{Classe \cls{CreateMatchingQ}}
Presenta la pagina da cui sarà possibile creare una nuova domanda a collegamenti.
\begin{figure}[H]
\centering
\noindent\makebox[\textwidth]{\includegraphics[width=\textwidth]{Img/quizzipedia-client-viewclient-viewquestionmanager-creatematchingq.pdf}}
\caption[Schema Classe CreateMatchingQ]{Schema Classe Quizzipedia:: Client:: ViewClient:: ViewQuestionManager:: CreateMatchingQ}
\end{figure}
\subsubsection{Classe \cls{CreateMultipleChoiceQ}}
Presenta la pagina da cui sarà possibile creare una nuova domanda a scelta multipla.
\begin{figure}[H]
\centering
\noindent\makebox[\textwidth]{\includegraphics[width=\textwidth]{Img/quizzipedia-client-viewclient-viewquestionmanager-createmultiplechoiceq.pdf}}
\caption[Schema Classe CreateMultipleChoiceQ]{Schema Classe Quizzipedia:: Client:: ViewClient:: ViewQuestionManager:: CreateMultipleChoiceQ}
\end{figure}
\subsubsection{Classe \cls{CreateShortAnswerQ}}
Presenta la pagina da cui sarà possibile creare una nuova domanda aperta.
\begin{figure}[H]
\centering
\noindent\makebox[\textwidth]{\includegraphics[width=\textwidth]{Img/quizzipedia-client-viewclient-viewquestionmanager-createshortanswerq.pdf}}
\caption[Schema Classe CreateShortAnswerQ]{Schema Classe Quizzipedia:: Client:: ViewClient:: ViewQuestionManager:: CreateShortAnswerQ}
\end{figure}
\subsubsection{Classe \cls{CreateTrueFalseQ}}
Presenta la pagina da cui sarà possibile creare una nuova domanda di tipo vero/falso.
\begin{figure}[H]
\centering
\noindent\makebox[\textwidth]{\includegraphics[width=\textwidth]{Img/quizzipedia-client-viewclient-viewquestionmanager-createtruefalseq.pdf}}
\caption[Schema Classe CreateTrueFalseQ]{Schema Classe Quizzipedia:: Client:: ViewClient:: ViewQuestionManager:: CreateTrueFalseQ}
\end{figure}
\subsubsection{Classe \cls{QuestionMgmt}}
Presenta all'utente il pannello da cui sarà possibile creare una nuova domanda e visualizzare una lista di informazioni riassuntive sulle domande già create, con possibilità di modifica e rimozione.
\begin{figure}[H]
\centering
\noindent\makebox[\textwidth]{\includegraphics[width=\textwidth]{Img/quizzipedia-client-viewclient-viewquestionmanager-questionmgmt.pdf}}
\caption[Schema Classe QuestionMgmt]{Schema Classe Quizzipedia:: Client:: ViewClient:: ViewQuestionManager:: QuestionMgmt}
\end{figure}
\paragraph{Relazioni con altre classi}
\subparagraph{Uscenti}
\begin{itemize}
\item usa \cls{Quizzipedia::Client::ViewModelClient::CtrlServices::CtrlQuestions::CtrlQuestionManager} per visualizzare la lista delle domande del docente, permettendogli di modificarle o rimuoverle
\end{itemize}
\subsection{\pkg{Quizzipedia:: Client:: ViewClient:: ViewQuizManager}}
Qui sono raccolte le classi responsabili della presentazione delle pagine da cui sarà possibile gestire i quiz.
\begin{figure}[H]
\centering
\noindent\makebox[\textwidth]{\includegraphics[width=\textwidth]{Img/quizzipedia-client-viewclient-viewquizmanager.pdf}}
\caption[Schema Componente ViewQuizManager]{Schema Componente Quizzipedia:: Client:: ViewClient:: ViewQuizManager}
\end{figure}
\subsubsection{Interazioni con altre componenti}
\begin{itemize}
\item \bold{Uscenti}
\begin{itemize}
\item usa \pkg{Quizzipedia::Client::ViewModelClient::CtrlServices} per aggiungi
\end{itemize}
\end{itemize}
\subsubsection{Classe \cls{ViewCreateQuiz}}
Presenta la pagina da cui sarà possibile creare un nuovo quiz.
\begin{figure}[H]
\centering
\noindent\makebox[\textwidth]{\includegraphics[width=\textwidth]{Img/quizzipedia-client-viewclient-viewquizmanager-viewcreatequiz.pdf}}
\caption[Schema Classe ViewCreateQuiz]{Schema Classe Quizzipedia:: Client:: ViewClient:: ViewQuizManager:: ViewCreateQuiz}
\end{figure}
\paragraph{Relazioni con altre classi}
\subparagraph{Uscenti}
\begin{itemize}
\item usa \cls{Quizzipedia::Client::ViewModelClient::CtrlServices::CtrlQuiz} per memorizzare le informazioni date dal docente durante la creazione del quiz
\end{itemize}
\subsubsection{Classe \cls{ViewEditQuiz}}
Presenta all'utente una pagina da cui è possibile modificare nel dettaglio un quiz esistente, intervenendo sulle domande che lo compongono.
\begin{figure}[H]
\centering
\noindent\makebox[\textwidth]{\includegraphics[width=\textwidth]{Img/quizzipedia-client-viewclient-viewquizmanager-vieweditquiz.pdf}}
\caption[Schema Classe ViewEditQuiz]{Schema Classe Quizzipedia:: Client:: ViewClient:: ViewQuizManager:: ViewEditQuiz}
\end{figure}
\paragraph{Relazioni con altre classi}
\subparagraph{Uscenti}
\begin{itemize}
\item usa \cls{Quizzipedia::Client::ViewModelClient::CtrlServices::CtrlQuiz} per fare in modo che i cambiamenti che l'utente apporta al quiz siano consistenti
\end{itemize}
\subsubsection{Classe \cls{ViewQuizMgmt}}
Carica una pagina contenente una lista con informazioni riassuntive sui quiz e un pannello da cui sarà possibile svolgere operazioni di modifica e rimozione. Da qui è inoltre possibile creare nuovi quiz.
\begin{figure}[H]
\centering
\noindent\makebox[\textwidth]{\includegraphics[width=\textwidth]{Img/quizzipedia-client-viewclient-viewquizmanager-viewquizmgmt.pdf}}
\caption[Schema Classe ViewQuizMgmt]{Schema Classe Quizzipedia:: Client:: ViewClient:: ViewQuizManager:: ViewQuizMgmt}
\end{figure}
\paragraph{Relazioni con altre classi}
\subparagraph{Uscenti}
\begin{itemize}
\item usa \cls{Quizzipedia::Client::ViewModelClient::CtrlServices::CtrlQuiz} per aggiungi
\item usa \cls{Quizzipedia::Client::ViewModelClient::CtrlServices::CtrlQuizManager} per fornire al docente la lista dei propri quiz con possibilità di modifica e rimozione
\end{itemize}
\subsection{\pkg{Quizzipedia:: Client:: ViewClient:: ViewQuizSolver}}
Il componente raccoglie le classi necessarie alla visualizzazione delle pagine da cui sarà possibile svolgere quiz.
\begin{figure}[H]
\centering
\noindent\makebox[\textwidth]{\includegraphics[width=\textwidth]{Img/quizzipedia-client-viewclient-viewquizsolver.pdf}}
\caption[Schema Componente ViewQuizSolver]{Schema Componente Quizzipedia:: Client:: ViewClient:: ViewQuizSolver}
\end{figure}
\subsubsection{Componenti contenute}
\begin{itemize}
\item \pkg{Quizzipedia::Client::ViewClient::ViewQuizSolver::ViewQuestionSolver}
\end{itemize}
\subsubsection{Interazioni con altre componenti}
\begin{itemize}
\item \bold{Uscenti}
\begin{itemize}
\item usa \pkg{Quizzipedia::Client::ViewModelClient::CtrlServices} per aggiungi
\end{itemize}
\end{itemize}
\subsubsection{Classe \cls{ViewQuizExec}}
Pagina da cui sarà possibile svolgere un quiz. Verranno caricate le domande che lo compongono fino al risultato finale.
\begin{figure}[H]
\centering
\noindent\makebox[\textwidth]{\includegraphics[width=\textwidth]{Img/quizzipedia-client-viewclient-viewquizsolver-viewquizexec.pdf}}
\caption[Schema Classe ViewQuizExec]{Schema Classe Quizzipedia:: Client:: ViewClient:: ViewQuizSolver:: ViewQuizExec}
\end{figure}
\paragraph{Relazioni con altre classi}
\subparagraph{Uscenti}
\begin{itemize}
\item usa \cls{Quizzipedia::Client::ViewClient::ViewQuizSolver::ViewQuestionSolver::ViewCompletionQ} per visualizzare correttamente le domande a completamento
\item usa \cls{Quizzipedia::Client::ViewClient::ViewQuizSolver::ViewQuestionSolver::ViewMatchingQ} per visualizzare correttamente le domande a collegamenti
\item usa \cls{Quizzipedia::Client::ViewClient::ViewQuizSolver::ViewQuestionSolver::ViewMultipleChoiceQ} per visualizzare correttamente le domande a scelta multipla
\item usa \cls{Quizzipedia::Client::ViewClient::ViewQuizSolver::ViewQuestionSolver::ViewShortAnswerQ} per visualizzare correttamente le domande aperte
\item usa \cls{Quizzipedia::Client::ViewClient::ViewQuizSolver::ViewQuestionSolver::ViewTrueFalseQ} per visualizzare correttamente le domande di tipo vero/falso
\item usa \cls{Quizzipedia::Client::ViewModelClient::CtrlServices::CtrlQuizExec} per ottenere i dati da presentare all'utente durante lo svolgimento di quiz e memorizzare le sue risposte
\end{itemize}
\subsection{\pkg{Quizzipedia:: Client:: ViewClient:: ViewQuizSolver:: ViewQuestionSolver}}
Il componente raccoglie le classi necessarie alla visualizzazione delle pagine da cui sarà possibile rispondere alle singole domande.
\begin{figure}[H]
\centering
\noindent\makebox[\textwidth]{\includegraphics[width=\textwidth]{Img/quizzipedia-client-viewclient-viewquizsolver-viewquestionsolver.pdf}}
\caption[Schema Componente ViewQuestionSolver]{Schema Componente Quizzipedia:: Client:: ViewClient:: ViewQuizSolver:: ViewQuestionSolver}
\end{figure}
\subsubsection{Interazioni con altre componenti}
\begin{itemize}
\item \bold{Uscenti}
\begin{itemize}
\item usa \pkg{Quizzipedia::Client::ViewModelClient::CtrlServices} per 
\end{itemize}
\end{itemize}
\subsubsection{Classe \cls{ViewCompletionQ}}
Presenta all'utente la pagina da cui sarà possibile rispondere a una domanda a completamento.
\begin{figure}[H]
\centering
\noindent\makebox[\textwidth]{\includegraphics[width=\textwidth]{Img/quizzipedia-client-viewclient-viewquizsolver-viewquestionsolver-viewcompletionq.pdf}}
\caption[Schema Classe ViewCompletionQ]{Schema Classe Quizzipedia:: Client:: ViewClient:: ViewQuizSolver:: ViewQuestionSolver:: ViewCompletionQ}
\end{figure}
\paragraph{Relazioni con altre classi}
\subparagraph{Entranti}
\begin{itemize}
\item usata da \cls{Quizzipedia::Client::ViewClient::ViewQuizSolver::ViewQuizExec} per visualizzare correttamente le domande a completamento
\end{itemize}
\subsubsection{Classe \cls{ViewMatchingQ}}
Presenta all'utente la pagina da cui sarà possibile rispondere a una domanda a collegamenti.
\begin{figure}[H]
\centering
\noindent\makebox[\textwidth]{\includegraphics[width=\textwidth]{Img/quizzipedia-client-viewclient-viewquizsolver-viewquestionsolver-viewmatchingq.pdf}}
\caption[Schema Classe ViewMatchingQ]{Schema Classe Quizzipedia:: Client:: ViewClient:: ViewQuizSolver:: ViewQuestionSolver:: ViewMatchingQ}
\end{figure}
\paragraph{Relazioni con altre classi}
\subparagraph{Entranti}
\begin{itemize}
\item usata da \cls{Quizzipedia::Client::ViewClient::ViewQuizSolver::ViewQuizExec} per visualizzare correttamente le domande a collegamenti
\end{itemize}
\subsubsection{Classe \cls{ViewMultipleChoiceQ}}
Presenta all'utente la pagina da cui sarà possibile rispondere a una domanda a risposta multipla.
\begin{figure}[H]
\centering
\noindent\makebox[\textwidth]{\includegraphics[width=\textwidth]{Img/quizzipedia-client-viewclient-viewquizsolver-viewquestionsolver-viewmultiplechoiceq.pdf}}
\caption[Schema Classe ViewMultipleChoiceQ]{Schema Classe Quizzipedia:: Client:: ViewClient:: ViewQuizSolver:: ViewQuestionSolver:: ViewMultipleChoiceQ}
\end{figure}
\paragraph{Relazioni con altre classi}
\subparagraph{Entranti}
\begin{itemize}
\item usata da \cls{Quizzipedia::Client::ViewClient::ViewQuizSolver::ViewQuizExec} per visualizzare correttamente le domande a scelta multipla
\end{itemize}
\subsubsection{Classe \cls{ViewShortAnswerQ}}
Presenta all'utente la pagina da cui sarà possibile rispondere a una domanda a risposta aperta.
\begin{figure}[H]
\centering
\noindent\makebox[\textwidth]{\includegraphics[width=\textwidth]{Img/quizzipedia-client-viewclient-viewquizsolver-viewquestionsolver-viewshortanswerq.pdf}}
\caption[Schema Classe ViewShortAnswerQ]{Schema Classe Quizzipedia:: Client:: ViewClient:: ViewQuizSolver:: ViewQuestionSolver:: ViewShortAnswerQ}
\end{figure}
\paragraph{Relazioni con altre classi}
\subparagraph{Entranti}
\begin{itemize}
\item usata da \cls{Quizzipedia::Client::ViewClient::ViewQuizSolver::ViewQuizExec} per visualizzare correttamente le domande aperte
\end{itemize}
\subsubsection{Classe \cls{ViewTrueFalseQ}}
Presenta all'utente la pagina da cui sarà possibile rispondere a una domanda di tipo vero/falso.
\begin{figure}[H]
\centering
\noindent\makebox[\textwidth]{\includegraphics[width=\textwidth]{Img/quizzipedia-client-viewclient-viewquizsolver-viewquestionsolver-viewtruefalseq.pdf}}
\caption[Schema Classe ViewTrueFalseQ]{Schema Classe Quizzipedia:: Client:: ViewClient:: ViewQuizSolver:: ViewQuestionSolver:: ViewTrueFalseQ}
\end{figure}
\paragraph{Relazioni con altre classi}
\subparagraph{Entranti}
\begin{itemize}
\item usata da \cls{Quizzipedia::Client::ViewClient::ViewQuizSolver::ViewQuizExec} per visualizzare correttamente le domande di tipo vero/falso
\end{itemize}
\subsection{\pkg{Quizzipedia:: Client:: ViewClient:: ViewRequests}}
Qui sono raccolte le pagine che permettono all'utente di inviare e gestire le richieste di ruolo e classe.
\begin{figure}[H]
\centering
\noindent\makebox[\textwidth]{\includegraphics[width=\textwidth]{Img/quizzipedia-client-viewclient-viewrequests.pdf}}
\caption[Schema Componente ViewRequests]{Schema Componente Quizzipedia:: Client:: ViewClient:: ViewRequests}
\end{figure}
\subsubsection{Interazioni con altre componenti}
\begin{itemize}
\item \bold{Uscenti}
\begin{itemize}
\item usa \pkg{Quizzipedia::Client::ViewModelClient::CtrlRequests} per aggiungi
\end{itemize}
\end{itemize}
\subsubsection{Classe \cls{SendRequests}}
È la pagina da cui l'utente potrà richiedere di entrare in una classe o che gli venga assegnato un ruolo in un ente.
\begin{figure}[H]
\centering
\noindent\makebox[\textwidth]{\includegraphics[width=\textwidth]{Img/quizzipedia-client-viewclient-viewrequests-sendrequests.pdf}}
\caption[Schema Classe SendRequests]{Schema Classe Quizzipedia:: Client:: ViewClient:: ViewRequests:: SendRequests}
\end{figure}
\paragraph{Relazioni con altre classi}
\subparagraph{Uscenti}
\begin{itemize}
\item usa \cls{Quizzipedia::Client::ViewModelClient::CtrlRequests::CtrlRequestClass} per aggiungi
\end{itemize}
\subsubsection{Classe \cls{ViewPendingRequests}}
Visualizza il pannello da cui sarà possibile gestire le richieste di inserimento in una classe e le richieste di assegnazione di ruolo pendenti.
\begin{figure}[H]
\centering
\noindent\makebox[\textwidth]{\includegraphics[width=\textwidth]{Img/quizzipedia-client-viewclient-viewrequests-viewpendingrequests.pdf}}
\caption[Schema Classe ViewPendingRequests]{Schema Classe Quizzipedia:: Client:: ViewClient:: ViewRequests:: ViewPendingRequests}
\end{figure}
\paragraph{Relazioni con altre classi}
\subparagraph{Uscenti}
\begin{itemize}
\item usa \cls{Quizzipedia::Client::ViewModelClient::CtrlRequests::CtrlRequestClass} per aggiungi
\end{itemize}
\subsection{\pkg{Quizzipedia:: Client:: ViewClient:: ViewSearch}}
Il componente contiene le classi responsabili della creazione delle pagine da cui sarà possibile ricercare domande, quiz, istituti e classi all'interno degli enti.
\begin{figure}[H]
\centering
\noindent\makebox[\textwidth]{\includegraphics[width=\textwidth]{Img/quizzipedia-client-viewclient-viewsearch.pdf}}
\caption[Schema Componente ViewSearch]{Schema Componente Quizzipedia:: Client:: ViewClient:: ViewSearch}
\end{figure}
\subsubsection{Interazioni con altre componenti}
\begin{itemize}
\item \bold{Uscenti}
\begin{itemize}
\item usa \pkg{Quizzipedia::Client::ViewModelClient::CtrlOrganization} per aggiungi
\item usa \pkg{Quizzipedia::Client::ViewModelClient::CtrlServices} per aggiungi
\end{itemize}
\end{itemize}
\subsubsection{Classe \cls{ViewSearchClass}}
La classe carica la pagina da cui sarà possibile ricercare enti e classi all'interno di un ente.
\begin{figure}[H]
\centering
\noindent\makebox[\textwidth]{\includegraphics[width=\textwidth]{Img/quizzipedia-client-viewclient-viewsearch-viewsearchclass.pdf}}
\caption[Schema Classe ViewSearchClass]{Schema Classe Quizzipedia:: Client:: ViewClient:: ViewSearch:: ViewSearchClass}
\end{figure}
\paragraph{Relazioni con altre classi}
\subparagraph{Uscenti}
\begin{itemize}
\item usa \cls{Quizzipedia::Client::ViewModelClient::CtrlOrganization::CtrlInstitution} per 
\end{itemize}
\subsubsection{Classe \cls{ViewSearchQuestions}}
Classe che ha il compito di caricare la pagina da cui sarà possibile effettuare la ricerca di domande.
\begin{figure}[H]
\centering
\noindent\makebox[\textwidth]{\includegraphics[width=\textwidth]{Img/quizzipedia-client-viewclient-viewsearch-viewsearchquestions.pdf}}
\caption[Schema Classe ViewSearchQuestions]{Schema Classe Quizzipedia:: Client:: ViewClient:: ViewSearch:: ViewSearchQuestions}
\end{figure}
\paragraph{Relazioni con altre classi}
\subparagraph{Uscenti}
\begin{itemize}
\item usa \cls{Quizzipedia::Client::ViewModelClient::CtrlServices::CtrlSearchQuestion} per scambiare dati col controller e col model sulla domanda cercata dall'utente
\end{itemize}
\subsubsection{Classe \cls{ViewSearchQuiz}}
Raccoglie i metodi necessari alla creazione della pagina da cui sarà possibile cercare un quiz.
\begin{figure}[H]
\centering
\noindent\makebox[\textwidth]{\includegraphics[width=\textwidth]{Img/quizzipedia-client-viewclient-viewsearch-viewsearchquiz.pdf}}
\caption[Schema Classe ViewSearchQuiz]{Schema Classe Quizzipedia:: Client:: ViewClient:: ViewSearch:: ViewSearchQuiz}
\end{figure}
\paragraph{Relazioni con altre classi}
\subparagraph{Uscenti}
\begin{itemize}
\item usa \cls{Quizzipedia::Client::ViewModelClient::CtrlServices::CtrlSearchQuiz} per poter passare i dati inseriti dal'utente perché gli venga restituito un quiz corretto
\end{itemize}
\subsection{\pkg{Quizzipedia:: Client:: ViewClient:: ViewStatistics}}
Componente che gestisce le pagine in cui verranno visualizzate le statistiche richieste dall'utente.
\begin{figure}[H]
\centering
\noindent\makebox[\textwidth]{\includegraphics[width=\textwidth]{Img/quizzipedia-client-viewclient-viewstatistics.pdf}}
\caption[Schema Componente ViewStatistics]{Schema Componente Quizzipedia:: Client:: ViewClient:: ViewStatistics}
\end{figure}
\subsubsection{Interazioni con altre componenti}
\begin{itemize}
\item \bold{Uscenti}
\begin{itemize}
\item usa \pkg{Quizzipedia::Client::ViewModelClient::CtrlStatistics} per aggiungi
\end{itemize}
\end{itemize}
\subsubsection{Classe \cls{ViewQuestionStats}}
Vengono rappresentate le statistiche generali relative alle domande.
\begin{figure}[H]
\centering
\noindent\makebox[\textwidth]{\includegraphics[width=\textwidth]{Img/quizzipedia-client-viewclient-viewstatistics-viewquestionstats.pdf}}
\caption[Schema Classe ViewQuestionStats]{Schema Classe Quizzipedia:: Client:: ViewClient:: ViewStatistics:: ViewQuestionStats}
\end{figure}
\paragraph{Relazioni con altre classi}
\subparagraph{Uscenti}
\begin{itemize}
\item usa \cls{Quizzipedia::Client::ViewModelClient::CtrlStatistics::CtrlQuestionStatistics} per presentare all'utente le statistiche generali sulle domande che ha richiesto
\end{itemize}
\subsubsection{Classe \cls{ViewQuizResults}}
Da qui è possibile vedere la lista di studenti che hanno risolto un quiz e il risultato ottenuto.
\begin{figure}[H]
\centering
\noindent\makebox[\textwidth]{\includegraphics[width=\textwidth]{Img/quizzipedia-client-viewclient-viewstatistics-viewquizresults.pdf}}
\caption[Schema Classe ViewQuizResults]{Schema Classe Quizzipedia:: Client:: ViewClient:: ViewStatistics:: ViewQuizResults}
\end{figure}
\paragraph{Relazioni con altre classi}
\subparagraph{Uscenti}
\begin{itemize}
\item usa \cls{Quizzipedia::Client::ViewModelClient::CtrlStatistics::CtrlStudentsQuizStats} per avere le informazioni necessarie alla visualizzazione dei risultati dei quiz
\end{itemize}
\subsubsection{Classe \cls{ViewQuizStats}}
Vengono rappresentate le statistiche generali riguardanti i quiz.
\begin{figure}[H]
\centering
\noindent\makebox[\textwidth]{\includegraphics[width=\textwidth]{Img/quizzipedia-client-viewclient-viewstatistics-viewquizstats.pdf}}
\caption[Schema Classe ViewQuizStats]{Schema Classe Quizzipedia:: Client:: ViewClient:: ViewStatistics:: ViewQuizStats}
\end{figure}
\paragraph{Relazioni con altre classi}
\subparagraph{Uscenti}
\begin{itemize}
\item usa \cls{Quizzipedia::Client::ViewModelClient::CtrlStatistics::CtrlQuizStatistics} per permettere all'utente di visualizzare le statistichegenerali di un quiz
\end{itemize}
\subsubsection{Classe \cls{ViewStudentsStats}}
Vengono rappresentate le statistiche relative a una singola classe in relazione a un particolare quiz.
\begin{figure}[H]
\centering
\noindent\makebox[\textwidth]{\includegraphics[width=\textwidth]{Img/quizzipedia-client-viewclient-viewstatistics-viewstudentsstats.pdf}}
\caption[Schema Classe ViewStudentsStats]{Schema Classe Quizzipedia:: Client:: ViewClient:: ViewStatistics:: ViewStudentsStats}
\end{figure}
\paragraph{Relazioni con altre classi}
\subparagraph{Uscenti}
\begin{itemize}
\item usa \cls{Quizzipedia::Client::ViewModelClient::CtrlStatistics::CtrlStudentsQuizStats} per avere i dati necessari a visualizzare le statistiche degli studenti relativamente a un quiz
\end{itemize}
\subsubsection{Classe \cls{ViewTeachersStats}}
Vengono rappresentate le statistiche riguardo i quiz e le domande create dai docenti.
\begin{figure}[H]
\centering
\noindent\makebox[\textwidth]{\includegraphics[width=\textwidth]{Img/quizzipedia-client-viewclient-viewstatistics-viewteachersstats.pdf}}
\caption[Schema Classe ViewTeachersStats]{Schema Classe Quizzipedia:: Client:: ViewClient:: ViewStatistics:: ViewTeachersStats}
\end{figure}
\paragraph{Relazioni con altre classi}
\subparagraph{Uscenti}
\begin{itemize}
\item usa \cls{Quizzipedia::Client::ViewModelClient::CtrlStatistics::CtrlTeachersStats} per poter ottenere i dati da fornire all'utente per la visualizzazione delle statistiche
\end{itemize}
\subsection{\pkg{Quizzipedia:: Client:: ViewClient:: ViewTopicManager}}
Qui sono raccolte le classi responsabili della presentazione delle pagine da cui sarà possibile gestire gli argomenti di domande e quiz.
\begin{figure}[H]
\centering
\noindent\makebox[\textwidth]{\includegraphics[width=\textwidth]{Img/quizzipedia-client-viewclient-viewtopicmanager.pdf}}
\caption[Schema Componente ViewTopicManager]{Schema Componente Quizzipedia:: Client:: ViewClient:: ViewTopicManager}
\end{figure}
\subsubsection{Interazioni con altre componenti}
\begin{itemize}
\item \bold{Uscenti}
\begin{itemize}
\item usa \pkg{Quizzipedia::Client::ViewModelClient::CtrlServices} per aggiungi
\end{itemize}
\end{itemize}
\subsubsection{Classe \cls{ViewTopicsManager}}
È la pagina da cui sarà possibile vedere la lista degli argomenti disponibili. Da qui sarà inoltre possibile creare nuovi argomenti o cancellare quelli già inseriti.
\begin{figure}[H]
\centering
\noindent\makebox[\textwidth]{\includegraphics[width=\textwidth]{Img/quizzipedia-client-viewclient-viewtopicmanager-viewtopicsmanager.pdf}}
\caption[Schema Classe ViewTopicsManager]{Schema Classe Quizzipedia:: Client:: ViewClient:: ViewTopicManager:: ViewTopicsManager}
\end{figure}
\paragraph{Relazioni con altre classi}
\subparagraph{Uscenti}
\begin{itemize}
\item usa \cls{Quizzipedia::Client::ViewModelClient::CtrlServices::CtrlTopics} per passare salvare i dati e le modifiche apprtati dall'utente agli argomenti
\end{itemize}
\subsection{\pkg{Quizzipedia:: Client:: ViewClient:: ViewUsers}}
Raccoglie le classi necessarie a presentare all'utente le pagine da cui visualizzare le informazioni che lo riguardano e compiere le funzioni principali.
\begin{figure}[H]
\centering
\noindent\makebox[\textwidth]{\includegraphics[width=\textwidth]{Img/quizzipedia-client-viewclient-viewusers.pdf}}
\caption[Schema Componente ViewUsers]{Schema Componente Quizzipedia:: Client:: ViewClient:: ViewUsers}
\end{figure}
\subsubsection{Interazioni con altre componenti}
\begin{itemize}
\item \bold{Uscenti}
\begin{itemize}
\item usa \pkg{Quizzipedia::Client::ViewModelClient::CtrlUsers} per aggiungi
\end{itemize}
\end{itemize}
\subsubsection{Classe \cls{ChangePw}}
Da qui, grazie ai metodi della classe, l'utente potrà modificare la propria password.
\begin{figure}[H]
\centering
\noindent\makebox[\textwidth]{\includegraphics[width=\textwidth]{Img/quizzipedia-client-viewclient-viewusers-changepw.pdf}}
\caption[Schema Classe ChangePw]{Schema Classe Quizzipedia:: Client:: ViewClient:: ViewUsers:: ChangePw}
\end{figure}
\paragraph{Relazioni con altre classi}
\subparagraph{Uscenti}
\begin{itemize}
\item usa \cls{Quizzipedia::Client::ViewModelClient::CtrlUsers::CtrlUserManager} per passare i dati inseriti dall'utente e necessari alla modifica della password (vecchia password, nuova password e conferma della nuova)
\end{itemize}
\subsubsection{Classe \cls{Login}}
Presenta la pagina necessaria per effettuare il login nel sistema.
\begin{figure}[H]
\centering
\noindent\makebox[\textwidth]{\includegraphics[width=\textwidth]{Img/quizzipedia-client-viewclient-viewusers-login.pdf}}
\caption[Schema Classe Login]{Schema Classe Quizzipedia:: Client:: ViewClient:: ViewUsers:: Login}
\end{figure}
\paragraph{Relazioni con altre classi}
\subparagraph{Uscenti}
\begin{itemize}
\item usa \cls{Quizzipedia::Client::ViewModelClient::CtrlUsers::CtrlLogin} per passare le informazioni immesse dall'utente durante il login al sistema perché vengano verificate
\end{itemize}
\subsubsection{Classe \cls{PrivateQuizList}}
pagina da cui l'utente potrà vedere la lista dei quiz privati e selezionare quali fare.
\begin{figure}[H]
\centering
\noindent\makebox[\textwidth]{\includegraphics[width=\textwidth]{Img/quizzipedia-client-viewclient-viewusers-privatequizlist.pdf}}
\caption[Schema Classe PrivateQuizList]{Schema Classe Quizzipedia:: Client:: ViewClient:: ViewUsers:: PrivateQuizList}
\end{figure}
\paragraph{Relazioni con altre classi}
\subparagraph{Uscenti}
\begin{itemize}
\item usa \cls{Quizzipedia::Client::ViewModelClient::CtrlServices::CtrlQuizSelection} per permettere all'utente di ottenere i quiz privati a cui ha accesso e di svolgerne
\end{itemize}
\subsubsection{Classe \cls{Profile}}
La classe presenta all'utente la pagina da cui prendere visione delle proprie informazioni personali.
\begin{figure}[H]
\centering
\noindent\makebox[\textwidth]{\includegraphics[width=\textwidth]{Img/quizzipedia-client-viewclient-viewusers-profile.pdf}}
\caption[Schema Classe Profile]{Schema Classe Quizzipedia:: Client:: ViewClient:: ViewUsers:: Profile}
\end{figure}
\paragraph{Relazioni con altre classi}
\subparagraph{Uscenti}
\begin{itemize}
\item usa \cls{Quizzipedia::Client::ViewModelClient::CtrlUsers::CtrlUserManager} per visualizzare le informazioni dell'utente e collegarlo alle funzioni principali
\end{itemize}
\subsubsection{Classe \cls{QuizHistory}}
pagina da cui l'utente potrà vedere lo storico dei quiz da lui svolti e i relativi risultati.
\begin{figure}[H]
\centering
\noindent\makebox[\textwidth]{\includegraphics[width=\textwidth]{Img/quizzipedia-client-viewclient-viewusers-quizhistory.pdf}}
\caption[Schema Classe QuizHistory]{Schema Classe Quizzipedia:: Client:: ViewClient:: ViewUsers:: QuizHistory}
\end{figure}
\subsubsection{Classe \cls{QuizHistorySingleResult}}
pagina da cui l'utente potrà vedere nel dettaglio le risposte da lui date in un quiz selezionato dallo storico.
\begin{figure}[H]
\centering
\noindent\makebox[\textwidth]{\includegraphics[width=\textwidth]{Img/quizzipedia-client-viewclient-viewusers-quizhistorysingleresult.pdf}}
\caption[Schema Classe QuizHistorySingleResult]{Schema Classe Quizzipedia:: Client:: ViewClient:: ViewUsers:: QuizHistorySingleResult}
\end{figure}
\subsubsection{Classe \cls{RecoveryPw}}
Da questa pagina sarà possibile inserire i dati per il recupero della password.
\begin{figure}[H]
\centering
\noindent\makebox[\textwidth]{\includegraphics[width=\textwidth]{Img/quizzipedia-client-viewclient-viewusers-recoverypw.pdf}}
\caption[Schema Classe RecoveryPw]{Schema Classe Quizzipedia:: Client:: ViewClient:: ViewUsers:: RecoveryPw}
\end{figure}
\paragraph{Relazioni con altre classi}
\subparagraph{Uscenti}
\begin{itemize}
\item usa \cls{Quizzipedia::Client::ViewModelClient::CtrlUsers::CtrlRecoveryPw} per passare al sistema le informazioni immesse dall'utente necessarie per il recupero della password
\end{itemize}
\subsubsection{Classe \cls{Registration}}
Presenta la pagina da cui effettuare la  registrazione al sistema.
\begin{figure}[H]
\centering
\noindent\makebox[\textwidth]{\includegraphics[width=\textwidth]{Img/quizzipedia-client-viewclient-viewusers-registration.pdf}}
\caption[Schema Classe Registration]{Schema Classe Quizzipedia:: Client:: ViewClient:: ViewUsers:: Registration}
\end{figure}
\paragraph{Relazioni con altre classi}
\subparagraph{Uscenti}
\begin{itemize}
\item usa \cls{Quizzipedia::Client::ViewModelClient::CtrlUsers::CtrlRegistration} per Raccogliere i dati inseriti dall'utente e passarli al controller
\end{itemize}
\subsubsection{Classe \cls{RemoveUser}}
Pagina da cui sarà possibile selezionare utenti da rimuovere dal sistema.
\begin{figure}[H]
\centering
\noindent\makebox[\textwidth]{\includegraphics[width=\textwidth]{Img/quizzipedia-client-viewclient-viewusers-removeuser.pdf}}
\caption[Schema Classe RemoveUser]{Schema Classe Quizzipedia:: Client:: ViewClient:: ViewUsers:: RemoveUser}
\end{figure}
\subsubsection{Classe \cls{SelectInst}}
pagina da cui l'utente, una volta autenticato, potrà selezionare l'ente a cui vuole accedere. Potrà scegliere tra quelli a cui è già iscritto, oppure scegliere di entrare in uno nuovo. Volendo, può anche proseguire come utente generico.
\begin{figure}[H]
\centering
\noindent\makebox[\textwidth]{\includegraphics[width=\textwidth]{Img/quizzipedia-client-viewclient-viewusers-selectinst.pdf}}
\caption[Schema Classe SelectInst]{Schema Classe Quizzipedia:: Client:: ViewClient:: ViewUsers:: SelectInst}
\end{figure}
\subsection{\pkg{Quizzipedia:: Client:: ViewModelClient}}
Raccoglie le classi responsabili della comunicazione tra il model e la view. Ha inoltre il compito di comunicare con il server per elaborare le richieste svolte dall'utente.
Con Angular.js il controller permette di tenere sempre aggiornato facilmente il model con le modifiche fatte nella view da parte dell'utente.
\begin{figure}[H]
\centering
\noindent\makebox[\textwidth]{\includegraphics[width=\textwidth]{Img/quizzipedia-client-viewmodelclient.pdf}}
\caption[Schema Componente ViewModelClient]{Schema Componente Quizzipedia:: Client:: ViewModelClient}
\end{figure}
\subsubsection{Componenti contenute}
\begin{itemize}
\item \pkg{Quizzipedia::Client::ViewModelClient::CtrlOrganization}
\item \pkg{Quizzipedia::Client::ViewModelClient::CtrlRequests}
\item \pkg{Quizzipedia::Client::ViewModelClient::CtrlServices}
\item \pkg{Quizzipedia::Client::ViewModelClient::CtrlStatistics}
\item \pkg{Quizzipedia::Client::ViewModelClient::CtrlUsers}
\end{itemize}
\subsubsection{Interazioni con altre componenti}
\begin{itemize}
\item \bold{Entranti}
\begin{itemize}
\item usata da \pkg{Quizzipedia::Client::ViewClient} per aggiungi
\end{itemize}
\item \bold{Uscenti}
\begin{itemize}
\item usa \pkg{Quizzipedia::Client::ModelClient} per avere accesso alla struttura degli oggetti che manipola
\end{itemize}
\end{itemize}
\subsection{\pkg{Quizzipedia:: Client:: ViewModelClient:: CtrlOrganization}}
Raccoglie le classi che si occupano delle comunicazioni necessarie per la creazione e la gestione di enti e classi.
\begin{figure}[H]
\centering
\noindent\makebox[\textwidth]{\includegraphics[width=\textwidth]{Img/quizzipedia-client-viewmodelclient-ctrlorganization.pdf}}
\caption[Schema Componente CtrlOrganization]{Schema Componente Quizzipedia:: Client:: ViewModelClient:: CtrlOrganization}
\end{figure}
\subsubsection{Interazioni con altre componenti}
\begin{itemize}
\item \bold{Entranti}
\begin{itemize}
\item usata da \pkg{Quizzipedia::Client::ViewClient::ViewOrgManager} per aggiungi
\item usata da \pkg{Quizzipedia::Client::ViewClient::ViewSearch} per aggiungi
\end{itemize}
\end{itemize}
\subsubsection{Classe \cls{CtrlInstitution}}
La classe in questione permette di modificare oppure eliminare le classi in un ente presente all'interno del sistema.
Sono presenti pertanto i metodi necessari a svolgere tali scopi e per eseguire il caricamento o il salvataggio di un ente e delle sue classi.
\begin{figure}[H]
\centering
\noindent\makebox[\textwidth]{\includegraphics[width=\textwidth]{Img/quizzipedia-client-viewmodelclient-ctrlorganization-ctrlinstitution.pdf}}
\caption[Schema Classe CtrlInstitution]{Schema Classe Quizzipedia:: Client:: ViewModelClient:: CtrlOrganization:: CtrlInstitution}
\end{figure}
\paragraph{Relazioni con altre classi}
\subparagraph{Entranti}
\begin{itemize}
\item usata da \cls{Quizzipedia::Client::ViewClient::ViewOrgManager::ViewModifyOrg} per passare al controller i dati immessi dall'utente durante le operazioni di modifica, creazione o rimozione di una classe di un istituto
\item usata da \cls{Quizzipedia::Client::ViewClient::ViewSearch::ViewSearchClass} per 
\end{itemize}
\subparagraph{Uscenti}
\begin{itemize}
\item usa \cls{Quizzipedia::Client::ModelClient::Organizations::Class} per avere
accesso alla struttura della classe e potere quindi svolgere correttamente operazioni su di
essa
\item usa \cls{Quizzipedia::Client::ModelClient::Organizations::Institution} per avere accesso alla struttura dell'istituto e caricare correttamente l'istituto corrente
\item usa \cls{Quizzipedia::Client::ViewModelClient::CtrlOrganization::CtrlPagination} per visualizzare in modo ordinato e dinamico la lista delle classi. Dalla lista è poi possibile effettuare le operazioni di rimozione e modifica
\item usa \cls{Quizzipedia::Client::ViewModelClient::CtrlOrganization::MyClass} per memorizzare nello \$scope le informazioni necessarie alla modifica, eliminazione o creazione di una classe
\end{itemize}
\subsubsection{Classe \cls{CtrlPagination}}
Questa classe permette di creare e gestire dinamicamente la paginazione quando viene visualizzata la lista delle classe presenti in un ente.
\begin{figure}[H]
\centering
\noindent\makebox[\textwidth]{\includegraphics[width=\textwidth]{Img/quizzipedia-client-viewmodelclient-ctrlorganization-ctrlpagination.pdf}}
\caption[Schema Classe CtrlPagination]{Schema Classe Quizzipedia:: Client:: ViewModelClient:: CtrlOrganization:: CtrlPagination}
\end{figure}
\paragraph{Relazioni con altre classi}
\subparagraph{Entranti}
\begin{itemize}
\item usata da \cls{Quizzipedia::Client::ViewModelClient::CtrlOrganization::CtrlInstitution} per visualizzare in modo ordinato e dinamico la lista delle classi. Dalla lista è poi possibile effettuare le operazioni di rimozione e modifica
\end{itemize}
\subsubsection{Classe \cls{MyClass}}
La sua istanziazione avviene nella variabile \$scope di Angular e permette di comunicare con la view. Questa classe permette di creare, eliminare, modificare oppure gestire una classe, relativa ad un ente, presente all'interno del sistema.
Sono presenti i metodi necessari all'inserimento e alla rimozione di uno studente o un docente da una classe, oltre ai metodi necessari alla creazione, eliminazione e modifica di una classe.
\begin{figure}[H]
\centering
\noindent\makebox[\textwidth]{\includegraphics[width=\textwidth]{Img/quizzipedia-client-viewmodelclient-ctrlorganization-myclass.pdf}}
\caption[Schema Classe MyClass]{Schema Classe Quizzipedia:: Client:: ViewModelClient:: CtrlOrganization:: MyClass}
\end{figure}
\paragraph{Relazioni con altre classi}
\subparagraph{Entranti}
\begin{itemize}
\item usata da \cls{Quizzipedia::Client::ViewModelClient::CtrlOrganization::CtrlInstitution} per memorizzare nello \$scope le informazioni necessarie alla modifica, eliminazione o creazione di una classe
\end{itemize}
\subsection{\pkg{Quizzipedia:: Client:: ViewModelClient:: CtrlRequests}}
Questo componente contiene classi necessarie alla gestione delle richieste di ruolo o classe fatte da un utente autenticato.
\begin{figure}[H]
\centering
\noindent\makebox[\textwidth]{\includegraphics[width=\textwidth]{Img/quizzipedia-client-viewmodelclient-ctrlrequests.pdf}}
\caption[Schema Componente CtrlRequests]{Schema Componente Quizzipedia:: Client:: ViewModelClient:: CtrlRequests}
\end{figure}
\subsubsection{Interazioni con altre componenti}
\begin{itemize}
\item \bold{Entranti}
\begin{itemize}
\item usata da \pkg{Quizzipedia::Client::ViewClient::ViewRequests} per aggiungi
\end{itemize}
\end{itemize}
\subsubsection{Classe \cls{CtrlRequestClass}}
Si occupa delle comunicazioni necessarie per la gestione delle richieste di inserimento in una classe.
\begin{figure}[H]
\centering
\noindent\makebox[\textwidth]{\includegraphics[width=\textwidth]{Img/quizzipedia-client-viewmodelclient-ctrlrequests-ctrlrequestclass.pdf}}
\caption[Schema Classe CtrlRequestClass]{Schema Classe Quizzipedia:: Client:: ViewModelClient:: CtrlRequests:: CtrlRequestClass}
\end{figure}
\paragraph{Relazioni con altre classi}
\subparagraph{Entranti}
\begin{itemize}
\item usata da \cls{Quizzipedia::Client::ViewClient::ViewRequests::SendRequests} per aggiungi
\item usata da \cls{Quizzipedia::Client::ViewClient::ViewRequests::ViewPendingRequests} per aggiungi
\end{itemize}
\subparagraph{Uscenti}
\begin{itemize}
\item usa \cls{Quizzipedia::Client::ModelClient::Requests::ClassList} per avere accesso alla lista di richieste e poterle gestire correttamente
\end{itemize}
\subsubsection{Classe \cls{CtrlRequestRole}}
Si occupa delle comunicazioni necessarie per la gestione delle richieste di ruolo.
\begin{figure}[H]
\centering
\noindent\makebox[\textwidth]{\includegraphics[width=\textwidth]{Img/quizzipedia-client-viewmodelclient-ctrlrequests-ctrlrequestrole.pdf}}
\caption[Schema Classe CtrlRequestRole]{Schema Classe Quizzipedia:: Client:: ViewModelClient:: CtrlRequests:: CtrlRequestRole}
\end{figure}
\paragraph{Relazioni con altre classi}
\subparagraph{Uscenti}
\begin{itemize}
\item usa \cls{Quizzipedia::Client::ModelClient::Requests::RoleList} per gestire le richieste di ruolo pendenti
\end{itemize}
\subsection{\pkg{Quizzipedia:: Client:: ViewModelClient:: CtrlServices}}
Raccoglie gli elementi necessari alla creazione, svolgimento e ricerca di quiz e domande.
\begin{figure}[H]
\centering
\noindent\makebox[\textwidth]{\includegraphics[width=\textwidth]{Img/quizzipedia-client-viewmodelclient-ctrlservices.pdf}}
\caption[Schema Componente CtrlServices]{Schema Componente Quizzipedia:: Client:: ViewModelClient:: CtrlServices}
\end{figure}
\subsubsection{Componenti contenute}
\begin{itemize}
\item \pkg{Quizzipedia::Client::ViewModelClient::CtrlServices::CtrlQuestions}
\end{itemize}
\subsubsection{Interazioni con altre componenti}
\begin{itemize}
\item \bold{Entranti}
\begin{itemize}
\item usata da \pkg{Quizzipedia::Client::ViewClient::ViewQuestionManager} per aggiungi
\item usata da \pkg{Quizzipedia::Client::ViewClient::ViewQuizManager} per aggiungi
\item usata da \pkg{Quizzipedia::Client::ViewClient::ViewQuizSolver} per aggiungi
\item usata da \pkg{Quizzipedia::Client::ViewClient::ViewQuizSolver::ViewQuestionSolver} per 
\item usata da \pkg{Quizzipedia::Client::ViewClient::ViewSearch} per aggiungi
\item usata da \pkg{Quizzipedia::Client::ViewClient::ViewTopicManager} per aggiungi
\end{itemize}
\item \bold{Uscenti}
\begin{itemize}
\item usa \pkg{Quizzipedia::Client::ModelClient::Services::Answers} per avere accesso alla struttura e poter ricavare le risposte degli utenti a quiz e domande
\item usa \pkg{Quizzipedia::Client::ModelClient::Services::Questions} per avere accesso alla corretta struttura di domande e quiz
\end{itemize}
\end{itemize}
\subsubsection{Classe \cls{CtrlQuiz}}
Raccoglie i metodi necessari alle comunicazioni tra view e model richieste per la creazione di quiz.
\begin{figure}[H]
\centering
\noindent\makebox[\textwidth]{\includegraphics[width=\textwidth]{Img/quizzipedia-client-viewmodelclient-ctrlservices-ctrlquiz.pdf}}
\caption[Schema Classe CtrlQuiz]{Schema Classe Quizzipedia:: Client:: ViewModelClient:: CtrlServices:: CtrlQuiz}
\end{figure}
\paragraph{Relazioni con altre classi}
\subparagraph{Entranti}
\begin{itemize}
\item usata da \cls{Quizzipedia::Client::ViewClient::ViewQuizManager::ViewCreateQuiz} per memorizzare le informazioni date dal docente durante la creazione del quiz
\item usata da \cls{Quizzipedia::Client::ViewClient::ViewQuizManager::ViewEditQuiz} per fare in modo che i cambiamenti che l'utente apporta al quiz siano consistenti
\item usata da \cls{Quizzipedia::Client::ViewClient::ViewQuizManager::ViewQuizMgmt} per aggiungi
\end{itemize}
\subparagraph{Uscenti}
\begin{itemize}
\item usa \cls{Quizzipedia::Client::ModelClient::Organizations::Class} per fornire al docente la lista delle classi a cui poter assegnare il quiz
\item usa \cls{Quizzipedia::Client::ModelClient::Services::Questions::GenericQuestion} per poter aggiungere domande all'interno del quiz
\item usa \cls{Quizzipedia::Client::ModelClient::Services::Quiz} per avere un riferimento alla struttura dei quiz durante la creazione
\item usa \cls{Quizzipedia::Client::ModelClient::Services::Topics} per fornire al docente una lista di possibili argomenti tra cui scegliere
\item usa \cls{Quizzipedia::Client::ModelClient::Users::Teacher} per riconoscere l'utente come docente e memorizzarlo come creatore del quiz
\item usa \cls{Quizzipedia::Client::ViewModelClient::CtrlServices::MyQuiz} per memorizzare nello \$scope le informazioni necessarie creazione di un nuovo quiz
\end{itemize}
\subsubsection{Classe \cls{CtrlQuizExec}}
Gestisce la procedura di risoluzione quiz, andando a caricare le singole domande e memorizzando le risposte date dall'utente.
\begin{figure}[H]
\centering
\noindent\makebox[\textwidth]{\includegraphics[width=\textwidth]{Img/quizzipedia-client-viewmodelclient-ctrlservices-ctrlquizexec.pdf}}
\caption[Schema Classe CtrlQuizExec]{Schema Classe Quizzipedia:: Client:: ViewModelClient:: CtrlServices:: CtrlQuizExec}
\end{figure}
\paragraph{Relazioni con altre classi}
\subparagraph{Entranti}
\begin{itemize}
\item usata da \cls{Quizzipedia::Client::ViewClient::ViewQuizSolver::ViewQuizExec} per ottenere i dati da presentare all'utente durante lo svolgimento di quiz e memorizzare le sue risposte
\end{itemize}
\subparagraph{Uscenti}
\begin{itemize}
\item usa \cls{Quizzipedia::Client::ModelClient::Services::Answers::AnswerQuestion} per memorizzare le risposte date dagli utenti alle singole domande
\item usa \cls{Quizzipedia::Client::ModelClient::Services::Answers::AnswerQuiz} per memorizzare la soluzione del quiz fornita dall'utente
\item usa \cls{Quizzipedia::Client::ModelClient::Services::Questions::GenericQuestion} per gestire le singole domande contenute nel quiz
\item usa \cls{Quizzipedia::Client::ViewModelClient::CtrlServices::CtrlQuizSelection} per avere il riferimento al quiz che l'utente ha deciso di svolgere
\end{itemize}
\subsubsection{Classe \cls{CtrlQuizManager}}
Fornisce al docente i metodi necessari per poter visualizzare, modificare e rimuovere quiz dalla lista di quiz da loro creati.
\begin{figure}[H]
\centering
\noindent\makebox[\textwidth]{\includegraphics[width=\textwidth]{Img/quizzipedia-client-viewmodelclient-ctrlservices-ctrlquizmanager.pdf}}
\caption[Schema Classe CtrlQuizManager]{Schema Classe Quizzipedia:: Client:: ViewModelClient:: CtrlServices:: CtrlQuizManager}
\end{figure}
\paragraph{Relazioni con altre classi}
\subparagraph{Entranti}
\begin{itemize}
\item usata da \cls{Quizzipedia::Client::ViewClient::ViewQuizManager::ViewQuizMgmt} per fornire al docente la lista dei propri quiz con possibilità di modifica e rimozione
\end{itemize}
\subparagraph{Uscenti}
\begin{itemize}
\item usa \cls{Quizzipedia::Client::ModelClient::Services::Quiz} per dare al docente lista di quiz di cui è proprietario con possibilità di modifica e rimozione
\item usa \cls{Quizzipedia::Client::ModelClient::Users::Teacher} per riconoscere il docente e fornirgli una lista dei quiz da lui creati
\end{itemize}
\subsubsection{Classe \cls{CtrlQuizSelection}}
Segue l'utente nella procedura di scelta del quiz che si desidera risolvere.
\begin{figure}[H]
\centering
\noindent\makebox[\textwidth]{\includegraphics[width=\textwidth]{Img/quizzipedia-client-viewmodelclient-ctrlservices-ctrlquizselection.pdf}}
\caption[Schema Classe CtrlQuizSelection]{Schema Classe Quizzipedia:: Client:: ViewModelClient:: CtrlServices:: CtrlQuizSelection}
\end{figure}
\paragraph{Relazioni con altre classi}
\subparagraph{Entranti}
\begin{itemize}
\item usata da \cls{Quizzipedia::Client::ViewClient::ViewUsers::PrivateQuizList} per permettere all'utente di ottenere i quiz privati a cui ha accesso e di svolgerne
\item usata da \cls{Quizzipedia::Client::ViewModelClient::CtrlServices::CtrlQuizExec} per avere il riferimento al quiz che l'utente ha deciso di svolgere
\end{itemize}
\subparagraph{Uscenti}
\begin{itemize}
\item usa \cls{Quizzipedia::Client::ModelClient::Organizations::Class} per ottenere le classi a cui l'utente è iscritto
\item usa \cls{Quizzipedia::Client::ModelClient::Services::Quiz} per avere un riferimento al quiz che l'utente intende risolvere
\item usa \cls{Quizzipedia::Client::ModelClient::Users::User} per tenere traccia dell'utente che sta svolgendo le operazioni
\end{itemize}
\subsubsection{Classe \cls{CtrlSearchQuestion}}
la classe fornisce i metodi necessari alla ricerca di domande nella fase di creazione del quiz.
\begin{figure}[H]
\centering
\noindent\makebox[\textwidth]{\includegraphics[width=\textwidth]{Img/quizzipedia-client-viewmodelclient-ctrlservices-ctrlsearchquestion.pdf}}
\caption[Schema Classe CtrlSearchQuestion]{Schema Classe Quizzipedia:: Client:: ViewModelClient:: CtrlServices:: CtrlSearchQuestion}
\end{figure}
\paragraph{Relazioni con altre classi}
\subparagraph{Entranti}
\begin{itemize}
\item usata da \cls{Quizzipedia::Client::ViewClient::ViewSearch::ViewSearchQuestions} per scambiare dati col controller e col model sulla domanda cercata dall'utente
\end{itemize}
\subparagraph{Uscenti}
\begin{itemize}
\item usa \cls{Quizzipedia::Client::ModelClient::Services::Questions::GenericQuestion} per comprendere e utilizzare correttamente la struttura della domanda
\item usa \cls{Quizzipedia::Client::ModelClient::Services::Topics} per fornire all'utente la lista di argomenti tra cui scegliere quello desiderato
\item usa \cls{Quizzipedia::Client::ModelClient::Users::Teacher} per identificare il docente che sta effettuando la ricerca
\item usa \cls{Quizzipedia::Client::ViewModelClient::CtrlServices::MySearchQuestion} per memorizzare nello \$scope le informazioni necessarie alla ricerca di domande
\end{itemize}
\subsubsection{Classe \cls{CtrlSearchQuiz}}
La classe contiene i metodi necessari alla comunicazione tra model e view nell'effettuare la ricerca di quiz.
\begin{figure}[H]
\centering
\noindent\makebox[\textwidth]{\includegraphics[width=\textwidth]{Img/quizzipedia-client-viewmodelclient-ctrlservices-ctrlsearchquiz.pdf}}
\caption[Schema Classe CtrlSearchQuiz]{Schema Classe Quizzipedia:: Client:: ViewModelClient:: CtrlServices:: CtrlSearchQuiz}
\end{figure}
\paragraph{Relazioni con altre classi}
\subparagraph{Entranti}
\begin{itemize}
\item usata da \cls{Quizzipedia::Client::ViewClient::ViewSearch::ViewSearchQuiz} per poter passare i dati inseriti dal'utente perché gli venga restituito un quiz corretto
\end{itemize}
\subparagraph{Uscenti}
\begin{itemize}
\item usa \cls{Quizzipedia::Client::ModelClient::Services::Quiz} per poter restituire all'utente un quiz che rispetti i criteri specificati
\item usa \cls{Quizzipedia::Client::ModelClient::Services::Topics} per fornire all'utente la lista degli argomenti disponibili tra cui scegliere
\item usa \cls{Quizzipedia::Client::ModelClient::Users::User} per identificare l'utente che sta effettuando la ricerca
\item usa \cls{Quizzipedia::Client::ViewModelClient::CtrlServices::MySearchQuiz} per istanziare un oggetto che raccolga le informazioni necessarie alla ricerca di quiz
\item usa \cls{Quizzipedia::Server::RoutingManager::SearchRouter} per aggiungi
\end{itemize}
\subsubsection{Classe \cls{CtrlTopics}}
Fornisce i metodi necessari alle comunicazioni richieste per la gestione degli argomenti.
\begin{figure}[H]
\centering
\noindent\makebox[\textwidth]{\includegraphics[width=\textwidth]{Img/quizzipedia-client-viewmodelclient-ctrlservices-ctrltopics.pdf}}
\caption[Schema Classe CtrlTopics]{Schema Classe Quizzipedia:: Client:: ViewModelClient:: CtrlServices:: CtrlTopics}
\end{figure}
\paragraph{Relazioni con altre classi}
\subparagraph{Entranti}
\begin{itemize}
\item usata da \cls{Quizzipedia::Client::ViewClient::ViewTopicManager::ViewTopicsManager} per passare salvare i dati e le modifiche apprtati dall'utente agli argomenti
\end{itemize}
\subparagraph{Uscenti}
\begin{itemize}
\item usa \cls{Quizzipedia::Client::ModelClient::Services::Topics} per gestire l'inserimento e la rimozione di argomenti
\end{itemize}
\subsubsection{Classe \cls{MyQuiz}}
La sua istanziazione avviene nella variabile \$scope di Angular e permette di comunicare con la view. Questa classe permette di raccogliere le informazioni necessarie alla creazione di un nuovo quiz.
\begin{figure}[H]
\centering
\noindent\makebox[\textwidth]{\includegraphics[width=\textwidth]{Img/quizzipedia-client-viewmodelclient-ctrlservices-myquiz.pdf}}
\caption[Schema Classe MyQuiz]{Schema Classe Quizzipedia:: Client:: ViewModelClient:: CtrlServices:: MyQuiz}
\end{figure}
\paragraph{Relazioni con altre classi}
\subparagraph{Entranti}
\begin{itemize}
\item usata da \cls{Quizzipedia::Client::ViewModelClient::CtrlServices::CtrlQuiz} per memorizzare nello \$scope le informazioni necessarie creazione di un nuovo quiz
\end{itemize}
\subsubsection{Classe \cls{MySearchQuestion}}
La sua istanziazione avviene nella variabile \$scope di Angular e permette di comunicare con la view. Questa classe raccoglie le informazioni immesse dall'utente durante la ricerca di domande.
\begin{figure}[H]
\centering
\noindent\makebox[\textwidth]{\includegraphics[width=\textwidth]{Img/quizzipedia-client-viewmodelclient-ctrlservices-mysearchquestion.pdf}}
\caption[Schema Classe MySearchQuestion]{Schema Classe Quizzipedia:: Client:: ViewModelClient:: CtrlServices:: MySearchQuestion}
\end{figure}
\paragraph{Relazioni con altre classi}
\subparagraph{Entranti}
\begin{itemize}
\item usata da \cls{Quizzipedia::Client::ViewModelClient::CtrlServices::CtrlSearchQuestion} per memorizzare nello \$scope le informazioni necessarie alla ricerca di domande
\end{itemize}
\subsubsection{Classe \cls{MySearchQuiz}}
La sua istanziazione avviene nella variabile \$scope di Angular e permette di comunicare con la view. Questa classe raccoglie le informazioni immesse dall'utente durante la ricerca di quiz.
\begin{figure}[H]
\centering
\noindent\makebox[\textwidth]{\includegraphics[width=\textwidth]{Img/quizzipedia-client-viewmodelclient-ctrlservices-mysearchquiz.pdf}}
\caption[Schema Classe MySearchQuiz]{Schema Classe Quizzipedia:: Client:: ViewModelClient:: CtrlServices:: MySearchQuiz}
\end{figure}
\paragraph{Relazioni con altre classi}
\subparagraph{Entranti}
\begin{itemize}
\item usata da \cls{Quizzipedia::Client::ViewModelClient::CtrlServices::CtrlSearchQuiz} per istanziare un oggetto che raccolga le informazioni necessarie alla ricerca di quiz
\end{itemize}
\subsection{\pkg{Quizzipedia:: Client:: ViewModelClient:: CtrlServices:: CtrlQuestions}}
Raccoglie le classi necessarie al collegamento tra view e model nelle fasi di gestione delle domande.
\begin{figure}[H]
\centering
\noindent\makebox[\textwidth]{\includegraphics[width=\textwidth]{Img/quizzipedia-client-viewmodelclient-ctrlservices-ctrlquestions.pdf}}
\caption[Schema Componente CtrlQuestions]{Schema Componente Quizzipedia:: Client:: ViewModelClient:: CtrlServices:: CtrlQuestions}
\end{figure}
\subsubsection{Classe \cls{CtrlQuestion}}
Si occupa di gestire la creazione delle domande da parte dell'utente, raccogliendo i dati dalla view.
\begin{figure}[H]
\centering
\noindent\makebox[\textwidth]{\includegraphics[width=\textwidth]{Img/quizzipedia-client-viewmodelclient-ctrlservices-ctrlquestions-ctrlquestion.pdf}}
\caption[Schema Classe CtrlQuestion]{Schema Classe Quizzipedia:: Client:: ViewModelClient:: CtrlServices:: CtrlQuestions:: CtrlQuestion}
\end{figure}
\paragraph{Relazioni con altre classi}
\subparagraph{Entranti}
\begin{itemize}
\item usata da \cls{Quizzipedia::Client::ViewClient::ViewQuestionManager::CreateGenericQ} per passare al controller i dati immessi dall'utente durante la creazione della domanda
\end{itemize}
\subparagraph{Uscenti}
\begin{itemize}
\item usa \cls{Quizzipedia::Client::ModelClient::Services::Questions::GenericQuestion} per avere la struttura della domanda durante la sua creazione
\item usa \cls{Quizzipedia::Client::ModelClient::Services::Topics} per fornire all'utente la lista di argomenti tra cui scegliere quello per la propria domanda
\item usa \cls{Quizzipedia::Client::ViewModelClient::CtrlServices::CtrlQuestions::MyGenericQ} per memorizzare nello \$scope le informazioni necessarie creazione di una nuova domanda
\end{itemize}
\subsubsection{Classe \cls{CtrlQuestionManager}}
Fornisce i metodi necessari per la comunicazione tra view e model. Permette al docente di visualizzare la lista delle proprie domande e ne consente la modifica e la rimozione.
\begin{figure}[H]
\centering
\noindent\makebox[\textwidth]{\includegraphics[width=\textwidth]{Img/quizzipedia-client-viewmodelclient-ctrlservices-ctrlquestions-ctrlquestionmanager.pdf}}
\caption[Schema Classe CtrlQuestionManager]{Schema Classe Quizzipedia:: Client:: ViewModelClient:: CtrlServices:: CtrlQuestions:: CtrlQuestionManager}
\end{figure}
\paragraph{Relazioni con altre classi}
\subparagraph{Entranti}
\begin{itemize}
\item usata da \cls{Quizzipedia::Client::ViewClient::ViewQuestionManager::QuestionMgmt} per visualizzare la lista delle domande del docente, permettendogli di modificarle o rimuoverle
\end{itemize}
\subparagraph{Uscenti}
\begin{itemize}
\item usa \cls{Quizzipedia::Client::ModelClient::Services::Questions::GenericQuestion} per poter modificare o cancellare le domande
\item usa \cls{Quizzipedia::Client::ModelClient::Users::Teacher} per identificare il docente in modo da fornirgli la lista con le sole domande da lui create
\end{itemize}
\subsubsection{Classe \cls{MyCompletionQ}}
permette di memorizzare la risposta data dall'utente in caso di domanda a completamento.
\begin{figure}[H]
\centering
\noindent\makebox[\textwidth]{\includegraphics[width=\textwidth]{Img/quizzipedia-client-viewmodelclient-ctrlservices-ctrlquestions-mycompletionq.pdf}}
\caption[Schema Classe MyCompletionQ]{Schema Classe Quizzipedia:: Client:: ViewModelClient:: CtrlServices:: CtrlQuestions:: MyCompletionQ}
\end{figure}
\paragraph{Relazioni con altre classi}
\subparagraph{Uscenti}
\begin{itemize}
\item usa \cls{Quizzipedia::Client::ViewModelClient::CtrlServices::CtrlQuestions::MyGenericQ} per aggiungere alle funzionalità condivise da tutte le domande quelle proprie della domanda a completamento
\end{itemize}
\subsubsection{Classe \cls{MyGenericQ}}
La sua istanziazione avviene nella variabile \$scope di Angular e permette di comunicare con la view. Questa classe permette di creare la parte comune a tutti i tipi di domanda.
\begin{figure}[H]
\centering
\noindent\makebox[\textwidth]{\includegraphics[width=\textwidth]{Img/quizzipedia-client-viewmodelclient-ctrlservices-ctrlquestions-mygenericq.pdf}}
\caption[Schema Classe MyGenericQ]{Schema Classe Quizzipedia:: Client:: ViewModelClient:: CtrlServices:: CtrlQuestions:: MyGenericQ}
\end{figure}
\paragraph{Relazioni con altre classi}
\subparagraph{Entranti}
\begin{itemize}
\item usata da \cls{Quizzipedia::Client::ViewModelClient::CtrlServices::CtrlQuestions::CtrlQuestion} per memorizzare nello \$scope le informazioni necessarie creazione di una nuova domanda
\item usata da \cls{Quizzipedia::Client::ViewModelClient::CtrlServices::CtrlQuestions::MyCompletionQ} per aggiungere alle funzionalità condivise da tutte le domande quelle proprie della domanda a completamento
\item usata da \cls{Quizzipedia::Client::ViewModelClient::CtrlServices::CtrlQuestions::MyMatchingQ} per aggiungere alle funzionalità condivise da tutte le domande quelle proprie della domanda a collegamenti
\item usata da \cls{Quizzipedia::Client::ViewModelClient::CtrlServices::CtrlQuestions::MyMutipleChoiceQ} per aggiungere alle funzionalità condivise da tutte le domande quelle proprie della domanda a scelta multipla
\item usata da \cls{Quizzipedia::Client::ViewModelClient::CtrlServices::CtrlQuestions::MyShortAnswerQ} per aggiungere alle funzionalità condivise da tutte le domande quelle proprie della domanda aperta
\item usata da \cls{Quizzipedia::Client::ViewModelClient::CtrlServices::CtrlQuestions::MyTrueFalseQ} per aggiungere alle funzionalità condivise da tutte le domande quelle proprie della domanda di tipo vero/falso
\end{itemize}
\subsubsection{Classe \cls{MyMatchingQ}}
permette di memorizzare la risposta data dall'utente in caso di domanda a collegamenti.
\begin{figure}[H]
\centering
\noindent\makebox[\textwidth]{\includegraphics[width=\textwidth]{Img/quizzipedia-client-viewmodelclient-ctrlservices-ctrlquestions-mymatchingq.pdf}}
\caption[Schema Classe MyMatchingQ]{Schema Classe Quizzipedia:: Client:: ViewModelClient:: CtrlServices:: CtrlQuestions:: MyMatchingQ}
\end{figure}
\paragraph{Relazioni con altre classi}
\subparagraph{Uscenti}
\begin{itemize}
\item usa \cls{Quizzipedia::Client::ViewModelClient::CtrlServices::CtrlQuestions::MyGenericQ} per aggiungere alle funzionalità condivise da tutte le domande quelle proprie della domanda a collegamenti
\end{itemize}
\subsubsection{Classe \cls{MyMutipleChoiceQ}}
permette di memorizzare la risposta data dall'utente in caso di domanda a scelta multipla.
\begin{figure}[H]
\centering
\noindent\makebox[\textwidth]{\includegraphics[width=\textwidth]{Img/quizzipedia-client-viewmodelclient-ctrlservices-ctrlquestions-mymutiplechoiceq.pdf}}
\caption[Schema Classe MyMutipleChoiceQ]{Schema Classe Quizzipedia:: Client:: ViewModelClient:: CtrlServices:: CtrlQuestions:: MyMutipleChoiceQ}
\end{figure}
\paragraph{Relazioni con altre classi}
\subparagraph{Uscenti}
\begin{itemize}
\item usa \cls{Quizzipedia::Client::ViewModelClient::CtrlServices::CtrlQuestions::MyGenericQ} per aggiungere alle funzionalità condivise da tutte le domande quelle proprie della domanda a scelta multipla
\end{itemize}
\subsubsection{Classe \cls{MyShortAnswerQ}}
permette di memorizzare la risposta data dall'utente in caso di domanda aperta.
\begin{figure}[H]
\centering
\noindent\makebox[\textwidth]{\includegraphics[width=\textwidth]{Img/quizzipedia-client-viewmodelclient-ctrlservices-ctrlquestions-myshortanswerq.pdf}}
\caption[Schema Classe MyShortAnswerQ]{Schema Classe Quizzipedia:: Client:: ViewModelClient:: CtrlServices:: CtrlQuestions:: MyShortAnswerQ}
\end{figure}
\paragraph{Relazioni con altre classi}
\subparagraph{Uscenti}
\begin{itemize}
\item usa \cls{Quizzipedia::Client::ViewModelClient::CtrlServices::CtrlQuestions::MyGenericQ} per aggiungere alle funzionalità condivise da tutte le domande quelle proprie della domanda aperta
\end{itemize}
\subsubsection{Classe \cls{MyTrueFalseQ}}
permette di memorizzare la risposta data dall'utente in caso di domanda di tipo vero/falso.
\begin{figure}[H]
\centering
\noindent\makebox[\textwidth]{\includegraphics[width=\textwidth]{Img/quizzipedia-client-viewmodelclient-ctrlservices-ctrlquestions-mytruefalseq.pdf}}
\caption[Schema Classe MyTrueFalseQ]{Schema Classe Quizzipedia:: Client:: ViewModelClient:: CtrlServices:: CtrlQuestions:: MyTrueFalseQ}
\end{figure}
\paragraph{Relazioni con altre classi}
\subparagraph{Uscenti}
\begin{itemize}
\item usa \cls{Quizzipedia::Client::ViewModelClient::CtrlServices::CtrlQuestions::MyGenericQ} per aggiungere alle funzionalità condivise da tutte le domande quelle proprie della domanda di tipo vero/falso
\end{itemize}
\subsection{\pkg{Quizzipedia:: Client:: ViewModelClient:: CtrlStatistics}}
Raccoglie le classi necessarie a recuperare le statistiche da presentare all'utente.
\begin{figure}[H]
\centering
\noindent\makebox[\textwidth]{\includegraphics[width=\textwidth]{Img/quizzipedia-client-viewmodelclient-ctrlstatistics.pdf}}
\caption[Schema Componente CtrlStatistics]{Schema Componente Quizzipedia:: Client:: ViewModelClient:: CtrlStatistics}
\end{figure}
\subsubsection{Interazioni con altre componenti}
\begin{itemize}
\item \bold{Entranti}
\begin{itemize}
\item usata da \pkg{Quizzipedia::Client::ViewClient::ViewStatistics} per aggiungi
\end{itemize}
\end{itemize}
\subsubsection{Classe \cls{CtrlQuestionStatistics}}
La classe serve a caricare, per renderle visibili all'utente, le statistiche generali relative a una domanda.
\begin{figure}[H]
\centering
\noindent\makebox[\textwidth]{\includegraphics[width=\textwidth]{Img/quizzipedia-client-viewmodelclient-ctrlstatistics-ctrlquestionstatistics.pdf}}
\caption[Schema Classe CtrlQuestionStatistics]{Schema Classe Quizzipedia:: Client:: ViewModelClient:: CtrlStatistics:: CtrlQuestionStatistics}
\end{figure}
\paragraph{Relazioni con altre classi}
\subparagraph{Entranti}
\begin{itemize}
\item usata da \cls{Quizzipedia::Client::ViewClient::ViewStatistics::ViewQuestionStats} per presentare all'utente le statistiche generali sulle domande che ha richiesto
\end{itemize}
\subparagraph{Uscenti}
\begin{itemize}
\item usa \cls{Quizzipedia::Client::ModelClient::Statistics::QuestionStatistics} per ottenere le statistiche sulle domande richieste dall'utente
\end{itemize}
\subsubsection{Classe \cls{CtrlQuizStatistics}}
La classe serve a caricare, per renderle visibili all'utente, le statistiche generali di un quiz.
\begin{figure}[H]
\centering
\noindent\makebox[\textwidth]{\includegraphics[width=\textwidth]{Img/quizzipedia-client-viewmodelclient-ctrlstatistics-ctrlquizstatistics.pdf}}
\caption[Schema Classe CtrlQuizStatistics]{Schema Classe Quizzipedia:: Client:: ViewModelClient:: CtrlStatistics:: CtrlQuizStatistics}
\end{figure}
\paragraph{Relazioni con altre classi}
\subparagraph{Entranti}
\begin{itemize}
\item usata da \cls{Quizzipedia::Client::ViewClient::ViewStatistics::ViewQuizStats} per permettere all'utente di visualizzare le statistichegenerali di un quiz
\end{itemize}
\subparagraph{Uscenti}
\begin{itemize}
\item usa \cls{Quizzipedia::Client::ModelClient::Statistics::QuizStatistics} per ottenere le statistiche generali richieste sui quiz richieste dall'utente
\end{itemize}
\subsubsection{Classe \cls{CtrlStudentsQuizStats}}
Si occupa di ottenere i dati necessari all'utente per visualizzare le statistiche degli studenti relativamente a un quiz.
\begin{figure}[H]
\centering
\noindent\makebox[\textwidth]{\includegraphics[width=\textwidth]{Img/quizzipedia-client-viewmodelclient-ctrlstatistics-ctrlstudentsquizstats.pdf}}
\caption[Schema Classe CtrlStudentsQuizStats]{Schema Classe Quizzipedia:: Client:: ViewModelClient:: CtrlStatistics:: CtrlStudentsQuizStats}
\end{figure}
\paragraph{Relazioni con altre classi}
\subparagraph{Entranti}
\begin{itemize}
\item usata da \cls{Quizzipedia::Client::ViewClient::ViewStatistics::ViewQuizResults} per avere le informazioni necessarie alla visualizzazione dei risultati dei quiz
\item usata da \cls{Quizzipedia::Client::ViewClient::ViewStatistics::ViewStudentsStats} per avere i dati necessari a visualizzare le statistiche degli studenti relativamente a un quiz
\end{itemize}
\subparagraph{Uscenti}
\begin{itemize}
\item usa \cls{Quizzipedia::Client::ModelClient::Organizations::Class} per fornire all'utente la lista di classi tra cui scegliere
\item usa \cls{Quizzipedia::Client::ModelClient::Services::Quiz} per selezionare il quiz di cui si vogliono avere le statistiche
\item usa \cls{Quizzipedia::Client::ModelClient::Statistics::StudentsStatisticsQuiz} per presentare le statistiche degli studenti relativamente a un quiz
\item usa \cls{Quizzipedia::Client::ViewModelClient::CtrlStatistics::MyStudentsStats} per memorizzare nello \$scope le informazioni necessarie alla selezione e alla richiesta di statistiche per un quiz
\end{itemize}
\subsubsection{Classe \cls{CtrlTeachersStats}}
Si occupa di fornire al docente le statistiche riguardanti le proprie domande e i propri quiz.
\begin{figure}[H]
\centering
\noindent\makebox[\textwidth]{\includegraphics[width=\textwidth]{Img/quizzipedia-client-viewmodelclient-ctrlstatistics-ctrlteachersstats.pdf}}
\caption[Schema Classe CtrlTeachersStats]{Schema Classe Quizzipedia:: Client:: ViewModelClient:: CtrlStatistics:: CtrlTeachersStats}
\end{figure}
\paragraph{Relazioni con altre classi}
\subparagraph{Entranti}
\begin{itemize}
\item usata da \cls{Quizzipedia::Client::ViewClient::ViewStatistics::ViewTeachersStats} per poter ottenere i dati da fornire all'utente per la visualizzazione delle statistiche
\end{itemize}
\subparagraph{Uscenti}
\begin{itemize}
\item usa \cls{Quizzipedia::Client::ModelClient::Services::Questions::GenericQuestion} per poter manipolare le domande create dal docente
\item usa \cls{Quizzipedia::Client::ModelClient::Users::Teacher} per identificare il docente di cui si vuole avere le statistiche
\end{itemize}
\subsubsection{Classe \cls{MyStudentsStats}}
La sua istanziazione avviene nella variabile \$scope di Angular e permette di comunicare con la view. Questa classe raccoglie le informazioni necessarie a visualizzare le statistiche degli studenti.
\begin{figure}[H]
\centering
\noindent\makebox[\textwidth]{\includegraphics[width=\textwidth]{Img/quizzipedia-client-viewmodelclient-ctrlstatistics-mystudentsstats.pdf}}
\caption[Schema Classe MyStudentsStats]{Schema Classe Quizzipedia:: Client:: ViewModelClient:: CtrlStatistics:: MyStudentsStats}
\end{figure}
\paragraph{Relazioni con altre classi}
\subparagraph{Entranti}
\begin{itemize}
\item usata da \cls{Quizzipedia::Client::ViewModelClient::CtrlStatistics::CtrlStudentsQuizStats} per memorizzare nello \$scope le informazioni necessarie alla selezione e alla richiesta di statistiche per un quiz
\end{itemize}
\subsection{\pkg{Quizzipedia:: Client:: ViewModelClient:: CtrlUsers}}
Il componente raccoglie le classi che permettono la comunicazione per quanto riguarda funzioni e dati dell'utente.
\begin{figure}[H]
\centering
\noindent\makebox[\textwidth]{\includegraphics[width=\textwidth]{Img/quizzipedia-client-viewmodelclient-ctrlusers.pdf}}
\caption[Schema Componente CtrlUsers]{Schema Componente Quizzipedia:: Client:: ViewModelClient:: CtrlUsers}
\end{figure}
\subsubsection{Interazioni con altre componenti}
\begin{itemize}
\item \bold{Entranti}
\begin{itemize}
\item usata da \pkg{Quizzipedia::Client::ViewClient::ViewUsers} per aggiungi
\end{itemize}
\item \bold{Uscenti}
\begin{itemize}
\item usa \pkg{Quizzipedia::Client::ModelClient::Users} per avere accesso alla struttura dei vari tipi di utente. In questo modo potrà compiere operazioni di modifica e manutenzione
\end{itemize}
\end{itemize}
\subsubsection{Classe \cls{CtrlHeader}}
La classe si occupa di caricare lo header corretto a seconda del tipo di utente che visita la pagina.
\begin{figure}[H]
\centering
\noindent\makebox[\textwidth]{\includegraphics[width=\textwidth]{Img/quizzipedia-client-viewmodelclient-ctrlusers-ctrlheader.pdf}}
\caption[Schema Classe CtrlHeader]{Schema Classe Quizzipedia:: Client:: ViewModelClient:: CtrlUsers:: CtrlHeader}
\end{figure}
\paragraph{Relazioni con altre classi}
\subparagraph{Entranti}
\begin{itemize}
\item usata da \cls{Quizzipedia::Client::ViewClient::Shared::Header::HeaderLogged} per ricevere dati su che tipologia di utente è collegata al sito e caricare, quindi, lo header corretto
\item usata da \cls{Quizzipedia::Client::ViewClient::Shared::Header::HeaderNotLogged} per ricevere informazioni sul tipo di utente connesso e caricare lo header correttamente
\end{itemize}
\subparagraph{Uscenti}
\begin{itemize}
\item usa \cls{Quizzipedia::Client::ModelClient::Users::User} per comprendere la tipologia di utente che visita la pagina e identificare quindi lo header corretto
\end{itemize}
\subsubsection{Classe \cls{CtrlLogin}}
Gestisce la comunicazione necessaria per l'autenticazione dell'utente in fase di login.
\begin{figure}[H]
\centering
\noindent\makebox[\textwidth]{\includegraphics[width=\textwidth]{Img/quizzipedia-client-viewmodelclient-ctrlusers-ctrllogin.pdf}}
\caption[Schema Classe CtrlLogin]{Schema Classe Quizzipedia:: Client:: ViewModelClient:: CtrlUsers:: CtrlLogin}
\end{figure}
\paragraph{Relazioni con altre classi}
\subparagraph{Entranti}
\begin{itemize}
\item usata da \cls{Quizzipedia::Client::ViewClient::ViewUsers::Login} per passare le informazioni immesse dall'utente durante il login al sistema perché vengano verificate
\end{itemize}
\subparagraph{Uscenti}
\begin{itemize}
\item usa \cls{Quizzipedia::Server::RoutingManager::AuthenticationRouter} per passare le credenziali di login al server perché vengano validate
\end{itemize}
\subsubsection{Classe \cls{CtrlRecoveryPw}}
La classe gestisce il recupero della password in caso di smarrimento.
\begin{figure}[H]
\centering
\noindent\makebox[\textwidth]{\includegraphics[width=\textwidth]{Img/quizzipedia-client-viewmodelclient-ctrlusers-ctrlrecoverypw.pdf}}
\caption[Schema Classe CtrlRecoveryPw]{Schema Classe Quizzipedia:: Client:: ViewModelClient:: CtrlUsers:: CtrlRecoveryPw}
\end{figure}
\paragraph{Relazioni con altre classi}
\subparagraph{Entranti}
\begin{itemize}
\item usata da \cls{Quizzipedia::Client::ViewClient::ViewUsers::RecoveryPw} per passare al sistema le informazioni immesse dall'utente necessarie per il recupero della password
\end{itemize}
\subsubsection{Classe \cls{CtrlRegistration}}
Si occupa di registrare un nuovo utente nel sistema. Per fare ciò utilizza l'oggetto MyUser, definito all'interno dello \$scope.
\begin{figure}[H]
\centering
\noindent\makebox[\textwidth]{\includegraphics[width=\textwidth]{Img/quizzipedia-client-viewmodelclient-ctrlusers-ctrlregistration.pdf}}
\caption[Schema Classe CtrlRegistration]{Schema Classe Quizzipedia:: Client:: ViewModelClient:: CtrlUsers:: CtrlRegistration}
\end{figure}
\paragraph{Relazioni con altre classi}
\subparagraph{Entranti}
\begin{itemize}
\item usata da \cls{Quizzipedia::Client::ViewClient::ViewUsers::Registration} per Raccogliere i dati inseriti dall'utente e passarli al controller
\end{itemize}
\subparagraph{Uscenti}
\begin{itemize}
\item usa \cls{Quizzipedia::Client::ViewModelClient::CtrlUsers::MyUser} per memorizzare, nello \$scope, le informazioni necessarie alla registrazione dell'utente
\item usa \cls{Quizzipedia::Server::RoutingManager::AuthenticationRouter} per passare al server le informazioni necessarie per la registrazione
\end{itemize}
\subsubsection{Classe \cls{CtrlUserManager}}
La classe, tramite variabili definite nello \$scope, permette agli utenti di gestire e visualizzare il proprio profilo personale.
\begin{figure}[H]
\centering
\noindent\makebox[\textwidth]{\includegraphics[width=\textwidth]{Img/quizzipedia-client-viewmodelclient-ctrlusers-ctrlusermanager.pdf}}
\caption[Schema Classe CtrlUserManager]{Schema Classe Quizzipedia:: Client:: ViewModelClient:: CtrlUsers:: CtrlUserManager}
\end{figure}
\paragraph{Relazioni con altre classi}
\subparagraph{Entranti}
\begin{itemize}
\item usata da \cls{Quizzipedia::Client::ViewClient::ViewUsers::ChangePw} per passare i dati inseriti dall'utente e necessari alla modifica della password (vecchia password, nuova password e conferma della nuova)
\item usata da \cls{Quizzipedia::Client::ViewClient::ViewUsers::Profile} per visualizzare le informazioni dell'utente e collegarlo alle funzioni principali
\end{itemize}
\subparagraph{Uscenti}
\begin{itemize}
\item usa \cls{Quizzipedia::Client::ModelClient::Users::User} per caricare un utente del tipo corretto. A seconda del tipo di utente ne verrà creato uno appropriato tra studente, docente, responsabile o utente senza ruolo
\end{itemize}
\subsubsection{Classe \cls{MyUser}}
La sua istanziazione avviene nella variabile \$scope di Angular e permette di comunicare con la view. Questa classe serve a raccogliere le informazioni dell'utente che si sta registrando.
\begin{figure}[H]
\centering
\noindent\makebox[\textwidth]{\includegraphics[width=\textwidth]{Img/quizzipedia-client-viewmodelclient-ctrlusers-myuser.pdf}}
\caption[Schema Classe MyUser]{Schema Classe Quizzipedia:: Client:: ViewModelClient:: CtrlUsers:: MyUser}
\end{figure}
\paragraph{Relazioni con altre classi}
\subparagraph{Entranti}
\begin{itemize}
\item usata da \cls{Quizzipedia::Client::ViewModelClient::CtrlUsers::CtrlRegistration} per memorizzare, nello \$scope, le informazioni necessarie alla registrazione dell'utente
\end{itemize}
\subsection{\pkg{Quizzipedia:: Server}}
Racchiude tutte le componenti necessarie per il back-end del prodotto. Comunica con il database non relazionale MongoDB per ottenere le informazioni richieste dal client a cui verranno inviate successivamente.
Contiene inoltre le componenti che si occupano del linguaggio di markup QML.
\begin{figure}[H]
\centering
\noindent\makebox[\textwidth]{\includegraphics[width=\textwidth]{Img/quizzipedia-server.pdf}}
\caption[Schema Componente Server]{Schema Componente Quizzipedia:: Server}
\end{figure}
\subsection{\pkg{Quizzipedia:: Server:: ModelServer}}
Rappresenta il modello dei dati che verranno utilizzati dal sistema lato server. 
Il controller del server avrà il compito di interagire col model per ottenere il modello che gli serve a seconda delle richieste fatte dal client.
\begin{figure}[H]
\centering
\noindent\makebox[\textwidth]{\includegraphics[width=\textwidth]{Img/quizzipedia-server-modelserver.pdf}}
\caption[Schema Componente ModelServer]{Schema Componente Quizzipedia:: Server:: ModelServer}
\end{figure}
\subsubsection{Componenti contenute}
\begin{itemize}
\item \pkg{Quizzipedia::Server::ModelServer::Organizations}
\item \pkg{Quizzipedia::Server::ModelServer::Requests}
\item \pkg{Quizzipedia::Server::ModelServer::Services}
\item \pkg{Quizzipedia::Server::ModelServer::Statistics}
\item \pkg{Quizzipedia::Server::ModelServer::Users}
\end{itemize}
\subsubsection{Interazioni con altre componenti}
\begin{itemize}
\item \bold{Entranti}
\begin{itemize}
\item usata da \pkg{Quizzipedia::Server::ControllerServer} per memorizzare i dati temporanei utili alla sessione corrente dell'utente nell'utilizzo del sistema Quizzipedia
\end{itemize}
\end{itemize}
\subsection{\pkg{Quizzipedia:: Server:: ModelServer:: Organizations}}
La componente gestisce, lato server, le classi e gli enti, ovvero il sistema in base a cui sono organizzati gli utenti nel sistema.
\begin{figure}[H]
\centering
\noindent\makebox[\textwidth]{\includegraphics[width=\textwidth]{Img/quizzipedia-server-modelserver-organizations.pdf}}
\caption[Schema Componente Organizations]{Schema Componente Quizzipedia:: Server:: ModelServer:: Organizations}
\end{figure}
\subsubsection{Interazioni con altre componenti}
\begin{itemize}
\item \bold{Entranti}
\begin{itemize}
\item usata da \pkg{Quizzipedia::Server::ControllerServer::ClassManager} per organizzare gli utenti all'interno di classi
\item usata da \pkg{Quizzipedia::Server::ControllerServer::InstitutionManager} per organizzare gli utenti all'interno di enti
\item usata da \pkg{Quizzipedia::Server::ModelServer::Users} per organizzare gli utenti all'interno di enti e classi
\end{itemize}
\end{itemize}
\subsubsection{Classe \cls{Class}}
Rappresenta una classe all'interno di un ente. Memorizza le informazioni che definiscono ogni classe,
informazioni che saranno utilizzate per la visualizzazione e per la gestione della classe stessa.
\begin{figure}[H]
\centering
\noindent\makebox[\textwidth]{\includegraphics[width=\textwidth]{Img/quizzipedia-server-modelserver-organizations-class.pdf}}
\caption[Schema Classe Class]{Schema Classe Quizzipedia:: Server:: ModelServer:: Organizations:: Class}
\end{figure}
\paragraph{Relazioni con altre classi}
\subparagraph{Entranti}
\begin{itemize}
\item usata da \cls{Quizzipedia::Server::ControllerServer::ClassManager::ClassAdder} per 
\item usata da \cls{Quizzipedia::Server::ControllerServer::ClassManager::ClassDeleter} per 
\item usata da \cls{Quizzipedia::Server::ControllerServer::ClassManager::ClassUpdater} per 
\item usata da \cls{Quizzipedia::Server::ControllerServer::ClassManager::FromClassRemover} per 
\item usata da \cls{Quizzipedia::Server::ControllerServer::ClassManager::InClassAdder} per 
\item usata da \cls{Quizzipedia::Server::ControllerServer::ClassManager::StudentsClassFetcher} per 
\item usata da \cls{Quizzipedia::Server::ControllerServer::ClassManager::TeachersClassFetcher} per 
\item usata da \cls{Quizzipedia::Server::ModelServer::Organizations::Institution} per memorizzare la lista
delle classi presenti in un ente
\item usata da \cls{Quizzipedia::Server::ModelServer::Requests::RequestClass} per identificare la classe
in cui l'utente vuole essere inserito
\item usata da \cls{Quizzipedia::Server::ModelServer::Services::Quiz} per tenere traccia delle classi per
cui il quiz è stato creato. Se un quiz è assegnato a delle classi specifiche, allora è privato e
accessibile dai soli studenti delle classi
\end{itemize}
\subsubsection{Classe \cls{Institution}}
Tale classe rappresenta un ente. Contiene le informazioni relative alla struttura dell'ente che saranno visualizzate dall'utente e gestiste dal controller, come ad esempio la lista delle classi presenti nell'ente, la lista degli studenti e degli insegnati all'interno dell'ente.
\begin{figure}[H]
\centering
\noindent\makebox[\textwidth]{\includegraphics[width=\textwidth]{Img/quizzipedia-server-modelserver-organizations-institution.pdf}}
\caption[Schema Classe Institution]{Schema Classe Quizzipedia:: Server:: ModelServer:: Organizations:: Institution}
\end{figure}
\paragraph{Relazioni con altre classi}
\subparagraph{Entranti}
\begin{itemize}
\item usata da \cls{Quizzipedia::Server::ControllerServer::InstitutionManager::InstitutionUpdater} per 
\end{itemize}
\subparagraph{Uscenti}
\begin{itemize}
\item usa \cls{Quizzipedia::Server::ModelServer::Organizations::Class} per memorizzare la lista
delle classi presenti in un ente
\item usa \cls{Quizzipedia::Server::ModelServer::Requests::ClassList} per gestire la lista delle richieste di accesso a una delle proprie classi
\item usa \cls{Quizzipedia::Server::ModelServer::Requests::RoleList} per gestire la lista delle richieste di assegnazione di ruolo degli utenti
\item usa \cls{Quizzipedia::Server::ModelServer::Services::Topics} per avere un elenco degli argomenti possibili all'interno dell'ente
\item usa \cls{Quizzipedia::Server::ModelServer::Users::Director} per identificare il responsabile dell'ente
\end{itemize}
\subsection{\pkg{Quizzipedia:: Server:: ModelServer:: Requests}}
Questo componente contiene le classi necessarie a gestire le richieste di ruolo e di classe degli utenti autenticati.
\begin{figure}[H]
\centering
\noindent\makebox[\textwidth]{\includegraphics[width=\textwidth]{Img/quizzipedia-server-modelserver-requests.pdf}}
\caption[Schema Componente Requests]{Schema Componente Quizzipedia:: Server:: ModelServer:: Requests}
\end{figure}
\subsubsection{Interazioni con altre componenti}
\begin{itemize}
\item \bold{Entranti}
\begin{itemize}
\item usata da \pkg{Quizzipedia::Server::ControllerServer::RequestsManager} per gestire tutte le richieste avanzate da parte degli utenti
\item usata da \pkg{Quizzipedia::Server::ModelServer::Users} per permettere agli utenti di effettuare richieste. Le richieste possono essere di ruolo o di classe
\end{itemize}
\end{itemize}
\subsubsection{Classe \cls{ClassList}}
Questa classe gestisce le richieste da parte di docenti o studenti per l'assegnazione a una specifica classe. Contiene i metodi da cui è possibile accettare o rifiutare la richiesta dell'utente.
\begin{figure}[H]
\centering
\noindent\makebox[\textwidth]{\includegraphics[width=\textwidth]{Img/quizzipedia-server-modelserver-requests-classlist.pdf}}
\caption[Schema Classe ClassList]{Schema Classe Quizzipedia:: Server:: ModelServer:: Requests:: ClassList}
\end{figure}
\paragraph{Relazioni con altre classi}
\subparagraph{Entranti}
\begin{itemize}
\item usata da \cls{Quizzipedia::Server::ControllerServer::RequestsManager::InsertClassRequestsAdder} per 
\item usata da \cls{Quizzipedia::Server::ControllerServer::RequestsManager::RequestsFetcher} per 
\item usata da \cls{Quizzipedia::Server::ModelServer::Organizations::Institution} per gestire la lista delle richieste di accesso a una delle proprie classi
\end{itemize}
\subparagraph{Uscenti}
\begin{itemize}
\item usa \cls{Quizzipedia::Server::ModelServer::Requests::RequestClass} per creare una lista in cui compaiano gli utenti e le classi in cui desiderano entrare
\end{itemize}
\subsubsection{Classe \cls{RequestClass}}
La classe memorizza l'utente che invia la richiesta di inserimento in una classe e la classe per cui la richiesta è stata effettuata.
\begin{figure}[H]
\centering
\noindent\makebox[\textwidth]{\includegraphics[width=\textwidth]{Img/quizzipedia-server-modelserver-requests-requestclass.pdf}}
\caption[Schema Classe RequestClass]{Schema Classe Quizzipedia:: Server:: ModelServer:: Requests:: RequestClass}
\end{figure}
\paragraph{Relazioni con altre classi}
\subparagraph{Entranti}
\begin{itemize}
\item usata da \cls{Quizzipedia::Server::ModelServer::Requests::ClassList} per creare una lista in cui compaiano gli utenti e le classi in cui desiderano entrare
\end{itemize}
\subparagraph{Uscenti}
\begin{itemize}
\item usa \cls{Quizzipedia::Server::ModelServer::Organizations::Class} per identificare la classe
in cui l'utente vuole essere inserito
\end{itemize}
\subsubsection{Classe \cls{RequestRole}}
La classe memorizza l'utente che invia una richiesta di ruolo e il ruolo che vuole ricoprire.
\begin{figure}[H]
\centering
\noindent\makebox[\textwidth]{\includegraphics[width=\textwidth]{Img/quizzipedia-server-modelserver-requests-requestrole.pdf}}
\caption[Schema Classe RequestRole]{Schema Classe Quizzipedia:: Server:: ModelServer:: Requests:: RequestRole}
\end{figure}
\paragraph{Relazioni con altre classi}
\subparagraph{Entranti}
\begin{itemize}
\item usata da \cls{Quizzipedia::Server::ModelServer::Requests::RoleList} per memorizzare una lista delle richieste di ruolo degli utenti di un istituto
\end{itemize}
\subsubsection{Classe \cls{RoleList}}
Gli utenti senza ruolo inviano le proprie richieste per l'assegnazione al ruolo di studente o docente al responsabile di un ente. Questa classe gestisce tali richieste.
\begin{figure}[H]
\centering
\noindent\makebox[\textwidth]{\includegraphics[width=\textwidth]{Img/quizzipedia-server-modelserver-requests-rolelist.pdf}}
\caption[Schema Classe RoleList]{Schema Classe Quizzipedia:: Server:: ModelServer:: Requests:: RoleList}
\end{figure}
\paragraph{Relazioni con altre classi}
\subparagraph{Entranti}
\begin{itemize}
\item usata da \cls{Quizzipedia::Server::ControllerServer::RequestsManager::RequestsFetcher} per 
\item usata da \cls{Quizzipedia::Server::ControllerServer::RequestsManager::RoleAccepter} per 
\item usata da \cls{Quizzipedia::Server::ControllerServer::RequestsManager::RoleRequestAdder} per 
\item usata da \cls{Quizzipedia::Server::ModelServer::Organizations::Institution} per gestire la lista delle richieste di assegnazione di ruolo degli utenti
\end{itemize}
\subparagraph{Uscenti}
\begin{itemize}
\item usa \cls{Quizzipedia::Server::ModelServer::Requests::RequestRole} per memorizzare una lista delle richieste di ruolo degli utenti di un istituto
\end{itemize}
\subsection{\pkg{Quizzipedia:: Server:: ModelServer:: Services}}
Il componente racchiude i modelli necessari alla creazione di domande e quiz, i servizi principali offerti dal nostro prodotto.
\begin{figure}[H]
\centering
\noindent\makebox[\textwidth]{\includegraphics[width=\textwidth]{Img/quizzipedia-server-modelserver-services.pdf}}
\caption[Schema Componente Services]{Schema Componente Quizzipedia:: Server:: ModelServer:: Services}
\end{figure}
\subsubsection{Componenti contenute}
\begin{itemize}
\item \pkg{Quizzipedia::Server::ModelServer::Services::Answers}
\item \pkg{Quizzipedia::Server::ModelServer::Services::Questions}
\end{itemize}
\subsubsection{Interazioni con altre componenti}
\begin{itemize}
\item \bold{Entranti}
\begin{itemize}
\item usata da \pkg{Quizzipedia::Server::ControllerServer::QuestionsManager} per gestire le domande
\item usata da \pkg{Quizzipedia::Server::ControllerServer::QuizManager} per gestire i quiz
\item usata da \pkg{Quizzipedia::Server::ControllerServer::SearchManager} per gestire le ricerche di quiz e domande da parte degli utenti
\item usata da \pkg{Quizzipedia::Server::ControllerServer::TopicManager} per gestire gli argomenti
\item usata da \pkg{Quizzipedia::Server::ModelServer::Users} per permettere agli utenti di tenere traccia dello storico dei quiz svolti
\end{itemize}
\end{itemize}
\subsubsection{Classe \cls{Quiz}}
Include la struttura del quiz.
\begin{figure}[H]
\centering
\noindent\makebox[\textwidth]{\includegraphics[width=\textwidth]{Img/quizzipedia-server-modelserver-services-quiz.pdf}}
\caption[Schema Classe Quiz]{Schema Classe Quizzipedia:: Server:: ModelServer:: Services:: Quiz}
\end{figure}
\paragraph{Relazioni con altre classi}
\subparagraph{Entranti}
\begin{itemize}
\item usata da \cls{Quizzipedia::Server::ControllerServer::ProfileManager::PersonalQuizFetcher} per 
\item usata da \cls{Quizzipedia::Server::ControllerServer::QuizManager::QuizCreator} per 
\item usata da \cls{Quizzipedia::Server::ControllerServer::QuizManager::QuizEraser} per 
\item usata da \cls{Quizzipedia::Server::ControllerServer::QuizManager::QuizFetcher} per 
\item usata da \cls{Quizzipedia::Server::ControllerServer::QuizManager::QuizUpdater} per 
\item usata da \cls{Quizzipedia::Server::ControllerServer::QuizManager::ResultsUpdater} per 
\item usata da \cls{Quizzipedia::Server::ControllerServer::SearchManager::QuizSearcher} per 
\item usata da \cls{Quizzipedia::Server::ModelServer::Services::Answers::AnswerQuiz} per individuare il quiz a cui associare le risposte date dall'utente
\item usata da \cls{Quizzipedia::Server::ModelServer::Users::Teacher} per tenere traccia dei quiz da lui creati
\end{itemize}
\subparagraph{Uscenti}
\begin{itemize}
\item usa \cls{Quizzipedia::Server::ModelServer::Organizations::Class} per tenere traccia delle classi per
cui il quiz è stato creato. Se un quiz è assegnato a delle classi specifiche, allora è privato e
accessibile dai soli studenti delle classi
\item usa \cls{Quizzipedia::Server::ModelServer::Services::Questions::GenericQuestion} per memorizzare la lista delle domande che compongono un quiz
\item usa \cls{Quizzipedia::Server::ModelServer::Statistics::QuizStatistics} per memorizzare le statistiche generali riguardo al quiz
\item usa \cls{Quizzipedia::Server::ModelServer::Statistics::StudentsStatisticsQuiz} per memorizzare le statistiche relative agli studenti che hanno svolto il quiz
\end{itemize}
\subsubsection{Classe \cls{Topics}}
Modella la struttura necessaria a memorizzare la lista di argomenti. A ogni domanda e a ogni quiz verranno poi associati i relativi argomenti.
\begin{figure}[H]
\centering
\noindent\makebox[\textwidth]{\includegraphics[width=\textwidth]{Img/quizzipedia-server-modelserver-services-topics.pdf}}
\caption[Schema Classe Topics]{Schema Classe Quizzipedia:: Server:: ModelServer:: Services:: Topics}
\end{figure}
\paragraph{Relazioni con altre classi}
\subparagraph{Entranti}
\begin{itemize}
\item usata da \cls{Quizzipedia::Server::ControllerServer::TopicManager::TopicCreator} per 
\item usata da \cls{Quizzipedia::Server::ControllerServer::TopicManager::TopicEraser} per 
\item usata da \cls{Quizzipedia::Server::ModelServer::Organizations::Institution} per avere un elenco degli argomenti possibili all'interno dell'ente
\end{itemize}
\subsection{\pkg{Quizzipedia:: Server:: ModelServer:: Services:: Answers}}
Raccoglie tutte le componenti necessarie alla gestione della verifica dei quiz e delle domande svolte.
\begin{figure}[H]
\centering
\noindent\makebox[\textwidth]{\includegraphics[width=\textwidth]{Img/quizzipedia-server-modelserver-services-answers.pdf}}
\caption[Schema Componente Answers]{Schema Componente Quizzipedia:: Server:: ModelServer:: Services:: Answers}
\end{figure}
\subsubsection{Interazioni con altre componenti}
\begin{itemize}
\item \bold{Uscenti}
\begin{itemize}
\item usa \pkg{Quizzipedia::Server::ModelServer::Services::Questions} per associare le risposte date dall'utente alla domanda giusta
\end{itemize}
\end{itemize}
\subsubsection{Classe \cls{AnswerColumn}}
Classe utilizzata da AnswerMatchingQ per salvare le risposte della domanda MatchingQ.
\begin{figure}[H]
\centering
\noindent\makebox[\textwidth]{\includegraphics[width=\textwidth]{Img/quizzipedia-server-modelserver-services-answers-answercolumn.pdf}}
\caption[Schema Classe AnswerColumn]{Schema Classe Quizzipedia:: Server:: ModelServer:: Services:: Answers:: AnswerColumn}
\end{figure}
\paragraph{Relazioni con altre classi}
\subparagraph{Entranti}
\begin{itemize}
\item usata da \cls{Quizzipedia::Server::ModelServer::Services::Answers::AnswerMatchingQ} per memorizzare le risposte date dall'utente a una domanda a collegamenti
\end{itemize}
\subsubsection{Classe \cls{AnswerCompletionQ}}
Classe che si occupa di memorizzare le informazioni di un domanda CompletionQ risolta.
\begin{figure}[H]
\centering
\noindent\makebox[\textwidth]{\includegraphics[width=\textwidth]{Img/quizzipedia-server-modelserver-services-answers-answercompletionq.pdf}}
\caption[Schema Classe AnswerCompletionQ]{Schema Classe Quizzipedia:: Server:: ModelServer:: Services:: Answers:: AnswerCompletionQ}
\end{figure}
\paragraph{Relazioni con altre classi}
\subparagraph{Entranti}
\begin{itemize}
\item usata da \cls{Quizzipedia::Server::ModelServer::Services::Questions::CompletionQ} per poter impostare la risposta data dall'utente
\end{itemize}
\subparagraph{Uscenti}
\begin{itemize}
\item usa \cls{Quizzipedia::Server::ModelServer::Services::Answers::AnswerQuestion} per concretizzarla, ereditandone così i campi dati e i metodi concreti. Inoltre, la classe concretizza il metodo check e memorizza correttamente la risposta data in caso di risposta a domanda a completamento
\end{itemize}
\subsubsection{Classe \cls{AnswerMatchingQ}}
Classe che si occupa di memorizzare le informazioni di un domanda MatchingQ risolta.
\begin{figure}[H]
\centering
\noindent\makebox[\textwidth]{\includegraphics[width=\textwidth]{Img/quizzipedia-server-modelserver-services-answers-answermatchingq.pdf}}
\caption[Schema Classe AnswerMatchingQ]{Schema Classe Quizzipedia:: Server:: ModelServer:: Services:: Answers:: AnswerMatchingQ}
\end{figure}
\paragraph{Relazioni con altre classi}
\subparagraph{Entranti}
\begin{itemize}
\item usata da \cls{Quizzipedia::Server::ModelServer::Services::Questions::MatchingQ} per poter impostare la risposta data dall'utente
\end{itemize}
\subparagraph{Uscenti}
\begin{itemize}
\item usa \cls{Quizzipedia::Server::ModelServer::Services::Answers::AnswerColumn} per memorizzare le risposte date dall'utente a una domanda a collegamenti
\item usa \cls{Quizzipedia::Server::ModelServer::Services::Answers::AnswerQuestion} per concretizzarla, ereditandone così i campi dati e i metodi concreti. Inoltre, la classe concretizza il metodo check e memorizza correttamente la risposta data in caso di risposta a domanda a collegamento
\end{itemize}
\subsubsection{Classe \cls{AnswerMultipleChoiceQ}}
Classe che si occupa di memorizzare le informazioni di un domanda MultipleChoiceQ risolta.
\begin{figure}[H]
\centering
\noindent\makebox[\textwidth]{\includegraphics[width=\textwidth]{Img/quizzipedia-server-modelserver-services-answers-answermultiplechoiceq.pdf}}
\caption[Schema Classe AnswerMultipleChoiceQ]{Schema Classe Quizzipedia:: Server:: ModelServer:: Services:: Answers:: AnswerMultipleChoiceQ}
\end{figure}
\paragraph{Relazioni con altre classi}
\subparagraph{Entranti}
\begin{itemize}
\item usata da \cls{Quizzipedia::Server::ModelServer::Services::Questions::MultipleChoiceQ} per poter impostare la risposta data dall'utente
\end{itemize}
\subparagraph{Uscenti}
\begin{itemize}
\item usa \cls{Quizzipedia::Server::ModelServer::Services::Answers::AnswerQuestion} per concretizzarla, ereditandone così i campi dati e i metodi concreti. Inoltre, la classe concretizza il metodo check e memorizza correttamente la risposta data in caso di risposta a domanda a scelta multipla
\end{itemize}
\subsubsection{Classe \cls{AnswerQuestion}}
Classe che si occupa di memorizzare le informazioni di un domanda risolta.
\begin{figure}[H]
\centering
\noindent\makebox[\textwidth]{\includegraphics[width=\textwidth]{Img/quizzipedia-server-modelserver-services-answers-answerquestion.pdf}}
\caption[Schema Classe AnswerQuestion]{Schema Classe Quizzipedia:: Server:: ModelServer:: Services:: Answers:: AnswerQuestion}
\end{figure}
\paragraph{Relazioni con altre classi}
\subparagraph{Entranti}
\begin{itemize}
\item usata da \cls{Quizzipedia::Server::ModelServer::Services::Answers::AnswerCompletionQ} per concretizzarla, ereditandone così i campi dati e i metodi concreti. Inoltre, la classe concretizza il metodo check e memorizza correttamente la risposta data in caso di risposta a domanda a completamento
\item usata da \cls{Quizzipedia::Server::ModelServer::Services::Answers::AnswerMatchingQ} per concretizzarla, ereditandone così i campi dati e i metodi concreti. Inoltre, la classe concretizza il metodo check e memorizza correttamente la risposta data in caso di risposta a domanda a collegamento
\item usata da \cls{Quizzipedia::Server::ModelServer::Services::Answers::AnswerMultipleChoiceQ} per concretizzarla, ereditandone così i campi dati e i metodi concreti. Inoltre, la classe concretizza il metodo check e memorizza correttamente la risposta data in caso di risposta a domanda a scelta multipla
\item usata da \cls{Quizzipedia::Server::ModelServer::Services::Answers::AnswerQuiz} per memorizzare e verificare le risposte date dagli utenti alle domande che compongono il quiz
\item usata da \cls{Quizzipedia::Server::ModelServer::Services::Answers::AnswerShortAnswerQ} per concretizzarla, ereditandone così i campi dati e i metodi concreti. Inoltre, la classe concretizza il metodo check e memorizza correttamente la risposta data in caso di risposta a domanda aperta
\item usata da \cls{Quizzipedia::Server::ModelServer::Services::Answers::AnswerTrueFalseQ} per concretizzarla, ereditandone così i campi dati e i metodi concreti. Inoltre, la classe concretizza il metodo check e memorizza correttamente la risposta data in caso di risposta a domanda di tipo vero/falso
\end{itemize}
\subparagraph{Uscenti}
\begin{itemize}
\item usa \cls{Quizzipedia::Server::ModelServer::Services::Questions::GenericQuestion} per tenere traccia della domanda a cui corrisponde la risposta in questione
\item usa \cls{Quizzipedia::Server::ModelServer::Users::User} per tenere traccia dell'utente che sta svolgendo la domanda
\end{itemize}
\subsubsection{Classe \cls{AnswerQuiz}}
Classe che si occupa di memorizzare le informazioni di un quiz completato e gestisce la verifica del suo superamento.
\begin{figure}[H]
\centering
\noindent\makebox[\textwidth]{\includegraphics[width=\textwidth]{Img/quizzipedia-server-modelserver-services-answers-answerquiz.pdf}}
\caption[Schema Classe AnswerQuiz]{Schema Classe Quizzipedia:: Server:: ModelServer:: Services:: Answers:: AnswerQuiz}
\end{figure}
\paragraph{Relazioni con altre classi}
\subparagraph{Entranti}
\begin{itemize}
\item usata da \cls{Quizzipedia::Server::ModelServer::Users::NoRole} per tenere traccia dei quiz che ha già svolto e del loro esito
\item usata da \cls{Quizzipedia::Server::ModelServer::Users::Student} per tenere traccia dei quiz che ha già svolto e del loro esito
\end{itemize}
\subparagraph{Uscenti}
\begin{itemize}
\item usa \cls{Quizzipedia::Server::ModelServer::Services::Answers::AnswerQuestion} per memorizzare e verificare le risposte date dagli utenti alle domande che compongono il quiz
\item usa \cls{Quizzipedia::Server::ModelServer::Services::Quiz} per individuare il quiz a cui associare le risposte date dall'utente
\end{itemize}
\subsubsection{Classe \cls{AnswerShortAnswerQ}}
Classe che si occupa di memorizzare le informazioni di un domanda ShortAnswerQ risolta.
\begin{figure}[H]
\centering
\noindent\makebox[\textwidth]{\includegraphics[width=\textwidth]{Img/quizzipedia-server-modelserver-services-answers-answershortanswerq.pdf}}
\caption[Schema Classe AnswerShortAnswerQ]{Schema Classe Quizzipedia:: Server:: ModelServer:: Services:: Answers:: AnswerShortAnswerQ}
\end{figure}
\paragraph{Relazioni con altre classi}
\subparagraph{Entranti}
\begin{itemize}
\item usata da \cls{Quizzipedia::Server::ModelServer::Services::Questions::ShortAnswerQ} per poter impostare la risposta data dall'utente
\end{itemize}
\subparagraph{Uscenti}
\begin{itemize}
\item usa \cls{Quizzipedia::Server::ModelServer::Services::Answers::AnswerQuestion} per concretizzarla, ereditandone così i campi dati e i metodi concreti. Inoltre, la classe concretizza il metodo check e memorizza correttamente la risposta data in caso di risposta a domanda aperta
\end{itemize}
\subsubsection{Classe \cls{AnswerTrueFalseQ}}
Classe che si occupa di memorizzare le informazioni di un domanda TrueFalseQ risolta.
\begin{figure}[H]
\centering
\noindent\makebox[\textwidth]{\includegraphics[width=\textwidth]{Img/quizzipedia-server-modelserver-services-answers-answertruefalseq.pdf}}
\caption[Schema Classe AnswerTrueFalseQ]{Schema Classe Quizzipedia:: Server:: ModelServer:: Services:: Answers:: AnswerTrueFalseQ}
\end{figure}
\paragraph{Relazioni con altre classi}
\subparagraph{Entranti}
\begin{itemize}
\item usata da \cls{Quizzipedia::Server::ModelServer::Services::Questions::TrueFalseQ} per poter impostare la risposta data dall'utente
\end{itemize}
\subparagraph{Uscenti}
\begin{itemize}
\item usa \cls{Quizzipedia::Server::ModelServer::Services::Answers::AnswerQuestion} per concretizzarla, ereditandone così i campi dati e i metodi concreti. Inoltre, la classe concretizza il metodo check e memorizza correttamente la risposta data in caso di risposta a domanda di tipo vero/falso
\end{itemize}
\subsection{\pkg{Quizzipedia:: Server:: ModelServer:: Services:: Questions}}
Descrive il modo in cui sono strutturati i vari tipi di domande che l'utente può incontrare durante la creazione o la compilazione di quiz.
\begin{figure}[H]
\centering
\noindent\makebox[\textwidth]{\includegraphics[width=\textwidth]{Img/quizzipedia-server-modelserver-services-questions.pdf}}
\caption[Schema Componente Questions]{Schema Componente Quizzipedia:: Server:: ModelServer:: Services:: Questions}
\end{figure}
\subsubsection{Interazioni con altre componenti}
\begin{itemize}
\item \bold{Entranti}
\begin{itemize}
\item usata da \pkg{Quizzipedia::Server::ControllerServer::QuestionsManager} per gestire le domande
\item usata da \pkg{Quizzipedia::Server::ModelServer::Services::Answers} per associare le risposte date dall'utente alla domanda giusta
\end{itemize}
\end{itemize}
\subsubsection{Classe \cls{Cell}}
La classe descrive ogni singola riga (quindi ogni opzione) della colonna della domanda a collegamento.
\begin{figure}[H]
\centering
\noindent\makebox[\textwidth]{\includegraphics[width=\textwidth]{Img/quizzipedia-server-modelserver-services-questions-cell.pdf}}
\caption[Schema Classe Cell]{Schema Classe Quizzipedia:: Server:: ModelServer:: Services:: Questions:: Cell}
\end{figure}
\paragraph{Relazioni con altre classi}
\subparagraph{Entranti}
\begin{itemize}
\item usata da \cls{Quizzipedia::Server::ModelServer::Services::Questions::Column} per implementare correttamente le celle delle colonne richieste per la domanda a collegamento
\end{itemize}
\subsubsection{Classe \cls{Column}}
La classe descrive le colonne della domanda a collegamenti.
\begin{figure}[H]
\centering
\noindent\makebox[\textwidth]{\includegraphics[width=\textwidth]{Img/quizzipedia-server-modelserver-services-questions-column.pdf}}
\caption[Schema Classe Column]{Schema Classe Quizzipedia:: Server:: ModelServer:: Services:: Questions:: Column}
\end{figure}
\paragraph{Relazioni con altre classi}
\subparagraph{Entranti}
\begin{itemize}
\item usata da \cls{Quizzipedia::Server::ModelServer::Services::Questions::MatchingQ} per implementare correttamente le colonne richieste dalla domanda a collegamenti
\end{itemize}
\subparagraph{Uscenti}
\begin{itemize}
\item usa \cls{Quizzipedia::Server::ModelServer::Services::Questions::Cell} per implementare correttamente le celle delle colonne richieste per la domanda a collegamento
\end{itemize}
\subsubsection{Classe \cls{CompletionQ}}
Descrive le domande a completamento. Il docente fornirà un testo incompleto e una lista di possibili completamenti; lo studente dovrà inserire le parole adeguate nella giusta posizione.
\begin{figure}[H]
\centering
\noindent\makebox[\textwidth]{\includegraphics[width=\textwidth]{Img/quizzipedia-server-modelserver-services-questions-completionq.pdf}}
\caption[Schema Classe CompletionQ]{Schema Classe Quizzipedia:: Server:: ModelServer:: Services:: Questions:: CompletionQ}
\end{figure}
\paragraph{Relazioni con altre classi}
\subparagraph{Uscenti}
\begin{itemize}
\item usa \cls{Quizzipedia::Server::ModelServer::Services::Answers::AnswerCompletionQ} per poter impostare la risposta data dall'utente
\item usa \cls{Quizzipedia::Server::ModelServer::Services::Questions::GenericQuestion} per per concretizzarla, ereditandone così i campi dati e i metodi concreti
\end{itemize}
\subsubsection{Classe \cls{GenericQuestion}}
Descrive le parti comuni a tutti i tipi di domanda presenti nel sistema.
\begin{figure}[H]
\centering
\noindent\makebox[\textwidth]{\includegraphics[width=\textwidth]{Img/quizzipedia-server-modelserver-services-questions-genericquestion.pdf}}
\caption[Schema Classe GenericQuestion]{Schema Classe Quizzipedia:: Server:: ModelServer:: Services:: Questions:: GenericQuestion}
\end{figure}
\paragraph{Relazioni con altre classi}
\subparagraph{Entranti}
\begin{itemize}
\item usata da \cls{Quizzipedia::Server::ControllerServer::QuestionsManager::QuestionCreator} per 
\item usata da \cls{Quizzipedia::Server::ControllerServer::QuestionsManager::QuestionEraser} per 
\item usata da \cls{Quizzipedia::Server::ControllerServer::QuestionsManager::QuestionUpdater} per 
\item usata da \cls{Quizzipedia::Server::ControllerServer::SearchManager::QuestionsSearcher} per 
\item usata da \cls{Quizzipedia::Server::ModelServer::Services::Answers::AnswerQuestion} per tenere traccia della domanda a cui corrisponde la risposta in questione
\item usata da \cls{Quizzipedia::Server::ModelServer::Services::Questions::CompletionQ} per per concretizzarla, ereditandone così i campi dati e i metodi concreti
\item usata da \cls{Quizzipedia::Server::ModelServer::Services::Questions::MatchingQ} per per concretizzarla, ereditandone così i campi dati e i metodi concreti
\item usata da \cls{Quizzipedia::Server::ModelServer::Services::Questions::MultipleChoiceQ} per per concretizzarla, ereditandone così i campi dati e i metodi concreti
\item usata da \cls{Quizzipedia::Server::ModelServer::Services::Questions::ShortAnswerQ} per per concretizzarla, ereditandone così i campi dati e i metodi concreti
\item usata da \cls{Quizzipedia::Server::ModelServer::Services::Questions::TrueFalseQ} per per concretizzarla, ereditandone così i campi dati e i metodi concreti
\item usata da \cls{Quizzipedia::Server::ModelServer::Services::Quiz} per memorizzare la lista delle domande che compongono un quiz
\item usata da \cls{Quizzipedia::Server::ModelServer::Users::Teacher} per 
\end{itemize}
\subparagraph{Uscenti}
\begin{itemize}
\item usa \cls{Quizzipedia::Server::ModelServer::Statistics::QuestionStatistics} per ottenere le proprie statistiche
\end{itemize}
\subsubsection{Classe \cls{MatchingQ}}
La struttura descrive le domande a collegamento. L'utente dovrà formare la risposta collegando le entrate da un numero variabile di colonne.
\begin{figure}[H]
\centering
\noindent\makebox[\textwidth]{\includegraphics[width=\textwidth]{Img/quizzipedia-server-modelserver-services-questions-matchingq.pdf}}
\caption[Schema Classe MatchingQ]{Schema Classe Quizzipedia:: Server:: ModelServer:: Services:: Questions:: MatchingQ}
\end{figure}
\paragraph{Relazioni con altre classi}
\subparagraph{Uscenti}
\begin{itemize}
\item usa \cls{Quizzipedia::Server::ModelServer::Services::Answers::AnswerMatchingQ} per poter impostare la risposta data dall'utente
\item usa \cls{Quizzipedia::Server::ModelServer::Services::Questions::Column} per implementare correttamente le colonne richieste dalla domanda a collegamenti
\item usa \cls{Quizzipedia::Server::ModelServer::Services::Questions::GenericQuestion} per per concretizzarla, ereditandone così i campi dati e i metodi concreti
\end{itemize}
\subsubsection{Classe \cls{MultipleChoiceQ}}
La struttura descrive le domande a scelta multipla; viene presentata una lista di opzioni tra cui scegliere quelle corrette.
\begin{figure}[H]
\centering
\noindent\makebox[\textwidth]{\includegraphics[width=\textwidth]{Img/quizzipedia-server-modelserver-services-questions-multiplechoiceq.pdf}}
\caption[Schema Classe MultipleChoiceQ]{Schema Classe Quizzipedia:: Server:: ModelServer:: Services:: Questions:: MultipleChoiceQ}
\end{figure}
\paragraph{Relazioni con altre classi}
\subparagraph{Uscenti}
\begin{itemize}
\item usa \cls{Quizzipedia::Server::ModelServer::Services::Answers::AnswerMultipleChoiceQ} per poter impostare la risposta data dall'utente
\item usa \cls{Quizzipedia::Server::ModelServer::Services::Questions::GenericQuestion} per per concretizzarla, ereditandone così i campi dati e i metodi concreti
\end{itemize}
\subsubsection{Classe \cls{ShortAnswerQ}}
La struttura descrive le domande aperte, ovvero quelle la cui risposta consiste in un termine o una frase specifici.
\begin{figure}[H]
\centering
\noindent\makebox[\textwidth]{\includegraphics[width=\textwidth]{Img/quizzipedia-server-modelserver-services-questions-shortanswerq.pdf}}
\caption[Schema Classe ShortAnswerQ]{Schema Classe Quizzipedia:: Server:: ModelServer:: Services:: Questions:: ShortAnswerQ}
\end{figure}
\paragraph{Relazioni con altre classi}
\subparagraph{Uscenti}
\begin{itemize}
\item usa \cls{Quizzipedia::Server::ModelServer::Services::Answers::AnswerShortAnswerQ} per poter impostare la risposta data dall'utente
\item usa \cls{Quizzipedia::Server::ModelServer::Services::Questions::GenericQuestion} per per concretizzarla, ereditandone così i campi dati e i metodi concreti
\end{itemize}
\subsubsection{Classe \cls{TrueFalseQ}}
Viene descritta la struttura delle domande che prevedono di decidere la veridicità di un'affermazione.
\begin{figure}[H]
\centering
\noindent\makebox[\textwidth]{\includegraphics[width=\textwidth]{Img/quizzipedia-server-modelserver-services-questions-truefalseq.pdf}}
\caption[Schema Classe TrueFalseQ]{Schema Classe Quizzipedia:: Server:: ModelServer:: Services:: Questions:: TrueFalseQ}
\end{figure}
\paragraph{Relazioni con altre classi}
\subparagraph{Uscenti}
\begin{itemize}
\item usa \cls{Quizzipedia::Server::ModelServer::Services::Answers::AnswerTrueFalseQ} per poter impostare la risposta data dall'utente
\item usa \cls{Quizzipedia::Server::ModelServer::Services::Questions::GenericQuestion} per per concretizzarla, ereditandone così i campi dati e i metodi concreti
\end{itemize}
\subsection{\pkg{Quizzipedia:: Server:: ModelServer:: Statistics}}
Qui sono raccolte le classi contenenti le informazioni sulle statistiche di domande, quiz e studenti di ogni classe.
\begin{figure}[H]
\centering
\noindent\makebox[\textwidth]{\includegraphics[width=\textwidth]{Img/quizzipedia-server-modelserver-statistics.pdf}}
\caption[Schema Componente Statistics]{Schema Componente Quizzipedia:: Server:: ModelServer:: Statistics}
\end{figure}
\subsubsection{Interazioni con altre componenti}
\begin{itemize}
\item \bold{Entranti}
\begin{itemize}
\item usata da \pkg{Quizzipedia::Server::ControllerServer::StatisticsManager} per mostrare e gestire le statistiche relative a quiz e domande
\item usata da \pkg{Quizzipedia::Server::ModelServer::Users} per per permettere agli utenti di visualizzare statistiche relative ai quiz
\end{itemize}
\end{itemize}
\subsubsection{Classe \cls{QuestionStatistics}}
La classe raccoglie le statistiche principali riguardanti una singola domanda. Da qui è poi possibile risalire alla domanda.
\begin{figure}[H]
\centering
\noindent\makebox[\textwidth]{\includegraphics[width=\textwidth]{Img/quizzipedia-server-modelserver-statistics-questionstatistics.pdf}}
\caption[Schema Classe QuestionStatistics]{Schema Classe Quizzipedia:: Server:: ModelServer:: Statistics:: QuestionStatistics}
\end{figure}
\paragraph{Relazioni con altre classi}
\subparagraph{Entranti}
\begin{itemize}
\item usata da \cls{Quizzipedia::Server::ControllerServer::StatisticsManager::QuestionStatisticsFetcher} per recuperare le statistiche che chiede l'utente riguardo una domanda
\item usata da \cls{Quizzipedia::Server::ModelServer::Services::Questions::GenericQuestion} per ottenere le proprie statistiche
\end{itemize}
\subsubsection{Classe \cls{QuizStatistics}}
La classe raccoglie le statistiche principali riguardanti un singolo quiz. Da qui è poi possibile ottenere il quiz in questione.
\begin{figure}[H]
\centering
\noindent\makebox[\textwidth]{\includegraphics[width=\textwidth]{Img/quizzipedia-server-modelserver-statistics-quizstatistics.pdf}}
\caption[Schema Classe QuizStatistics]{Schema Classe Quizzipedia:: Server:: ModelServer:: Statistics:: QuizStatistics}
\end{figure}
\paragraph{Relazioni con altre classi}
\subparagraph{Entranti}
\begin{itemize}
\item usata da \cls{Quizzipedia::Server::ControllerServer::StatisticsManager::QuizStatisticsFetcher} per 
\item usata da \cls{Quizzipedia::Server::ModelServer::Services::Quiz} per memorizzare le statistiche generali riguardo al quiz
\end{itemize}
\subsubsection{Classe \cls{StudentsStatisticsQuiz}}
Classe usata per visualizzare gli studenti che hanno svolto un quiz specifico e i loro voti per il quiz stesso.
\begin{figure}[H]
\centering
\noindent\makebox[\textwidth]{\includegraphics[width=\textwidth]{Img/quizzipedia-server-modelserver-statistics-studentsstatisticsquiz.pdf}}
\caption[Schema Classe StudentsStatisticsQuiz]{Schema Classe Quizzipedia:: Server:: ModelServer:: Statistics:: StudentsStatisticsQuiz}
\end{figure}
\paragraph{Relazioni con altre classi}
\subparagraph{Entranti}
\begin{itemize}
\item usata da \cls{Quizzipedia::Server::ModelServer::Services::Quiz} per memorizzare le statistiche relative agli studenti che hanno svolto il quiz
\end{itemize}
\subparagraph{Uscenti}
\begin{itemize}
\item usa \cls{Quizzipedia::Server::ModelServer::Users::Student} per tenere traccia degli utenti a cui le statistiche si riferiscono
\end{itemize}
\subsection{\pkg{Quizzipedia:: Server:: ModelServer:: Users}}
Raccoglie le classi necessarie a descrivere le diverse tipologie di utente presenti nel sistema.
\begin{figure}[H]
\centering
\noindent\makebox[\textwidth]{\includegraphics[width=\textwidth]{Img/quizzipedia-server-modelserver-users.pdf}}
\caption[Schema Componente Users]{Schema Componente Quizzipedia:: Server:: ModelServer:: Users}
\end{figure}
\subsubsection{Interazioni con altre componenti}
\begin{itemize}
\item \bold{Entranti}
\begin{itemize}
\item usata da \pkg{Quizzipedia::Server::ControllerServer::AuthenticationManager} per gestire gli utenti autenticati nel sistema
\item usata da \pkg{Quizzipedia::Server::ControllerServer::ProfileManager} per gestire il profilo degli utenti
\end{itemize}
\item \bold{Uscenti}
\begin{itemize}
\item usa \pkg{Quizzipedia::Server::ModelServer::Organizations} per organizzare gli utenti all'interno di enti e classi
\item usa \pkg{Quizzipedia::Server::ModelServer::Requests} per permettere agli utenti di effettuare richieste. Le richieste possono essere di ruolo o di classe
\item usa \pkg{Quizzipedia::Server::ModelServer::Services} per permettere agli utenti di tenere traccia dello storico dei quiz svolti
\item usa \pkg{Quizzipedia::Server::ModelServer::Statistics} per per permettere agli utenti di visualizzare statistiche relative ai quiz
\end{itemize}
\end{itemize}
\subsubsection{Classe \cls{AuthenticationData}}
Questa classe gestisce le informazioni di autenticazione comuni a tutti gli utenti.
\begin{figure}[H]
\centering
\noindent\makebox[\textwidth]{\includegraphics[width=\textwidth]{Img/quizzipedia-server-modelserver-users-authenticationdata.pdf}}
\caption[Schema Classe AuthenticationData]{Schema Classe Quizzipedia:: Server:: ModelServer:: Users:: AuthenticationData}
\end{figure}
\paragraph{Relazioni con altre classi}
\subparagraph{Entranti}
\begin{itemize}
\item usata da \cls{Quizzipedia::Server::ControllerServer::AuthenticationManager::LoggerIn} per 
\item usata da \cls{Quizzipedia::Server::ControllerServer::AuthenticationManager::LoggerOut} per 
\item usata da \cls{Quizzipedia::Server::ControllerServer::AuthenticationManager::PasswordRecover} per 
\item usata da \cls{Quizzipedia::Server::ControllerServer::AuthenticationManager::Register} per 
\item usata da \cls{Quizzipedia::Server::ModelServer::Users::User} per memorizzare le informazioni di autenticazione comuni a tutti gli utenti registrati
\end{itemize}
\subsubsection{Classe \cls{Director}}
Rappresenta un responsabile, ovvero colui che gestisce docenti e studenti per ogni ente del sistema.
\begin{figure}[H]
\centering
\noindent\makebox[\textwidth]{\includegraphics[width=\textwidth]{Img/quizzipedia-server-modelserver-users-director.pdf}}
\caption[Schema Classe Director]{Schema Classe Quizzipedia:: Server:: ModelServer:: Users:: Director}
\end{figure}
\paragraph{Relazioni con altre classi}
\subparagraph{Entranti}
\begin{itemize}
\item usata da \cls{Quizzipedia::Server::ModelServer::Organizations::Institution} per identificare il responsabile dell'ente
\end{itemize}
\subparagraph{Uscenti}
\begin{itemize}
\item usa \cls{Quizzipedia::Server::ModelServer::Users::User} per ereditarne i campi dato e i metodi comuni a tutti gli utenti
\end{itemize}
\subsubsection{Classe \cls{NoRole}}
Rappresenta gli utenti senza ruolo del sistema; coloro che si sono registrati e autenticati ma non hanno ancora fatto richiesta per l'assegnazione ad alcun ruolo.
\begin{figure}[H]
\centering
\noindent\makebox[\textwidth]{\includegraphics[width=\textwidth]{Img/quizzipedia-server-modelserver-users-norole.pdf}}
\caption[Schema Classe NoRole]{Schema Classe Quizzipedia:: Server:: ModelServer:: Users:: NoRole}
\end{figure}
\paragraph{Relazioni con altre classi}
\subparagraph{Uscenti}
\begin{itemize}
\item usa \cls{Quizzipedia::Server::ModelServer::Services::Answers::AnswerQuiz} per tenere traccia dei quiz che ha già svolto e del loro esito
\item usa \cls{Quizzipedia::Server::ModelServer::Users::User} per ereditarne i campi dato e i metodi comuni a tutti gli utenti
\end{itemize}
\subsubsection{Classe \cls{Student}}
Rappresenta uno studente del sistema e implementa le sue funzioni specifiche oltre a quelle ereditate da utente.
\begin{figure}[H]
\centering
\noindent\makebox[\textwidth]{\includegraphics[width=\textwidth]{Img/quizzipedia-server-modelserver-users-student.pdf}}
\caption[Schema Classe Student]{Schema Classe Quizzipedia:: Server:: ModelServer:: Users:: Student}
\end{figure}
\paragraph{Relazioni con altre classi}
\subparagraph{Entranti}
\begin{itemize}
\item usata da \cls{Quizzipedia::Server::ModelServer::Statistics::StudentsStatisticsQuiz} per tenere traccia degli utenti a cui le statistiche si riferiscono
\end{itemize}
\subparagraph{Uscenti}
\begin{itemize}
\item usa \cls{Quizzipedia::Server::ModelServer::Services::Answers::AnswerQuiz} per tenere traccia dei quiz che ha già svolto e del loro esito
\item usa \cls{Quizzipedia::Server::ModelServer::Users::User} per ereditarne i campi dato e i metodi comuni a tutti gli utenti
\end{itemize}
\subsubsection{Classe \cls{Teacher}}
Rappresenta un docente del sistema e ne implementa le funzionalità specifiche in aggiunta a quelle comuni a tutti gli utenti.
\begin{figure}[H]
\centering
\noindent\makebox[\textwidth]{\includegraphics[width=\textwidth]{Img/quizzipedia-server-modelserver-users-teacher.pdf}}
\caption[Schema Classe Teacher]{Schema Classe Quizzipedia:: Server:: ModelServer:: Users:: Teacher}
\end{figure}
\paragraph{Relazioni con altre classi}
\subparagraph{Uscenti}
\begin{itemize}
\item usa \cls{Quizzipedia::Server::ModelServer::Services::Questions::GenericQuestion} per 
\item usa \cls{Quizzipedia::Server::ModelServer::Services::Quiz} per tenere traccia dei quiz da lui creati
\item usa \cls{Quizzipedia::Server::ModelServer::Users::User} per ereditarne i campi dato e i metodi comuni a tutti gli utenti
\end{itemize}
\subsubsection{Classe \cls{User}}
Questa è una classe astratta e raccoglie le funzionalità comuni a tutti gli utenti.
\begin{figure}[H]
\centering
\noindent\makebox[\textwidth]{\includegraphics[width=\textwidth]{Img/quizzipedia-server-modelserver-users-user.pdf}}
\caption[Schema Classe User]{Schema Classe Quizzipedia:: Server:: ModelServer:: Users:: User}
\end{figure}
\paragraph{Relazioni con altre classi}
\subparagraph{Entranti}
\begin{itemize}
\item usata da \cls{Quizzipedia::Server::ControllerServer::ProfileManager::AccountDeleter} per 
\item usata da \cls{Quizzipedia::Server::ControllerServer::ProfileManager::PasswordSetter} per 
\item usata da \cls{Quizzipedia::Server::ControllerServer::ProfileManager::PersonalDataFetcher} per 
\item usata da \cls{Quizzipedia::Server::ModelServer::Services::Answers::AnswerQuestion} per tenere traccia dell'utente che sta svolgendo la domanda
\item usata da \cls{Quizzipedia::Server::ModelServer::Users::Director} per ereditarne i campi dato e i metodi comuni a tutti gli utenti
\item usata da \cls{Quizzipedia::Server::ModelServer::Users::NoRole} per ereditarne i campi dato e i metodi comuni a tutti gli utenti
\item usata da \cls{Quizzipedia::Server::ModelServer::Users::Student} per ereditarne i campi dato e i metodi comuni a tutti gli utenti
\item usata da \cls{Quizzipedia::Server::ModelServer::Users::Teacher} per ereditarne i campi dato e i metodi comuni a tutti gli utenti
\end{itemize}
\subparagraph{Uscenti}
\begin{itemize}
\item usa \cls{Quizzipedia::Server::ModelServer::Users::AuthenticationData} per memorizzare le informazioni di autenticazione comuni a tutti gli utenti registrati
\end{itemize}
\subsection{\pkg{Quizzipedia:: Server:: ControllerServer}}
Questo componente contiene tutti i servizi che permettono di isolare il più possibile l'accesso al database non relazionale. Avviene sempre un controllo dell'utente che genera una determinata richiesta al server affinchè sia abilitato per farla. 
Una volta ottenuta l'informazione richiesta dal database, lo stesso controller si occuperà di inviarla al client richiedente.
\begin{figure}[H]
\centering
\noindent\makebox[\textwidth]{\includegraphics[width=\textwidth]{Img/quizzipedia-server-controllerserver.pdf}}
\caption[Schema Componente ControllerServer]{Schema Componente Quizzipedia:: Server:: ControllerServer}
\end{figure}
\subsubsection{Componenti contenute}
\begin{itemize}
\item \pkg{Quizzipedia::Server::ControllerServer::AuthenticationManager}
\item \pkg{Quizzipedia::Server::ControllerServer::ClassManager}
\item \pkg{Quizzipedia::Server::ControllerServer::InstitutionManager}
\item \pkg{Quizzipedia::Server::ControllerServer::ProfileManager}
\item \pkg{Quizzipedia::Server::ControllerServer::QuestionsManager}
\item \pkg{Quizzipedia::Server::ControllerServer::QuizManager}
\item \pkg{Quizzipedia::Server::ControllerServer::RequestsManager}
\item \pkg{Quizzipedia::Server::ControllerServer::SearchManager}
\item \pkg{Quizzipedia::Server::ControllerServer::StatisticsManager}
\item \pkg{Quizzipedia::Server::ControllerServer::TopicManager}
\end{itemize}
\subsubsection{Interazioni con altre componenti}
\begin{itemize}
\item \bold{Entranti}
\begin{itemize}
\item usata da \pkg{Quizzipedia::Server::RoutingManager} per instradare e inoltrare le chiamate REST provenienti dal client ai moduli corretti in grado di soddisfarle
\end{itemize}
\item \bold{Uscenti}
\begin{itemize}
\item usa \pkg{Quizzipedia::Server::ModelServer} per memorizzare i dati temporanei utili alla sessione corrente dell'utente nell'utilizzo del sistema Quizzipedia
\end{itemize}
\end{itemize}
\subsubsection{Classe \cls{SessionController}}
Effettua il controllo sull'utente per verificare che egli sia in possesso dell'autorizzazione necessaria per compiere determinate richieste alla base di dati.
\begin{figure}[H]
\centering
\noindent\makebox[\textwidth]{\includegraphics[width=\textwidth]{Img/quizzipedia-server-controllerserver-sessioncontroller.pdf}}
\caption[Schema Classe SessionController]{Schema Classe Quizzipedia:: Server:: ControllerServer:: SessionController}
\end{figure}
\paragraph{Relazioni con altre classi}
\subparagraph{Entranti}
\begin{itemize}
\item usata da \cls{Quizzipedia::Server::ControllerServer::ClassManager::ClassAdder} per verificare che l'utente sia un Responsabile
\item usata da \cls{Quizzipedia::Server::ControllerServer::ClassManager::ClassDeleter} per verificare che l'utente sia un Responsabile
\item usata da \cls{Quizzipedia::Server::ControllerServer::ClassManager::ClassUpdater} per verificare che l'utente sia un Responsabile
\item usata da \cls{Quizzipedia::Server::ControllerServer::ClassManager::FromClassRemover} per verificare che l'utente sia almeno un Docente
\item usata da \cls{Quizzipedia::Server::ControllerServer::ClassManager::InClassAdder} per verificare che l'utente sia almeno un Docente
\item usata da \cls{Quizzipedia::Server::ControllerServer::ClassManager::StudentsClassFetcher} per verificare che l'utente sia un Docente
\item usata da \cls{Quizzipedia::Server::ControllerServer::ClassManager::TeachersClassFetcher} per verificare che l'utente sia un Responsabile
\item usata da \cls{Quizzipedia::Server::ControllerServer::InstitutionManager::InstitutionUpdater} per verificare che l'utente sia un Responsabile
\item usata da \cls{Quizzipedia::Server::ControllerServer::QuestionsManager::QuestionCreator} per verificare che l'utente sia un Docente
\item usata da \cls{Quizzipedia::Server::ControllerServer::QuestionsManager::QuestionEraser} per verificare che l'utente sia un Docente
\item usata da \cls{Quizzipedia::Server::ControllerServer::QuestionsManager::QuestionUpdater} per verificare che l'utente sia un Docente
\item usata da \cls{Quizzipedia::Server::ControllerServer::QuizManager::QuizCreator} per verificare che l'utente sia un Docente
\item usata da \cls{Quizzipedia::Server::ControllerServer::QuizManager::QuizEraser} per verificare che l'utente sia un Docente
\item usata da \cls{Quizzipedia::Server::ControllerServer::QuizManager::QuizUpdater} per verificare che l'utente sia un Docente
\item usata da \cls{Quizzipedia::Server::ControllerServer::RequestsManager::RequestsFetcher} per verificare che l'utente sia un Responsabile
\item usata da \cls{Quizzipedia::Server::ControllerServer::RequestsManager::RoleAccepter} per verificare che l'utente sia un Responsabile
\item usata da \cls{Quizzipedia::Server::ControllerServer::StatisticsManager::QuestionStatisticsFetcher} per verificare che l'utente sia un Docente
\item usata da \cls{Quizzipedia::Server::ControllerServer::StatisticsManager::QuizStatisticsFetcher} per verificare che l'utente sia un Docente
\item usata da \cls{Quizzipedia::Server::ControllerServer::StatisticsManager::StudentStatisticsFetcher} per verificare che l'utente sia un Docente
\item usata da \cls{Quizzipedia::Server::ControllerServer::TopicManager::TopicCreator} per verificare che l'utente sia un Docente
\item usata da \cls{Quizzipedia::Server::ControllerServer::TopicManager::TopicEraser} per verificare che l'utente sia un Docente
\end{itemize}
\subsection{\pkg{Quizzipedia:: Server:: ControllerServer:: AuthenticationManager}}
Componente che permette di gestire le funzioni base per una corretta autenticazione al sistema.
\begin{figure}[H]
\centering
\noindent\makebox[\textwidth]{\includegraphics[width=\textwidth]{Img/quizzipedia-server-controllerserver-authenticationmanager.pdf}}
\caption[Schema Componente AuthenticationManager]{Schema Componente Quizzipedia:: Server:: ControllerServer:: AuthenticationManager}
\end{figure}
\subsubsection{Interazioni con altre componenti}
\begin{itemize}
\item \bold{Uscenti}
\begin{itemize}
\item usa \pkg{Quizzipedia::Server::ModelServer::Users} per gestire gli utenti autenticati nel sistema
\end{itemize}
\end{itemize}
\subsubsection{Classe \cls{LoggerIn}}
Permette l'autenticazione nel sistema da parte di utenti preventivamente registrati.
\begin{figure}[H]
\centering
\noindent\makebox[\textwidth]{\includegraphics[width=\textwidth]{Img/quizzipedia-server-controllerserver-authenticationmanager-loggerin.pdf}}
\caption[Schema Classe LoggerIn]{Schema Classe Quizzipedia:: Server:: ControllerServer:: AuthenticationManager:: LoggerIn}
\end{figure}
\paragraph{Relazioni con altre classi}
\subparagraph{Entranti}
\begin{itemize}
\item usata da \cls{Quizzipedia::Server::RoutingManager::AuthenticationRouter} per aggiungi
\end{itemize}
\subparagraph{Uscenti}
\begin{itemize}
\item usa \cls{Quizzipedia::Server::ModelServer::Users::AuthenticationData} per 
\end{itemize}
\subsubsection{Classe \cls{LoggerOut}}
Permette l'uscita dal sistema ad utenti autenticati.
\begin{figure}[H]
\centering
\noindent\makebox[\textwidth]{\includegraphics[width=\textwidth]{Img/quizzipedia-server-controllerserver-authenticationmanager-loggerout.pdf}}
\caption[Schema Classe LoggerOut]{Schema Classe Quizzipedia:: Server:: ControllerServer:: AuthenticationManager:: LoggerOut}
\end{figure}
\paragraph{Relazioni con altre classi}
\subparagraph{Entranti}
\begin{itemize}
\item usata da \cls{Quizzipedia::Server::RoutingManager::AuthenticationRouter} per aggiungi
\end{itemize}
\subparagraph{Uscenti}
\begin{itemize}
\item usa \cls{Quizzipedia::Server::ModelServer::Users::AuthenticationData} per 
\end{itemize}
\subsubsection{Classe \cls{PasswordRecover}}
Permette il recupero della password da parte di un utente in caso di smarrimento o dimenticanza.
\begin{figure}[H]
\centering
\noindent\makebox[\textwidth]{\includegraphics[width=\textwidth]{Img/quizzipedia-server-controllerserver-authenticationmanager-passwordrecover.pdf}}
\caption[Schema Classe PasswordRecover]{Schema Classe Quizzipedia:: Server:: ControllerServer:: AuthenticationManager:: PasswordRecover}
\end{figure}
\paragraph{Relazioni con altre classi}
\subparagraph{Entranti}
\begin{itemize}
\item usata da \cls{Quizzipedia::Server::RoutingManager::AuthenticationRouter} per aggiungi
\end{itemize}
\subparagraph{Uscenti}
\begin{itemize}
\item usa \cls{Quizzipedia::Server::ModelServer::Users::AuthenticationData} per 
\end{itemize}
\subsubsection{Classe \cls{Register}}
Permette la registrazione di un utente nel sistema.
\begin{figure}[H]
\centering
\noindent\makebox[\textwidth]{\includegraphics[width=\textwidth]{Img/quizzipedia-server-controllerserver-authenticationmanager-register.pdf}}
\caption[Schema Classe Register]{Schema Classe Quizzipedia:: Server:: ControllerServer:: AuthenticationManager:: Register}
\end{figure}
\paragraph{Relazioni con altre classi}
\subparagraph{Entranti}
\begin{itemize}
\item usata da \cls{Quizzipedia::Server::RoutingManager::AuthenticationRouter} per aggiungi
\end{itemize}
\subparagraph{Uscenti}
\begin{itemize}
\item usa \cls{Quizzipedia::Server::ModelServer::Users::AuthenticationData} per 
\end{itemize}
\subsection{\pkg{Quizzipedia:: Server:: ControllerServer:: ClassManager}}
Componente che racchiude tutte le funzionalità adibite al salvataggio e alla visualizzazione delle informazioni riguardanti le classi di un ente.
\begin{figure}[H]
\centering
\noindent\makebox[\textwidth]{\includegraphics[width=\textwidth]{Img/quizzipedia-server-controllerserver-classmanager.pdf}}
\caption[Schema Componente ClassManager]{Schema Componente Quizzipedia:: Server:: ControllerServer:: ClassManager}
\end{figure}
\subsubsection{Interazioni con altre componenti}
\begin{itemize}
\item \bold{Uscenti}
\begin{itemize}
\item usa \pkg{Quizzipedia::Server::ModelServer::Organizations} per organizzare gli utenti all'interno di classi
\end{itemize}
\end{itemize}
\subsubsection{Classe \cls{ClassAdder}}
Permette la creazione di una nuova classe.
\begin{figure}[H]
\centering
\noindent\makebox[\textwidth]{\includegraphics[width=\textwidth]{Img/quizzipedia-server-controllerserver-classmanager-classadder.pdf}}
\caption[Schema Classe ClassAdder]{Schema Classe Quizzipedia:: Server:: ControllerServer:: ClassManager:: ClassAdder}
\end{figure}
\paragraph{Relazioni con altre classi}
\subparagraph{Entranti}
\begin{itemize}
\item usata da \cls{Quizzipedia::Server::RoutingManager::ClassRouter} per aggiungi
\end{itemize}
\subparagraph{Uscenti}
\begin{itemize}
\item usa \cls{Quizzipedia::Server::ControllerServer::SessionController} per verificare che l'utente sia un Responsabile
\item usa \cls{Quizzipedia::Server::ModelServer::Organizations::Class} per 
\end{itemize}
\subsubsection{Classe \cls{ClassDeleter}}
Permette la rimozione delle classi dal sistema.
\begin{figure}[H]
\centering
\noindent\makebox[\textwidth]{\includegraphics[width=\textwidth]{Img/quizzipedia-server-controllerserver-classmanager-classdeleter.pdf}}
\caption[Schema Classe ClassDeleter]{Schema Classe Quizzipedia:: Server:: ControllerServer:: ClassManager:: ClassDeleter}
\end{figure}
\paragraph{Relazioni con altre classi}
\subparagraph{Entranti}
\begin{itemize}
\item usata da \cls{Quizzipedia::Server::RoutingManager::ClassRouter} per aggiungi
\end{itemize}
\subparagraph{Uscenti}
\begin{itemize}
\item usa \cls{Quizzipedia::Server::ControllerServer::SessionController} per verificare che l'utente sia un Responsabile
\item usa \cls{Quizzipedia::Server::ModelServer::Organizations::Class} per 
\end{itemize}
\subsubsection{Classe \cls{ClassUpdater}}
Permette la modifica delle informazioni di base di una determinata classe.
\begin{figure}[H]
\centering
\noindent\makebox[\textwidth]{\includegraphics[width=\textwidth]{Img/quizzipedia-server-controllerserver-classmanager-classupdater.pdf}}
\caption[Schema Classe ClassUpdater]{Schema Classe Quizzipedia:: Server:: ControllerServer:: ClassManager:: ClassUpdater}
\end{figure}
\paragraph{Relazioni con altre classi}
\subparagraph{Entranti}
\begin{itemize}
\item usata da \cls{Quizzipedia::Server::RoutingManager::ClassRouter} per aggiungi
\end{itemize}
\subparagraph{Uscenti}
\begin{itemize}
\item usa \cls{Quizzipedia::Server::ControllerServer::SessionController} per verificare che l'utente sia un Responsabile
\item usa \cls{Quizzipedia::Server::ModelServer::Organizations::Class} per 
\end{itemize}
\subsubsection{Classe \cls{FromClassRemover}}
Permette la rimozione di un utente da una determinata classe.
\begin{figure}[H]
\centering
\noindent\makebox[\textwidth]{\includegraphics[width=\textwidth]{Img/quizzipedia-server-controllerserver-classmanager-fromclassremover.pdf}}
\caption[Schema Classe FromClassRemover]{Schema Classe Quizzipedia:: Server:: ControllerServer:: ClassManager:: FromClassRemover}
\end{figure}
\paragraph{Relazioni con altre classi}
\subparagraph{Entranti}
\begin{itemize}
\item usata da \cls{Quizzipedia::Server::RoutingManager::ClassRouter} per aggiungi
\end{itemize}
\subparagraph{Uscenti}
\begin{itemize}
\item usa \cls{Quizzipedia::Server::ControllerServer::SessionController} per verificare che l'utente sia almeno un Docente
\item usa \cls{Quizzipedia::Server::ModelServer::Organizations::Class} per 
\end{itemize}
\subsubsection{Classe \cls{InClassAdder}}
Permette l'inserimento di un utente in una specifica classe.
\begin{figure}[H]
\centering
\noindent\makebox[\textwidth]{\includegraphics[width=\textwidth]{Img/quizzipedia-server-controllerserver-classmanager-inclassadder.pdf}}
\caption[Schema Classe InClassAdder]{Schema Classe Quizzipedia:: Server:: ControllerServer:: ClassManager:: InClassAdder}
\end{figure}
\paragraph{Relazioni con altre classi}
\subparagraph{Entranti}
\begin{itemize}
\item usata da \cls{Quizzipedia::Server::RoutingManager::ClassRouter} per aggiungi
\end{itemize}
\subparagraph{Uscenti}
\begin{itemize}
\item usa \cls{Quizzipedia::Server::ControllerServer::SessionController} per verificare che l'utente sia almeno un Docente
\item usa \cls{Quizzipedia::Server::ModelServer::Organizations::Class} per 
\end{itemize}
\subsubsection{Classe \cls{StudentsClassFetcher}}
Recupera la lista degli studenti appartenenti ad una determinata classe.
\begin{figure}[H]
\centering
\noindent\makebox[\textwidth]{\includegraphics[width=\textwidth]{Img/quizzipedia-server-controllerserver-classmanager-studentsclassfetcher.pdf}}
\caption[Schema Classe StudentsClassFetcher]{Schema Classe Quizzipedia:: Server:: ControllerServer:: ClassManager:: StudentsClassFetcher}
\end{figure}
\paragraph{Relazioni con altre classi}
\subparagraph{Entranti}
\begin{itemize}
\item usata da \cls{Quizzipedia::Server::RoutingManager::ClassRouter} per aggiungi
\end{itemize}
\subparagraph{Uscenti}
\begin{itemize}
\item usa \cls{Quizzipedia::Server::ControllerServer::SessionController} per verificare che l'utente sia un Docente
\item usa \cls{Quizzipedia::Server::ModelServer::Organizations::Class} per 
\end{itemize}
\subsubsection{Classe \cls{TeachersClassFetcher}}
Recupera la lista degli insegnanti relativi ad una determinata classe.
\begin{figure}[H]
\centering
\noindent\makebox[\textwidth]{\includegraphics[width=\textwidth]{Img/quizzipedia-server-controllerserver-classmanager-teachersclassfetcher.pdf}}
\caption[Schema Classe TeachersClassFetcher]{Schema Classe Quizzipedia:: Server:: ControllerServer:: ClassManager:: TeachersClassFetcher}
\end{figure}
\paragraph{Relazioni con altre classi}
\subparagraph{Entranti}
\begin{itemize}
\item usata da \cls{Quizzipedia::Server::RoutingManager::ClassRouter} per aggiungi
\end{itemize}
\subparagraph{Uscenti}
\begin{itemize}
\item usa \cls{Quizzipedia::Server::ControllerServer::SessionController} per verificare che l'utente sia un Responsabile
\item usa \cls{Quizzipedia::Server::ModelServer::Organizations::Class} per 
\end{itemize}
\subsection{\pkg{Quizzipedia:: Server:: ControllerServer:: InstitutionManager}}
Contiene le classi che permettono la gestione dell'ente da parte del relativo responsabile.
\begin{figure}[H]
\centering
\noindent\makebox[\textwidth]{\includegraphics[width=\textwidth]{Img/quizzipedia-server-controllerserver-institutionmanager.pdf}}
\caption[Schema Componente InstitutionManager]{Schema Componente Quizzipedia:: Server:: ControllerServer:: InstitutionManager}
\end{figure}
\subsubsection{Interazioni con altre componenti}
\begin{itemize}
\item \bold{Uscenti}
\begin{itemize}
\item usa \pkg{Quizzipedia::Server::ModelServer::Organizations} per organizzare gli utenti all'interno di enti
\end{itemize}
\end{itemize}
\subsubsection{Classe \cls{InstitutionUpdater}}
Permette la modifica dell'ente da parte del responsabile.
\begin{figure}[H]
\centering
\noindent\makebox[\textwidth]{\includegraphics[width=\textwidth]{Img/quizzipedia-server-controllerserver-institutionmanager-institutionupdater.pdf}}
\caption[Schema Classe InstitutionUpdater]{Schema Classe Quizzipedia:: Server:: ControllerServer:: InstitutionManager:: InstitutionUpdater}
\end{figure}
\paragraph{Relazioni con altre classi}
\subparagraph{Entranti}
\begin{itemize}
\item usata da \cls{Quizzipedia::Server::RoutingManager::InstitutionRouter} per aggiungi
\end{itemize}
\subparagraph{Uscenti}
\begin{itemize}
\item usa \cls{Quizzipedia::Server::ControllerServer::SessionController} per verificare che l'utente sia un Responsabile
\item usa \cls{Quizzipedia::Server::ModelServer::Organizations::Institution} per 
\end{itemize}
\subsection{\pkg{Quizzipedia:: Server:: ControllerServer:: ProfileManager}}
Componente che racchiude tutte le funzionalità adibite al salvataggio e alla visualizzazione delle informazioni personali da parte di un utente autenticato.
\begin{figure}[H]
\centering
\noindent\makebox[\textwidth]{\includegraphics[width=\textwidth]{Img/quizzipedia-server-controllerserver-profilemanager.pdf}}
\caption[Schema Componente ProfileManager]{Schema Componente Quizzipedia:: Server:: ControllerServer:: ProfileManager}
\end{figure}
\subsubsection{Interazioni con altre componenti}
\begin{itemize}
\item \bold{Uscenti}
\begin{itemize}
\item usa \pkg{Quizzipedia::Server::ModelServer::Users} per gestire il profilo degli utenti
\end{itemize}
\end{itemize}
\subsubsection{Classe \cls{AccountDeleter}}
Permette la rimozione di un account dal sistema.
\begin{figure}[H]
\centering
\noindent\makebox[\textwidth]{\includegraphics[width=\textwidth]{Img/quizzipedia-server-controllerserver-profilemanager-accountdeleter.pdf}}
\caption[Schema Classe AccountDeleter]{Schema Classe Quizzipedia:: Server:: ControllerServer:: ProfileManager:: AccountDeleter}
\end{figure}
\paragraph{Relazioni con altre classi}
\subparagraph{Entranti}
\begin{itemize}
\item usata da \cls{Quizzipedia::Server::RoutingManager::ProfileRouter} per aggiungi
\end{itemize}
\subparagraph{Uscenti}
\begin{itemize}
\item usa \cls{Quizzipedia::Server::ModelServer::Users::User} per 
\end{itemize}
\subsubsection{Classe \cls{PasswordSetter}}
Permette ad un utente di impostare una nuova password relativa al proprio account.
\begin{figure}[H]
\centering
\noindent\makebox[\textwidth]{\includegraphics[width=\textwidth]{Img/quizzipedia-server-controllerserver-profilemanager-passwordsetter.pdf}}
\caption[Schema Classe PasswordSetter]{Schema Classe Quizzipedia:: Server:: ControllerServer:: ProfileManager:: PasswordSetter}
\end{figure}
\paragraph{Relazioni con altre classi}
\subparagraph{Entranti}
\begin{itemize}
\item usata da \cls{Quizzipedia::Server::RoutingManager::ProfileRouter} per aggiungi
\end{itemize}
\subparagraph{Uscenti}
\begin{itemize}
\item usa \cls{Quizzipedia::Server::ModelServer::Users::User} per 
\end{itemize}
\subsubsection{Classe \cls{PersonalDataFetcher}}
Ritorna tutte le informazioni personali riferite all'utente che ne effettua la richiesta.
\begin{figure}[H]
\centering
\noindent\makebox[\textwidth]{\includegraphics[width=\textwidth]{Img/quizzipedia-server-controllerserver-profilemanager-personaldatafetcher.pdf}}
\caption[Schema Classe PersonalDataFetcher]{Schema Classe Quizzipedia:: Server:: ControllerServer:: ProfileManager:: PersonalDataFetcher}
\end{figure}
\paragraph{Relazioni con altre classi}
\subparagraph{Entranti}
\begin{itemize}
\item usata da \cls{Quizzipedia::Server::RoutingManager::ProfileRouter} per aggiungi
\end{itemize}
\subparagraph{Uscenti}
\begin{itemize}
\item usa \cls{Quizzipedia::Server::ModelServer::Users::User} per 
\end{itemize}
\subsubsection{Classe \cls{PersonalQuizFetcher}}
Ritorna una lista contenente tutti i quiz che un utente autenticato ha svolto fino a quel momento.
\begin{figure}[H]
\centering
\noindent\makebox[\textwidth]{\includegraphics[width=\textwidth]{Img/quizzipedia-server-controllerserver-profilemanager-personalquizfetcher.pdf}}
\caption[Schema Classe PersonalQuizFetcher]{Schema Classe Quizzipedia:: Server:: ControllerServer:: ProfileManager:: PersonalQuizFetcher}
\end{figure}
\paragraph{Relazioni con altre classi}
\subparagraph{Entranti}
\begin{itemize}
\item usata da \cls{Quizzipedia::Server::RoutingManager::ProfileRouter} per aggiungi
\end{itemize}
\subparagraph{Uscenti}
\begin{itemize}
\item usa \cls{Quizzipedia::Server::ModelServer::Services::Quiz} per 
\end{itemize}
\subsection{\pkg{Quizzipedia:: Server:: ControllerServer:: QuestionsManager}}
Pacchetto relativo alla gestione delle domande, la loro creazione, il loro aggiornamento oppure la loro eliminazione.
\begin{figure}[H]
\centering
\noindent\makebox[\textwidth]{\includegraphics[width=\textwidth]{Img/quizzipedia-server-controllerserver-questionsmanager.pdf}}
\caption[Schema Componente QuestionsManager]{Schema Componente Quizzipedia:: Server:: ControllerServer:: QuestionsManager}
\end{figure}
\subsubsection{Interazioni con altre componenti}
\begin{itemize}
\item \bold{Uscenti}
\begin{itemize}
\item usa \pkg{Quizzipedia::Server::ModelServer::Services} per gestire le domande
\item usa \pkg{Quizzipedia::Server::ModelServer::Services::Questions} per gestire le domande
\end{itemize}
\end{itemize}
\subsubsection{Classe \cls{QuestionCreator}}
Permette il salvataggio nella base di dati di una nuova domanda.
\begin{figure}[H]
\centering
\noindent\makebox[\textwidth]{\includegraphics[width=\textwidth]{Img/quizzipedia-server-controllerserver-questionsmanager-questioncreator.pdf}}
\caption[Schema Classe QuestionCreator]{Schema Classe Quizzipedia:: Server:: ControllerServer:: QuestionsManager:: QuestionCreator}
\end{figure}
\paragraph{Relazioni con altre classi}
\subparagraph{Entranti}
\begin{itemize}
\item usata da \cls{Quizzipedia::Server::RoutingManager::QuestionRouter} per aggiungi
\end{itemize}
\subparagraph{Uscenti}
\begin{itemize}
\item usa \cls{Quizzipedia::Server::ControllerServer::SessionController} per verificare che l'utente sia un Docente
\item usa \cls{Quizzipedia::Server::ModelServer::Services::Questions::GenericQuestion} per 
\end{itemize}
\subsubsection{Classe \cls{QuestionEraser}}
Permette la cancellazione di una domanda dalla base di dati.
\begin{figure}[H]
\centering
\noindent\makebox[\textwidth]{\includegraphics[width=\textwidth]{Img/quizzipedia-server-controllerserver-questionsmanager-questioneraser.pdf}}
\caption[Schema Classe QuestionEraser]{Schema Classe Quizzipedia:: Server:: ControllerServer:: QuestionsManager:: QuestionEraser}
\end{figure}
\paragraph{Relazioni con altre classi}
\subparagraph{Entranti}
\begin{itemize}
\item usata da \cls{Quizzipedia::Server::RoutingManager::QuestionRouter} per aggiungi
\end{itemize}
\subparagraph{Uscenti}
\begin{itemize}
\item usa \cls{Quizzipedia::Server::ControllerServer::SessionController} per verificare che l'utente sia un Docente
\item usa \cls{Quizzipedia::Server::ModelServer::Services::Questions::GenericQuestion} per 
\end{itemize}
\subsubsection{Classe \cls{QuestionUpdater}}
Permette la modifica di una domanda già esistente.
\begin{figure}[H]
\centering
\noindent\makebox[\textwidth]{\includegraphics[width=\textwidth]{Img/quizzipedia-server-controllerserver-questionsmanager-questionupdater.pdf}}
\caption[Schema Classe QuestionUpdater]{Schema Classe Quizzipedia:: Server:: ControllerServer:: QuestionsManager:: QuestionUpdater}
\end{figure}
\paragraph{Relazioni con altre classi}
\subparagraph{Entranti}
\begin{itemize}
\item usata da \cls{Quizzipedia::Server::RoutingManager::QuestionRouter} per aggiungi
\end{itemize}
\subparagraph{Uscenti}
\begin{itemize}
\item usa \cls{Quizzipedia::Server::ControllerServer::SessionController} per verificare che l'utente sia un Docente
\item usa \cls{Quizzipedia::Server::ModelServer::Services::Questions::GenericQuestion} per 
\end{itemize}
\subsubsection{Classe \cls{StatisticsQuestionUpdater}}
Aggiorna le statistiche relative ad una domanda quando viene svolto un quiz in esso contenuta.
\begin{figure}[H]
\centering
\noindent\makebox[\textwidth]{\includegraphics[width=\textwidth]{Img/quizzipedia-server-controllerserver-questionsmanager-statisticsquestionupdater.pdf}}
\caption[Schema Classe StatisticsQuestionUpdater]{Schema Classe Quizzipedia:: Server:: ControllerServer:: QuestionsManager:: StatisticsQuestionUpdater}
\end{figure}
\paragraph{Relazioni con altre classi}
\subparagraph{Entranti}
\begin{itemize}
\item usata da \cls{Quizzipedia::Server::RoutingManager::QuestionRouter} per 
\end{itemize}
\subsection{\pkg{Quizzipedia:: Server:: ControllerServer:: QuizManager}}
Componente che racchiude tutte le funzionalità adibite alla creazione, modifica e al recupero di quiz per lo svolgimento da parte di un utente.
\begin{figure}[H]
\centering
\noindent\makebox[\textwidth]{\includegraphics[width=\textwidth]{Img/quizzipedia-server-controllerserver-quizmanager.pdf}}
\caption[Schema Componente QuizManager]{Schema Componente Quizzipedia:: Server:: ControllerServer:: QuizManager}
\end{figure}
\subsubsection{Componenti contenute}
\begin{itemize}
\item \pkg{Quizzipedia::Server::ControllerServer::QuizManager::QMLAgent}
\end{itemize}
\subsubsection{Interazioni con altre componenti}
\begin{itemize}
\item \bold{Uscenti}
\begin{itemize}
\item usa \pkg{Quizzipedia::Server::ModelServer::Services} per gestire i quiz
\end{itemize}
\end{itemize}
\subsubsection{Classe \cls{QuizCreator}}
Permette il salvataggio nella base di dati di un nuovo quiz.
\begin{figure}[H]
\centering
\noindent\makebox[\textwidth]{\includegraphics[width=\textwidth]{Img/quizzipedia-server-controllerserver-quizmanager-quizcreator.pdf}}
\caption[Schema Classe QuizCreator]{Schema Classe Quizzipedia:: Server:: ControllerServer:: QuizManager:: QuizCreator}
\end{figure}
\paragraph{Relazioni con altre classi}
\subparagraph{Entranti}
\begin{itemize}
\item usata da \cls{Quizzipedia::Server::RoutingManager::QuizRouter} per aggiungi
\end{itemize}
\subparagraph{Uscenti}
\begin{itemize}
\item usa \cls{Quizzipedia::Server::ControllerServer::QuizManager::QMLAgent::QMLGenerator} per aggiungi
\item usa \cls{Quizzipedia::Server::ControllerServer::SessionController} per verificare che l'utente sia un Docente
\item usa \cls{Quizzipedia::Server::ModelServer::Services::Quiz} per 
\end{itemize}
\subsubsection{Classe \cls{QuizEraser}}
Permette la cancellazione di un quiz dalla base di dati.
\begin{figure}[H]
\centering
\noindent\makebox[\textwidth]{\includegraphics[width=\textwidth]{Img/quizzipedia-server-controllerserver-quizmanager-quizeraser.pdf}}
\caption[Schema Classe QuizEraser]{Schema Classe Quizzipedia:: Server:: ControllerServer:: QuizManager:: QuizEraser}
\end{figure}
\paragraph{Relazioni con altre classi}
\subparagraph{Entranti}
\begin{itemize}
\item usata da \cls{Quizzipedia::Server::RoutingManager::QuizRouter} per aggiungi
\end{itemize}
\subparagraph{Uscenti}
\begin{itemize}
\item usa \cls{Quizzipedia::Server::ControllerServer::SessionController} per verificare che l'utente sia un Docente
\item usa \cls{Quizzipedia::Server::ModelServer::Services::Quiz} per 
\end{itemize}
\subsubsection{Classe \cls{QuizFetcher}}
Ritorna un determinato quiz pronto per essere svolto da un utente .
\begin{figure}[H]
\centering
\noindent\makebox[\textwidth]{\includegraphics[width=\textwidth]{Img/quizzipedia-server-controllerserver-quizmanager-quizfetcher.pdf}}
\caption[Schema Classe QuizFetcher]{Schema Classe Quizzipedia:: Server:: ControllerServer:: QuizManager:: QuizFetcher}
\end{figure}
\paragraph{Relazioni con altre classi}
\subparagraph{Entranti}
\begin{itemize}
\item usata da \cls{Quizzipedia::Server::RoutingManager::QuizRouter} per aggiungi
\end{itemize}
\subparagraph{Uscenti}
\begin{itemize}
\item usa \cls{Quizzipedia::Server::ControllerServer::QuizManager::QMLAgent::QMLParser} per aggiungi
\item usa \cls{Quizzipedia::Server::ModelServer::Services::Quiz} per 
\end{itemize}
\subsubsection{Classe \cls{QuizUpdater}}
Permette la modifica di un quiz già presente nella base di dati.
\begin{figure}[H]
\centering
\noindent\makebox[\textwidth]{\includegraphics[width=\textwidth]{Img/quizzipedia-server-controllerserver-quizmanager-quizupdater.pdf}}
\caption[Schema Classe QuizUpdater]{Schema Classe Quizzipedia:: Server:: ControllerServer:: QuizManager:: QuizUpdater}
\end{figure}
\paragraph{Relazioni con altre classi}
\subparagraph{Entranti}
\begin{itemize}
\item usata da \cls{Quizzipedia::Server::RoutingManager::QuizRouter} per aggiungi
\end{itemize}
\subparagraph{Uscenti}
\begin{itemize}
\item usa \cls{Quizzipedia::Server::ControllerServer::QuizManager::QMLAgent::QMLGenerator} per aggiungi
\item usa \cls{Quizzipedia::Server::ControllerServer::SessionController} per verificare che l'utente sia un Docente
\item usa \cls{Quizzipedia::Server::ModelServer::Services::Quiz} per 
\end{itemize}
\subsubsection{Classe \cls{ResultsUpdater}}
Aggiorna i risultati dei quiz ad ogni svolgimento degli stessi da parte di un utente.
\begin{figure}[H]
\centering
\noindent\makebox[\textwidth]{\includegraphics[width=\textwidth]{Img/quizzipedia-server-controllerserver-quizmanager-resultsupdater.pdf}}
\caption[Schema Classe ResultsUpdater]{Schema Classe Quizzipedia:: Server:: ControllerServer:: QuizManager:: ResultsUpdater}
\end{figure}
\paragraph{Relazioni con altre classi}
\subparagraph{Entranti}
\begin{itemize}
\item usata da \cls{Quizzipedia::Server::RoutingManager::QuizRouter} per aggiungi
\end{itemize}
\subparagraph{Uscenti}
\begin{itemize}
\item usa \cls{Quizzipedia::Server::ModelServer::Services::Quiz} per 
\end{itemize}
\subsubsection{Classe \cls{StatisticsQuizUpdater}}
Aggiorna le statistiche relative ad un quiz ogni volta che viene svolto da parte degli utenti.
\begin{figure}[H]
\centering
\noindent\makebox[\textwidth]{\includegraphics[width=\textwidth]{Img/quizzipedia-server-controllerserver-quizmanager-statisticsquizupdater.pdf}}
\caption[Schema Classe StatisticsQuizUpdater]{Schema Classe Quizzipedia:: Server:: ControllerServer:: QuizManager:: StatisticsQuizUpdater}
\end{figure}
\paragraph{Relazioni con altre classi}
\subparagraph{Entranti}
\begin{itemize}
\item usata da \cls{Quizzipedia::Server::RoutingManager::QuizRouter} per aggiungi
\end{itemize}
\subsection{\pkg{Quizzipedia:: Server:: ControllerServer:: QuizManager:: QMLAgent}}
Questo componente racchiude i moduli necessari alla traduzione, da QML ad un formato comprensibile dal sistema, delle informazioni estratte dal database per la generazione delle pagine HTML relative ad un quiz e viceversa.
\begin{figure}[H]
\centering
\noindent\makebox[\textwidth]{\includegraphics[width=\textwidth]{Img/quizzipedia-server-controllerserver-quizmanager-qmlagent.pdf}}
\caption[Schema Componente QMLAgent]{Schema Componente Quizzipedia:: Server:: ControllerServer:: QuizManager:: QMLAgent}
\end{figure}
\subsubsection{Classe \cls{QMLGenerator}}
Permette la traduzione in formato QML di un quiz nel caso si voglia procedere al salvataggio dello stesso all'interno del database.
\begin{figure}[H]
\centering
\noindent\makebox[\textwidth]{\includegraphics[width=\textwidth]{Img/quizzipedia-server-controllerserver-quizmanager-qmlagent-qmlgenerator.pdf}}
\caption[Schema Classe QMLGenerator]{Schema Classe Quizzipedia:: Server:: ControllerServer:: QuizManager:: QMLAgent:: QMLGenerator}
\end{figure}
\paragraph{Relazioni con altre classi}
\subparagraph{Entranti}
\begin{itemize}
\item usata da \cls{Quizzipedia::Server::ControllerServer::QuizManager::QuizCreator} per aggiungi
\item usata da \cls{Quizzipedia::Server::ControllerServer::QuizManager::QuizUpdater} per aggiungi
\end{itemize}
\subsubsection{Classe \cls{QMLParser}}
Permette la traduzione di un quiz dal formato QML ad uno comprensibile dal sistema per la generazione delle relative pagine HTML.
\begin{figure}[H]
\centering
\noindent\makebox[\textwidth]{\includegraphics[width=\textwidth]{Img/quizzipedia-server-controllerserver-quizmanager-qmlagent-qmlparser.pdf}}
\caption[Schema Classe QMLParser]{Schema Classe Quizzipedia:: Server:: ControllerServer:: QuizManager:: QMLAgent:: QMLParser}
\end{figure}
\paragraph{Relazioni con altre classi}
\subparagraph{Entranti}
\begin{itemize}
\item usata da \cls{Quizzipedia::Server::ControllerServer::QuizManager::QuizFetcher} per aggiungi
\end{itemize}
\subsection{\pkg{Quizzipedia:: Server:: ControllerServer:: RequestsManager}}
Componente che si occupa di memorizzare richieste da parte degli utenti, mostrarle al responsabile e permettergli di accettarle o meno.
\begin{figure}[H]
\centering
\noindent\makebox[\textwidth]{\includegraphics[width=\textwidth]{Img/quizzipedia-server-controllerserver-requestsmanager.pdf}}
\caption[Schema Componente RequestsManager]{Schema Componente Quizzipedia:: Server:: ControllerServer:: RequestsManager}
\end{figure}
\subsubsection{Interazioni con altre componenti}
\begin{itemize}
\item \bold{Uscenti}
\begin{itemize}
\item usa \pkg{Quizzipedia::Server::ModelServer::Requests} per gestire tutte le richieste avanzate da parte degli utenti
\end{itemize}
\end{itemize}
\subsubsection{Classe \cls{InsertClassRequestsAdder}}
Permette la memorizzazione nella base di dati di tutte le richieste, fatte da parte degli utenti, di essere inseriti in una determinata classe.
\begin{figure}[H]
\centering
\noindent\makebox[\textwidth]{\includegraphics[width=\textwidth]{Img/quizzipedia-server-controllerserver-requestsmanager-insertclassrequestsadder.pdf}}
\caption[Schema Classe InsertClassRequestsAdder]{Schema Classe Quizzipedia:: Server:: ControllerServer:: RequestsManager:: InsertClassRequestsAdder}
\end{figure}
\paragraph{Relazioni con altre classi}
\subparagraph{Entranti}
\begin{itemize}
\item usata da \cls{Quizzipedia::Server::RoutingManager::RequestsRouter} per aggiungi
\end{itemize}
\subparagraph{Uscenti}
\begin{itemize}
\item usa \cls{Quizzipedia::Server::ModelServer::Requests::ClassList} per 
\end{itemize}
\subsubsection{Classe \cls{RequestsFetcher}}
Permette la visualizzazione di tutte le richieste da parte degli utenti.
\begin{figure}[H]
\centering
\noindent\makebox[\textwidth]{\includegraphics[width=\textwidth]{Img/quizzipedia-server-controllerserver-requestsmanager-requestsfetcher.pdf}}
\caption[Schema Classe RequestsFetcher]{Schema Classe Quizzipedia:: Server:: ControllerServer:: RequestsManager:: RequestsFetcher}
\end{figure}
\paragraph{Relazioni con altre classi}
\subparagraph{Entranti}
\begin{itemize}
\item usata da \cls{Quizzipedia::Server::RoutingManager::RequestsRouter} per aggiungi
\end{itemize}
\subparagraph{Uscenti}
\begin{itemize}
\item usa \cls{Quizzipedia::Server::ControllerServer::SessionController} per verificare che l'utente sia un Responsabile
\item usa \cls{Quizzipedia::Server::ModelServer::Requests::ClassList} per 
\item usa \cls{Quizzipedia::Server::ModelServer::Requests::RoleList} per 
\end{itemize}
\subsubsection{Classe \cls{RoleAccepter}}
Permette l'accettazione o la negazione dell'assegnazione di un ruolo ad un utente dopo che ne ha fatto richiesta.
\begin{figure}[H]
\centering
\noindent\makebox[\textwidth]{\includegraphics[width=\textwidth]{Img/quizzipedia-server-controllerserver-requestsmanager-roleaccepter.pdf}}
\caption[Schema Classe RoleAccepter]{Schema Classe Quizzipedia:: Server:: ControllerServer:: RequestsManager:: RoleAccepter}
\end{figure}
\paragraph{Relazioni con altre classi}
\subparagraph{Entranti}
\begin{itemize}
\item usata da \cls{Quizzipedia::Server::RoutingManager::RequestsRouter} per aggiungi
\end{itemize}
\subparagraph{Uscenti}
\begin{itemize}
\item usa \cls{Quizzipedia::Server::ControllerServer::SessionController} per verificare che l'utente sia un Responsabile
\item usa \cls{Quizzipedia::Server::ModelServer::Requests::RoleList} per 
\end{itemize}
\subsubsection{Classe \cls{RoleRequestAdder}}
Memorizza richieste da parte degli utenti di assumere un determinato ruolo all'interno del sistema.
\begin{figure}[H]
\centering
\noindent\makebox[\textwidth]{\includegraphics[width=\textwidth]{Img/quizzipedia-server-controllerserver-requestsmanager-rolerequestadder.pdf}}
\caption[Schema Classe RoleRequestAdder]{Schema Classe Quizzipedia:: Server:: ControllerServer:: RequestsManager:: RoleRequestAdder}
\end{figure}
\paragraph{Relazioni con altre classi}
\subparagraph{Entranti}
\begin{itemize}
\item usata da \cls{Quizzipedia::Server::RoutingManager::RequestsRouter} per aggiungi
\end{itemize}
\subparagraph{Uscenti}
\begin{itemize}
\item usa \cls{Quizzipedia::Server::ModelServer::Requests::RoleList} per 
\end{itemize}
\subsection{\pkg{Quizzipedia:: Server:: ControllerServer:: SearchManager}}
Questo componente permette di effettuare una ricerca nel database di quiz o domande richiesti dall'utente e ritornare una lista che corrisponde ai parametri desiderati.
\begin{figure}[H]
\centering
\noindent\makebox[\textwidth]{\includegraphics[width=\textwidth]{Img/quizzipedia-server-controllerserver-searchmanager.pdf}}
\caption[Schema Componente SearchManager]{Schema Componente Quizzipedia:: Server:: ControllerServer:: SearchManager}
\end{figure}
\subsubsection{Interazioni con altre componenti}
\begin{itemize}
\item \bold{Uscenti}
\begin{itemize}
\item usa \pkg{Quizzipedia::Server::ModelServer::Services} per gestire le ricerche di quiz e domande da parte degli utenti
\end{itemize}
\end{itemize}
\subsubsection{Classe \cls{QuestionsSearcher}}
Ritorna una lista di domande che corrispondono ai parametri di ricerca impostati dall'utente.
\begin{figure}[H]
\centering
\noindent\makebox[\textwidth]{\includegraphics[width=\textwidth]{Img/quizzipedia-server-controllerserver-searchmanager-questionssearcher.pdf}}
\caption[Schema Classe QuestionsSearcher]{Schema Classe Quizzipedia:: Server:: ControllerServer:: SearchManager:: QuestionsSearcher}
\end{figure}
\paragraph{Relazioni con altre classi}
\subparagraph{Entranti}
\begin{itemize}
\item usata da \cls{Quizzipedia::Server::RoutingManager::SearchRouter} per aggiungi
\end{itemize}
\subparagraph{Uscenti}
\begin{itemize}
\item usa \cls{Quizzipedia::Server::ModelServer::Services::Questions::GenericQuestion} per 
\end{itemize}
\subsubsection{Classe \cls{QuizSearcher}}
Ritorna una lista di quiz che corrispondono ai parametri di ricerca impostati dall'utente.
\begin{figure}[H]
\centering
\noindent\makebox[\textwidth]{\includegraphics[width=\textwidth]{Img/quizzipedia-server-controllerserver-searchmanager-quizsearcher.pdf}}
\caption[Schema Classe QuizSearcher]{Schema Classe Quizzipedia:: Server:: ControllerServer:: SearchManager:: QuizSearcher}
\end{figure}
\paragraph{Relazioni con altre classi}
\subparagraph{Entranti}
\begin{itemize}
\item usata da \cls{Quizzipedia::Server::RoutingManager::SearchRouter} per aggiungi
\end{itemize}
\subparagraph{Uscenti}
\begin{itemize}
\item usa \cls{Quizzipedia::Server::ModelServer::Services::Quiz} per 
\end{itemize}
\subsection{\pkg{Quizzipedia:: Server:: ControllerServer:: StatisticsManager}}
Questo componente ha il compito di recuperare tutte le informazioni sottoforma di statistiche relative ad un quiz, una domanda in particolare o ad un utente.
\begin{figure}[H]
\centering
\noindent\makebox[\textwidth]{\includegraphics[width=\textwidth]{Img/quizzipedia-server-controllerserver-statisticsmanager.pdf}}
\caption[Schema Componente StatisticsManager]{Schema Componente Quizzipedia:: Server:: ControllerServer:: StatisticsManager}
\end{figure}
\subsubsection{Interazioni con altre componenti}
\begin{itemize}
\item \bold{Uscenti}
\begin{itemize}
\item usa \pkg{Quizzipedia::Server::ModelServer::Statistics} per mostrare e gestire le statistiche relative a quiz e domande
\end{itemize}
\end{itemize}
\subsubsection{Classe \cls{QuestionStatisticsFetcher}}
Ritorna le statistiche riferite ad una domanda.
\begin{figure}[H]
\centering
\noindent\makebox[\textwidth]{\includegraphics[width=\textwidth]{Img/quizzipedia-server-controllerserver-statisticsmanager-questionstatisticsfetcher.pdf}}
\caption[Schema Classe QuestionStatisticsFetcher]{Schema Classe Quizzipedia:: Server:: ControllerServer:: StatisticsManager:: QuestionStatisticsFetcher}
\end{figure}
\paragraph{Relazioni con altre classi}
\subparagraph{Entranti}
\begin{itemize}
\item usata da \cls{Quizzipedia::Server::RoutingManager::StatisticsRouter} per aggiungi
\end{itemize}
\subparagraph{Uscenti}
\begin{itemize}
\item usa \cls{Quizzipedia::Server::ControllerServer::SessionController} per verificare che l'utente sia un Docente
\item usa \cls{Quizzipedia::Server::ModelServer::Statistics::QuestionStatistics} per recuperare le statistiche che chiede l'utente riguardo una domanda
\end{itemize}
\subsubsection{Classe \cls{QuizStatisticsFetcher}}
Ritorna le statistiche riferite ad un quiz.
\begin{figure}[H]
\centering
\noindent\makebox[\textwidth]{\includegraphics[width=\textwidth]{Img/quizzipedia-server-controllerserver-statisticsmanager-quizstatisticsfetcher.pdf}}
\caption[Schema Classe QuizStatisticsFetcher]{Schema Classe Quizzipedia:: Server:: ControllerServer:: StatisticsManager:: QuizStatisticsFetcher}
\end{figure}
\paragraph{Relazioni con altre classi}
\subparagraph{Entranti}
\begin{itemize}
\item usata da \cls{Quizzipedia::Server::RoutingManager::StatisticsRouter} per aggiungi
\end{itemize}
\subparagraph{Uscenti}
\begin{itemize}
\item usa \cls{Quizzipedia::Server::ControllerServer::SessionController} per verificare che l'utente sia un Docente
\item usa \cls{Quizzipedia::Server::ModelServer::Statistics::QuizStatistics} per 
\end{itemize}
\subsubsection{Classe \cls{StudentStatisticsFetcher}}
Ritorna le statistiche riferite ad uno studente.
\begin{figure}[H]
\centering
\noindent\makebox[\textwidth]{\includegraphics[width=\textwidth]{Img/quizzipedia-server-controllerserver-statisticsmanager-studentstatisticsfetcher.pdf}}
\caption[Schema Classe StudentStatisticsFetcher]{Schema Classe Quizzipedia:: Server:: ControllerServer:: StatisticsManager:: StudentStatisticsFetcher}
\end{figure}
\paragraph{Relazioni con altre classi}
\subparagraph{Entranti}
\begin{itemize}
\item usata da \cls{Quizzipedia::Server::RoutingManager::StatisticsRouter} per aggiungi
\end{itemize}
\subparagraph{Uscenti}
\begin{itemize}
\item usa \cls{Quizzipedia::Server::ControllerServer::SessionController} per verificare che l'utente sia un Docente
\end{itemize}
\subsection{\pkg{Quizzipedia:: Server:: ControllerServer:: TopicManager}}
Componente che permette la creazione di un nuovo argomento o l'eliminazione di uno già esistente.
\begin{figure}[H]
\centering
\noindent\makebox[\textwidth]{\includegraphics[width=\textwidth]{Img/quizzipedia-server-controllerserver-topicmanager.pdf}}
\caption[Schema Componente TopicManager]{Schema Componente Quizzipedia:: Server:: ControllerServer:: TopicManager}
\end{figure}
\subsubsection{Interazioni con altre componenti}
\begin{itemize}
\item \bold{Uscenti}
\begin{itemize}
\item usa \pkg{Quizzipedia::Server::ModelServer::Services} per gestire gli argomenti
\end{itemize}
\end{itemize}
\subsubsection{Classe \cls{TopicCreator}}
Permette la creazione di un nuovo argomento.
\begin{figure}[H]
\centering
\noindent\makebox[\textwidth]{\includegraphics[width=\textwidth]{Img/quizzipedia-server-controllerserver-topicmanager-topiccreator.pdf}}
\caption[Schema Classe TopicCreator]{Schema Classe Quizzipedia:: Server:: ControllerServer:: TopicManager:: TopicCreator}
\end{figure}
\paragraph{Relazioni con altre classi}
\subparagraph{Entranti}
\begin{itemize}
\item usata da \cls{Quizzipedia::Server::RoutingManager::TopicRouter} per aggiungi
\end{itemize}
\subparagraph{Uscenti}
\begin{itemize}
\item usa \cls{Quizzipedia::Server::ControllerServer::SessionController} per verificare che l'utente sia un Docente
\item usa \cls{Quizzipedia::Server::ModelServer::Services::Topics} per 
\end{itemize}
\subsubsection{Classe \cls{TopicEraser}}
Permette l'eliminazione di un argomento.
\begin{figure}[H]
\centering
\noindent\makebox[\textwidth]{\includegraphics[width=\textwidth]{Img/quizzipedia-server-controllerserver-topicmanager-topiceraser.pdf}}
\caption[Schema Classe TopicEraser]{Schema Classe Quizzipedia:: Server:: ControllerServer:: TopicManager:: TopicEraser}
\end{figure}
\paragraph{Relazioni con altre classi}
\subparagraph{Entranti}
\begin{itemize}
\item usata da \cls{Quizzipedia::Server::RoutingManager::TopicRouter} per aggiungi
\end{itemize}
\subparagraph{Uscenti}
\begin{itemize}
\item usa \cls{Quizzipedia::Server::ControllerServer::SessionController} per verificare che l'utente sia un Docente
\item usa \cls{Quizzipedia::Server::ModelServer::Services::Topics} per 
\end{itemize}
\subsection{\pkg{Quizzipedia:: Server:: RoutingManager}}
Questo pacchetto costituisce lo strato superiore a ControllerServer e contiene tutte le API necessarie per gestire le comunicazioni provenienti dal client tramite Socket.IO. 
Con l'uso di Express.js le richieste provenienti dal client verranno indirizzate alla corretta classe di questo componente, che si occuperà di inviare la richiesta al relativo services per la sua esecuzione..
\begin{figure}[H]
\centering
\noindent\makebox[\textwidth]{\includegraphics[width=\textwidth]{Img/quizzipedia-server-routingmanager.pdf}}
\caption[Schema Componente RoutingManager]{Schema Componente Quizzipedia:: Server:: RoutingManager}
\end{figure}
\subsubsection{Interazioni con altre componenti}
\begin{itemize}
\item \bold{Uscenti}
\begin{itemize}
\item usa \pkg{Quizzipedia::Server::ControllerServer} per instradare e inoltrare le chiamate REST provenienti dal client ai moduli corretti in grado di soddisfarle
\end{itemize}
\end{itemize}
\subsubsection{Classe \cls{AbstractRouter}}
Classe astratta supertipo delle altre classi del pacchetto.
\begin{figure}[H]
\centering
\noindent\makebox[\textwidth]{\includegraphics[width=\textwidth]{Img/quizzipedia-server-routingmanager-abstractrouter.pdf}}
\caption[Schema Classe AbstractRouter]{Schema Classe Quizzipedia:: Server:: RoutingManager:: AbstractRouter}
\end{figure}
\subsubsection{Classe \cls{AuthenticationRouter}}
Invocato dal client per interagire con AuthenticationManager.
\begin{figure}[H]
\centering
\noindent\makebox[\textwidth]{\includegraphics[width=\textwidth]{Img/quizzipedia-server-routingmanager-authenticationrouter.pdf}}
\caption[Schema Classe AuthenticationRouter]{Schema Classe Quizzipedia:: Server:: RoutingManager:: AuthenticationRouter}
\end{figure}
\paragraph{Relazioni con altre classi}
\subparagraph{Entranti}
\begin{itemize}
\item usata da \cls{Quizzipedia::Client::ViewModelClient::CtrlUsers::CtrlLogin} per passare le credenziali di login al server perché vengano validate
\item usata da \cls{Quizzipedia::Client::ViewModelClient::CtrlUsers::CtrlRegistration} per passare al server le informazioni necessarie per la registrazione
\end{itemize}
\subparagraph{Uscenti}
\begin{itemize}
\item usa \cls{Quizzipedia::Server::ControllerServer::AuthenticationManager::LoggerIn} per aggiungi
\item usa \cls{Quizzipedia::Server::ControllerServer::AuthenticationManager::LoggerOut} per aggiungi
\item usa \cls{Quizzipedia::Server::ControllerServer::AuthenticationManager::PasswordRecover} per aggiungi
\item usa \cls{Quizzipedia::Server::ControllerServer::AuthenticationManager::Register} per aggiungi
\end{itemize}
\subsubsection{Classe \cls{ClassRouter}}
Invocato dal client per interagire con ClassManager.
\begin{figure}[H]
\centering
\noindent\makebox[\textwidth]{\includegraphics[width=\textwidth]{Img/quizzipedia-server-routingmanager-classrouter.pdf}}
\caption[Schema Classe ClassRouter]{Schema Classe Quizzipedia:: Server:: RoutingManager:: ClassRouter}
\end{figure}
\paragraph{Relazioni con altre classi}
\subparagraph{Uscenti}
\begin{itemize}
\item usa \cls{Quizzipedia::Server::ControllerServer::ClassManager::ClassAdder} per aggiungi
\item usa \cls{Quizzipedia::Server::ControllerServer::ClassManager::ClassDeleter} per aggiungi
\item usa \cls{Quizzipedia::Server::ControllerServer::ClassManager::ClassUpdater} per aggiungi
\item usa \cls{Quizzipedia::Server::ControllerServer::ClassManager::FromClassRemover} per aggiungi
\item usa \cls{Quizzipedia::Server::ControllerServer::ClassManager::InClassAdder} per aggiungi
\item usa \cls{Quizzipedia::Server::ControllerServer::ClassManager::StudentsClassFetcher} per aggiungi
\item usa \cls{Quizzipedia::Server::ControllerServer::ClassManager::TeachersClassFetcher} per aggiungi
\end{itemize}
\subsubsection{Classe \cls{InstitutionRouter}}
Invocato dal client per interagire con InstitutionManager.
\begin{figure}[H]
\centering
\noindent\makebox[\textwidth]{\includegraphics[width=\textwidth]{Img/quizzipedia-server-routingmanager-institutionrouter.pdf}}
\caption[Schema Classe InstitutionRouter]{Schema Classe Quizzipedia:: Server:: RoutingManager:: InstitutionRouter}
\end{figure}
\paragraph{Relazioni con altre classi}
\subparagraph{Uscenti}
\begin{itemize}
\item usa \cls{Quizzipedia::Server::ControllerServer::InstitutionManager::InstitutionUpdater} per aggiungi
\end{itemize}
\subsubsection{Classe \cls{ProfileRouter}}
Invocato dal client per interagire con ProfileManager.
\begin{figure}[H]
\centering
\noindent\makebox[\textwidth]{\includegraphics[width=\textwidth]{Img/quizzipedia-server-routingmanager-profilerouter.pdf}}
\caption[Schema Classe ProfileRouter]{Schema Classe Quizzipedia:: Server:: RoutingManager:: ProfileRouter}
\end{figure}
\paragraph{Relazioni con altre classi}
\subparagraph{Uscenti}
\begin{itemize}
\item usa \cls{Quizzipedia::Server::ControllerServer::ProfileManager::AccountDeleter} per aggiungi
\item usa \cls{Quizzipedia::Server::ControllerServer::ProfileManager::PasswordSetter} per aggiungi
\item usa \cls{Quizzipedia::Server::ControllerServer::ProfileManager::PersonalDataFetcher} per aggiungi
\item usa \cls{Quizzipedia::Server::ControllerServer::ProfileManager::PersonalQuizFetcher} per aggiungi
\end{itemize}
\subsubsection{Classe \cls{QuestionRouter}}
Invocato dal client per interagire con QuestionsManager.
\begin{figure}[H]
\centering
\noindent\makebox[\textwidth]{\includegraphics[width=\textwidth]{Img/quizzipedia-server-routingmanager-questionrouter.pdf}}
\caption[Schema Classe QuestionRouter]{Schema Classe Quizzipedia:: Server:: RoutingManager:: QuestionRouter}
\end{figure}
\paragraph{Relazioni con altre classi}
\subparagraph{Uscenti}
\begin{itemize}
\item usa \cls{Quizzipedia::Server::ControllerServer::QuestionsManager::QuestionCreator} per aggiungi
\item usa \cls{Quizzipedia::Server::ControllerServer::QuestionsManager::QuestionEraser} per aggiungi
\item usa \cls{Quizzipedia::Server::ControllerServer::QuestionsManager::QuestionUpdater} per aggiungi
\item usa \cls{Quizzipedia::Server::ControllerServer::QuestionsManager::StatisticsQuestionUpdater} per 
\end{itemize}
\subsubsection{Classe \cls{QuizRouter}}
Invocato dal client per interagire con QuizManager.
\begin{figure}[H]
\centering
\noindent\makebox[\textwidth]{\includegraphics[width=\textwidth]{Img/quizzipedia-server-routingmanager-quizrouter.pdf}}
\caption[Schema Classe QuizRouter]{Schema Classe Quizzipedia:: Server:: RoutingManager:: QuizRouter}
\end{figure}
\paragraph{Relazioni con altre classi}
\subparagraph{Uscenti}
\begin{itemize}
\item usa \cls{Quizzipedia::Server::ControllerServer::QuizManager::QuizCreator} per aggiungi
\item usa \cls{Quizzipedia::Server::ControllerServer::QuizManager::QuizEraser} per aggiungi
\item usa \cls{Quizzipedia::Server::ControllerServer::QuizManager::QuizFetcher} per aggiungi
\item usa \cls{Quizzipedia::Server::ControllerServer::QuizManager::QuizUpdater} per aggiungi
\item usa \cls{Quizzipedia::Server::ControllerServer::QuizManager::ResultsUpdater} per aggiungi
\item usa \cls{Quizzipedia::Server::ControllerServer::QuizManager::StatisticsQuizUpdater} per aggiungi
\end{itemize}
\subsubsection{Classe \cls{RequestsRouter}}
Invocato dal client per interagire con RequestsManager.
\begin{figure}[H]
\centering
\noindent\makebox[\textwidth]{\includegraphics[width=\textwidth]{Img/quizzipedia-server-routingmanager-requestsrouter.pdf}}
\caption[Schema Classe RequestsRouter]{Schema Classe Quizzipedia:: Server:: RoutingManager:: RequestsRouter}
\end{figure}
\paragraph{Relazioni con altre classi}
\subparagraph{Uscenti}
\begin{itemize}
\item usa \cls{Quizzipedia::Server::ControllerServer::RequestsManager::InsertClassRequestsAdder} per aggiungi
\item usa \cls{Quizzipedia::Server::ControllerServer::RequestsManager::RequestsFetcher} per aggiungi
\item usa \cls{Quizzipedia::Server::ControllerServer::RequestsManager::RoleAccepter} per aggiungi
\item usa \cls{Quizzipedia::Server::ControllerServer::RequestsManager::RoleRequestAdder} per aggiungi
\end{itemize}
\subsubsection{Classe \cls{SearchRouter}}
Invocato dal client per interagire con SearchManager.
\begin{figure}[H]
\centering
\noindent\makebox[\textwidth]{\includegraphics[width=\textwidth]{Img/quizzipedia-server-routingmanager-searchrouter.pdf}}
\caption[Schema Classe SearchRouter]{Schema Classe Quizzipedia:: Server:: RoutingManager:: SearchRouter}
\end{figure}
\paragraph{Relazioni con altre classi}
\subparagraph{Entranti}
\begin{itemize}
\item usata da \cls{Quizzipedia::Client::ViewModelClient::CtrlServices::CtrlSearchQuiz} per aggiungi
\end{itemize}
\subparagraph{Uscenti}
\begin{itemize}
\item usa \cls{Quizzipedia::Server::ControllerServer::SearchManager::QuestionsSearcher} per aggiungi
\item usa \cls{Quizzipedia::Server::ControllerServer::SearchManager::QuizSearcher} per aggiungi
\end{itemize}
\subsubsection{Classe \cls{StatisticsRouter}}
Invocato dal client per interagire con StatisticsManager.
\begin{figure}[H]
\centering
\noindent\makebox[\textwidth]{\includegraphics[width=\textwidth]{Img/quizzipedia-server-routingmanager-statisticsrouter.pdf}}
\caption[Schema Classe StatisticsRouter]{Schema Classe Quizzipedia:: Server:: RoutingManager:: StatisticsRouter}
\end{figure}
\paragraph{Relazioni con altre classi}
\subparagraph{Uscenti}
\begin{itemize}
\item usa \cls{Quizzipedia::Server::ControllerServer::StatisticsManager::QuestionStatisticsFetcher} per aggiungi
\item usa \cls{Quizzipedia::Server::ControllerServer::StatisticsManager::QuizStatisticsFetcher} per aggiungi
\item usa \cls{Quizzipedia::Server::ControllerServer::StatisticsManager::StudentStatisticsFetcher} per aggiungi
\end{itemize}
\subsubsection{Classe \cls{TopicRouter}}
Invocato dal client per interagire con TopicManager.
\begin{figure}[H]
\centering
\noindent\makebox[\textwidth]{\includegraphics[width=\textwidth]{Img/quizzipedia-server-routingmanager-topicrouter.pdf}}
\caption[Schema Classe TopicRouter]{Schema Classe Quizzipedia:: Server:: RoutingManager:: TopicRouter}
\end{figure}
\paragraph{Relazioni con altre classi}
\subparagraph{Uscenti}
\begin{itemize}
\item usa \cls{Quizzipedia::Server::ControllerServer::TopicManager::TopicCreator} per aggiungi
\item usa \cls{Quizzipedia::Server::ControllerServer::TopicManager::TopicEraser} per aggiungi
\end{itemize}
