% Nome del file: SpecificaTecnica.tex
% Percorso: \gl{template}
% Autore: Vault-Tech
% Data creazione: 17.03.2016
% E-mail: vaulttech.swe@gmail.comcom
%
% Diario delle modifiche: interno al file.

\documentclass[a4paper, titlepage]{article}


\usepackage[margin=3cm]{geometry}
\usepackage{float}
\usepackage{grffile}
\usepackage{../../Stile}
\usepackage{../../Comandi}

\setcounter{secnumdepth}{5}
\setcounter{tocdepth}{5}

\def\NOME{Specifica Tecnica}
\def\VERSIONE{1.0}
\def\DATA{17.03.2016}
\def\REDATTORE{?}
\def\VERIFICATORE{?}
\def\RESPONSABILE{?}
\def\USO{Esterno}
\def\DISTRIBUZIONE{\COMMITTENTE \\ & \CARDIN \\ & \PROPONENTE}

\begin{document}
\pagestyle{fancy}	
\pagenumbering{Roman}
\rfoot{Pagina \thepage{} di \pageref{lastromanpage}}

\maketitle

\begin{diario}
	\recap{Prima modifica}{?}{?}{17.03.2016}{0.1}
\end{diario}

\newpage
\tableofcontents\label{lastromanpage}

\newpage
\listoffigures

\newpage
\listoftables

\newpage
\clearpage	
\pagenumbering{arabic}
\rfoot{Pagina \thepage{} di \pageref*{LastPage}}

\section{Introduzione}
\subsection{Scopo del documento}
Lo scopo di questo documento è quello di individuare e descrivere i requisiti necessari per lo svolgimento del progetto Quizzipedia.

\subsection{Scopo del prodotto}
\SCOPO

\subsection{Glossario}
\GLOSSARIO

\subsection{Riferimenti}
\subsubsection{Riferimenti normativi}
\begin{itemize}
\item \bold{Norme di Progetto:} \doc{Norme di Progetto v1.0}.
\end{itemize}

\subsubsection{Riferimenti informativi}
\begin{itemize}
\item \bold{Capitolato d'appalto C5:} \italics{Quizzipedia: \gl{software} per la gestione di questionari} (\href{http://www.math.unipd.it/~tullio/IS-1/2015/Progetto/C5.pdf}{http://www.math.unipd.it/~tullio/IS-1/2015/Progetto/C5.pdf});

\item \bold{Studio di fattibilità:} \doc{Studio di Fattibilità v1.0};

\item \bold{Glossario:} \doc{Glossario v1.0};

\item \bold{Lezione L06:} \italics{L06: Ingegneria dei requisiti} (\href{http://www.math.unipd.it/~tullio/IS-1/2015/Dispense/L06.pdf}{http://www.math.unipd.it/~tullio/IS-1/2015/Dispense/L06.pdf});

\item \bold{IEEE 830-1998} \italics{Commented IEEE 830-1998} (\href{https://www.cs.purdue.edu/homes/apm/courses/BITSC461-fall03/miller-guidelines/IEEE830-1998.html}{https://www.cs.purdue.edu/homes/apm/courses/BITSC461-fall03/miller-guidelines/IEEE830-1998.html}).

\end{itemize}

\newpage

\section{Descrizione generale}
\subsection{Contesto d'uso del prodotto}
Quizzipedia si pone l'obiettivo di creare un sistema di gestione di questionari su diversi argomenti, composti da diverse tipologie di domande. Gli utenti potranno sia creare e modificare i questionari, sia svolgerli e ottenere la propria valutazione.


\end{document}