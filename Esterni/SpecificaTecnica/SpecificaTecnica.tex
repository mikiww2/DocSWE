% Nome del file: SpecificaTecnica.tex
% Percorso: \gl{template}
% Autore: Vault-Tech
% Data creazione: 17.03.2016
% E-mail: vaulttech.swe@gmail.comcom
%
% Diario delle modifiche: interno al file.

\documentclass[a4paper, titlepage]{article}

\usepackage[margin=3cm]{geometry}
\usepackage{float}
\usepackage{grffile}
\usepackage{../../Stile}
\usepackage{../../Comandi}

\setcounter{secnumdepth}{5}
\setcounter{tocdepth}{5}

\def\NOME{Specifica Tecnica}
\def\VERSIONE{0.1}
\def\DATA{05.03.2016}
\def\REDATTORE{Giacomo Beltrame \\ & Simone Boccato \\ & Michela De Bortoli \\ & Vassilikì Menarin \\ & Miki Violetto}
\def\VERIFICATORE{Rudy Berton \\ & Filippo Tesser}
\def\RESPONSABILE{Michela De Bortoli \\ & Filippo Tesser}
\def\USO{Esterno}
\def\DISTRIBUZIONE{\AUTORE}

\begin{document}
\pagestyle{fancy}	
\pagenumbering{Roman}
\rfoot{Pagina \thepage{} di \pageref{lastromanpage}}

\maketitle

\begin{diario}
	\recap{Stesura della struttura del documento}{Giacomo Beltrame}{Progettista}{05.03.2016}{0.1}
\end{diario}

\newpage
\tableofcontents

\newpage
\listoffigures \label{lastromanpage}

\newpage
\clearpage	
\pagenumbering{arabic}
\rfoot{Pagina \thepage{} di \pageref*{LastPage}}
\hypersetup{linkcolor=blue}

\section{Introduzione}
\subsection{Scopo del documento}
In tale documento verrà definita la progettazione ad alto livello del prodotto Quizzipedia.
Per tale scopo vengono descritte le componenti, le classi e i design pattern utilizzati per la realizzazione del prodotto. Inoltre viene presentato il tracciamento tra le componenti e i requisiti individuati.

\subsection{Scopo del prodotto}
\SCOPO

\subsection{Glossario}
\GLOSSARIO

\subsection{Riferimenti}
\subsubsection{Riferimenti normativi}
\begin{itemize}
\item \bold{Norme di Progetto:} \NdPdoc.
\end{itemize}

\subsubsection{Riferimenti informativi}
\begin{itemize}
\item \bold{Capitolato d'appalto C5:} \italics{Quizzipedia: \gl{software} per la gestione di questionari} \url{http://www.math.unipd.it/~tullio/IS-1/2015/Progetto/C5.pdf};

\item \bold{Glossario:} \Gldoc;

\item \bold{Analisi dei requisiti: } \ARdoc;

\item \bold{HTML5: } \url{https://www.w3.org/TR/html5/};

\item \bold{CSS3: } \url{https://www.w3.org/TR/CSS/};

\item \bold{Javascript: } \url{http://www.ecma-international.org/publications/files/ECMA-ST/Ecma-262.pdf};

\item \bold{PostgreSQL: } \url{http://www.postgresql.org/docs/};

\item \bold{AngularJS: } \url{https://angular.io/docs/ts/latest/};

\item \bold{Node.js: } \url{https://nodejs.org/api/};

\item \bold{Express.js: } \url{http://expressjs.com/en/api.html};

\item \bold{Bootstrap: } \url{http://getbootstrap.com/css/};

\item \bold{JSON: } \url{http://www.ecma-international.org/publications/files/ECMA-ST/ECMA-404.pdf};

\item \bold{Fabric.js: } \url{http://fabricjs.com/docs/};

\item \bold{Socket.IO: } \url{http://socket.io/docs/}.

\end{itemize}

\newpage

\section{Tecnologie utilizzate}
In questa sezione vengono presentate le tecnologie scelte per sviluppare il progetto. Per ognuna di esse verrà fornita una breve descrizione, i principali punti a favore e sfavore e, ove necessario, la versione utilizzata.

\subsection{AngularJS 2.0}
Framework Javascript open source, patrocinato da Google, che si ispira al pattern MVC.
Semplifica la realizzazione di applicazioni web favorendo un approccio dichiarativo allo sviluppo client-side e alla creazione interfacce utente.

\subsubsection{Vantaggi}

\begin{itemize}
	\item Riduzione verbosità del codice;
	\item ampia documentazione;
	\item data binding bidirezionale;
	\item dependency injection permette di isolare componenti e rimpiazzarli facilmente;
	\item test facilitati.
\end{itemize}

\subsubsection{Svantaggi}

\begin{itemize}
	\item Codice articolato;
	\item curva di apprendimento più ripida rispetto ad altri framework.
\end{itemize}

\subsection{Node.js}
L'utilizzo di Node.js è stato consigliato dal proponente insieme a Tomcat. Si tratta di un sistema run-time crossplatform che utilizza l’engine Google V8 JavaScript per eseguire il codice; viene utilizzato nello sviluppo di applicazioni real-time, lato server e di rete.

\subsubsection{Vantaggi}

\begin{itemize}
	\item Fornisce una semplice soluzione per lo sviluppo di programmi scalabili;
	\item essendo leggero ed efficiente è particolarmente indicato per applicazioni real-time con uso intensivo di dati;
	\item usa un sistema I/O asincrono basato su eventi che permette di sviluppare più facilmente sistemi responsivi.
\end{itemize}

\subsubsection{Svantaggi}

\begin{itemize}
	\item Essendo di recente creazione, alcuni moduli possono essere instabili.
\end{itemize}

\subsection{Express.js}
Express.js è un framework di Node.js che mette a disposizione un middleware utilizzabile per la realizzazione dell’infrastruttura web.

\subsubsection{Vantaggi}

\begin{itemize}
	\item Permette di estendere il modulo http di Node.js rendendo più facile la gestione dell’indirizzamento del server, delle risposte, dei cookie e delle richieste di stato HTTP.
\end{itemize}

\subsubsection{Svantaggi}

\begin{itemize}
	\item Non sono stati rilevati grossi svantaggi nell'uso di Express.js, che risulta essere un buon framework per estendere il modulo HTTP di Node.js.
\end{itemize}

\subsection{PostgreSQL}
Si è deciso di utilizzare PostgreSQL come database. PostgreSQL è un avanzato database relazionale ad oggetti, molto completo e open source.
\subsubsection{Vantaggi}

\begin{itemize}
	\item Più familiarità con database relazionali;
	\item query più flessibili di MongoDB;
	\item supporto di transazioni.
\end{itemize}

\subsubsection{Svantaggi}

\begin{itemize}
	\item Relativamente poco performante;
	\item mai usato dai componenti del team.
\end{itemize}

\subsection{HTML5}
Si è deciso di usare HTML5 come linguaggio di markup per la creazione delle pagine web.

\subsubsection{Vantaggi}

\begin{itemize}
	\item Facilità di creare pagine adattabili;
	\item maggior flessibilità rispetto alle versioni precedenti;
	\item supporta l'elemento grafico canvas che permette di utilizzare Javascript per creare animazioni e grafica.
\end{itemize}

\subsubsection{Svantaggi}

\begin{itemize}
	\item Tecnologia non pienamente supportata da tutti i browser.
\end{itemize}

\subsection{CSS3}
Si è deciso di utilizzare CSS3 come linguaggio per la formattazione dei documenti HTML.

\subsubsection{Vantaggi}

\begin{itemize}
	\item Separazione struttura-contenuti-presentazione;
	\item controllo più preciso e completo dell'aspetto grafico;
	\item la manutenzione grafica del sito risulta molto più facile.
\end{itemize}

\subsubsection{Svantaggi}

\begin{itemize}
	\item L’unico svantaggio rilevato dall’uso di CSS3 è dato dal fatto che, essendo un linguaggio ancora in via di sviluppo, è supportato pienamente solo dai web browser più recenti.
\end{itemize}

\subsection{Bootstrap}
Si tratta di uno dei più noti framework per lo sviluppo di interfacce web. Bootstrap contiene modelli di progettazione basati su HTML e CSS, sia per la tipografia, che per le varie componenti dell'interfaccia, come moduli, bottoni e navigazione, e altri componenti dell'interfaccia, così come alcune estensioni opzionali di Javascript.

\subsubsection{Vantaggi}

\begin{itemize}
	\item Offre una utile struttura standardizzata;
	\item supporta maggiori browser e risolve problemi di compatibilità;
	\item lightweight;
	\item vari plugin di Javascript inclusi;
	\item buona documentazione.
\end{itemize}

\subsubsection{Svantaggi}

\begin{itemize}
	\item Verbosità degli stili;
	\item in caso di customizzazione carente i siti hanno aspetto simile.
\end{itemize}

\subsection{Javascript}
Javascript è un linguaggio di scripting orientato agli oggetti e agli eventi, comunemente utilizzato nella programmazione Web lato client per la creazione, in siti web e applicazioni web, di effetti dinamici interattivi tramite funzioni di script invocate da eventi innescati a loro volta in vari modi dall'utente sulla pagina web in uso.

\subsubsection{Vantaggi}

\begin{itemize}
	\item Velocità di risposta dell'applicazione migliore, a causa dell'esecuzione locale;
	\item libertà espressiva;
	\item possibilità di validare in prima istanza i dati inseriti dall'utente;
	\item migliore interazione tra utente e pagine;
	\item gestione dei canvas.
\end{itemize}

\subsubsection{Svantaggi}

\begin{itemize}
	\item Mancanza di tipizzazione, errori di programmazione potenzialmente più dannosi;
	\item difficoltà del creare i test;
	\item il codice può essere letto da tutti.
\end{itemize}

\subsection{JSON}
Si è deciso di usare JSON come formato per lo scambio dati tra client e server.

\subsubsection{Vantaggi}

\begin{itemize}
	\item Semplice utilizzo attraverso Javascript;
	\item parsing molto facile nei linguaggi più noti.
\end{itemize}

\subsubsection{Svantaggi}

\subsection{Fabric.js}
Si è stabilito di usare Fabric.js per la gestione dei canvas.

\subsubsection{Vantaggi}

\subsubsection{Svantaggi}

\subsection{Socket.IO}
È una libreria Javascript che permette comunicazioni real-time bidirezionali e basate su eventi.

\subsubsection{Vantaggi}

\begin{itemize}
	\item Permette di creare comunicazioni bidirezionali tra client e server;
	\item gestisce la connessione in modo trasparente.
\end{itemize}

\subsubsection{Svantaggi}

\begin{itemize}
	\item essendo una tecnologia di recente creazione, può soffrire ancora di alcuni bug.
\end{itemize}

\section{Componenti e classi}
\subsection{Client}
Racchiude tutte le componenti necessarie per il front-end del prodotto. Visualizza i dati dell'utente e invia richieste al server.
\begin{figure}[H]
\centering
\noindent\makebox[\textwidth]{\includegraphics[width=\textwidth]{ImgST/quizzipedia-client.pdf}}
\caption{Schema Componente Quizzipedia::Client}
\end{figure}
\subsubsection{Componente Quizzipedia::Client::ModelClient}
Rappresenta il modello dei dati che verranno utilizzati dal sistema lato client.
\begin{figure}[H]
\centering
\noindent\makebox[\textwidth]{\includegraphics[width=\textwidth]{ImgST/quizzipedia-client-modelclient.pdf}}
\caption{Schema Componente Quizzipedia::Client::ModelClient}
\end{figure}
\myparagraph{Componenti contenute}
\begin{itemize}
\item Quizzipedia::Client::ModelClient::Organization
\item Quizzipedia::Client::ModelClient::Requests
\item Quizzipedia::Client::ModelClient::Services
\item Quizzipedia::Client::ModelClient::Statistics
\item Quizzipedia::Client::ModelClient::Users
\end{itemize}
\subsubsection{Componente Quizzipedia::Client::ModelClient::Organization}
La componente gestisce le classi e gli enti, ovvero il sistema in base a cui sono organizzati gli utenti nel sistema.
\begin{figure}[H]
\centering
\noindent\makebox[\textwidth]{\includegraphics[width=\textwidth]{ImgST/quizzipedia-client-modelclient-organization.pdf}}
\caption{Schema Componente Quizzipedia::Client::ModelClient::Organization}
\end{figure}
\myparagraph{Classe Class}
Contiene informazioni relative alla struttura delle classi.
\begin{figure}[H]
\centering
\noindent\makebox[\textwidth]{\includegraphics[width=\textwidth]{ImgST/quizzipedia-client-modelclient-organization-class.pdf}}
\caption{Schema Componente Quizzipedia::Client::ModelClient::Organization::Class}
\end{figure}
\mysubparagraph{Relazioni con altre classi}
\begin{itemize}
\item Quizzipedia::Client::ModelClient::Organization::Institution
\end{itemize}
\subsubsection{Componente Quizzipedia::Client::ModelClient::Requests}
Questo package contiene le classi necessarie a gestire le richieste di ruolo e di classe degli utenti.
\begin{figure}[H]
\centering
\noindent\makebox[\textwidth]{\includegraphics[width=\textwidth]{ImgST/quizzipedia-client-modelclient-requests.pdf}}
\caption{Schema Componente Quizzipedia::Client::ModelClient::Requests}
\end{figure}
\myparagraph{Classe ClassList}
Questa classe gestisce le richieste da parte di Docenti o Studenti per l'assegnazione a una specifica classe.
\begin{figure}[H]
\centering
\noindent\makebox[\textwidth]{\includegraphics[width=\textwidth]{ImgST/quizzipedia-client-modelclient-requests-classlist.pdf}}
\caption{Schema Componente Quizzipedia::Client::ModelClient::Requests::ClassList}
\end{figure}
\mysubparagraph{Relazioni con altre classi}
\begin{itemize}
\item Quizzipedia::Client::ModelClient::Requests::Request
\end{itemize}
\subsubsection{Componente Quizzipedia::Client::ModelClient::Services}
Il package racchiude i modelli necessari alla creazione di domande e quiz, i servizi principali offerti dal nostro prodotto.
\begin{figure}[H]
\centering
\noindent\makebox[\textwidth]{\includegraphics[width=\textwidth]{ImgST/quizzipedia-client-modelclient-services.pdf}}
\caption{Schema Componente Quizzipedia::Client::ModelClient::Services}
\end{figure}
\myparagraph{Componenti contenute}
\begin{itemize}
\item Quizzipedia::Client::ModelClient::Services::Questions
\end{itemize}
\myparagraph{Classe Info}
Riassume le informazioni principali su quiz e domande, necessarie per una presentazione sintetica e puntuale all'utente. È poi possibile risalire alla domanda o al quiz completi.
\begin{figure}[H]
\centering
\noindent\makebox[\textwidth]{\includegraphics[width=\textwidth]{ImgST/quizzipedia-client-modelclient-services-info.pdf}}
\caption{Schema Componente Quizzipedia::Client::ModelClient::Services::Info}
\end{figure}
\subsubsection{Componente Quizzipedia::Client::ModelClient::Services::Questions}
Descrive il modo in cui sono strutturati i vari tipi di domande che l'utente può incontrare durante la creazione o la compilazione di quiz.
\begin{figure}[H]
\centering
\noindent\makebox[\textwidth]{\includegraphics[width=\textwidth]{ImgST/quizzipedia-client-modelclient-services-questions.pdf}}
\caption{Schema Componente Quizzipedia::Client::ModelClient::Services::Questions}
\end{figure}
\myparagraph{Classe Cell}
La classe descrive ogni singola riga (quindi ogni opzione) della colonna della domanda a collegamento.
\begin{figure}[H]
\centering
\noindent\makebox[\textwidth]{\includegraphics[width=\textwidth]{ImgST/quizzipedia-client-modelclient-services-questions-cell.pdf}}
\caption{Schema Componente Quizzipedia::Client::ModelClient::Services::Questions::Cell}
\end{figure}
\subsubsection{Componente Quizzipedia::Client::ModelClient::Statistics}
Qui sono raccolte le classi con il compito di reperire informazioni sulle statistiche dal server e presentarle al'utente finale. Sono disponibili statistiche per le domande, per i quiz e per gli studenti di ogni classe.
\begin{figure}[H]
\centering
\noindent\makebox[\textwidth]{\includegraphics[width=\textwidth]{ImgST/quizzipedia-client-modelclient-statistics.pdf}}
\caption{Schema Componente Quizzipedia::Client::ModelClient::Statistics}
\end{figure}
\myparagraph{Classe ClassQuiz}
La classe raccoglie le statistiche riguardanti gli studenti di una classe relativamente a un quiz assegnato.
\begin{figure}[H]
\centering
\noindent\makebox[\textwidth]{\includegraphics[width=\textwidth]{ImgST/quizzipedia-client-modelclient-statistics-classquiz.pdf}}
\caption{Schema Componente Quizzipedia::Client::ModelClient::Statistics::ClassQuiz}
\end{figure}
\mysubparagraph{Relazioni con altre classi}
\begin{itemize}
\item Quizzipedia::Client::ModelClient::Statistics::StudentStatistics
\end{itemize}
\subsubsection{Componente Quizzipedia::Client::ModelClient::Users}
Raccoglie le classi necessarie a descrivere le diverse tipologie di utente che possono accedere al sistema.
\begin{figure}[H]
\centering
\noindent\makebox[\textwidth]{\includegraphics[width=\textwidth]{ImgST/quizzipedia-client-modelclient-users.pdf}}
\caption{Schema Componente Quizzipedia::Client::ModelClient::Users}
\end{figure}
\myparagraph{Classe AutheniticationData}
Questa classe gestisce le informazioni di autenticazione comuni a tutti gli utenti.
\begin{figure}[H]
\centering
\noindent\makebox[\textwidth]{\includegraphics[width=\textwidth]{ImgST/quizzipedia-client-modelclient-users-autheniticationdata.pdf}}
\caption{Schema Componente Quizzipedia::Client::ModelClient::Users::AutheniticationData}
\end{figure}
\mysubparagraph{Relazioni con altre classi}
\begin{itemize}
\item Quizzipedia::Client::ModelClient::Users::User
\end{itemize}
\subsubsection{Componente Quizzipedia::Client::ViewClient}
Racchiude tutte le componenti necessarie per presentare il prodotto all'utente.
\begin{figure}[H]
\centering
\noindent\makebox[\textwidth]{\includegraphics[width=\textwidth]{ImgST/quizzipedia-client-viewclient.pdf}}
\caption{Schema Componente Quizzipedia::Client::ViewClient}
\end{figure}
\myparagraph{Componenti contenute}
\begin{itemize}
\item Quizzipedia::Client::ViewClient::ViewClassManager
\item Quizzipedia::Client::ViewClient::ViewQuestionManager
\item Quizzipedia::Client::ViewClient::ViewQuizManager
\item Quizzipedia::Client::ViewClient::ViewQuizSolver
\item Quizzipedia::Client::ViewClient::ViewRequests
\item Quizzipedia::Client::ViewClient::ViewSearch
\item Quizzipedia::Client::ViewClient::ViewStatistics
\item Quizzipedia::Client::ViewClient::ViewTopicManager
\item Quizzipedia::Client::ViewClient::ViewUsers
\end{itemize}
\subsubsection{Componente Quizzipedia::Client::ViewClient::ViewClassManager}
Qui sono raccolte le classi responsabili della presentazione delle pagine da cui sarà possibile gestire le classi.
\begin{figure}[H]
\centering
\noindent\makebox[\textwidth]{\includegraphics[width=\textwidth]{ImgST/quizzipedia-client-viewclient-viewclassmanager.pdf}}
\caption{Schema Componente Quizzipedia::Client::ViewClient::ViewClassManager}
\end{figure}
\myparagraph{Classe ViewCreateClass}
Classe responsabile di creare la pagina da cui sarà possibile creare una nuova classe.
\begin{figure}[H]
\centering
\noindent\makebox[\textwidth]{\includegraphics[width=\textwidth]{ImgST/quizzipedia-client-viewclient-viewclassmanager-viewcreateclass.pdf}}
\caption{Schema Componente Quizzipedia::Client::ViewClient::ViewClassManager::ViewCreateClass}
\end{figure}
\subsubsection{Componente Quizzipedia::Client::ViewClient::ViewQuestionManager}
Qui sono raccolte le classi responsabili della presentazione delle pagine da cui sarà possibile gestire le domande.
\begin{figure}[H]
\centering
\noindent\makebox[\textwidth]{\includegraphics[width=\textwidth]{ImgST/quizzipedia-client-viewclient-viewquestionmanager.pdf}}
\caption{Schema Componente Quizzipedia::Client::ViewClient::ViewQuestionManager}
\end{figure}
\myparagraph{Classe ViewCreateQuestion}
Presenta la pagina da cui sarà possibile creare una nuova domanda.
\begin{figure}[H]
\centering
\noindent\makebox[\textwidth]{\includegraphics[width=\textwidth]{ImgST/quizzipedia-client-viewclient-viewquestionmanager-viewcreatequestion.pdf}}
\caption{Schema Componente Quizzipedia::Client::ViewClient::ViewQuestionManager::ViewCreateQuestion}
\end{figure}
\subsubsection{Componente Quizzipedia::Client::ViewClient::ViewQuizManager}
Qui sono raccolte le classi responsabili della presentazione delle pagine da cui sarà possibile gestire i quiz.
\begin{figure}[H]
\centering
\noindent\makebox[\textwidth]{\includegraphics[width=\textwidth]{ImgST/quizzipedia-client-viewclient-viewquizmanager.pdf}}
\caption{Schema Componente Quizzipedia::Client::ViewClient::ViewQuizManager}
\end{figure}
\myparagraph{Classe ViewCreateQuiz}
Presenta la pagina da cui sarà possibile creare un nuovo quiz.
\begin{figure}[H]
\centering
\noindent\makebox[\textwidth]{\includegraphics[width=\textwidth]{ImgST/quizzipedia-client-viewclient-viewquizmanager-viewcreatequiz.pdf}}
\caption{Schema Componente Quizzipedia::Client::ViewClient::ViewQuizManager::ViewCreateQuiz}
\end{figure}
\subsubsection{Componente Quizzipedia::Client::ViewClient::ViewQuizSolver}
Il package raccoglie le classi necessarie alla visualizzazione delle pagine da cui sarà possibile svolgere quiz.
\begin{figure}[H]
\centering
\noindent\makebox[\textwidth]{\includegraphics[width=\textwidth]{ImgST/quizzipedia-client-viewclient-viewquizsolver.pdf}}
\caption{Schema Componente Quizzipedia::Client::ViewClient::ViewQuizSolver}
\end{figure}
\myparagraph{Componenti contenute}
\begin{itemize}
\item Quizzipedia::Client::ViewClient::ViewQuizSolver::ViewQuestionSolver
\end{itemize}
\myparagraph{Classe ViewResults}
La classe ha il compito di costruire la pagina da cui sarà possibile vedere l'esito di un quiz.
\begin{figure}[H]
\centering
\noindent\makebox[\textwidth]{\includegraphics[width=\textwidth]{ImgST/quizzipedia-client-viewclient-viewquizsolver-viewresults.pdf}}
\caption{Schema Componente Quizzipedia::Client::ViewClient::ViewQuizSolver::ViewResults}
\end{figure}
\subsubsection{Componente Quizzipedia::Client::ViewClient::ViewQuizSolver::ViewQuestionSolver}
Il package raccoglie le classi necessarie alla visualizzazione delle pagine da cui sarà possibile rispondere alle singole domande.
\begin{figure}[H]
\centering
\noindent\makebox[\textwidth]{\includegraphics[width=\textwidth]{ImgST/quizzipedia-client-viewclient-viewquizsolver-viewquestionsolver.pdf}}
\caption{Schema Componente Quizzipedia::Client::ViewClient::ViewQuizSolver::ViewQuestionSolver}
\end{figure}
\myparagraph{Classe ViewCompletionQ}
Presenta all'utente la pagina da cui sarà possibile rispondere a una domanda a completamento.
\begin{figure}[H]
\centering
\noindent\makebox[\textwidth]{\includegraphics[width=\textwidth]{ImgST/quizzipedia-client-viewclient-viewquizsolver-viewquestionsolver-viewcompletionq.pdf}}
\caption{Schema Componente Quizzipedia::Client::ViewClient::ViewQuizSolver::ViewQuestionSolver::ViewCompletionQ}
\end{figure}
\subsubsection{Componente Quizzipedia::Client::ViewClient::ViewRequests}
Qui sono raccolte le pagine che permettono all'utente di gestire le richieste di ruolo e classe.
\begin{figure}[H]
\centering
\noindent\makebox[\textwidth]{\includegraphics[width=\textwidth]{ImgST/quizzipedia-client-viewclient-viewrequests.pdf}}
\caption{Schema Componente Quizzipedia::Client::ViewClient::ViewRequests}
\end{figure}
\myparagraph{Classe RequestClass}
Costruisce la pagina da cui l'utente potrà richiedere di entrare in una classe.
\begin{figure}[H]
\centering
\noindent\makebox[\textwidth]{\includegraphics[width=\textwidth]{ImgST/quizzipedia-client-viewclient-viewrequests-requestclass.pdf}}
\caption{Schema Componente Quizzipedia::Client::ViewClient::ViewRequests::RequestClass}
\end{figure}
\subsubsection{Componente Quizzipedia::Client::ViewClient::ViewSearch}
Il package contiene le classi responsabili della creazione delle pagine da cui sarà possibile ricercare domande, quiz e classi .
\begin{figure}[H]
\centering
\noindent\makebox[\textwidth]{\includegraphics[width=\textwidth]{ImgST/quizzipedia-client-viewclient-viewsearch.pdf}}
\caption{Schema Componente Quizzipedia::Client::ViewClient::ViewSearch}
\end{figure}
\myparagraph{Classe ViewSearchClass}
La classe carica la pagina da cui sarà possibile ricercare classi all'interno di un ente.
\begin{figure}[H]
\centering
\noindent\makebox[\textwidth]{\includegraphics[width=\textwidth]{ImgST/quizzipedia-client-viewclient-viewsearch-viewsearchclass.pdf}}
\caption{Schema Componente Quizzipedia::Client::ViewClient::ViewSearch::ViewSearchClass}
\end{figure}
\subsubsection{Componente Quizzipedia::Client::ViewClient::ViewStatistics}
Package che gestisce le pagine in cui verranno visualizzate le statistiche.
\begin{figure}[H]
\centering
\noindent\makebox[\textwidth]{\includegraphics[width=\textwidth]{ImgST/quizzipedia-client-viewclient-viewstatistics.pdf}}
\caption{Schema Componente Quizzipedia::Client::ViewClient::ViewStatistics}
\end{figure}
\myparagraph{Classe VieewQuizStats}
Vengono rappresentate le statistiche generali riguardanti i quiz.
\begin{figure}[H]
\centering
\noindent\makebox[\textwidth]{\includegraphics[width=\textwidth]{ImgST/quizzipedia-client-viewclient-viewstatistics-vieewquizstats.pdf}}
\caption{Schema Componente Quizzipedia::Client::ViewClient::ViewStatistics::VieewQuizStats}
\end{figure}
\subsubsection{Componente Quizzipedia::Client::ViewClient::ViewTopicManager}
Qui sono raccolte le classi responsabili della presentazione delle pagine da cui sarà possibile gestire gli argomenti di domande e quiz.
\begin{figure}[H]
\centering
\noindent\makebox[\textwidth]{\includegraphics[width=\textwidth]{ImgST/quizzipedia-client-viewclient-viewtopicmanager.pdf}}
\caption{Schema Componente Quizzipedia::Client::ViewClient::ViewTopicManager}
\end{figure}
\myparagraph{Classe ViewCreateTopic}
La classe caricherà la pagina da cui sarà possibile creare un nuovo argomento.
\begin{figure}[H]
\centering
\noindent\makebox[\textwidth]{\includegraphics[width=\textwidth]{ImgST/quizzipedia-client-viewclient-viewtopicmanager-viewcreatetopic.pdf}}
\caption{Schema Componente Quizzipedia::Client::ViewClient::ViewTopicManager::ViewCreateTopic}
\end{figure}
\subsubsection{Componente Quizzipedia::Client::ViewClient::ViewUsers}
Raccoglie le classi necessarie a presentare all'utente le pagine da cui visualizzare le informazioni che lo riguardano e compiere le funzioni principali.
\begin{figure}[H]
\centering
\noindent\makebox[\textwidth]{\includegraphics[width=\textwidth]{ImgST/quizzipedia-client-viewclient-viewusers.pdf}}
\caption{Schema Componente Quizzipedia::Client::ViewClient::ViewUsers}
\end{figure}
\myparagraph{Classe Login}
Presenta la pagina necessaria per effettuare il login nel sistema.
\begin{figure}[H]
\centering
\noindent\makebox[\textwidth]{\includegraphics[width=\textwidth]{ImgST/quizzipedia-client-viewclient-viewusers-login.pdf}}
\caption{Schema Componente Quizzipedia::Client::ViewClient::ViewUsers::Login}
\end{figure}
\subsubsection{Componente Quizzipedia::Client::ControllerClient}
Raccoglie le classi responsabili della comunicazione tra il model e la view.
\begin{figure}[H]
\centering
\noindent\makebox[\textwidth]{\includegraphics[width=\textwidth]{ImgST/quizzipedia-client-controllerclient.pdf}}
\caption{Schema Componente Quizzipedia::Client::ControllerClient}
\end{figure}
\myparagraph{Componenti contenute}
\begin{itemize}
\item Quizzipedia::Client::ControllerClient::CtrlOrganization
\item Quizzipedia::Client::ControllerClient::CtrlRequests
\item Quizzipedia::Client::ControllerClient::CtrlServices
\item Quizzipedia::Client::ControllerClient::CtrlStatistics
\item Quizzipedia::Client::ControllerClient::CtrlUsers
\end{itemize}
\subsubsection{Componente Quizzipedia::Client::ControllerClient::CtrlOrganization}
Raccoglie le classi che si occupano delle comunicazioni necessarie per la creazione di enti e classi.
\begin{figure}[H]
\centering
\noindent\makebox[\textwidth]{\includegraphics[width=\textwidth]{ImgST/quizzipedia-client-controllerclient-ctrlorganization.pdf}}
\caption{Schema Componente Quizzipedia::Client::ControllerClient::CtrlOrganization}
\end{figure}
\myparagraph{Classe CtrlClass}
Vi sono presenti metodi necessari alla creazione delle classi.
\begin{figure}[H]
\centering
\noindent\makebox[\textwidth]{\includegraphics[width=\textwidth]{ImgST/quizzipedia-client-controllerclient-ctrlorganization-ctrlclass.pdf}}
\caption{Schema Componente Quizzipedia::Client::ControllerClient::CtrlOrganization::CtrlClass}
\end{figure}
\subsubsection{Componente Quizzipedia::Client::ControllerClient::CtrlRequests}
Questo package contiene classi necessarie alla gestione delle richieste di ruolo o classe.
\begin{figure}[H]
\centering
\noindent\makebox[\textwidth]{\includegraphics[width=\textwidth]{ImgST/quizzipedia-client-controllerclient-ctrlrequests.pdf}}
\caption{Schema Componente Quizzipedia::Client::ControllerClient::CtrlRequests}
\end{figure}
\myparagraph{Classe CtrlRequestClass}
Si occupa delle comunicazioni necessarie per la gestione delle richieste di inserimento in una classe.
\begin{figure}[H]
\centering
\noindent\makebox[\textwidth]{\includegraphics[width=\textwidth]{ImgST/quizzipedia-client-controllerclient-ctrlrequests-ctrlrequestclass.pdf}}
\caption{Schema Componente Quizzipedia::Client::ControllerClient::CtrlRequests::CtrlRequestClass}
\end{figure}
\subsubsection{Componente Quizzipedia::Client::ControllerClient::CtrlServices}
Raccoglie gli elementi necessari alla creazione, svolgimento e ricerca di quiz e domande.
\begin{figure}[H]
\centering
\noindent\makebox[\textwidth]{\includegraphics[width=\textwidth]{ImgST/quizzipedia-client-controllerclient-ctrlservices.pdf}}
\caption{Schema Componente Quizzipedia::Client::ControllerClient::CtrlServices}
\end{figure}
\myparagraph{Classe CtrlQuestion}
Fornisce i metodi necessari per la comunicazione tra view e model durante la creazione, modifica e svolgimento di una domanda.
\begin{figure}[H]
\centering
\noindent\makebox[\textwidth]{\includegraphics[width=\textwidth]{ImgST/quizzipedia-client-controllerclient-ctrlservices-ctrlquestion.pdf}}
\caption{Schema Componente Quizzipedia::Client::ControllerClient::CtrlServices::CtrlQuestion}
\end{figure}
\subsubsection{Componente Quizzipedia::Client::ControllerClient::CtrlStatistics}
Raccoglie le classi necessarie a recuperare le statistiche da presentare all'utente.
\begin{figure}[H]
\centering
\noindent\makebox[\textwidth]{\includegraphics[width=\textwidth]{ImgST/quizzipedia-client-controllerclient-ctrlstatistics.pdf}}
\caption{Schema Componente Quizzipedia::Client::ControllerClient::CtrlStatistics}
\end{figure}
\myparagraph{Classe CtrlStats}
Classe necessaria al caricamento delle statistiche relative ai quiz, alle domande e alle classi.
\begin{figure}[H]
\centering
\noindent\makebox[\textwidth]{\includegraphics[width=\textwidth]{ImgST/quizzipedia-client-controllerclient-ctrlstatistics-ctrlstats.pdf}}
\caption{Schema Componente Quizzipedia::Client::ControllerClient::CtrlStatistics::CtrlStats}
\end{figure}
\subsubsection{Componente Quizzipedia::Client::ControllerClient::CtrlUsers}
Il package raccoglie le classi che permettono la comunicazione per quanto riguarda funzioni e dati dell'utente.
\begin{figure}[H]
\centering
\noindent\makebox[\textwidth]{\includegraphics[width=\textwidth]{ImgST/quizzipedia-client-controllerclient-ctrlusers.pdf}}
\caption{Schema Componente Quizzipedia::Client::ControllerClient::CtrlUsers}
\end{figure}
\myparagraph{Classe CtrlData}
La classe raccoglie i metodi per gestire le informazioni personali di tutti gli utenti autenticati.
\begin{figure}[H]
\centering
\noindent\makebox[\textwidth]{\includegraphics[width=\textwidth]{ImgST/quizzipedia-client-controllerclient-ctrlusers-ctrldata.pdf}}
\caption{Schema Componente Quizzipedia::Client::ControllerClient::CtrlUsers::CtrlData}
\end{figure}
\subsection{Server}
Racchiude tutte le componenti necessarie per il back-end del prodotto. Contiene anche le componenti che si occupano del QML.
\begin{figure}[H]
\centering
\noindent\makebox[\textwidth]{\includegraphics[width=\textwidth]{ImgST/quizzipedia-server.pdf}}
\caption{Schema Componente Quizzipedia::Server}
\end{figure}
\subsubsection{Componente Quizzipedia::Server::ModelServer}
Rappresenta il modello dei dati che verranno utilizzati dal sistema lato server.
\begin{figure}[H]
\centering
\noindent\makebox[\textwidth]{\includegraphics[width=\textwidth]{ImgST/quizzipedia-server-modelserver.pdf}}
\caption{Schema Componente Quizzipedia::Server::ModelServer}
\end{figure}
\myparagraph{Componenti contenute}
\begin{itemize}
\item Quizzipedia::Server::ModelServer::Services1
\end{itemize}
\subsubsection{Componente Quizzipedia::Server::ModelServer::Services1}
Il package racchiude i modelli necessari alla creazione di domande e quiz, i servizi principali offerti dal nostro prodotto.
\begin{figure}[H]
\centering
\noindent\makebox[\textwidth]{\includegraphics[width=\textwidth]{ImgST/quizzipedia-server-modelserver-services1.pdf}}
\caption{Schema Componente Quizzipedia::Server::ModelServer::Services1}
\end{figure}
\myparagraph{Componenti contenute}
\begin{itemize}
\item Quizzipedia::Server::ModelServer::Services1::Questions1
\end{itemize}
\myparagraph{Classe Topics1}
Modella la struttura necessaria a memorizzare la lista di argomenti. A ogni domanda e a ogni quiz verranno poi associati i relativi argomenti.
\begin{figure}[H]
\centering
\noindent\makebox[\textwidth]{\includegraphics[width=\textwidth]{ImgST/quizzipedia-server-modelserver-services1-topics1.pdf}}
\caption{Schema Componente Quizzipedia::Server::ModelServer::Services1::Topics1}
\end{figure}
\subsubsection{Componente Quizzipedia::Server::ModelServer::Services1::Questions1}
Descrive il modo in cui sono strutturati i vari tipi di domande che l'utente può incontrare durante la creazione o la compilazione di quiz.
\begin{figure}[H]
\centering
\noindent\makebox[\textwidth]{\includegraphics[width=\textwidth]{ImgST/quizzipedia-server-modelserver-services1-questions1.pdf}}
\caption{Schema Componente Quizzipedia::Server::ModelServer::Services1::Questions1}
\end{figure}
\subsubsection{Componente Quizzipedia::Server::ControllerServer}
Questo package contiene tutti i servizi che ricalcano il pattern architetturale DAO in modo da isolare l'accesso al database relazionale, controllando sempre che l'utente che genera una determinata richiesta al server sia abilitato per farla.
\begin{figure}[H]
\centering
\noindent\makebox[\textwidth]{\includegraphics[width=\textwidth]{ImgST/quizzipedia-server-controllerserver.pdf}}
\caption{Schema Componente Quizzipedia::Server::ControllerServer}
\end{figure}
\myparagraph{Componenti contenute}
\begin{itemize}
\item Quizzipedia::Server::ControllerServer::AuthenticationManager
\item Quizzipedia::Server::ControllerServer::ClassManager
\item Quizzipedia::Server::ControllerServer::ProfileManager
\item Quizzipedia::Server::ControllerServer::QuizManager
\item Quizzipedia::Server::ControllerServer::RequestsManager
\item Quizzipedia::Server::ControllerServer::SearchManager
\item Quizzipedia::Server::ControllerServer::StatisticsManager
\item Quizzipedia::Server::ControllerServer::TopicManager
\end{itemize}
\subsubsection{Componente Quizzipedia::Server::ControllerServer::AuthenticationManager}
Package che permette di gestire le funzioni basi per una corretta autenticazione al sistema.
\begin{figure}[H]
\centering
\noindent\makebox[\textwidth]{\includegraphics[width=\textwidth]{ImgST/quizzipedia-server-controllerserver-authenticationmanager.pdf}}
\caption{Schema Componente Quizzipedia::Server::ControllerServer::AuthenticationManager}
\end{figure}
\myparagraph{Classe LoggerIn}
Permette l'autenticazione nel sistema da parte di utenti preventivamente registrati.
\begin{figure}[H]
\centering
\noindent\makebox[\textwidth]{\includegraphics[width=\textwidth]{ImgST/quizzipedia-server-controllerserver-authenticationmanager-loggerin.pdf}}
\caption{Schema Componente Quizzipedia::Server::ControllerServer::AuthenticationManager::LoggerIn}
\end{figure}
\subsubsection{Componente Quizzipedia::Server::ControllerServer::ClassManager}
Package che racchiude tutte le funzionalità adibite al salvataggio e alla visualizzazione delle informazioni riguardanti le classi di un ente.
\begin{figure}[H]
\centering
\noindent\makebox[\textwidth]{\includegraphics[width=\textwidth]{ImgST/quizzipedia-server-controllerserver-classmanager.pdf}}
\caption{Schema Componente Quizzipedia::Server::ControllerServer::ClassManager}
\end{figure}
\myparagraph{Classe AddClass}
Permette la creazione di una nuova classe.
\begin{figure}[H]
\centering
\noindent\makebox[\textwidth]{\includegraphics[width=\textwidth]{ImgST/quizzipedia-server-controllerserver-classmanager-addclass.pdf}}
\caption{Schema Componente Quizzipedia::Server::ControllerServer::ClassManager::AddClass}
\end{figure}
\subsubsection{Componente Quizzipedia::Server::ControllerServer::ProfileManager}
Package che racchiude tutte le funzionalità adibite al salvataggio e alla visualizzazione delle informazioni personali da parte di un utente autenticato.
\begin{figure}[H]
\centering
\noindent\makebox[\textwidth]{\includegraphics[width=\textwidth]{ImgST/quizzipedia-server-controllerserver-profilemanager.pdf}}
\caption{Schema Componente Quizzipedia::Server::ControllerServer::ProfileManager}
\end{figure}
\myparagraph{Classe AccountDeleter}
Permette la rimozione di un account dal sistema.
\begin{figure}[H]
\centering
\noindent\makebox[\textwidth]{\includegraphics[width=\textwidth]{ImgST/quizzipedia-server-controllerserver-profilemanager-accountdeleter.pdf}}
\caption{Schema Componente Quizzipedia::Server::ControllerServer::ProfileManager::AccountDeleter}
\end{figure}
\subsubsection{Componente Quizzipedia::Server::ControllerServer::QuizManager}
Package che racchiude tutte le funzionalità adibite alla creazione, modifica e al recupero di quiz per lo svolgimento da parte di un utente.
\begin{figure}[H]
\centering
\noindent\makebox[\textwidth]{\includegraphics[width=\textwidth]{ImgST/quizzipedia-server-controllerserver-quizmanager.pdf}}
\caption{Schema Componente Quizzipedia::Server::ControllerServer::QuizManager}
\end{figure}
\myparagraph{Componenti contenute}
\begin{itemize}
\item Quizzipedia::Server::ControllerServer::QuizManager::QMLAgent
\end{itemize}
\myparagraph{Classe QuizCreator}
Permette il salvataggio nella base di dati di un nuovo quiz.
\begin{figure}[H]
\centering
\noindent\makebox[\textwidth]{\includegraphics[width=\textwidth]{ImgST/quizzipedia-server-controllerserver-quizmanager-quizcreator.pdf}}
\caption{Schema Componente Quizzipedia::Server::ControllerServer::QuizManager::QuizCreator}
\end{figure}
\subsubsection{Componente Quizzipedia::Server::ControllerServer::QuizManager::QMLAgent}
Questo package racchiude i moduli necessari alla traduzione da QML ad un formato comprensibile dal sistema le informazioni estratte dal database per la generazione delle pagine HTML relative ad un quiz e viceversa.
\begin{figure}[H]
\centering
\noindent\makebox[\textwidth]{\includegraphics[width=\textwidth]{ImgST/quizzipedia-server-controllerserver-quizmanager-qmlagent.pdf}}
\caption{Schema Componente Quizzipedia::Server::ControllerServer::QuizManager::QMLAgent}
\end{figure}
\myparagraph{Classe QMLGenerator}
Permette la traduzione in formato QML di un quiz nel caso si voglia procedere al salvataggio dello stesso all'interno del database.
\begin{figure}[H]
\centering
\noindent\makebox[\textwidth]{\includegraphics[width=\textwidth]{ImgST/quizzipedia-server-controllerserver-quizmanager-qmlagent-qmlgenerator.pdf}}
\caption{Schema Componente Quizzipedia::Server::ControllerServer::QuizManager::QMLAgent::QMLGenerator}
\end{figure}
\subsubsection{Componente Quizzipedia::Server::ControllerServer::RequestsManager}
Package che si occupa di memorizzare richieste da parte degli utenti, mostrarle al responsabile e permettergli di accettarle o meno.
\begin{figure}[H]
\centering
\noindent\makebox[\textwidth]{\includegraphics[width=\textwidth]{ImgST/quizzipedia-server-controllerserver-requestsmanager.pdf}}
\caption{Schema Componente Quizzipedia::Server::ControllerServer::RequestsManager}
\end{figure}
\myparagraph{Classe ClassRequestsAdder}
Permette la memorizzazione delle richieste di creazione di una classe.
\begin{figure}[H]
\centering
\noindent\makebox[\textwidth]{\includegraphics[width=\textwidth]{ImgST/quizzipedia-server-controllerserver-requestsmanager-classrequestsadder.pdf}}
\caption{Schema Componente Quizzipedia::Server::ControllerServer::RequestsManager::ClassRequestsAdder}
\end{figure}
\subsubsection{Componente Quizzipedia::Server::ControllerServer::SearchManager}
Questo package permette di effettuare una ricerca nel database di quiz o domande richiesti dall'utente e ritornare una lista che corrisponde ai parametri desiderati.
\begin{figure}[H]
\centering
\noindent\makebox[\textwidth]{\includegraphics[width=\textwidth]{ImgST/quizzipedia-server-controllerserver-searchmanager.pdf}}
\caption{Schema Componente Quizzipedia::Server::ControllerServer::SearchManager}
\end{figure}
\myparagraph{Classe QuestionSearcher}
Ritorna una lista di domande che corrispondono ai parametri di ricerca impostati dall'utente.
\begin{figure}[H]
\centering
\noindent\makebox[\textwidth]{\includegraphics[width=\textwidth]{ImgST/quizzipedia-server-controllerserver-searchmanager-questionsearcher.pdf}}
\caption{Schema Componente Quizzipedia::Server::ControllerServer::SearchManager::QuestionSearcher}
\end{figure}
\subsubsection{Componente Quizzipedia::Server::ControllerServer::StatisticsManager}
Questo package ha il compito di recuperare tutte le informazioni sottoforma di statistiche relative ad un quiz, una domanda in particolare o ad un utente.
\begin{figure}[H]
\centering
\noindent\makebox[\textwidth]{\includegraphics[width=\textwidth]{ImgST/quizzipedia-server-controllerserver-statisticsmanager.pdf}}
\caption{Schema Componente Quizzipedia::Server::ControllerServer::StatisticsManager}
\end{figure}
\myparagraph{Classe PersonalStatisticsFetcher}
Ritorna le statistiche personali.
\begin{figure}[H]
\centering
\noindent\makebox[\textwidth]{\includegraphics[width=\textwidth]{ImgST/quizzipedia-server-controllerserver-statisticsmanager-personalstatisticsfetcher.pdf}}
\caption{Schema Componente Quizzipedia::Server::ControllerServer::StatisticsManager::PersonalStatisticsFetcher}
\end{figure}
\subsubsection{Componente Quizzipedia::Server::ControllerServer::TopicManager}
Package che permette la creazione di un nuovo argomento o l'eliminazione di uno già esistente.
\begin{figure}[H]
\centering
\noindent\makebox[\textwidth]{\includegraphics[width=\textwidth]{ImgST/quizzipedia-server-controllerserver-topicmanager.pdf}}
\caption{Schema Componente Quizzipedia::Server::ControllerServer::TopicManager}
\end{figure}
\myparagraph{Classe SessionController8}
Effettua il controllo sull'utente per verificare che egli sia in possesso dell'autorizzazione necessaria per compiere determinate richieste alla base di dati.
\begin{figure}[H]
\centering
\noindent\makebox[\textwidth]{\includegraphics[width=\textwidth]{ImgST/quizzipedia-server-controllerserver-topicmanager-sessioncontroller8.pdf}}
\caption{Schema Componente Quizzipedia::Server::ControllerServer::TopicManager::SessionController8}
\end{figure}
\subsubsection{Componente Quizzipedia::Server::RoutingManager}
Questo pachetto costituisce lo strato superiore a ControllerServer e contiene tutte le API necessarie per comunicare con il client tramite socket; esso ha il compito di indirizzare le richieste ai vari services un base alla richiesta da parte dell'utente.
\begin{figure}[H]
\centering
\noindent\makebox[\textwidth]{\includegraphics[width=\textwidth]{ImgST/quizzipedia-server-routingmanager.pdf}}
\caption{Schema Componente Quizzipedia::Server::RoutingManager}
\end{figure}
\myparagraph{Classe AuthenticationRouter}
Invocato dal client per interagire con AuthenticationManager.
\begin{figure}[H]
\centering
\noindent\makebox[\textwidth]{\includegraphics[width=\textwidth]{ImgST/quizzipedia-server-routingmanager-authenticationrouter.pdf}}
\caption{Schema Componente Quizzipedia::Server::RoutingManager::AuthenticationRouter}
\end{figure}


\newpage
\section{Diagrammi di attività}
Questa sezione descriverà le operazioni permesse ai vari tipi di utente durante l’utilizzo di Quizzipedia. Ogni operazione principale sarà accompagnata da un diagramma di attività, le operazioni più complesse sono indicate con un " * " e sfondo bianco e sono ulteriormente descritte da sotto-diagrammi.

\subsection{Attività utente}
\begin{figure}[H]
	\centering
	\noindent\makebox[\textwidth]{\includegraphics[scale=0.7]{Img/attivita_utente.pdf}}
	\caption{Diagramma di attività utente}
\end{figure}
All’avvio dell’applicazione l’utente non è autenticato nel sistema, e gli è possibile:
\begin{itemize}
	\item registrarsi;
	\item autenticarsi;
	\item effettuare il recupero della password;
	\item svolgere quiz pubblici.
\end{itemize}
Una volta effettuata un’operazione, l’utente può svolgerne un’altra o interrompere l’utilizzo di Quizzipedia.

\newpage
\subsubsection{Registrazione}
\begin{figure}[H]
	\centering
	\noindent\makebox[\textwidth]{\includegraphics[scale=0.5]{Img/registrazione.pdf}}
	\caption{Diagramma di registrazione}
\end{figure}
Un utente può registrarsi nel sistema. Per effettuare la registrazione è necessario compilare un apposito form, inserendo:
\begin{itemize}
	\item nome;
	\item cognome;
	\item indirizzo email;
	\item conferma dell'indirizzo email;
	\item password;
	\item conferma della password.
\end{itemize}
Dovrà infine confermare la richiesta di registrazione. 
Se l’indirizzo email inserito è non valido o corrisponde a un account già esistente, la password non è valida o le conferme di email e password non corrispondono ai valori precedentemente inseriti, la registrazione non sarà completata e verrà visualizzato un messaggio di errore. 
In caso contrario, la registrazione andrà a buon fine e l’utente otterrà un account a cui potrà accedere autenticandosi con le credenziali fornite. 

\newpage
\subsubsection{Recupero password}
\begin{figure}[H]
	\centering
	\noindent\makebox[\textwidth]{\includegraphics[width=\textwidth]{Img/recupero_password.pdf}}
	\caption{Diagramma di recupero password}
\end{figure}
L’utente, se non riesce ad autenticarsi, può effettuare il recupero della password. Per recuperare la propria password, l’utente deve inserire il proprio indirizzo email in un apposito form ed effettuare la richiesta di recupero. 
Se l’indirizzo email corrisponde a un account registrato, una password generata casualmente sarà inviata all’indirizzo email inserito. L’utente potrà poi autenticarsi con quella password al posto di quella dimenticata. Se l’indirizzo email è scorretto verrà visualizzato un messaggio di errore e la password non sarà inviata.

\newpage
\subsubsection{Autenticazione}
\begin{figure}[H]
	\centering
	\noindent\makebox[\textwidth]{\includegraphics[scale=0.8]{Img/autenticazione.pdf}}
	\caption{Diagramma di autenticazione}
\end{figure}
Un utente si può autenticare nel sistema. Per autenticarsi l'utente deve inserire il proprio indirizzo email e la propria password in un apposito form di login e premere il pulsante di login. Se i dati sono corretti l'utente sarà autenticato nel sistema e potrà svolgere le attività permesse agli utenti autenticati, altrimenti visualizzerà un messaggio di errore che lo avviserà che le credenziali inserite sono errate. 

\newpage
\subsubsection{Svolgimento quiz}
\begin{figure}[H]
	\centering
	\noindent\makebox[\textwidth]{\includegraphics[scale=0.7]{Img/svolgimento_quiz.pdf}}
	\caption{Diagramma di svolgimento quiz}
\end{figure}
Gli utenti possono svolgere dei quiz. Se l’utente è uno studente e fa parte di qualche classe, egli potrà visualizzare i quiz privati disponibili per ogni classe a cui appartiene.
Utenti di qualsiasi tipo, anche non registrati, possono invece cercare quiz pubblici. 
L’utente, una volta scelto il quiz da svolgere, dovrà selezionarlo e confermare l’inizio dello svolgimento. Dovrà poi risolvere ogni domanda e confermare il termine dello svolgimento del quiz, e potrà infine visualizzare il proprio esito.

\newpage
\subsection{Attività utente autenticato}
\begin{figure}[H]
	\centering
	\noindent\makebox[\textwidth]{\includegraphics[width=\textwidth]{Img/attivita_utente_autenticato.pdf}}
	\caption{Diagramma di attività utente autenticato}
\end{figure}
L’utente autenticato è in possesso di un account e si è autenticato nel sistema. Egli può:
\begin{itemize} 
	\item cercare e svolgere quiz in base ai propri permessi; 
	\item visualizzare le proprie informazioni personali;
	\item modificare la propria password;
	\item chiedere ulteriori funzionalità al responsabile;
	\item effettuare il logout.
\end{itemize} 
Ogni utente autenticato potrà inoltre svolgere le attività riservate alla sua tipologia (studente, docente, responsabile, utente senza ruolo).

\newpage
\subsubsection{Modifica password}
\begin{figure}[H]
	\centering
	\noindent\makebox[\textwidth]{\includegraphics[scale=0.6]{Img/modifica_password.pdf}}
	\caption{Diagramma di modifica password}
\end{figure}
Ogni utente in possesso di un account e autenticato può modificare la propria password. La modifica della password richiede l’inserimento della password correntemente valida (scelta dall’utente o fornita dal sistema grazie al recupero password), seguito dall’inserimento della nuova password e la conferma della nuova password. Se la vecchia password è corretta, la nuova password ha un formato valido e la conferma della nuova password corrisponde a quella precedentemente inserita, la password verrà modificata correttamente, altrimenti l’utente visualizzerà un messaggio di errore.

\newpage
\subsubsection{Visualizzazione informazioni profilo}
\begin{figure}[H]
	\centering
	\noindent\makebox[\textwidth]{\includegraphics[width=\textwidth]{Img/visualizzazione_info_profilo.pdf}}
	\caption{Diagramma di visualizzazione informazioni profilo}
\end{figure}
Ogni utente autenticato può visualizzare le proprie informazioni personali: 
\begin{itemize}
	\item nome;
	\item cognome;
	\item indirizzo email.
\end{itemize}

\newpage
\subsubsection{Invio richiesta di ruolo}
\begin{figure}[H]
	\centering
	\noindent\makebox[\textwidth]{\includegraphics[scale=0.6]{Img/invio_richiesta.pdf}}
	\caption{Diagramma di invio richiesta di ruolo}
\end{figure}
Gli utenti autenticati possono effettuare delle richieste. Per prima cosa cercheranno l’ente a cui vogliono appartenere, visualizzeranno i risultati e selezioneranno l’ente. In seguito, se l’utente che sta effettuando la richiesta non ha un ruolo associato all’ente selezionato, egli potrà effettuare una richiesta di ruolo scegliendo fra studente e docente. Se l’utente è già in possesso di un ruolo nell’ente selezionato, egli dovrà cercare una classe interna all’ente e selezionarla per richiedere di essere associato a quella classe secondo il suo ruolo (studente o docente). Una volta selezionati tutti i parametri necessari, l’utente dovrà confermare l’invio della propria richiesta.

\newpage
\subsection{Attività studente}

\subsubsection{Visualizzazione storico quiz}
\begin{figure}[H]
	\centering
	\noindent\makebox[\textwidth]{\includegraphics[width=\textwidth]{Img/visualizzazione_storico_quiz.pdf}}
	\caption{Diagramma di visualizzazione storico quiz}
\end{figure}
Gli studenti possono accedere a uno storico dei quiz svolti, sia pubblici che privati. Una volta visualizzata la lista dei quiz svolti e selezionato uno dei quiz, lo studente potrà vedere il risultato corrispondente ottenuto. 

\newpage
\subsection{Attività docente}
\begin{figure}[H]
	\centering
	\noindent\makebox[\textwidth]{\includegraphics[width=\textwidth]{Img/attivita_docente.pdf}}
	\caption{Diagramma di attività docente}
\end{figure}
Al docente è stato approvato il ruolo di docente da parte del responsabile. Un docente può essere docente di classi multiple. La richiesta di associazione a una classe deve essere prima approvata dal responsabile. 
Un docente può:
\begin{itemize}
	\item gestire le proprie domande;
	\item gestire i propri quiz;
	\item gestire le richieste degli studenti;
	\item visualizzare vari tipi di statistiche.
\end{itemize}

\newpage
\subsubsection{Gestione quiz}
\begin{figure}[H]
	\centering
	\noindent\makebox[\textwidth]{\includegraphics[width=\textwidth]{Img/gestione_quiz.pdf}}
	\caption{Diagramma di gestione quiz}
\end{figure}
Il docente può creare nuovi quiz e modificare o eliminare i quiz creati in precedenza. Per modificare o eliminare un quiz, il docente deve prima visualizzare la lista dei quiz creati e selezionare il quiz desiderato.

\newpage
\myparagraph{Creazione quiz}
\begin{figure}[H]
	\centering
	\noindent\makebox[\textwidth]{\includegraphics[scale=0.67]{Img/creazione_quiz.pdf}}
	\caption{Diagramma di creazione quiz}
\end{figure}
Il docente può creare un quiz, inserendo:
\begin{itemize}
	\item il titolo del quiz; 
	\item l’argomento;
	\item una descrizione facoltativa;
	\item le domande (sia proprie che di altri docenti). 
\end{itemize}
Per inserire domande di altri docenti è necessario effettuare una ricerca delle domande e selezionare la domanda da inserire fra i risultati. Al termine dell’inserimento delle domande, il docente dovrà specificare il permesso del quiz. Se il docente seleziona il permesso pubblico il quiz sarà accessibile da chiunque, se seleziona il permesso privato dovrà specificare ulteriormente le classi a cui è riservato l’accesso al quiz. Il quiz potrà essere svolto dagli studenti di quelle classi. Se i vari campi sono inseriti correttamente e il numero di domande inserite è corretto, il quiz sarà creato con successo, altrimenti il docente visualizzerà un messaggio di errore.

\newpage
\myparagraph{Modifica quiz}
\begin{figure}[H]
	\centering
	\noindent\makebox[\textwidth]{\includegraphics[width=\textwidth]{Img/modifica_quiz.pdf}}
	\caption{Diagramma di modifica quiz}
\end{figure}
Il docente può modificare un quiz da lui creato, in particolare potrà modificare:
\begin{itemize}
	\item il titolo del quiz;
	\item l’argomento;
	\item una descrizione facoltativa;
	\item le domande (aggiungere domande o rimuoverle);
	\item il permesso del quiz.
\end{itemize}
Se i vari campi sono modificati correttamente e il numero di domande presenti è corretto, il quiz sarà modificato con successo, altrimenti il docente visualizzerà un messaggio di errore.

\newpage
\subsubsection{Gestione domande}
\begin{figure}[H]
	\centering
	\noindent\makebox[\textwidth]{\includegraphics[width=\textwidth]{Img/gestione_domande.pdf}}
	\caption{Diagramma di gestione domande}
\end{figure}
Il docente può creare nuove domande e modificare o eliminare le domande che ha creato in precedenza. Per modificare o eliminare una domanda, il docente deve prima visualizzare l’elenco di domande create e selezionare la domanda su cui effettuare l’operazione.

\newpage
\myparagraph{Creazione domanda}
\begin{figure}[H]
	\centering
	\noindent\makebox[\textwidth]{\includegraphics[width=\textwidth]{Img/creazione_domanda.pdf}}
	\caption{Diagramma di creazione domanda}
\end{figure}
Un docente può creare domande di vario tipo:
\begin{itemize}
	\item vero o falso;
	\item a risposta multipla;
	\item a collegamenti;
	\item a completamento testo;
	\item a risposta aperta.
\end{itemize}

\newpage
\mysubparagraph{Creazione domanda vero/falso}
\begin{figure}[H]
	\centering
	\noindent\makebox[\textwidth]{\includegraphics[scale=0.5]{Img/creazione_domanda_vf.pdf}}
	\caption{Diagramma di creazione domanda vero/falso}
\end{figure}
Un docente può creare una domanda vero o falso. Per creare la domanda il docente inserirà in un apposito form:
\begin{itemize}
	\item il titolo della domanda;
	\item una descrizione facoltativa;
	\item l'argomento della domanda; 
	\item il livello di difficoltà;
	\item un allegato (immagine, audio o video) facoltativo;
	\item delle parole chiave che faciliteranno la ricerca della domanda.
\end{itemize}
Il docente dovrà infine specificare la veridicità della domanda (se la risposta corretta è vero o falso). 
Se i vari campi sono inseriti correttamente, la domanda sarà creata con successo, altrimenti il docente visualizzerà un messaggio di errore.

\newpage
\mysubparagraph{Creazione domanda a completamento testo}
\begin{figure}[H]
	\centering
	\noindent\makebox[\textwidth]{\includegraphics[scale=0.58]{Img/creazione_domanda_completamento.pdf}}
	\caption{Diagramma di creazione domanda a completamento testo}
\end{figure}
Un docente può creare una domanda completamento testo. Per creare la domanda il docente inserirà in un apposito form:
\begin{itemize}
	\item il titolo della domanda;
	\item una descrizione facoltativa; 
	\item l'argomento della domanda; 
	\item il livello di difficoltà;
	\item un allegato (immagine, audio o video) facoltativo;
	\item delle parole chiave che faciliteranno la ricerca della domanda;
	\item il testo incompleto (compreso di spazi dove inserire le parole);
	\item le parole da inserire. 
\end{itemize}
Gli spazi dovranno poi essere collegati con le parole corrette. Se i vari campi sono inseriti correttamente, la domanda sarà creata con successo, altrimenti il docente visualizzerà un messaggio di errore.

\newpage
\mysubparagraph{Creazione domanda a collegamenti}
\begin{figure}[H]
	\centering
	\noindent\makebox[\textwidth]{\includegraphics[scale=0.5]{Img/creazione_domanda_collegamenti.pdf}}
	\caption{Diagramma di creazione domanda a collegamenti}
\end{figure}
Un docente può creare una domanda a collegamenti. Per creare la domanda il docente inserirà in un apposito form:
\begin{itemize} 
	\item il titolo della domanda;
	\item una descrizione facoltativa; 
	\item l'argomento della domanda;
	\item il livello di difficoltà;
	\item un allegato (immagine, audio o video) facoltativo;
	\item delle parole chiave che faciliteranno la ricerca della domanda;
	\item le ennuple che rappresentano le risposte corrette. 
\end{itemize}
Le ennuple sono composte da una parte iniziale e una finale, per ciascuna delle quale si può caricare un file multimediale (immagine, audio, video) o inserire del testo.
Se i vari campi sono inseriti correttamente, la domanda sarà creata con successo, altrimenti il docente visualizzerà un messaggio di errore.

\newpage
\mysubparagraph{Creazione domanda a risposta aperta}
\begin{figure}[H]
	\centering
	\noindent\makebox[\textwidth]{\includegraphics[scale=0.5]{Img/creazione_domanda_risposta_aperta.pdf}}
	\caption{Diagramma di creazione domanda a risposta aperta}
\end{figure}
Un docente può creare una domanda a risèpsta aperta. Per creare la domanda il docente inserirà in un apposito form:
\begin{itemize}
	\item il titolo della domanda;
	\item una descrizione facoltativa;
	\item l'argomento della domanda;
	\item il livello di difficoltà;
	\item un allegato (immagine, audio o video) facoltativo;
	\item delle parole chiave che faciliteranno la ricerca della domanda;
	\item la risposta corretta alla domanda.
\end{itemize}
Se i vari campi sono inseriti correttamente, la domanda sarà creata con successo, altrimenti il docente visualizzerà un messaggio di errore.

\newpage
\mysubparagraph{Creazione domanda a risposta multipla}
\begin{figure}[H]
	\centering
	\noindent\makebox[\textwidth]{\includegraphics[scale=0.57]{Img/creazione_domanda_risposta_multipla.pdf}}
	\caption{Diagramma di creazione domanda a risposta multipla}
\end{figure}
Un docente può creare una domanda a risposta multipla. Per creare la domanda il docente inserirà in un apposito form:
\begin{itemize}
	\item il titolo della domanda;
	\item una descrizione facoltativa;
	\item l'argomento della domanda;
	\item il livello di difficoltà;
	\item un allegato (immagine, audio o video) facoltativo; 
	\item delle parole chiave che faciliteranno la ricerca della domanda;
	\item le risposte possibili. 
\end{itemize}
Ogni risposta può contenere una casella di testo o un file multimediale (immagine, audio, video). Alla creazione di ogni risposta è necessario specificare se la risposta è corretta. Quando tutte le risposte possibili sono state create, il docente può concludere la creazione della domanda.
Se i vari campi sono inseriti correttamente e il numero di risposte possibili inserite è corretto, la domanda sarà creata con successo, altrimenti il docente visualizzerà un messaggio di errore.

\newpage
\myparagraph{Modifica domanda}
\begin{figure}[H]
	\centering
	\noindent\makebox[\textwidth]{\includegraphics[width=\textwidth]{Img/modifica_domanda.pdf}}
	\caption{Diagramma di modifica domanda}
\end{figure}
Il docente può modificare una domanda a collegamenti. In particolare, può modificarne le caratteristiche comuni a tutte le domande:
\begin{itemize}
	\item il titolo;
	\item la descrizione;
	\item l’argomento;
	\item il livello di difficoltà;
	\item le parole chiave;
	\item l’allegato.
\end{itemize}
Può inoltre modificare le ennuple delle risposte, cambiando gli allegati o il testo. 
Nel caso le modifiche effettuate rendano la domanda non valida, le modifiche non saranno confermate e il docente visualizzerà un messaggio di errore.

\newpage
\mysubparagraph{Modifica domanda vero/falso}
\begin{figure}[H]
	\centering
	\noindent\makebox[\textwidth]{\includegraphics[width=\textwidth]{Img/modifica_domanda_vf.pdf}}
	\caption{Diagramma di modifica domanda vero/falso}
\end{figure}
Il docente può modificare una domanda vero o falso. In particolare, può modificarne le caratteristiche comuni a tutte le domande:
\begin{itemize}
	\item il titolo;
	\item la descrizione;
	\item l’argomento;
	\item il livello di difficoltà;
	\item le parole chiave;
	\item l’allegato.
\end{itemize}
Può inoltre modificare la veridicità della domanda.
Nel caso le modifiche effettuate rendano la domanda non valida, le modifiche non saranno confermate e il docente visualizzerà un messaggio di errore.

\newpage
\mysubparagraph{Modifica domanda a completamento testo}
\begin{figure}[H]
	\centering
	\noindent\makebox[\textwidth]{\includegraphics[width=\textwidth]{Img/modifica_domanda_completamento.pdf}}
	\caption{Diagramma di modifica domanda a completamento testo}
\end{figure}
Il docente può modificare una domanda completamento testo. In particolare, può modificarne le caratteristiche comuni a tutte le domande:
\begin{itemize}
	\item il titolo;
	\item la descrizione;
	\item l’argomento;
	\item il livello di difficoltà;
	\item le parole chiave;
	\item l’allegato.
\end{itemize}
Può inoltre:
\begin{itemize}
	\item modificare il testo incompleto;
	\item aggiungere, rimuovere e spostare gli spazi;
	\item modificare le parole da inserire nel testo incompleto. 
\end{itemize}
Nel caso le modifiche effettuate rendano la domanda non valida, le modifiche non saranno confermate e il docente visualizzerà un messaggio di errore.


\newpage
\mysubparagraph{Modifica domanda a collegamenti}
\begin{figure}[H]
	\centering
	\noindent\makebox[\textwidth]{\includegraphics[width=\textwidth]{Img/modifica_domanda_collegamenti.pdf}}
	\caption{Diagramma di modifica domanda a collegamenti}
\end{figure}
Il docente può modificare una domanda a collegamenti. In particolare, può modificarne le caratteristiche comuni a tutte le domande:
\begin{itemize}
	\item il titolo;
	\item la descrizione;
	\item l’argomento;
	\item il livello di difficoltà; 
	\item le parole chiave;
	\item l’allegato.
\end{itemize}
Può inoltre modificare le ennuple delle risposte, cambiando gli allegati o il testo. 
Nel caso le modifiche effettuate rendano la domanda non valida, le modifiche non saranno confermate e il docente visualizzerà un messaggio di errore.

\newpage
\mysubparagraph{Modifica domanda a risposta aperta}
\begin{figure}[H]
	\centering
	\noindent\makebox[\textwidth]{\includegraphics[width=\textwidth]{Img/modifica_domanda_risposta_aperta.pdf}}
	\caption{Diagramma di modifica domanda a risposta aperta}
\end{figure}
Il docente può modificare una domanda a risposta aperta. In particolare, può modificarne le caratteristiche comuni a tutte le domande:
\begin{itemize}
	\item il titolo; 
	\item la descrizione;
	\item l’argomento;
	\item il livello di difficoltà; 
	\item le parole chiave;
	\item l’allegato.
\end{itemize}
Può inoltre modificare la risposta corretta alla domanda.
Nel caso le modifiche effettuate rendano la domanda non valida, le modifiche non saranno confermate e il docente visualizzerà un messaggio di errore.

\newpage
\mysubparagraph{Modifica domanda a risposta multipla}
\begin{figure}[H]
	\centering
	\noindent\makebox[\textwidth]{\includegraphics[width=\textwidth]{Img/modifica_domanda_risposta_multipla.pdf}}
	\caption{Diagramma di modifica domanda a risposta multipla}
\end{figure}
Il docente può modificare una domanda a risposta multipla. In particolare, può modificarne le caratteristiche comuni a tutte le domande:
\begin{itemize}
	\item il titolo;
	\item la descrizione;
	\item l’argomento;
	\item il livello di difficoltà;
	\item le parole chiave;
	\item l’allegato.
\end{itemize}
Può inoltre:
\begin{itemize}
	\item modificare le risposte possibili (e le caselle di testo o allegati corrispondenti), eliminarle e aggiungerne di nuove;
	\item modificare la correttezza delle risposte inserite.
\end{itemize}
Nel caso le modifiche effettuate rendano la domanda non valida, le modifiche non saranno confermate e il docente visualizzerà un messaggio di errore.

\newpage
\subsubsection{Gestione richieste studenti}
\begin{figure}[H]
	\centering
	\noindent\makebox[\textwidth]{\includegraphics[scale=0.7]{Img/gestione_richieste_studenti.pdf}}
	\caption{Diagramma di gestione richieste studenti}
\end{figure}
Il docente gestisce le richieste degli studenti. Per prima cosa visualizzerà la lista delle richieste correntemente in sospeso, poi selezionerà la richiesta da gestire e la approverà o rifiuterà. 

\newpage
\subsubsection{Visualizzazione statistiche}
\begin{figure}[H]
	\centering
	\noindent\makebox[\textwidth]{\includegraphics[width=\textwidth]{Img/visualizzazione_statistiche.pdf}}
	\caption{Diagramma di visualizzazione statistiche}
\end{figure}
Il docente può visualizzare vari tipi di statistiche: relative ai quiz, alle domande, agli studenti e agli altri docenti.

\newpage
\myparagraph{Visualizzazione statistiche studenti}
\begin{figure}[H]
	\centering
	\noindent\makebox[\textwidth]{\includegraphics[width=\textwidth]{Img/visualizzazione_statistiche_studenti.pdf}}
	\caption{Diagramma di visualizzazione statistiche studenti}
\end{figure}
Il docente può visualizzare le statistiche relative agli studenti: 
la lista dei quiz svolti e i risultati corrispondenti ottenuti.

\newpage
\myparagraph{Visualizzazione statistiche quiz}
\begin{figure}[H]
	\centering
	\noindent\makebox[\textwidth]{\includegraphics[width=\textwidth]{Img/visualizzazione_statistiche_quiz.pdf}}
	\caption{Diagramma di visualizzazione statistiche quiz}
\end{figure}
Il docente può visualizzare le statistiche relative ai quiz: 
\begin{itemize}
	\item il numero di volte in cui un quiz è stato risolto; 
	\item la percentuale di superamento da parte degli utenti;
	\item il risultato medio ottenuto dagli utenti.
\end{itemize}

\newpage
\myparagraph{Visualizzazione statistiche domande}
\begin{figure}[H]
	\centering
	\noindent\makebox[\textwidth]{\includegraphics[width=\textwidth]{Img/visualizzazione_statistiche_domande.pdf}}
	\caption{Diagramma di visualizzazione statistiche domande}
\end{figure}
Il docente può visualizzare le statistiche relative alle domande: 
\begin{itemize}
	\item il numero di quiz in cui sono state inserite;
	\item la percentuale di successo degli utenti nella risoluzione della domanda.
\end{itemize}

\newpage
\myparagraph{Visualizzazione statistiche docenti}
\begin{figure}[H]
	\centering
	\noindent\makebox[\textwidth]{\includegraphics[width=\textwidth]{Img/visualizzazione_statistiche_docenti.pdf}}
	\caption{Diagramma di visualizzazione statistiche docenti}
\end{figure}
Il docente può visualizzare le statistiche relative agli altri docenti:
\begin{itemize} 
	\item la lista dei quiz creati;
	\item la lista delle domande create.
\end{itemize}

\newpage
\subsection{Attività responsabile}
\begin{figure}[H]
	\centering
	\noindent\makebox[\textwidth]{\includegraphics[width=\textwidth]{Img/attivita_responsabile.pdf}}
	\caption{Diagramma di attività responsabile}
\end{figure}
Il responsabile può:
\begin{itemize}
	\item gestire le richieste dei docenti;
	\item gestire il proprio ente e le classi che contiene;
	\item gestire gli argomenti;
	\item rimuovere gli utenti del proprio ente. 
\end{itemize}

\newpage
\subsubsection{Gestione richieste docenti}
\begin{figure}[H]
	\centering
	\noindent\makebox[\textwidth]{\includegraphics[scale=0.7]{Img/gestione_richieste_docenti.pdf}}
	\caption{Diagramma di gestione richieste docenti}
\end{figure}
Il responsabile gestisce le richieste dei docenti. Per prima cosa visualizzerà la lista delle richieste correntemente in sospeso, poi selezionerà la richiesta da gestire e la approverà o rifiuterà.

\newpage
\subsubsection{Gestione ente e classi} 
\begin{figure}[H]
	\centering
	\noindent\makebox[\textwidth]{\includegraphics[width=\textwidth]{Img/gestione_ente_classi.pdf}}
	\caption{Diagramma di gestione ente e classi}
\end{figure}
Il responsabile può modificare la descrizione del proprio ente e gestire le sue classi: in particolare può creare nuove classi e modificare ed eliminare classi esistenti. Per modificare o eliminare una classe, il responsabile deve prima visualizzare la lista delle classi esistenti e selezionare la classe desiderata. 

\newpage
\myparagraph{Creazione classe}
\begin{figure}[H]
	\centering
	\noindent\makebox[\textwidth]{\includegraphics[scale=0.8]{Img/creazione_classe.pdf}}
	\caption{Diagramma di creazione classe}
\end{figure}
Il responsabile di un ente può creare una nuova classe all'interno dell'ente. Per creare una nuova classe è necessario inserire il nome della classe e il suo anno di creazione in un apposito form e confermare la creazione della classe. Se i dati sono inseriti correttamente la classe sarà creata, altrimenti il responsabile visualizzerà un apposito messaggio di errore.

\newpage
\myparagraph{Modifica classe}
\begin{figure}[H]
	\centering
	\noindent\makebox[\textwidth]{\includegraphics[width=\textwidth]{Img/modifica_classe.pdf}}
	\caption{Diagramma di modifica classe}
\end{figure}
Il docente e il responsabile possono modificare gli utenti associati a una classe, in particolare i docenti possono inserire e rimuovere studenti di una classe di cui sono insegnanti, mentre i responsabili possono inserire e rimuovere docenti in ogni classe del proprio ente (senza che venga inviata una richiesta da parte degli utenti). 

\newpage
\subsubsection{Rimozione utenti}
\begin{figure}[H]
	\centering
	\noindent\makebox[\textwidth]{\includegraphics[scale=0.7]{Img/rimozione_utenti.pdf}}
	\caption{Diagramma di rimozione studenti}
\end{figure}
Il responsabile di un ente può rimuovere gli utenti che appartengono al suo ente. 
L’eliminazione dell’utente non comporta la sua completa eliminazione dal sistema ma soltanto la sua rimozione dall’ente. Una volta selezionato l’utente da rimuovere il responsabile dovrà confermare la sua eliminazione, altrimenti l’operazione sarà annullata. Un responsabile non può rimuovere il proprio account.

\newpage
\subsubsection{Gestione argomenti}
\begin{figure}[H]
	\centering
	\noindent\makebox[\textwidth]{\includegraphics[scale=0.7]{Img/gestione_argomenti.pdf}}
	\caption{Diagramma di gestione argomenti}
\end{figure}
Il responsabile può gestire gli argomenti all’interno del sistema, creando nuovi argomenti ed eliminando argomenti esistenti.

\newpage
\subsection{Ricerca quiz}
\begin{figure}[H]
	\centering
	\noindent\makebox[\textwidth]{\includegraphics[scale=0.6]{Img/ricerca_quiz.pdf}}
	\caption{Diagramma di ricerca quiz}
\end{figure}
L’utente può cercare quiz nel sistema. Durante la ricerca avrà la possibilità di limitare la ricerca attraverso dei parametri: 
\begin{itemize}
	\item argomento;
	\item parola chiave; 
	\item livello di difficoltà; 
	\item autore;
	\item permesso (pubblico o privato). 
\end{itemize}
Solo gli utenti che hanno il ruolo di studente in qualche ente possono vedere i quiz privati delle classi a cui appartengono, e possono limitare ulteriormente la ricerca in base alla classe. I quiz pubblici sono invece accessibili a tutti gli utenti. Una volta selezionato i parametri desiderati, l’utente dovrà confermare l’inizio della ricerca e potrà visualizzare i risultati. 

\newpage
\subsection{Ricerca domande}
\begin{figure}[H]
	\centering
	\noindent\makebox[\textwidth]{\includegraphics[scale=0.6]{Img/ricerca_domande.pdf}}
	\caption{Diagramma di ricerca domande}
\end{figure}
Il docente può cercare domande nel sistema. Ha la possibilità di limitare la ricerca attraverso dei parametri: argomento, parola chiave, livello di difficoltà e autore. Una volta selezionato i parametri desiderati, il docente dovrà confermare l’inizio della ricerca e potrà visualizzare i risultati. 
\newpage


\section{Tracciamento}

\subsection{Componenti-Requisiti}
\begin{tabella}{l!{\VRule}>{\centering\arraybackslash}p{3cm}}
\color{white} \bold{Componenti} & \color{white} \bold{Requisito} \\
\endhead
\rowcolor{P}
Quizzipedia::Client::ControllerClient::CtrlOrganization & ROF12.1.1.5.1 \\
 & ROF14.4.1 \\
 & ROF17.1.1 \\
 & ROF17.1.2 \\
 & ROF17.3.1 \\
 & ROF17.3.2 \\
 & ROF19 \\
 & ROF19.2 \\
 & ROF21 \\
 & ROF21.1 \\
 & ROF21.1.1 \\
 & ROF21.1.2 \\
 & ROF21.1.4 \\
 & ROF21.2 \\
 & ROF21.2.2 \\
 & ROF21.3 \\
 & ROF21.4 \\
 & ROF21.5 \\
 & ROF21.6 \\
 & ROF21.6.1 \\
 & ROF21.6.2 \\
 & ROF21.6.3 \\
 & ROF21.6.4 \\
\rowcolor{D}
Quizzipedia::Client::ControllerClient::CtrlRequests & ROF16 \\
 & ROF16.1 \\
 & ROF16.2 \\
 & ROF16.3 \\
 & ROF16.4 \\
 & ROF17 \\
 & ROF17.1 \\
 & ROF17.1.1 \\
 & ROF17.1.2 \\
 & ROF17.2 \\
 & ROF17.2.1 \\
 & ROF17.2.2 \\
 & ROF17.3 \\
 & ROF17.3.1 \\
 & ROF17.3.2 \\
 & ROF17.4 \\
 & ROF17.4.1 \\
 & ROF17.4.2 \\
\rowcolor{P}
Quizzipedia::Client::ControllerClient::CtrlServices & RDF20.4.1 \\
 & RDF20.4.2 \\
 & ROF11 \\
 & ROF11.1 \\
 & ROF11.2 \\
 & ROF11.3 \\
 & ROF11.3.1 \\
 & ROF11.3.2 \\
 & ROF11.4 \\
 & ROF11.5 \\
 & ROF12 \\
 & ROF12.1 \\
 & ROF12.1.1 \\
 & ROF12.1.1.1 \\
 & ROF12.1.1.2 \\
 & ROF12.1.1.3 \\
 & ROF12.1.1.5 \\
 & ROF12.1.1.5.1 \\
 & ROF12.1.1.6 \\
 & ROF12.2 \\
 & ROF12.2.1 \\
 & ROF12.2.1.1 \\
 & ROF12.2.1.2 \\
 & ROF12.2.1.3 \\
 & ROF12.2.1.4 \\
 & ROF15 \\
 & ROF15.1 \\
 & ROF15.1.1 \\
 & ROF15.1.2 \\
 & ROF15.1.3 \\
 & ROF15.1.4 \\
 & ROF15.1.4.1 \\
 & ROF15.1.5 \\
 & ROF15.1.6 \\
 & ROF15.1.7 \\
 & ROF15.2 \\
 & ROF15.2.1 \\
 & ROF15.2.2 \\
 & ROF15.2.3 \\
 & ROF15.2.4 \\
 & ROF15.2.4.1 \\
 & ROF15.2.4.2 \\
 & ROF15.2.4.2.1 \\
 & ROF15.2.5 \\
 & ROF15.2.5.1 \\
 & ROF15.2.5.2 \\
 & ROF15.2.6 \\
 & ROF15.2.7 \\
 & ROF15.2.8 \\
 & ROF15.3 \\
 & ROF18 \\
 & ROF18.1 \\
 & ROF18.1.1 \\
 & ROF18.2 \\
 & ROF31 \\
 & ROF31.1 \\
 & ROF31.1.1 \\
 & ROF31.1.2 \\
 & ROF31.1.3 \\
 & ROF31.1.4 \\
 & ROF31.1.5 \\
 & ROF31.1.6 \\
 & ROF31.10 \\
 & ROF31.10.1 \\
 & ROF31.11 \\
 & ROF31.11.1 \\
 & ROF31.11.2 \\
 & ROF31.11.2.1 \\
 & ROF31.11.2.2 \\
 & ROF31.11.2.3 \\
 & ROF31.11.3 \\
 & ROF31.12 \\
 & ROF31.12.1 \\
 & ROF31.13 \\
 & ROF31.13.1 \\
 & ROF31.13.2 \\
 & ROF31.13.3 \\
 & ROF31.14 \\
 & ROF31.14.1 \\
 & ROF31.14.1.1 \\
 & ROF31.14.1.2 \\
 & ROF31.14.2 \\
 & ROF31.15 \\
 & ROF31.2 \\
 & ROF31.3 \\
 & ROF31.3.1 \\
 & ROF31.4 \\
 & ROF31.4.1 \\
 & ROF31.5 \\
 & ROF31.5.1 \\
 & ROF31.5.1.1 \\
 & ROF31.5.1.2 \\
 & ROF31.5.1.3 \\
 & ROF31.6 \\
 & ROF31.6.1 \\
 & ROF31.6.2 \\
 & ROF31.7 \\
 & ROF31.7.1 \\
 & ROF31.7.1.1 \\
 & ROF31.7.1.1.1 \\
 & ROF31.7.1.1.2 \\
 & ROF31.7.1.2 \\
 & ROF31.7.1.2.1 \\
 & ROF31.7.1.2.2 \\
 & ROF31.8 \\
 & ROF31.8.1 \\
 & ROF31.8.2 \\
 & ROF31.8.3 \\
 & ROF31.8.4 \\
 & ROF31.8.5 \\
 & ROF31.8.6 \\
 & ROF31.9 \\
\rowcolor{D}
Quizzipedia::Client::ControllerClient::CtrlStatistics & RDF20.4 \\
 & ROF12.1.1.4 \\
 & ROF20.1 \\
 & ROF20.1.1 \\
 & ROF20.1.2 \\
 & ROF20.1.2.1 \\
 & ROF20.1.3 \\
 & ROF20.2 \\
 & ROF20.2.1 \\
 & ROF20.3 \\
 & ROF20.3.1 \\
\rowcolor{P}
Quizzipedia::Client::ControllerClient::CtrlUsers & ROF1 \\
 & ROF1.1 \\
 & ROF1.2 \\
 & ROF1.3 \\
 & ROF1.3.1 \\
 & ROF1.3.2 \\
 & ROF1.3.3 \\
 & ROF1.4 \\
 & ROF1.4.1 \\
 & ROF1.4.2 \\
 & ROF1.5 \\
 & ROF13 \\
 & ROF14 \\
 & ROF14.1 \\
 & ROF14.1.1 \\
 & ROF14.1.2 \\
 & ROF14.1.3 \\
 & ROF14.2 \\
 & ROF14.3 \\
 & ROF14.3.3 \\
 & ROF14.3.3.1 \\
 & ROF14.3.3.2 \\
 & ROF14.3.3.3 \\
 & ROF14.3.3.4 \\
 & ROF14.4 \\
 & ROF14.4.1 \\
 & ROF2 \\
 & ROF2.1 \\
 & ROF2.2 \\
 & ROF2.3 \\
 & ROF3 \\
 & ROF3.1 \\
 & ROF3.2 \\
\rowcolor{D}
Quizzipedia::Client::ModelClient::Organization & ROF12.1.1.5.1 \\
 & ROF14.4.1 \\
 & ROF17.1.1 \\
 & ROF17.1.2 \\
 & ROF17.3.1 \\
 & ROF17.3.2 \\
 & ROF19 \\
 & ROF19.2 \\
 & ROF21 \\
 & ROF21.1 \\
 & ROF21.1.1 \\
 & ROF21.1.2 \\
 & ROF21.1.4 \\
 & ROF21.2 \\
 & ROF21.2.2 \\
 & ROF21.3 \\
 & ROF21.4 \\
 & ROF21.5 \\
 & ROF21.6 \\
 & ROF21.6.1 \\
 & ROF21.6.2 \\
 & ROF21.6.3 \\
 & ROF21.6.4 \\
\rowcolor{P}
Quizzipedia::Client::ModelClient::Requests & ROF16 \\
 & ROF16.1 \\
 & ROF16.2 \\
 & ROF16.3 \\
 & ROF16.4 \\
 & ROF17 \\
 & ROF17.1 \\
 & ROF17.1.1 \\
 & ROF17.1.2 \\
 & ROF17.2 \\
 & ROF17.2.1 \\
 & ROF17.2.2 \\
 & ROF17.3 \\
 & ROF17.3.1 \\
 & ROF17.3.2 \\
 & ROF17.4 \\
 & ROF17.4.1 \\
 & ROF17.4.2 \\
\rowcolor{D}
Quizzipedia::Client::ModelClient::Services & RDF20.4.1 \\
 & ROF11.3 \\
 & ROF11.4 \\
 & ROF11.5 \\
 & ROF12.1 \\
 & ROF12.1.1 \\
 & ROF12.1.1.1 \\
 & ROF12.2.1.1 \\
 & ROF15 \\
 & ROF15.1 \\
 & ROF15.1.1 \\
 & ROF15.1.2 \\
 & ROF15.1.3 \\
 & ROF15.1.4 \\
 & ROF15.1.4.1 \\
 & ROF15.1.5 \\
 & ROF15.1.6 \\
 & ROF15.1.7 \\
 & ROF15.2 \\
 & ROF15.2.1 \\
 & ROF15.2.2 \\
 & ROF15.2.3 \\
 & ROF15.2.4 \\
 & ROF15.2.4.1 \\
 & ROF15.2.4.2 \\
 & ROF15.2.4.2.1 \\
 & ROF15.2.5 \\
 & ROF15.2.5.1 \\
 & ROF15.2.5.2 \\
 & ROF15.2.6 \\
 & ROF15.2.7 \\
 & ROF15.2.8 \\
 & ROF15.3 \\
 & ROF18 \\
 & ROF18.1 \\
 & ROF18.1.1 \\
 & ROF18.2 \\
 & ROF31 \\
 & ROF31.2 \\
 & ROF31.8 \\
 & ROF31.9 \\
\rowcolor{P}
Quizzipedia::Client::ModelClient::Services::Questions & RDF20.4.2 \\
 & ROF11.3 \\
 & ROF11.3.1 \\
 & ROF12.2 \\
 & ROF15.2.5.2 \\
 & ROF31.1 \\
 & ROF31.1.1 \\
 & ROF31.1.2 \\
 & ROF31.1.3 \\
 & ROF31.1.4 \\
 & ROF31.1.5 \\
 & ROF31.1.6 \\
 & ROF31.10 \\
 & ROF31.10.1 \\
 & ROF31.11 \\
 & ROF31.11.1 \\
 & ROF31.11.2 \\
 & ROF31.11.2.1 \\
 & ROF31.11.2.2 \\
 & ROF31.11.2.3 \\
 & ROF31.11.3 \\
 & ROF31.12 \\
 & ROF31.12.1 \\
 & ROF31.13 \\
 & ROF31.13.1 \\
 & ROF31.13.2 \\
 & ROF31.13.3 \\
 & ROF31.14 \\
 & ROF31.14.1 \\
 & ROF31.14.1.1 \\
 & ROF31.14.1.2 \\
 & ROF31.14.2 \\
 & ROF31.15 \\
 & ROF31.3 \\
 & ROF31.3.1 \\
 & ROF31.4 \\
 & ROF31.4.1 \\
 & ROF31.5 \\
 & ROF31.5.1 \\
 & ROF31.5.1.1 \\
 & ROF31.5.1.2 \\
 & ROF31.5.1.3 \\
 & ROF31.6 \\
 & ROF31.6.1 \\
 & ROF31.7 \\
 & ROF31.7.1 \\
 & ROF31.7.1.1 \\
 & ROF31.7.1.1.1 \\
 & ROF31.7.1.1.2 \\
 & ROF31.7.1.2 \\
 & ROF31.7.1.2.1 \\
 & ROF31.7.1.2.2 \\
 & ROF31.8.1 \\
 & ROF31.8.2 \\
 & ROF31.8.3 \\
 & ROF31.8.4 \\
 & ROF31.8.5 \\
 & ROF31.8.6 \\
\rowcolor{D}
Quizzipedia::Client::ModelClient::Statistics & RDF20.4 \\
 & ROF20 \\
 & ROF20.1 \\
 & ROF20.1.1 \\
 & ROF20.1.2 \\
 & ROF20.1.2.1 \\
 & ROF20.1.3 \\
 & ROF20.2 \\
 & ROF20.2.1 \\
 & ROF20.3 \\
 & ROF20.3.1 \\
\rowcolor{P}
Quizzipedia::Client::ModelClient::Users & ROF1 \\
 & ROF1.1 \\
 & ROF1.2 \\
 & ROF1.3 \\
 & ROF1.4 \\
 & ROF1.5 \\
 & ROF11 \\
 & ROF12 \\
 & ROF12.1 \\
 & ROF12.1.1 \\
 & ROF12.1.1.1 \\
 & ROF12.1.1.2 \\
 & ROF12.1.1.3 \\
 & ROF12.1.1.4 \\
 & ROF12.1.1.5 \\
 & ROF12.1.1.5.1 \\
 & ROF12.1.1.6 \\
 & ROF12.2 \\
 & ROF12.2.1 \\
 & ROF12.2.1.1 \\
 & ROF12.2.1.2 \\
 & ROF12.2.1.3 \\
 & ROF12.2.1.4 \\
 & ROF13 \\
 & ROF14 \\
 & ROF14.1 \\
 & ROF14.1.1 \\
 & ROF14.1.2 \\
 & ROF14.1.3 \\
 & ROF14.2 \\
 & ROF14.3 \\
 & ROF14.3.3 \\
 & ROF14.3.3.1 \\
 & ROF14.3.3.2 \\
 & ROF14.3.3.4 \\
 & ROF14.4 \\
 & ROF14.4.1 \\
 & ROF2 \\
 & ROF2.1 \\
 & ROF2.2 \\
 & ROF2.3 \\
 & ROF3 \\
 & ROF3.1 \\
 & ROF3.2 \\
\rowcolor{D}
Quizzipedia::Client::ViewClient::ViewErrors & ROF1.5 \\
 & ROF14.3.3.4 \\
 & ROF15.2.7 \\
 & ROF2.3 \\
 & ROF3.2 \\
\rowcolor{P}
Quizzipedia::Client::ViewClient::ViewOrgManager & ROF12.1.1.5.1 \\
 & ROF19 \\
 & ROF19.2 \\
 & ROF21 \\
 & ROF21.1 \\
 & ROF21.1.1 \\
 & ROF21.1.2 \\
 & ROF21.1.4 \\
 & ROF21.2 \\
 & ROF21.2.2 \\
 & ROF21.3 \\
 & ROF21.4 \\
 & ROF21.5 \\
 & ROF21.6 \\
 & ROF21.6.1 \\
 & ROF21.6.2 \\
 & ROF21.6.3 \\
 & ROF21.6.4 \\
\rowcolor{D}
Quizzipedia::Client::ViewClient::ViewQuestionManager & RDF20.4.2 \\
 & ROF12.2 \\
 & ROF15.1.5 \\
 & ROF15.1.6 \\
 & ROF15.2.5.1 \\
 & ROF15.2.5.2 \\
 & ROF15.2.6 \\
 & ROF31 \\
 & ROF31.1 \\
 & ROF31.1.1 \\
 & ROF31.1.2 \\
 & ROF31.1.3 \\
 & ROF31.1.4 \\
 & ROF31.1.5 \\
 & ROF31.1.6 \\
 & ROF31.10 \\
 & ROF31.10.1 \\
 & ROF31.11 \\
 & ROF31.11.1 \\
 & ROF31.11.2 \\
 & ROF31.11.2.1 \\
 & ROF31.11.2.2 \\
 & ROF31.11.2.3 \\
 & ROF31.11.3 \\
 & ROF31.12 \\
 & ROF31.12.1 \\
 & ROF31.13 \\
 & ROF31.13.1 \\
 & ROF31.13.2 \\
 & ROF31.13.3 \\
 & ROF31.14 \\
 & ROF31.14.1 \\
 & ROF31.14.1.1 \\
 & ROF31.14.1.2 \\
 & ROF31.14.2 \\
 & ROF31.15 \\
 & ROF31.2 \\
 & ROF31.3 \\
 & ROF31.3.1 \\
 & ROF31.4 \\
 & ROF31.4.1 \\
 & ROF31.5 \\
 & ROF31.5.1 \\
 & ROF31.5.1.1 \\
 & ROF31.5.1.2 \\
 & ROF31.5.1.3 \\
 & ROF31.6 \\
 & ROF31.6.1 \\
 & ROF31.7 \\
 & ROF31.7.1 \\
 & ROF31.7.1.1 \\
 & ROF31.7.1.1.1 \\
 & ROF31.7.1.1.2 \\
 & ROF31.7.1.2 \\
 & ROF31.7.1.2.1 \\
 & ROF31.7.1.2.2 \\
 & ROF31.8 \\
 & ROF31.8.1 \\
 & ROF31.8.2 \\
 & ROF31.8.3 \\
 & ROF31.8.4 \\
 & ROF31.8.5 \\
 & ROF31.8.6 \\
 & ROF31.9 \\
\rowcolor{P}
Quizzipedia::Client::ViewClient::ViewQuizManager & RDF20.4.1 \\
 & ROF11.1 \\
 & ROF11.2 \\
 & ROF12.1 \\
 & ROF14.2 \\
 & ROF15 \\
 & ROF15.1 \\
 & ROF15.1.1 \\
 & ROF15.1.2 \\
 & ROF15.1.3 \\
 & ROF15.1.4 \\
 & ROF15.1.4.1 \\
 & ROF15.1.5 \\
 & ROF15.1.6 \\
 & ROF15.1.7 \\
 & ROF15.2 \\
 & ROF15.2.1 \\
 & ROF15.2.2 \\
 & ROF15.2.3 \\
 & ROF15.2.4 \\
 & ROF15.2.4.1 \\
 & ROF15.2.4.2 \\
 & ROF15.2.4.2.1 \\
 & ROF15.2.5 \\
 & ROF15.2.5.1 \\
 & ROF15.2.5.2 \\
 & ROF15.2.6 \\
 & ROF15.2.8 \\
 & ROF15.3 \\
\rowcolor{D}
Quizzipedia::Client::ViewClient::ViewQuizSolver & ROF11.3 \\
 & ROF11.3.2 \\
 & ROF11.4 \\
 & ROF11.5 \\
\rowcolor{P}
Quizzipedia::Client::ViewClient::ViewQuizSolver::ViewQuestionSolver & ROF11.3 \\
 & ROF11.3.1 \\
\rowcolor{D}
Quizzipedia::Client::ViewClient::ViewRequests & ROF16 \\
 & ROF16.1 \\
 & ROF16.2 \\
 & ROF16.3 \\
 & ROF16.4 \\
 & ROF17 \\
 & ROF17.1 \\
 & ROF17.1.1 \\
 & ROF17.1.2 \\
 & ROF17.2 \\
 & ROF17.2.1 \\
 & ROF17.2.2 \\
 & ROF17.3 \\
 & ROF17.3.1 \\
 & ROF17.3.2 \\
 & ROF17.4 \\
 & ROF17.4.1 \\
 & ROF17.4.2 \\
\rowcolor{P}
Quizzipedia::Client::ViewClient::ViewSearch & ROF11 \\
 & ROF12 \\
 & ROF12.1 \\
 & ROF12.1.1 \\
 & ROF12.1.1.1 \\
 & ROF12.1.1.2 \\
 & ROF12.1.1.3 \\
 & ROF12.1.1.4 \\
 & ROF12.1.1.5 \\
 & ROF12.1.1.5.1 \\
 & ROF12.1.1.6 \\
 & ROF12.2 \\
 & ROF12.2.1 \\
 & ROF12.2.1.1 \\
 & ROF12.2.1.2 \\
 & ROF12.2.1.3 \\
 & ROF12.2.1.4 \\
 & ROF15.1.4.1 \\
 & ROF15.1.5 \\
 & ROF15.2.4.2.1 \\
 & ROF15.2.5.1 \\
\rowcolor{D}
Quizzipedia::Client::ViewClient::ViewStatistics & RDF20.4 \\
 & ROF20 \\
 & ROF20.1 \\
 & ROF20.1.1 \\
 & ROF20.1.2 \\
 & ROF20.1.2.1 \\
 & ROF20.1.3 \\
 & ROF20.2 \\
 & ROF20.2.1 \\
 & ROF20.3 \\
 & ROF20.3.1 \\
\rowcolor{P}
Quizzipedia::Client::ViewClient::ViewTopicManager & ROF12.1.1.1 \\
 & ROF12.2.1.1 \\
 & ROF15.2.2 \\
 & ROF18 \\
 & ROF18.1 \\
 & ROF18.1.1 \\
 & ROF18.2 \\
 & ROF31.8.3 \\
\rowcolor{D}
Quizzipedia::Client::ViewClient::ViewUsers & ROF1 \\
 & ROF1.1 \\
 & ROF1.2 \\
 & ROF1.3 \\
 & ROF1.3.3 \\
 & ROF1.4 \\
 & ROF1.4.1 \\
 & ROF1.4.2 \\
 & ROF11 \\
 & ROF12 \\
 & ROF12.1 \\
 & ROF12.2 \\
 & ROF12.2.1.4 \\
 & ROF13 \\
 & ROF14 \\
 & ROF14.1 \\
 & ROF14.1.1 \\
 & ROF14.1.2 \\
 & ROF14.1.3 \\
 & ROF14.2 \\
 & ROF14.3 \\
 & ROF14.3.3 \\
 & ROF14.3.3.1 \\
 & ROF14.3.3.2 \\
 & ROF14.3.3.3 \\
 & ROF14.4 \\
 & ROF14.4.1 \\
 & ROF15 \\
 & ROF15.1 \\
 & ROF16 \\
 & ROF17 \\
 & ROF18 \\
 & ROF19 \\
 & ROF2 \\
 & ROF2.1 \\
 & ROF2.2 \\
 & ROF20 \\
 & ROF21 \\
 & ROF3 \\
 & ROF3.1 \\
\rowcolor{P}
Quizzipedia::Server::ControllerServer::AuthenticationManager & ROF1 \\
 & ROF1.1 \\
 & ROF1.2 \\
 & ROF1.3 \\
 & ROF1.3.1 \\
 & ROF1.3.2 \\
 & ROF1.3.3 \\
 & ROF1.4 \\
 & ROF1.4.1 \\
 & ROF1.4.2 \\
 & ROF1.5 \\
 & ROF13 \\
 & ROF2 \\
 & ROF2.1 \\
 & ROF2.2 \\
 & ROF2.3 \\
 & ROF3 \\
 & ROF3.1 \\
 & ROF3.2 \\
\rowcolor{D}
Quizzipedia::Server::ControllerServer::ClassManager & ROF17.1 \\
 & ROF17.1.1 \\
 & ROF17.1.2 \\
 & ROF17.3 \\
 & ROF17.3.1 \\
 & ROF17.3.2 \\
 & ROF21 \\
 & ROF21.1 \\
 & ROF21.1.1 \\
 & ROF21.1.2 \\
 & ROF21.1.4 \\
 & ROF21.2 \\
 & ROF21.2.2 \\
 & ROF21.3 \\
 & ROF21.4 \\
 & ROF21.5 \\
 & ROF21.6 \\
 & ROF21.6.1 \\
 & ROF21.6.2 \\
 & ROF21.6.3 \\
 & ROF21.6.4 \\
\rowcolor{P}
Quizzipedia::Server::ControllerServer::CompanyManager & ROF19 \\
 & ROF19.2 \\
\rowcolor{D}
Quizzipedia::Server::ControllerServer::ProfileManager & ROF14 \\
 & ROF14.1 \\
 & ROF14.1.1 \\
 & ROF14.1.2 \\
 & ROF14.1.3 \\
 & ROF14.2 \\
 & ROF14.3 \\
 & ROF14.3.3 \\
 & ROF14.3.3.1 \\
 & ROF14.3.3.2 \\
 & ROF14.3.3.3 \\
 & ROF14.3.3.4 \\
 & ROF14.4 \\
 & ROF14.4.1 \\
\rowcolor{P}
Quizzipedia::Server::ControllerServer::QuestionsManager & ROF31 \\
 & ROF31.1 \\
 & ROF31.1.1 \\
 & ROF31.1.2 \\
 & ROF31.1.3 \\
 & ROF31.1.4 \\
 & ROF31.10 \\
 & ROF31.10.1 \\
 & ROF31.11 \\
 & ROF31.11.1 \\
 & ROF31.11.2 \\
 & ROF31.11.3 \\
 & ROF31.12 \\
 & ROF31.12.1 \\
 & ROF31.13 \\
 & ROF31.13.1 \\
 & ROF31.13.2 \\
 & ROF31.13.3 \\
 & ROF31.14 \\
 & ROF31.14.1 \\
 & ROF31.14.2 \\
 & ROF31.15 \\
 & ROF31.3 \\
 & ROF31.4 \\
 & ROF31.5 \\
 & ROF31.6 \\
 & ROF31.7 \\
 & ROF31.8 \\
 & ROF31.8.1 \\
 & ROF31.8.2 \\
 & ROF31.8.3 \\
 & ROF31.8.4 \\
 & ROF31.8.5 \\
 & ROF31.8.6 \\
\rowcolor{D}
Quizzipedia::Server::ControllerServer::QuizManager & ROF11 \\
 & ROF11.4 \\
 & ROF11.5 \\
 & ROF15 \\
 & ROF15.1 \\
 & ROF15.1.1 \\
 & ROF15.1.2 \\
 & ROF15.1.3 \\
 & ROF15.1.4 \\
 & ROF15.1.5 \\
 & ROF15.1.6 \\
 & ROF15.1.7 \\
 & ROF15.2 \\
 & ROF15.2.1 \\
 & ROF15.2.2 \\
 & ROF15.2.3 \\
 & ROF15.2.4 \\
 & ROF15.2.5 \\
 & ROF15.2.6 \\
 & ROF15.2.7 \\
 & ROF15.2.8 \\
 & ROF15.3 \\
\rowcolor{P}
Quizzipedia::Server::ControllerServer::RequestsManager & ROF16 \\
 & ROF16.1 \\
 & ROF16.2 \\
 & ROF16.3 \\
 & ROF16.4 \\
 & ROF17 \\
 & ROF17.2 \\
 & ROF17.2.1 \\
 & ROF17.2.2 \\
 & ROF17.4 \\
 & ROF17.4.1 \\
 & ROF17.4.2 \\
\rowcolor{D}
Quizzipedia::Server::ControllerServer::SearchManager & ROF12 \\
 & ROF12.1 \\
 & ROF12.2 \\
\rowcolor{P}
Quizzipedia::Server::ControllerServer::StatisticsManager & RDF20.4 \\
 & ROF20 \\
 & ROF20.1 \\
 & ROF20.1.1 \\
 & ROF20.1.2 \\
 & ROF20.1.2.1 \\
 & ROF20.1.3 \\
 & ROF20.2 \\
 & ROF20.2.1 \\
 & ROF20.3 \\
 & ROF20.3.1 \\
\rowcolor{D}
Quizzipedia::Server::ControllerServer::TopicManager & ROF18 \\
 & ROF18.1 \\
 & ROF18.1.1 \\
 & ROF18.2 \\
\rowcolor{P}
Quizzipedia::Server::ModelServer::Services & RDF20.4.1 \\
 & ROF11.3 \\
 & ROF11.4 \\
 & ROF11.5 \\
 & ROF12.1 \\
 & ROF12.1.1 \\
 & ROF12.1.1.1 \\
 & ROF12.2.1.1 \\
 & ROF15 \\
 & ROF15.1 \\
 & ROF15.1.1 \\
 & ROF15.1.2 \\
 & ROF15.1.3 \\
 & ROF15.1.4 \\
 & ROF15.1.4.1 \\
 & ROF15.1.5 \\
 & ROF15.1.6 \\
 & ROF15.1.7 \\
 & ROF15.2 \\
 & ROF15.2.1 \\
 & ROF15.2.2 \\
 & ROF15.2.3 \\
 & ROF15.2.4 \\
 & ROF15.2.4.1 \\
 & ROF15.2.4.2 \\
 & ROF15.2.4.2.1 \\
 & ROF15.2.5 \\
 & ROF15.2.5.1 \\
 & ROF15.2.5.2 \\
 & ROF15.2.6 \\
 & ROF15.2.7 \\
 & ROF15.2.8 \\
 & ROF15.3 \\
 & ROF18 \\
 & ROF18.1 \\
 & ROF18.1.1 \\
 & ROF18.2 \\
 & ROF31 \\
 & ROF31.2 \\
 & ROF31.8 \\
 & ROF31.9 \\
\rowcolor{D}
Quizzipedia::Server::ModelServer::Services::Questions & RDF20.4.2 \\
 & ROF11.3 \\
 & ROF11.3.1 \\
 & ROF12.2 \\
 & ROF15.2.5.2 \\
 & ROF31.1 \\
 & ROF31.1.1 \\
 & ROF31.1.2 \\
 & ROF31.1.3 \\
 & ROF31.1.4 \\
 & ROF31.1.5 \\
 & ROF31.1.6 \\
 & ROF31.10 \\
 & ROF31.10.1 \\
 & ROF31.11 \\
 & ROF31.11.1 \\
 & ROF31.11.2 \\
 & ROF31.11.2.1 \\
 & ROF31.11.2.2 \\
 & ROF31.11.2.3 \\
 & ROF31.11.3 \\
 & ROF31.12 \\
 & ROF31.12.1 \\
 & ROF31.13 \\
 & ROF31.13.1 \\
 & ROF31.13.2 \\
 & ROF31.13.3 \\
 & ROF31.14 \\
 & ROF31.14.1 \\
 & ROF31.14.1.1 \\
 & ROF31.14.1.2 \\
 & ROF31.14.2 \\
 & ROF31.15 \\
 & ROF31.3 \\
 & ROF31.3.1 \\
 & ROF31.4 \\
 & ROF31.4.1 \\
 & ROF31.5 \\
 & ROF31.5.1 \\
 & ROF31.5.1.1 \\
 & ROF31.5.1.2 \\
 & ROF31.5.1.3 \\
 & ROF31.6 \\
 & ROF31.6.1 \\
 & ROF31.7 \\
 & ROF31.7.1 \\
 & ROF31.7.1.1 \\
 & ROF31.7.1.1.1 \\
 & ROF31.7.1.1.2 \\
 & ROF31.7.1.2 \\
 & ROF31.7.1.2.1 \\
 & ROF31.7.1.2.2 \\
 & ROF31.8.1 \\
 & ROF31.8.2 \\
 & ROF31.8.3 \\
 & ROF31.8.4 \\
 & ROF31.8.5 \\
 & ROF31.8.6 \\
\rowcolor{white}
\caption{Tracciamento componenti-requisiti}
\end{tabella}


\subsection{Requisiti-Componenti}
\begin{tabella}{l!{\VRule}>{\centering\arraybackslash}p{2.5 cm}}
\color{white} \bold{Requisito} & \color{white} \bold{Componenti} \\
\endhead
ROF11 & Quizzipedia::Client::ControllerClient::CtrlServices \linebreak Quizzipedia::Client::ControllerClient::CtrlServices \linebreak Quizzipedia::Client::ControllerClient::CtrlServices \linebreak Quizzipedia::Client::ControllerClient::CtrlServices \linebreak Quizzipedia::Client::ControllerClient::CtrlServices \linebreak Quizzipedia::Client::ControllerClient::CtrlServices \linebreak Quizzipedia::Client::ControllerClient::CtrlServices \linebreak Quizzipedia::Client::ControllerClient::CtrlServices \linebreak Quizzipedia::Client::ControllerClient::CtrlServices \linebreak Quizzipedia::Client::ControllerClient::CtrlServices \linebreak Quizzipedia::Client::ControllerClient::CtrlUsers \linebreak Quizzipedia::Client::ControllerClient::CtrlUsers \linebreak Quizzipedia::Client::ControllerClient::CtrlUsers \linebreak Quizzipedia::Client::ControllerClient::CtrlUsers \linebreak Quizzipedia::Client::ControllerClient::CtrlUsers \linebreak Quizzipedia::Client::ControllerClient::CtrlUsers \linebreak Quizzipedia::Client::ControllerClient::CtrlUsers \linebreak Quizzipedia::Client::ControllerClient::CtrlUsers \linebreak Quizzipedia::Client::ControllerClient::CtrlUsers \linebreak Quizzipedia::Client::ControllerClient::CtrlUsers \linebreak Quizzipedia::C \\
\rowcolor{white}
\caption{Tracciamento requisiti-componenti}
\end{tabella}


\end{document}
