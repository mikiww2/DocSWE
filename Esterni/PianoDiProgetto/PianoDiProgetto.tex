% Nome del file: documento.tex
% Percorso: \gl{template}
% Autore: Vault-Tech
% Data creazione: 27.12.2015
% E-mail: vaulttech.swe@gmail.comcom
%
% Diario delle modifiche: interno al file.

\documentclass[a4paper, titlepage]{article}

\usepackage[margin=3cm]{geometry}
\usepackage{../../Stile}
\usepackage{../../Comandi}

\setcounter{secnumdepth}{5}
\setcounter{tocdepth}{5}

\def\NOME{Piano di progetto}
\def\VERSIONE{1.0}
\def\DATA{17.01.2016}
\def\REDATTORE{ Giacomo Beltrame \\ & Vassilikì Menarin}
\def\VERIFICATORE{Rudy Berton \\ & Michela De Bortoli \\ & Filippo Tesser}
\def\RESPONSABILE{Vassilikì Menarin}
\def\USO{Esterno}
\def\DISTRIBUZIONE{\COMMITTENTE \\ & \CARDIN \\ & \PROPONENTE}

\begin{document}

\pagestyle{fancy}
\pagenumbering{Roman}
\rfoot{Pagina \thepage{} di \pageref{lastromanpage}}

\maketitle

\begin{diario}
	\recap{Approvazione del documento}{Vassilikì Menarin}{Responsabile}{17.01.2016}{1.0}
	\recap{Verifica documento}{Rudy Berton}{Verificatore}{17.01.2016}{0.15}
	\recap{Correzione del documento}{Vassilikì Menarin}{Responsabile}{16.01.2016}{0.14}
	\recap{Verifica documento}{Filippo Tesser}{Verificatore}{15.01.2016}{0.13}
	\recap{Stesura dell'organigramma}{Vassilikì Menarin}{Responsabile}{14.01.2016}{0.12}
  \recap{Inserimento grafici}{Giacomo Beltrame}{Amministratore}{13.01.2016}{0.11}
  \recap{Fine stesura del preventivo}{Vassilikì Menarin}{Responsabile}{13.01.2016}{0.10}
	\recap{Inizio stesura del preventivo}{Vassilikì Menarin}{Responsabile}{10.12.2015}{0.9}
	\recap{Correzione della pianificazione}{Vassilikì Menarin}{Responsabile}{11.01.2016}{0.8}
	\recap{Verifica della pianificazione}{Michela De Bortoli}{Verificatore}{10.01.2016}{0.7}
	\recap{Inserimento grafici di Gantt}{Giacomo Beltrame}{Amministratore}{08.01.2016}{0.6}
	\recap{Fine stesura della pianificazione}{Vassilikì Menarin}{Responsabile}{08.01.2016}{0.5}	
	\recap{Inizio stesura della pianificazione}{Vassilikì Menarin}{Responsabile}{06.01.2016}{0.4}	
	\recap{Stesura dell'analisi dei rischi}{Vassilikì Menarin}{Responsabile}{04.01.2016}{0.3}
	\recap{Stesura ciclo di vita e scadenze}{Vassilikì Menarin}{Responsabile}{03.01.2016}{0.2}
	\recap{Stesura di introduzione}{Vassilikì Menarin}{Responsabile}{02.01.2016}{0.1}
\end{diario}

\newpage
\tableofcontents
\newpage
\listoffigures
\newpage
\listoftables\label{lastromanpage}

\newpage
\clearpage	
\pagenumbering{arabic}
\rfoot{Pagina \thepage{} di \pageref*{LastPage}}
%Deve esserci per permettere i riferimenti incrociati di colore blu
\hypersetup{linkcolor=blue}

\section{Introduzione}
\subsection{Scopo del documento}
Con il seguente documento si intende pianificare il modo e i tempi in cui il gruppo Vault-Tech intende procedere per lo sviluppo del progetto Quizzipedia.
In particolare, gli scopi del documento sono:
\begin{itemize}
	\item presentare e motivare il ciclo di vita scelto per lo sviluppo del prodotto;
	\item fissare le attività e le relative sottoattività di sviluppo, il ruolo e le mansioni che ogni membro del gruppo avrà in esse;
	\item analizzare e gestire i possibili fattori di rischio;
	\item preventivare l'impiego delle risorse;
	\item definire i costo complessivi.
\end{itemize}


\subsection{Scopo del prodotto}
\SCOPO

\subsection{Glossario}
\GLOSSARIO
\subsection{Riferimenti}
\subsubsection{Riferimenti normativi}
\begin{itemize}
\item \bold{\gl{Capitolato} d'appalto C5:} Quizzipedia: \gl{software} per la gestione di questionari \url{http://www.math.unipd.it/~tullio/IS-1/2015/Progetto/C5.pdf};
\item \bold {Norme di progetto:} \doc{Norme di Progetto v1.00}.
\end{itemize}

\subsubsection{Riferimenti informativi}
\begin{itemize}
	\item \bold{Materiale presentato durante il corso di \italics{Ingegneria del \gl{software}:}} \url{http://www.math.unipd.it/~tullio/IS-1/2015/};
	\item \bold {\gl{Software} Engineering 9th Edition - Ian Sommerville - Chapter 2 Dependability and security e Chapter 4. \gl{\gl{Software} Management}};
	\item \bold{Guide to the \gl{Software} Engineering Body of Knowledge}: \url{http://www.computer.org/web/swebok/}
	\item \bold{Analisi dei requisiti:} \doc{Analisi dei requisiti v1.00};
	\item \bold{Piano di qualifica:} \doc{Piano di qualifica v1.00};
	\item \bold{Studio di fattibilità:} \doc{Studio di fattibilità v1.00};	
\end{itemize}

\newpage
\section{Ciclo di vita}\label{Ciclo di vita}
Il modello di ciclo di vita scelto per il progetto è quello incrementale.\\
Per avere controllo sull'avanzamento del progetto si è deciso di suddividere il lavoro in attività e sottoattività che terminano al raggiungimento di una \gl{milestone}. Questa \gl{milestone} può coincidere con scadenze di progetto o con incontri con il proponente.\\
Per permettere di avere un buon dialogo col committente e perché siano sempre chiari i requisiti richiesti e quelli soddisfatti, si è tentato di non fare intercorrere troppo tempo tra una \gl{milestone} e la successiva.\\
Si è cercato inoltre di lasciare un margine di \gl{slack} tra un'attività e l'altra, tenendo conto dei rischi che potrebbero insorgere, e si prevedono attività di verifica regolari all'interno di ogni fase.\\
Essendo il \gl{modello incrementale}, in ogni fase si prevede una revisione e aggiornamento impliciti, se necessario, dei documenti precedenti utilizzando, per una gestione ottimale dei processi, il modello \gl{PDCA}.\\
Per ogni attività si prevedono dei giorni di \gl{slack} prima della consegna dei documenti per accedere alle revisioni di avanzamento.
Le attività principali sono:

\begin{itemize}
\item \bold{Attività di Analisi dei requisiti utente:} dal 18.12.2015 al 22.01.2016\\
Durante questa attività vengono redatti i primi documenti e  si raccolgono i requisiti. Include due incontri col proponente per discutere e poi verificare i requisiti raccolti. Viene svolto un pesante lavoro di analisi che si conclude con la consegna dei documenti per la Revisione dei Requisiti.

\item \bold{Attività di Raffinamento dei requisiti:} dal 16.02.2016 al 23.02.2016\\
Comincia dopo aver saputo l'esito della Revisione dei Requisiti e dura  una settimana. Vengono apportate le correzioni opportune ai documenti già consegnati e si comincia a stendere la \doc{Specifica Tecnica}.

\item \bold{Attività di Progettazione architetturale:} dal 24.02.2016 al 11.04.2016\\
Segue immediatamente le attività di Raffinamento dei requisiti e si conclude con un incontro col proponente e la consegna dei documenti per accedere alla Revisione di Progettazione. L'attività principale è la redazione della \doc{Specifica Tecnica} e alla fine si presentano i documenti per la Revisione di Progettazione.

\item \bold{Attività di Progettazione di dettaglio e codifica}: dal 12.04.2016 al 16.05.2016\\
Avviene la codifica dei requisiti e finisce con la consegna dei documenti per accedere alla Revisione di Qualifica.

\item \bold{Attività di validazione}: Dal 17.05.2016 al 10.06.2016\\
L'intero progetto viene validato e collaudato. Si conclude con la Revisione di Accettazione.
\end{itemize}

Le varie sottoattività vengono illustrate più dettagliatamente, con relativo diagramma di Gantt, nella sezione \hyperref[Pianificazione]{Pianificazione} del documento corrente.

\newpage
\section{Scadenze}
Il gruppo ha deciso di rispettare le scadenze di consegna riportate in tabella:
\begin{tabella}{l!{\VRule}l}

	\color{white} \bold{Revisione} & \color{white} \bold{Data} \\
	\endfirsthead
	Revisione dei Requisiti (RR) & 22.01.2016 \\
	Revisione di Progresso (RP)	& 11.04.2016\\
	Revisione di Qualifica (RQ)  & 16.05.2016\\	
	Revisione di Accettazione (RA) & 10.06.2016\\
  
    
    \rowcolor{white}  
    \caption{Scadenze}	    	
	
\end{tabella}

\newpage
\section{Analisi dei rischi}

Al fine di migliorare la qualità del progetto viene presentata di seguito un'analisi realistica dei rischi che potrebbero insorgere nel corso dello sviluppo.\\
Ogni rischio individuato viene analizzato nel dettaglio, discutendo i seguenti punti:

\begin{itemize}
	\item \bold{Identificazione:} viene individuata la natura del rischio e ne viene data una breve descrizione.
	\item \bold{Analisi:} si fornisce la probabilità stimata di insorgenza e il livello di rischio che potrebbe conseguentemente portare.
	\item \bold{Pianificazione:} viene definito un piano d'azione in modo da rendere minima la probabilità di insorgenza del rischio. 
	\item \bold{Contenimento:} nel caso in cui il rischio, nonostante le misure adottate, dovesse comunque insorgere viene già deciso come agire per contenerlo.
\end{itemize}

Di seguito la tabella con i rischi individuati, divisi a seconda del livello di appartenenza:

\begin{tabella}{l!{\VRule}>{\centering\arraybackslash}p{6 cm}!{\VRule}>{\centering\arraybackslash}p{2 cm}!{\VRule}>{\centering\arraybackslash}p{2 cm}}
	%{l!{\VRule} p{20px} l ! {\VRule}  l !{\VRule}l}
		
	
	\color{white} \bold{Livello} & \color{white} \bold{Tipologia} & \color{white} \bold{Probabilità di insorgenza} & \color{white} \bold{Livello di rischio} \\
	\endfirsthead
	
	\cellcolor{P} & Scarsa conoscenza delle tecnologie & Media & Alto \\
	\cellcolor{P} & Guasti \gl{hardware} & Bassa & Bassa \\
	\multirow{-3}{*}{\cellcolor{P}Tecnologico}	& Malfunzionamenti \gl{software} & Bassa & Alto \\
	\hline
	
	\cellcolor{D} & Problemi personali dei membri & Media & Medio \\
	\multirow{-2}{*}{\cellcolor{D}Personale} & Problemi interni tra i membri & Bassa & Alto \\
	\hline
	
	\cellcolor{P} & Problemi di \gl{versionamento} & Media & Alto \\
	\multirow{-2}{*}{\cellcolor{P}Organizzativo} & Errata valutazione dei costi & Alta & Medio \\
	\hline
		
    \rowcolor{D}
	Strumenti & Mancante o insufficiente conoscenza degli strumenti & Alto & Alto \\	
	\hline	
	
	\rowcolor{P}
	Requisiti & Errata comprensione dei requisiti & Media & Alto\\
	\hline
		
	\rowcolor{white}  
	\caption{Analisi dei rischi}	    	
	
\end{tabella}

\subsection{Livello tecnologico}
\subsubsection{Scarsa conoscenza delle tecnologie}

\myparagraph {Identificazione}
  Alcune delle tecnologie utilizzate sono sconosciute a uno o più membri del gruppo; altre invece sono state viste solo in ambito teorico. In generale, esistono tecnologie con cui il gruppo non ha il grado di dimestichezza richiesto.
  
\myparagraph {Analisi}
	\begin{itemize}
		\item \bold{Probabilità di insorgenza:} Media.
		\item \bold{Livello di rischio:} Alto.
		\item \bold{Possibili conseguenze:} ritardi nei tempi prestabiliti, un maggiore numero di errori.
	\end{itemize}
	
	\myparagraph {Pianificazione}
	 I membri del gruppo, dopo aver concordato anticipatamente sulle tecnologie da utilizzare, si impegnano a documentarsi in modo autonomo e responsabile. Saranno seguiti dagli \italics{Amministratori}, che forniranno le documentazioni necessarie.
	 
	 \myparagraph {Contenimento} Nel caso in cui dovessero comunque presentarsi problemi, il \italics{Responsabile di Progetto} provvederà a sollevare momentaneamente il membro carente dal proprio incarico per permettergli di aggiornarsi nel minor tempo possibile. 


\subsubsection{Guasti hardware}
\myparagraph {Identificazione}
 Viene tenuto conto di possibili guasti dei dispositivi di lavoro dei membri del gruppo. Si includono possibili malfunzionamenti di PC e problemi alla linea internet.
 
\myparagraph {Analisi}
	\begin{itemize}
		\item \bold{Probabilità di insorgenza:} Bassa.
		\item \bold{Livello di rischio:} Basso.
		\item \bold{Possibili conseguenze:} ritardi nei tempi prestabiliti, impossibilità o ritardi per un membro di completare il proprio compito.
	\end{itemize}
	
\myparagraph {Pianificazione}
Per evitare perdita di dati significativi, i membri del gruppo eseguiranno backup regolari su \gl{repository}. In caso di guasti tutti i membri possono raggiungere Padova e disporre dei mezzi messi a disposizione dall'Università.

\myparagraph {Contenimento}
Se dovessero insorgere dei problemi si provvederà a recuperare la versione aggiornata del materiale da \gl{repository}.	Il membro interessato potrà utilizzare gli strumenti dell'Università nel tempo necessario alla riparazione/sostituzione.


\subsubsection{Malfunzionamenti software}
\myparagraph {Identificazione}
 È possibile che, nel corso del progetto, i \gl{software} utilizzati incorrano in dei malfunzionamenti che potrebbero causare perdita di dati o incompatibilità tra versioni in possesso di membri diversi.
 
\myparagraph {Analisi}
	\begin{itemize}
		\item \bold{Probabilità di insorgenza:} Bassa.
		\item \bold{Livello di rischio:} Alto.
		\item \bold{Possibili conseguenze:} ritardi nei tempi prestabiliti, impossibilità o ritardi per un membro di completare il proprio compito, possibile perdita di dati significativi.
	\end{itemize}

\myparagraph {Pianificazione}
 Oltre alle strategie di backup già illustrate precedentemente, gli \italics{Amministratori} si impegnano a garantire che tutti i membri del gruppo dispongano della stessa versione del \gl{software}. Vengono inoltre scelti per il progetto \gl{software} considerati affidabili.
 
\myparagraph {Contenimento}
 Qualora un membro dovesse rilevare problemi \gl{software} provvederà a comunicarlo tempestivamente agli \italics{Amministratori}; qualora il problema riguardasse invece l'intero gruppo, spetterà al \italics{Responsabile} decidere se cambiare \gl{software}.


\subsection{Livello personale}
\subsubsection{Problemi personali dei membri}
\myparagraph {Identificazione}
 Vengono presi in considerazione gli eventi imprevisti che potrebbero influire sulla disponibilità dei membri del gruppo, come per esempio periodi di malattia o complicazioni famigliari.
 
\myparagraph {Analisi}
	\begin{itemize}
		\item \bold{Probabilità di insorgenza:} Media.
		\item \bold{Livello di rischio:} Medio.
		\item \bold{Possibili conseguenze:} ritardi nei tempi prestabiliti.
	\end{itemize}
	
\myparagraph {Pianificazione}
 I membri del gruppo, sfruttando i numerosi canali di comunicazioni di cui dispongono, provvederanno a informare tempestivamente i propri compagni e, in particolare, il \italics{Responsabile di Progetto} in caso di imprevisti. In questo caso il \italics{Responsabile} provvederà a ridurre il carico lavorativo e a modificare la pianificazione. Per arginare questo rischio sono stati previsti, ove possibili, dei periodi di \gl{slack}.

\myparagraph {Contenimento}
 Qualora dovessero insorgere problemi, il \italics{Responsabile di Progetto} ripartirà il lavoro, andando a sfruttare, se necessario, i periodi di \gl{slack} prestabiliti. 

\subsubsection{Problemi interni tra i membri}
\myparagraph {Identificazione}
 Non tutti i membri del gruppo si conoscono, quindi è ragionevole preventivare delle difficoltà di lavoro che possono nascere da screzi personali. Inoltre, anche i membri che si conoscono non hanno mai lavorato in gruppi tanto numerosi e complessi.
 
\myparagraph {Analisi}
	\begin{itemize}
		\item \bold{Probabilità di insorgenza:} Bassa.
		\item \bold{Livello di rischio:} Alto.
		\item \bold{Possibili conseguenze:} ritardi nei tempi prestabiliti, blocco delle attività, peggioramento dell'ambiente lavorativo.
	\end{itemize}
	
\myparagraph {Pianificazione}
 I membri del gruppo si impegnano a tenere un atteggiamento responsabile e maturo, andando subito a esternare eventuali dissapori tra loro e conferendo con il \italics{Responsabile di Progetto} quando non riescano a gestire le loro incompatibilità.
 
\myparagraph {Contenimento}
 Se dovessero sorgere problemi per cui sia necessario l'intervento del \italics{Responsabile di Progetto}, questi cercherà di appianare le divergenze e, se necessario, provvederà a riorganizzare il lavoro, separando i membri coinvolti.


\subsection{Livello organizzativo}
\subsubsection{Problemi di versionamento}
\myparagraph {Identificazione}
 Poiché i membri del gruppo non hanno mai lavorato prima a progetti così complessi e che richiedessero coordinazione tra tante persone, esiste la possibilità che si creino problemi di \gl{versionamento} quando più persone sono incaricate di redigere o verificare lo stesso documento o la stessa parte di codice.
 
\myparagraph {Analisi}
	\begin{itemize}
		\item \bold{Probabilità di insorgenza:} Media.
		\item \bold{Livello di rischio:} Alto.
		\item \bold{Possibili conseguenze:} ritardi nei tempi prestabiliti, confusione e errori.
	\end{itemize}
	
\myparagraph {Pianificazione}
 È stato predisposto un \gl{tracker} in modo che ogni cambiamento sia sempre notificato e controllato. Il \italics{Responsabile di Progetto} fornisce una divisione di ruoli che i membri si impegnano a seguire senza accavallarsi; inoltre il \italics{Responsabile} potrà tenere traccia del lavoro dei membri tramite il \gl{software} \gl{Redmine}.
 
\myparagraph {Contenimento}
 In caso di problemi si provvederà e recuperare l'ultima versione corretta.


\subsubsection{Errata valutazione dei costi}
\myparagraph {Identificazione}
 Poiché i membri del gruppo non hanno esperienze precedenti, è possibile che, nella fase di pianificazione, vengano sottostimati i costi, non solo economici, ma anche in termini di tempo.
 
\myparagraph {Analisi}
	\begin{itemize}
		\item \bold{Probabilità di insorgenza:} Alta.
		\item \bold{Livello di rischio:} Medio.
		\item \bold{Possibili conseguenze:} ritardi nei tempi prestabiliti.
	\end{itemize}
	
\myparagraph {Pianificazione}
 Il \italics{Responsabile di Progetto} terrà sempre sotto controllo, tramite la \gl{dashboard}, lo stato di avanzamento delle varie fasi rispetto alla pianificazione iniziale. Dove possibile andrà a sfruttare i periodi di \gl{slack} predisposti, eventualmente ripianificando il lavoro del gruppo.
 
\myparagraph {Contenimento}
 In caso di ritardi il lavoro verrà ripianificato, cercando di rientrare nei tempi stabiliti.

\subsection{Strumenti}
\subsubsection{Mancante o insufficiente conoscenza degli strumenti}
\myparagraph {Identificazione}
 Lo sviluppo del progetto richiede l'utilizzo di numerosi strumenti mai utilizzati prima dal gruppo o il cui uso non è stato esaustivo. Per questo è ragionevole quantificare un iniziale periodo di apprendimento.

\myparagraph{Analisi}
	\begin{itemize}
		\item \bold{Probabilità di insorgenza:} Alta.
		\item \bold{Livello di rischio:} Alto.
		\item \bold{Possibili conseguenze:} ritardi nei tempi prestabiliti, rallentamenti nelle varie attività.
	\end{itemize}
	
\myparagraph {Pianificazione}
 Ogni volta che sarà ritenuto necessario, dal \italics{Responsabile di progetto} e dagli \italics{Amministratori}, utilizzare un nuovo strumento, tutti i membri del gruppo verranno subito avvertiti. L'introduzione di nuovi strumenti sarà sempre una scelta pesata e responsabile, consapevole degli alti livelli di rischio che può comportare. Verrà anche fornita la documentazione necessaria per comprendere lo strumento adottato e si chiede che i vari membri, per qualsiasi dubbio o incomprensione, avvertano subito il \italics{Responsabile}.\\
 Essendo questo un rischio ad alta probabilità e criticità, si intende sfruttare al meglio ogni membro del gruppo, non pretendendo da tutti lo stesso livello di conoscenza di tutti gli strumenti, ma andando a specializzare i singoli membri in base alle attitudini e alle esperienze personali.
	
\myparagraph {Contenimento}
 Se tutti i membri del gruppo dovessero riscontrare gravi difficoltà nell'apprendimento di uno strumento, o se il \italics{Responsabile} riterrà i tempi di apprendimento troppo lunghi, egli stesso avrà cura di selezionare un altro strumento più adatto.

\subsection{Requisiti}
\subsubsection{Errata comprensione dei requisiti}
\myparagraph{Identificazione}
 Si preventiva una mancata comprensione tra il gruppo e il proponente, che può portare a inesattezze nei requisiti e nella comprensione del prodotto che il proponente e il committente si aspettano.
 
\myparagraph{Analisi}
	\begin{itemize}
		\item \bold{Probabilità di insorgenza:} Media.
		\item \bold{Livello di rischio:} Alto.
		\item \bold{Possibili conseguenze:} ritardi nei tempi prestabiliti, rallentamenti nelle varie fasi, possibilità di dover tornare sui propri passi.
	\end{itemize}
	
\myparagraph {Pianificazione}\label{Pianificazione}
 Si è cercato di ridurre al minimo la possibilità di insorgenza di questo rischio predisponendo numerosi incontri col proponente e col committente. Soprattutto nella fase di Analisi, si vuole rendere il proponente partecipe, chiedendo spesso il suo parere e il suo feedback sia con incontri, sia tramite mail.

\myparagraph {Contenimento}
 Ogni qualvolta il committente solleverà perplessità si cercherà una soluzione ottimale per le parti; inoltre sono stati fissati incontri per mostrare la documentazione anche al proponente. Il gruppo provvederà a correggere gli errori eventualmente segnalati.

\section {Pianificazione}
Di seguito un'analisi dettagliata delle attività e delle relative sottoattività in base alla suddivisione presentata nella \hyperref[Ciclo di vita]{sezione 2} del documento corrente.
Sono stati inseriti dei periodi di \gl{slack}, per evitare di avere tempi troppo serrati e per lasciare un margine per arginare eventuali imprevisti.

\subsection{Attività  di Analisi requisiti utente}
\subsubsection{Periodo}
 Dal 18.12.2015 al 22.01.2016.\\
 Questa attività comincia con la creazione del gruppo e si conclude con la consegna dei documenti per accedere alla Revisione dei Requisiti.

\subsubsection{Sottattività}
\begin{itemize}
	\item \bold{Individuazione degli strumenti:} vengono discussi gli strumenti necessari al buon funzionamento del gruppo. I membri condividono le proprie conoscenze ed esperienze e selezionano gli strumenti  da utilizzare.
	\item \bold{Creazione documentazione necessaria al progetto}: in particolare, si procede alla prima stesura dei	seguenti documenti:
	\begin{description}
		\item \bold{Norme di Progetto:} viene redatto il documento \doc{Norme di Progetto v1.0}. In parallelo alla scelta del capitolato, vengono decise le regole che tutti i componenti del gruppo rispetteranno e le modalità di lavoro. Questo documento è indipendente dal capitolato scelto.
		\item \bold{Studio di Fattibilità:} dopo avere soppesato i pro e i contro di tutti i capitolati proposti, viene scelto quello che il gruppo si impegna a sviluppare. Viene quindi redatto il documento \doc{Studio di Fattibilità v1.0}.
		\item \bold{Piano di Progetto:} il \italics{Responsabile di Progetto} pianifica le attività del gruppo e stima tempi e spese necessari. Si crea il documento \doc{Piano di Progetto v1.0}.
		\item \bold{Analisi dei Requisiti:} in seguito alla discussione del capitolato e a un incontro col proponente, vengono raccolti i requisiti emersi e viene quindi stilato il documento \doc{Analisi dei Requisiti v1.0}.
		\item \bold{Piano di Qualifica:}  vengono definiti gli standard qualitativi e le metriche da utilizzare per verifica e validazione. Creazione di \doc{Piano di Qualifica v.1.0}.
		\item \bold{Glossario:} vengono individuate le parole da inserire nel glossario e queste vengono aggiunte al \gl{database}. Questa attività avverrà contemporaneamente alla stesura dei documenti. Quindi si procede alla creazione di \doc{Glossario v1.0}.
	\end{description}
	\item \bold{Incontri col proponente:} vengono fissati due incontri col proponente, prima e dopo la stesura dell'\doc{Analisi dei Requisiti}. Il primo incontro servirà per definire con chiarezza i requisiti richiesti, il secondo per verificarli.		 
\end{itemize}

%GANTT 1
\newpage
\begin{figure}
\includegraphics[scale=0.25]{Img/Grafici_Gantt/Analisi_dei_requisiti.pdf}
\caption{ Diagramma di Gantt: Attività di analisi dei requisiti}
\end{figure}

\subsection{Attività di Raffinamento dei requisiti}
\subsubsection{Periodo}
Dal 16.02.2016 al 23.02.2016.\\
Comincia dopo aver preso visione dell'esito della Revisione dei Requisiti e dura il tempo necessario per una revisione dei documenti.


\subsubsection{Sottoattività}
\begin{itemize}
	\item \bold{Correzione e incremento:} in base all'esito della Revisione dei Requisiti e agli appunti del proponente, vengono corretti e incrementati i documenti precedentemente prodotti. I documenti così approvati passano alla loro v2.0.
\end{itemize}

%GANTT 2
\newpage
\begin{figure}
    \includegraphics[scale=0.7]{Img/Grafici_Gantt/Raffinamento_requisiti.pdf}
	\caption{ \gl{Diagramma di Gantt}: Raffinamento dei requisiti}
\end{figure}

\subsection{Attività di Progettazione architetturale}
\subsubsection{Periodo}
Dal 24.02.2016 al 11.04.2016.\\
Segue immediatamente la fine dell'attività di Raffinamento dei requisiti  e termina con un incontro col proponente per illustrargli l'architettura scelta e con la consegna dei documenti per la Revisione di Progettazione.

\subsubsection{Sottattività}
\begin{itemize}
	\item \bold{Creazione di nuovi documenti:} viene redatto il documento \doc{Specifica Tecnica}. La sua stesura è la sottoattività principale;  vengono fatte scelte progettuali riguardanti il prodotto come, per esempio, i \gl{design pattern}.
	\item \bold{Incremento e verifica dei documenti precedenti:}
	\begin{description}
		\item \bold{Norme di Progetto:} vengono incrementate per poter redigere il documento \doc{Specifica Tecnica v1.0}. Dopo esser stato verificato e approvato il documento è nella sua versione \doc{Norme di Progetto v3.0}.
		\item \bold{Piano di Progetto:} se necessario, viene modificato alla luce del reale avanzamento delle attività;  vengono aggiunti i consuntivi delle attività terminate e il documento viene quindi approvato, passando alla versione \doc{Piano di Progetto v3.0}.
		\item \bold{Piano di Qualifica:} viene aggiornato con l'esito della Revisione dei Requisiti, vengono inoltre pianificati i test che il gruppo prevede di svolgere sul proprio prodotto. Si ottiene il documento \doc{Piano di Qualifica v.3.0}.
		\item \bold{Glossario:} vengono costantemente aggiunte le eventuali definizioni emerse durante le sottoattività in corso.  Il documento finale è il \doc{Glossario v3.0}.
	\end{description}	
	\item \bold{Incontro col proponente:} viene fissato un incontro col proponente per illustrargli l'architettura scelta.	 
\end{itemize}

%GANTT 3
\begin{figure}[!ht]
    \includegraphics[scale=0.2]{Img/Grafici_Gantt/Progettazione_architetturale.pdf}
	\caption{ \gl{Diagramma di Gantt}: Progettazione architetturale}
\end{figure}

\subsection{Attività di Progettazione di dettaglio e codifica }
\subsubsection{Periodo}
Dal 12.04.2016 al 16.05.2016.\\
Segue immediatamente l'attività precedente e termina con la consegna della documentazione per accedere alla Revisione di Qualifica.


\subsubsection{Sottattività}
\begin{itemize}
	\item \bold{Incremento e verifica dei documenti precedenti:} se necessario, verranno incrementati o corretti i documenti già scritti in modo simile a quanto già descritto nelle attività precedenti, i documenti così approvati passano alla versione successiva.
	\item \bold{Creazione di nuovi documenti:} vengono redatti i seguenti documenti:
	\begin{description}
		\item \bold{Definizione di Prodotto:} si procede alla stesura di \doc{Definizione di Prodotto v1.0}, che definisce la struttura del prodotto che il gruppo intende sviluppare seguendo le direttive della \doc{Specifica Tecnica}.
		\item \bold{Manuale Utente:} stesura documento \doc{Manuale Utente v1.0}.
		\item \bold{Manuale Amministratore:} stesura documento \doc{Manuale Amministratore v1.0}.
	\end{description}	
	\item \bold{Codifica:} i \italics{Programmatori} hanno il compito di codificare  requisiti seguendo le direttive della \doc{Definizione di prodotto}.		 
\end{itemize}

%GANTT 4
\newpage
\begin{figure}[!ht]
	\includegraphics[scale=0.25]{Img/Grafici_Gantt/Progettazione(dett-cod).pdf}
	\caption{ \gl{Diagramma di Gantt}: Progettazione di dettaglio e codifica}
\end{figure}

\subsection{Attività di validazione}
\subsubsection{Periodo}
Dal 17.05.2016 al 10.06.2016\\
Segue immediatamente le attività di progettazione di dettaglio  e termina con la consegna dei documenti necessari per la Revisione di Accettazione.

\subsubsection{Sottattività}
\begin{itemize}
	\item \bold{Incremento e verifica dei documenti precedenti:} se necessario, verranno incrementati o corretti i documenti già scritti in modo simile a quanto già descritto nelle attività precedenti, i documenti così approvati passano alla versione successiva.
	\item \bold{Esecuzione di test:} esecuzione di test di validazione come descritto nel \doc{Piano di Qualifica}, che verrà conseguentemente aggiornato.
	\item \bold{Individuazione e correzione di bug.} 
	\item \bold{Collaudo del prodotto finale prima della consegna.} 
\end{itemize}

%GANTT 5
\newpage
\begin{figure}[!ht]
	\includegraphics[scale=0.3]{Img/Grafici_Gantt/Validazione.pdf}
	\caption{ \gl{Diagramma di Gantt}: Validazione}
\end{figure}

\newpage
\section {Preventivo}\label{Preventivo}
Nonostante le attività di Analisi dei requisiti utente e di Raffinamento dei requisiti \bold{non} siano a carico del committente, sono state comunque riportate per completezza. Un riepilogo della divisione dei ruoli e dei costi rendicontati è disponibile, rispettivamente, nelle sottosezioni \hyperref[Prospetto orario rendicontato]{6.6.3} e \hyperref[Prospetto economico rendicontato totale]{6.6.4}.
\subsection{Analisi dei requisiti utente}
\subsubsection{Prospetto orario}
Durante questa prima attività ogni componente del gruppo ricoprirà i seguenti ruoli:

\begin{tabella}{l!{\VRule}c!{\VRule}c!{\VRule}c!{\VRule}c!{\VRule}c!{\VRule}c!{\VRule}c!{\VRule}c}
	
	\color{white} \bold{Nome} & \color{white} \bold{Responsabile} &\color{white} \bold{Amm} & \color{white} \bold{An} & \color{white} \bold{Pt} & \color{white} \bold{Pr} & \color{white} \bold{Ver} & \color{white} \bold{Ore totali persona} \\
	\endfirsthead
	Giacomo Beltrame & 0 & 22 & 0 & 0 & 0 & 10 & 32\\
	Rudy Berton & 0 & 0 & 20 & 0 & 0 & 10 & 30\\
    Simone Boccato & 0 & 0 & 30 & 0 & 0 & 8 & 38\\
    Michela De Bortoli & 0 & 0 & 26 & 0 & 0 & 12 & 38\\
    Vassilikì Menarin & 20 & 0 & 0 & 0 & 0 & 15 & 35\\
    Filippo Tesser & 0 & 17 & 0 & 0 & 0 & 18 & 35\\
    Miki Violetto & 0 & 0 & 26 & 0 & 0 & 10 & 36\\   
	
	\rowcolor{white}  
	\caption{Prospetto orario attività di analisi}	    	
	
\end{tabella}
\newpage
\begin{figure}[!ht]
	\centering
	\includegraphics[scale=0.5]{Img/Grafici/Ist01.pdf}
	\caption{ Istogramma: Prospetto orario attività di analisi}
\end{figure}

\newpage
\subsubsection{Prospetto economico}
Il prospetto economico per questa attività è illustrato in tabella. Notare che le spese per questa attività \bold{non} sono a carico del proponente.

\begin{tabella}{l!{\VRule}c!{\VRule}c}
	
	\color{white} \bold{Ruolo} & \color{white} \bold{Ore} &\color{white} \bold{Spese} \\
	\endfirsthead
	Responsabile & 20 & € 600 \\
	Amministraore & 39 & € 780 \\
	Analista & 102 & € 2550 \\
	Progettista & 0 & € 0 \\
	Programmatore & 0 & € 0 \\
	Verificatore & 83 & € 1245 \\
	Totale & 244  & € 5175\\
	
	\rowcolor{white}  
	\caption{Prospetto economico attività di analisi}	    	
	
\end{tabella}

\begin{figure}[!ht]
	\centering
	\includegraphics[scale=0.5]{Img/Grafici/Aer01.pdf}
	\caption{ Areogramma: Ore per ruolo durante l'attività di analisi}
\end{figure}

\newpage
\subsection{Raffinamento dei requisiti}
\subsubsection{Prospetto orario}
Durante questa attività ogni componente del gruppo ricoprirà i seguenti ruoli:

\begin{tabella}{l!{\VRule}c!{\VRule}c!{\VRule}c!{\VRule}c!{\VRule}c!{\VRule}c!{\VRule}c!{\VRule}c}
	
	\color{white} \bold{Nome} & \color{white} \bold{Responsabile} &\color{white} \bold{Amm} & \color{white} \bold{An} & \color{white} \bold{Pt} & \color{white} \bold{Pr} & \color{white} \bold{Ver} & \color{white} \bold{Ore totali persona} \\
	\endfirsthead
	Giacomo Beltrame & 0 & 0 & 4 & 0 & 0 & 7 & 11\\
	Rudy Berton & 0 & 7 & 0 & 0 & 0 & 6 & 13\\
	Simone Boccato & 0 & 6 & 0 & 0 & 0 & 5 & 11\\
	Michela De Bortoli & 0 & 0 & 0 & 0 & 0 & 10 & 10\\
	Vassilikì Menarin & 0 & 0 & 7 & 0 & 0 & 5 & 12\\
	Filippo Tesser & 0 & 0 & 4 & 0 & 0 & 7 & 11\\
	Miki Violetto & 12 & 3 & 0 & 0 & 0 & 3 & 18\\   
	
	\rowcolor{white}  
	\caption{Prospetto orario attività di raffinamento dei requisiti}	    	
	
\end{tabella}

\begin{figure}[!ht]
	\centering
	\includegraphics[scale=0.5]{Img/Grafici/Ist02.pdf}
	\caption{ Istogramma: Prospetto orario attività di raffinamento dei requisiti}
\end{figure}

\newpage
\subsubsection{Prospetto economico}
Il prospetto economico per questa attività è illustrato in tabella. Notare che le spese per questa attività \bold{non} sono a carico del proponente.

\begin{tabella}{l!{\VRule}c!{\VRule}c}
	
	\color{white} \bold{Ruolo} & \color{white} \bold{Ore} &\color{white} \bold{Spese} \\
	\endfirsthead
	Responsabile & 12 & € 360 \\
	Amministraore & 16 & € 320\\
	Analista & 15 & € 375 \\
	Progettista & 0 & € 0 \\
	Programmatore & 0 & € 0 \\
	Verificatore & 43 & € 645 \\
	Totale & 86  & € 1700\\
	
	\rowcolor{white}  
	\caption{Prospetto economico attività di raffinamento dei requisiti}	    	
	
\end{tabella}

\begin{figure}[!ht]
	\centering
	\includegraphics[scale=0.5]{Img/Grafici/Aer02.pdf}
	\caption{ Areogramma: Ore per ruolo durante l'attività di raffinamento dei requisiti}
\end{figure}

\newpage
\subsection{Progettazione architetturale}
\subsubsection{Prospetto orario}
Durante questa attività ogni componente del gruppo ricoprirà i seguenti ruoli:

\begin{tabella}{l!{\VRule}c!{\VRule}c!{\VRule}c!{\VRule}c!{\VRule}c!{\VRule}c!{\VRule}c!{\VRule}c}
	
	\color{white} \bold{Nome} & \color{white} \bold{Responsabile} &\color{white} \bold{Amm} & \color{white} \bold{An} & \color{white} \bold{Pt} & \color{white} \bold{Pr} & \color{white} \bold{Ver} & \color{white} \bold{Ore totali persona} \\
	\endfirsthead
	Giacomo Beltrame & 0 & 0 & 5 & 20 & 0 & 0 & 25\\
	Rudy Berton & 0 & 10 & 0 & 0 & 0 & 13 & 23\\
	Simone Boccato & 0 & 0 & 0 & 5 & 0 & 20 & 25\\
	Michela De Bortoli & 10 & 0 & 0 & 15 & 0 & 0 & 25\\
	Vassilikì Menarin & 0 & 0 & 0 & 15 & 0 & 12 & 27\\
	Filippo Tesser & 10 & 0 & 8 & 0 & 0 & 8 & 26\\
	Miki Violetto & 0 & 7 & 0 & 10 & 0 & 6 & 23\\   
	
	\rowcolor{white}  
	\caption{Prospetto orario attività di progettazione architetturale}	    	
	
\end{tabella}

\begin{figure}[!ht]
	\centering
		\includegraphics[scale=0.5]{Img/Grafici/Ist03.pdf}
	\caption{ Istogramma: Prospetto orario attività di progettazione architetturale}
\end{figure}

\newpage
\subsubsection{Prospetto economico}
Il prospetto economico per questa attività è illustrato in tabella. 

\begin{tabella}{l!{\VRule}c!{\VRule}c}
	
	\color{white} \bold{Ruolo} & \color{white} \bold{Ore} &\color{white} \bold{Spese} \\
	\endfirsthead
	Responsabile & 20 & € 600 \\
	Amministraore & 17 & € 340\\
	Analista & 13 & € 325 \\
	Progettista & 65 & € 1430 \\
	Programmatore & 0 & € 0 \\
	Verificatore & 59 & € 885 \\
	Totale & 174 & € 3580\\
	
	\rowcolor{white}  
	\caption{Prospetto economico attività di progettazione architetturale}	    	
	
\end{tabella}

\begin{figure}[!ht]
	\centering
		\includegraphics[scale=0.5]{Img/Grafici/Aer03.pdf}
	\caption{ Areogramma: Ore per ruolo durante l'attività di progettazione architetturale}
\end{figure}

\newpage
\subsection{Progettazione di dettaglio e codifica}
\subsubsection{Prospetto orario}
Durante questa attività ogni componente del gruppo ricoprirà i seguenti ruoli:

\begin{tabella}{l!{\VRule}c!{\VRule}c!{\VRule}c!{\VRule}c!{\VRule}c!{\VRule}c!{\VRule}c!{\VRule}c}
	
	\color{white} \bold{Nome} & \color{white} \bold{Responsabile} &\color{white} \bold{Amm} & \color{white} \bold{An} & \color{white} \bold{Pt} & \color{white} \bold{Pr} & \color{white} \bold{Ver} & \color{white} \bold{Ore totali persona} \\
	\endfirsthead
	Giacomo Beltrame & 13 & 0 & 6 & 0 & 20 & 18 & 57\\
	Rudy Berton & 0 & 0 & 0 & 30 & 0 & 22 & 52\\
	Simone Boccato & 0 & 6 & 0 & 17 & 25 & 7 & 55\\
	Michela De Bortoli & 0 & 10 & 0 & 20 & 25 & 0 & 55\\
	Vassilikì Menarin & 0 & 0 & 7 & 0 & 26 & 21 & 54\\
	Filippo Tesser & 0 & 0 & 0 & 20 & 15 & 16 & 51\\
	Miki Violetto & 0 & 0 & 0 & 12 & 29 & 13 & 54\\   
	
	\rowcolor{white}  
	\caption{Prospetto orario attività di progettazione di dettaglio e codifica}	    	
	
\end{tabella}

\begin{figure}[!ht]
	\centering
		\includegraphics[scale=0.5]{Img/Grafici/Ist04.pdf}
	\caption{ Istogramma: Prospetto orario attività di progettazione di dettaglio e codifica}
\end{figure}

\newpage
\subsubsection{Prospetto economico}
Il prospetto economico per questa attività è illustrato in tabella. 

\begin{tabella}{l!{\VRule}c!{\VRule}c}
	
	\color{white} \bold{Ruolo} & \color{white} \bold{Ore} &\color{white} \bold{Spese} \\
	\endfirsthead
	Responsabile & 13 & € 390 \\
	Amministraore & 16 & € 320\\
	Analista & 13 & € 325 \\
	Progettista & 99 & € 2178 \\
	Programmatore & 140 & € 2100 \\
	Verificatore & 97 & € 1455\\
	Totale & 378 & € 6768\\
	
	\rowcolor{white}  
	\caption{Prospetto economico attività di progettazione di dettaglio e codifica}	    	
	
\end{tabella}

\begin{figure}[!ht]
	\centering
		\includegraphics[scale=0.5]{Img/Grafici/Aer04.pdf}
	\caption{ Areogramma: Ore per ruolo durante l'attività di progettazione di dettaglio e codifica}
\end{figure}

\newpage
\subsection{Validazione}
\subsubsection{Prospetto orario}
Durante questa attività ogni componente del gruppo ricoprirà i seguenti ruoli:

\begin{tabella}{l!{\VRule}c!{\VRule}c!{\VRule}c!{\VRule}c!{\VRule}c!{\VRule}c!{\VRule}c!{\VRule}c}
	
	\color{white} \bold{Nome} & \color{white} \bold{Responsabile} &\color{white} \bold{Amm} & \color{white} \bold{An} & \color{white} \bold{Pt} & \color{white} \bold{Pr} & \color{white} \bold{Ver} & \color{white} \bold{Ore totali persona} \\
	\endfirsthead
	Giacomo Beltrame & 0 & 0 & 0 & 13 & 0 & 7 & 20\\
	Rudy Berton & 15 & 0 & 0 & 0 & 13 & 0 & 28\\
	Simone Boccato & 10 & 0 & 0 & 0 & 0 & 11 & 21\\
	Michela De Bortoli & 0 & 0 & 0 & 10 & 0 & 12 & 22\\
	Vassilikì Menarin & 0 & 11 & 0 & 10 & 0 & 0 & 21\\
	Filippo Tesser & 0 & 0 & 0 & 0 & 10 & 14 & 24\\
	Miki Violetto & 0 & 0 & 0 & 12 & 0 & 12 & 24\\   
	
	\rowcolor{white}  
	\caption{Prospetto orario attività di validazione}	    	
	
\end{tabella}

\begin{figure}[!ht]
	\centering
		\includegraphics[scale=0.5]{Img/Grafici/Ist05.pdf}
	\caption{ Istogramma: Prospetto orario attività di validazione}
\end{figure}

\newpage
\subsubsection{Prospetto economico}
Il prospetto economico per questa attività è illustrato in tabella. 

\begin{tabella}{l!{\VRule}c!{\VRule}c}
	
	\color{white} \bold{Ruolo} & \color{white} \bold{Ore} &\color{white} \bold{Spese} \\
	\endfirsthead
	Responsabile & 25 & € 750 \\
	Amministraore & 11 & € 220\\
	Analista & 0 & € 0 \\
	Progettista & 45 & € 990 \\
	Programmatore & 23 & € 345 \\
	Verificatore & 56 & € 840\\
	Totale & 160 & € 3145\\
	
	\rowcolor{white}  
	\caption{Prospetto economico attività di validazione}	    	
	
\end{tabella}

\begin{figure}[!ht]
	\centering
		\includegraphics[scale=0.5]{Img/Grafici/Aer05.pdf}
	\caption{ Areogramma: Ore per ruolo durante l'attività di validazione}
\end{figure}

\newpage
\subsection{Riepilogo conclusivo}\label{Riepilogo conclusivo rendicontato}
\subsubsection{Prospetto orario totale}
In tabella sono riassunte le ore totali (rendicontate e non) per ruolo che ogni membro del gruppo ricoprirà:

\begin{tabella}{l!{\VRule}c!{\VRule}c!{\VRule}c!{\VRule}c!{\VRule}c!{\VRule}c!{\VRule}c!{\VRule}c}
	
	\color{white} \bold{Nome} & \color{white} \bold{Responsabile} &\color{white} \bold{Amm} & \color{white} \bold{An} & \color{white} \bold{Pt} & \color{white} \bold{Pr} & \color{white} \bold{Ver} & \color{white} \bold{Ore totali persona} \\
	\endfirsthead
	Giacomo Beltrame & 13 & 22 & 15 & 33 & 20 & 42 & 145\\
	Rudy Berton & 15 & 17 & 20 & 30 & 13 & 51 & 146\\
	Simone Boccato & 10 & 12 & 30 & 22 & 25 & 51 & 150\\
	Michela De Bortoli & 10 & 10 & 26 & 45 & 25 & 34 & 150\\
	Vassilikì Menarin & 20 & 11 & 14 & 25 & 26 & 53 & 149\\
	Filippo Tesser & 10 & 17 & 12 & 20 & 25 & 63 & 147\\
	Miki Violetto & 12 & 10 & 26 & 34 & 29 & 44 & 155\\   
	
	\rowcolor{white}  
	\caption{Prospetto orario totale}	    	
	
\end{tabella}

\begin{figure}[!ht]
	\centering
		\includegraphics[scale=0.5]{Img/Grafici/Ist06.pdf}
	\caption{ Istogramma: Prospetto orario totale}
\end{figure}

\newpage
\subsubsection{Prospetto economico totale}
Il prospetto economico totale (sia per attività rendicontate, sia non) è illustrato in tabella e viene incluso per completezza. 

\begin{tabella}{l!{\VRule}c!{\VRule}c}
	
	\color{white} \bold{Ruolo} & \color{white} \bold{Ore} &\color{white} \bold{Spese} \\
	\endfirsthead
	Responsabile & 90 & € 2700 \\
	Amministraore & 99 & € 1980\\
	Analista & 143 & € 3575 \\
	Progettista & 209 & € 4598 \\
	Programmatore & 163 & € 2445 \\
	Verificatore & 338 & € 5070\\
	Totale & 1042 & € 20368\\
	
	\rowcolor{white}  
	\caption{Prospetto economico totale}	    	
	
\end{tabella}

\begin{figure}[!ht]
	\centering
		\includegraphics[scale=0.5]{Img/Grafici/Aer06.pdf}
	\caption{ Areogramma:Ore per ruolo totali}
\end{figure}

\newpage
\subsubsection{Prospetto orario rendicontato}\label{Prospetto orario rendicontato}
In tabella sono riassunte le ore totali rendicontate per ruolo che ogni membro del gruppo ricoprirà:

\begin{tabella}{l!{\VRule}c!{\VRule}c!{\VRule}c!{\VRule}c!{\VRule}c!{\VRule}c!{\VRule}c!{\VRule}c}
	
	\color{white} \bold{Nome} & \color{white} \bold{Responsabile} &\color{white} \bold{Amm} & \color{white} \bold{An} & \color{white} \bold{Pt} & \color{white} \bold{Pr} & \color{white} \bold{Ver} & \color{white} \bold{Ore totali persona} \\
	\endfirsthead
	Giacomo Beltrame & 13 & 0 & 11 & 33 & 20 & 25 & 102\\
	Rudy Berton & 15 & 10 & 0 & 30 & 13 & 35 & 103\\
	Simone Boccato & 10 & 6 & 0 & 22 & 25 & 38 & 101\\
	Michela De Bortoli & 10 & 10 & 0 & 45 & 25 & 12 & 102\\
	Vassilikì Menarin & 0 & 11 & 7 & 25 & 26 & 33 & 102\\
	Filippo Tesser & 10 & 0 & 8 & 20 & 25 & 38 & 101\\
	Miki Violetto & 0 & 7 & 0 & 34 & 29 & 31 & 101\\   
	
	\rowcolor{white}  
	\caption{Prospetto orario rendicontato totale}	    	
	
\end{tabella}

\begin{figure}[!ht]
	\centering
		\includegraphics[scale=0.5]{Img/Grafici/Ist07.pdf}
	\caption{ Istogramma: Prospetto orario rendicontato totale}
\end{figure}

\newpage
\subsubsection{Prospetto economico rendicontato}\label{Prospetto economico rendicontato totale}
Il prospetto economico per le ore rendicontate per ogni ruolo è illustrato in tabella. 

\begin{tabella}{l!{\VRule}c!{\VRule}c}
	
	\color{white} \bold{Ruolo} & \color{white} \bold{Ore} &\color{white} \bold{Spese} \\
	\endfirsthead
	Responsabile & 58 & € 1740 \\
	Amministraore & 44 & € 880\\
	Analista & 26 & € 650 \\
	Progettista & 209 & € 4598 \\
	Programmatore & 163 & € 2445 \\
	Verificatore & 212 & € 3180\\
	Totale & 712 & € 13493\\
	
	\rowcolor{white}  
	\caption{Prospetto economico rendicontato}	    	
	
\end{tabella}

\begin{figure}[!ht]
	\centering
		\includegraphics[scale=0.5]{Img/Grafici/Aer07.pdf}
	\caption{ Areogramma: Ore per ruolo rendicontate}
\end{figure}

\newpage
\appendix
\section{Organigramma}

\subsection{Redazione}

\begin{tabella}{l!{\VRule}c!{\VRule}c}
	
	\color{white} \bold{Nominativo} & \color{white} \bold{Data} &\color{white} \bold{Firma} \\
	\endfirsthead
	
	Vassilikì Menarin & 14.01.2016 & \includegraphics[scale=0.15]{Img/Firme/Viki.png} \\

\end{tabella}

\subsection{Approvazione}

\begin{tabella}{l!{\VRule}c!{\VRule}c}
	
	\color{white} \bold{Nominativo} & \color{white} \bold{Data} &\color{white} \bold{Firma} \\
	\endfirsthead
	
	Vassilikì Menarin & 17.01.2016 & \includegraphics[scale=0.15]{Img/Firme/Viki.png} \\
	Tullio Vardanega &  &  \\  		
	
\end{tabella}

\subsection{Accettazione dei componenti}

\begin{tabella}{l!{\VRule}c!{\VRule}c}
	
	\color{white} \bold{Nominativo} & \color{white} \bold{Data di accettazione} &\color{white} \bold{Firma} \\
	\endfirsthead
	
		Giacomo Beltrame & 18.12.2015 & \includegraphics[scale=0.15]{Img/Firme/Giacomo.png} \\
		Rudy Berton & 18.12.2015 & \includegraphics[scale=0.15]{Img/Firme/Rudy.png} \\
		Simone Boccato & 18.12.2015 & \includegraphics[scale=0.15]{Img/Firme/Simone.png} \\
		Michela De Bortoli & 18.12.2015 & \includegraphics[scale=0.15]{Img/Firme/Michela.png} \\
		Vassilikì Menarin & 18.12.2015 & \includegraphics[scale=0.15]{Img/Firme/Viki.png} \\
		Filippo Tesser& 18.12.2015 & \includegraphics[scale=0.15]{Img/Firme/Filippo.png} \\
		Miki Violetto & 18.12.2015 & \includegraphics[scale=0.15]{Img/Firme/Miki.png} \\	 		
	
\end{tabella}

\subsection{Componenti}

\begin{tabella}{l!{\VRule}c!{\VRule}c}
	
	\color{white} \bold{Nominativo} & \color{white} \bold{Matricola} &\color{white} \bold{Indirizzo e-mail} \\
	\endfirsthead
	
	Giacomo Beltrame & 1006153 & giacomo.beltrame.91@gmail.com \\
	Rudy Berton & 1049443 & rudyberton92@gmail.com \\
	Simone Boccato & 1047882 & boccato92@gmail.com \\
	Michela De Bortoli & 1027000 & shirun9215@gmail.com\\
	Vassilikì Menarin & 1049663 & vassiliki.menarin@gmail.com \\	Filippo Tesser& 1009236 & filippo90t@gmail.com \\
	Miki Violetto & 1029140 & mikiww2@gmail.com\\	 		
	
\end{tabella}

\subsection{Note}
I ruoli sono stati ripartiti come visibile nelle tabelle presenti nella sezione \hyperref[Preventivo]{Preventivo}. Si è fatto in modo che tutti i membri del gruppo ricoprissero tutti i ruoli e che i verificatori non verificassero mai le attività svolte da loro stessi.


\end{document}