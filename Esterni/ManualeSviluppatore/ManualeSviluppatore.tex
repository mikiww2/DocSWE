% Nome del file: ManualeSviluppatore.tex
% Percorso: \gl{template}
% Autore: Vault-Tech
% Data creazione: 10.05.2016
% E-mail: vaulttech.swe@gmail.comcom
%
% Diario delle modifiche: interno al file.

\documentclass[a4paper, titlepage]{article}

\usepackage[margin=3cm]{geometry}
\usepackage{../../Stile}
\usepackage{../../Comandi}

\setcounter{secnumdepth}{5}
\setcounter{tocdepth}{5}

\def\NOME{Manuale Sviluppatore}
\def\VERSIONE{1.0}
\def\DATA{???}
\def\REDATTORE{???}
\def\VERIFICATORE{???}
\def\RESPONSABILE{Giacomo Beltrame}
\def\USO{Esterno}
\def\DISTRIBUZIONE{\COMMITTENTE \\ & \CARDIN \\ & \PROPONENTE}


\begin{document}
	
	\pagestyle{fancy}	
	\pagenumbering{Roman}
	\rfoot{Pagina \thepage{} di \pageref{lastromanpage}}
	
	\maketitle
	
	\begin{diario}
	\recap{Approvazione del documento}{Vassilikì Menarin}{Responsabile}{24.12.2015}{1.0}
	\recap{Correzione errori}{Simone Boccato}{Analista}{23.12.2015}{0.3}
	\recap{Verifica del documento}{Filippo Tesser}{Verificatore}{22.12.2015}{0.2}
	\recap{Stesura del documento}{Simone Boccato}{Analista}{20.12.2015}{0.1}
\end{diario}
	
	\newpage
	\tableofcontents
%	\newpage
%	\listoffigures
%	\newpage
%	\listoftables\label{lastromanpage}
	
	\newpage
	\clearpage	
	\pagenumbering{arabic}
	\rfoot{Pagina \thepage{} di \pageref*{LastPage}}
	%Deve esserci per permettere i riferimenti incrociati di colore blu
	\hypersetup{linkcolor=blue}
	
	\section{Introduzione}
	\subsection{Scopo del documento}
	Questo documento ha lo scopo di indicare e spiegare quali sono i comandi da eseguire per installare correttamente l'applicativo Quizzipedia.
	
	\subsection{Scopo del prodotto}
	\SCOPO
	
	\subsection{Riferimenti}	
	\subsubsection{Riferimenti informativi}
	\begin{itemize}
		\item \textbf{Manuale utente \gl{Git}:} \url{https://git-scm.com/docs/user-manual.html};
		\item \textbf{Manuale installazione \gl{Node.js}:} \url{https://docs.npmjs.com/getting-started/installing-node};
		\item \textbf{Manuale installazione \gl{MongoDB}:} \url{https://docs.mongodb.com/master/installation/};
		\item \textbf{Front-end package manager Bower}: \url{http://bower.io/};
		\item \textbf{Grunt JavaScript task runner}: \url{http://gruntjs.com/}.
	\end{itemize}
	\newpage
	
	\section{Prerequisiti software}
	Per poter avviare Quizzipedia è richiesta l'installazione dei seguenti software.
	
	\subsection{Node.js e npm}
	Scarica e installa \gl{Node.js} e il package manager npm da \url{https://nodejs.org/en/download/current/}.
	
	Verificare che sia visibile la versione installata con i seguenti comandi:
	
	\texttt{\$ node -v}
	
	\texttt{\$ npm -v}

	\subsection{MongoDB}
	Scarica e installa \gl{MongoDB} da \url{https://www.mongodb.com/download-center}.
	
	Per avviare \gl{MongoDB} sarà necessario dare il seguente comando:
	
	\texttt{\$ mongod}
	
	\subsection{DA FARE: AGGIUNGERE GLI ALTRI MODULI LATO CLIENT}
	
	\section{Requisiti hardware}
	Il prodotto lato server richiede come requisiti minimi:
	\begin{itemize}
		\item processore dual core;
		\item 2GB di memoria RAM;
		\item 1GB di spazio libero su disco.
	\end{itemize}
	
	\section{Installazione}
	
	\subsection{Ottenere Quizzipedia}
	DA FARE
	
	\subsection{Installazione dipendenze}
	
	Una volta ottenuto il codice sorgente, spostarsi all'interno della root del repository, si controlli che sia presente il file package.json.
	
	Sempre nella directory corrente eseguire poi il seguente comando da terminale:
	
	\texttt{\$ npm install}
	
	in questo modo verrano installati automaticamente i moduli di Node.js necessari per l'esecuzione del prodotto.
	
	DA FARE: PROCEDIMENTO PER MODULI LATO CLIENT
	
	\subsection{Avvio}
	
	DA FARE: PROCEDURA PER ESEGUIRE SERVER, EVENTUALI AMBIENTI DI ESECUZIONE, E PROCEDURA PER FARLO ANDARE LOCALMENTE O IN REMOTO
	
\end{document}