% Nome del file: Verbale.tex
% Percorso: template
% Autore: Vault-Tech
% Data creazione: 23.12.2015
% E-mail: vaulttech.swe@gmail.com


\documentclass[a4paper]{article}

\usepackage[margin=3cm]{geometry}
\usepackage{Comandi}

%
% VARIABILI DA DEFINIRE
%
\def\DATA{18.01.2016}
\def\ORA{15:30}
\def\CITTA{Padova}
\def\PROVINCIA{PD}
\def\VIA{Via Cittadella,7}
\def\SEGRETARIO{Rudy Berton}
\def\RESPONSABILE{Vassilikì Menarin}
% NB:basta inserire i nomi qui sopra e poi dentro al documento li mette automaticamente!!!

\begin{document}

\begin{center}
\includegraphics[scale=0.5]{Img/logo.png}\\
\vspace{1cm}
{\Huge \PROGETTO}\\
\vspace{1cm}
\section*{Verbale V[\DATA]}
\end{center}

%
% INIZIO VERBALE
%
\bold{Responsabile:} \RESPONSABILE
\\ \bold{Segretario:} \SEGRETARIO
\\
\subsection*{Presenti}
Il segretario verbalizzante \SEGRETARIO{} indica come presenti all'appello:
\begin{itemize}
\item Vassilikì Menarin;
\item Rudy Berton;
\item Giacomo Beltrame;
\item Simone Boccato;
\item Filippo Tesser;
\item il proponente rappresentato da Gregorio Piccoli e Serena Lago.
\end{itemize}

\subsection*{Assenti}
Il segretario verbalizzante \SEGRETARIO{} indica come assenti all'appello:
\begin{itemize}
\item Miki Violetto;
\item Michela De Bortoli.
\end{itemize}

\subsection*{Ordine del giorno}
Il giorno \DATA{} alle ore \ORA{} presso \CITTA (\PROVINCIA) in \VIA{} il gruppo \AUTORE{} ha tenuto un incontro col proponente per discutere i seguenti punti:
%NB: se è solo una riunione interna tra noi cambiare la descrizione qua sopra!
\begin{itemize}
\item discutere e verificare i requisiti ed i casi d'uso realizzati;
\item ottenere un feedback da parte del proponente sull'analisi sviluppata.
\end{itemize}

\subsection*{Interventi}
Sono intervenuti tutti i presenti per chiarire alcuni dubbi sui requisiti analizzati e sviluppati durante questa prima attività.
\newline In particolare sono stati chiariti i ruoli che ricoprono gli attori all'interno del sistema e come ognuno debba interagire col sistema e in particolare come debba essere gestita l'iscrizione di un utente da parte di un amministratore.
\newline Il proponente ha proposto l'idea di avere utenti non necessariamente autenticati all'interno del sistema, considerazione prima non presa a carico dal gruppo, ma ora ritenuta positiva; si è deciso pertanto di seguire tale idea dedicando un diverso grado di funzionalità agli utenti a seconda della loro autenticazione o meno.
\newline Il proponente è risultato molto interessato alla tipologia di domande proposte dal gruppo ed ha apprezzato la diversità, indicando altri possibili modelli da seguire per la creazione di domande e la loro presentazione all'interno del sistema.
\newline Infine si è passati a discutere sulla raccolta dei dati per la realizzazione delle statistiche riguardanti i questionari: il proponente avrebbe il piacere che anche gli utenti non autenticati potessero portar del valore aggiunto alle statistiche.

\subsection*{Conclusioni}
L'incontro è stato molto produttivo e soddisfacente per ambedue le parti. Ha permesso al team di comprendere le limitazioni poste al sistema nello sviluppo dell'analisi dei requisiti e di modificare i requisiti e i casi d'uso per ottenere un miglioramento. Allo stesso tempo ha permesso di integrare i suggerimenti fatti dal proponente con idee nuove.

\subsection*{Problemi irrisolti}
Tutti i problemi sollevati sono stati risolti.

\vspace{2cm}
Il responsabile \RESPONSABILE{} approva il verbale.

\end{document}
