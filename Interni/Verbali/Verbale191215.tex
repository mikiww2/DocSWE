% Nome del file: Verbale.tex
% Percorso: template
% Autore: Vault-Tech
% Data creazione: 19.12.2015
% E-mail: vaulttech.swe@gmail.com


\documentclass[a4paper]{article}

\usepackage[margin=3cm]{geometry}
\usepackage{../../Comandi}

%
% VARIABILI DA DEFINIRE
%
\def\DATA{19.12.2015}
\def\ORA{10:30}
\def\CITTA{Padova}
\def\PROVINCIA{PD}
\def\VIA{via Luzzatti}
\def\SEGRETARIO{Filippo Tesser}
\def\RESPONSABILE{Vassilikì Menarin}
% NB:basta inserire i nomi qui sopra e poi dentro al documento li mette automaticamente!!!

\begin{document}

\begin{center}
\includegraphics[scale=0.5]{Img/logo.png}\\
\vspace{1cm}
{\Huge \PROGETTO}\\
\vspace{1cm}
\section*{Verbale V[\DATA]}
\end{center}

%
% INIZIO VERBALE
%
\bold{Responsabile:} \RESPONSABILE
\\ \bold{Segretario:} \SEGRETARIO
\\
\subsection*{Presenti}
Il segretario verbalizzante \SEGRETARIO{} indica come presenti all'appello:
\begin{itemize}
\item Filippo Tesser;
\item Miki Violetto;
\item Simone Boccato;
\item Rudy Berton;
\item Vassilikì Menarin;
\item Giacomo Beltrame.
\end{itemize}

\subsection*{Assenti}
Il segretario verbalizzante \SEGRETARIO{} indica come assenti all'appello:
\begin{itemize}
\item Michela De Bortoli.
\end{itemize}

\subsection*{Ordine del giorno}
Il giorno \DATA{} alle ore \ORA{} presso \CITTA (\PROVINCIA) in \VIA{} il gruppo \AUTORE{} ha tenuto una riunione interna per discutere i seguenti punti:
%NB: se è solo una riunione interna tra noi cambiare la descrizione qua sopra!
\begin{itemize}
\item scelta degli strumenti;
\item organizzazione generale del \gl{team}.
\end{itemize}

%NB: decidere se sia meglio fare a Q/A oppure un riassunto
\subsection*{Interventi}
Tutti gli interventi sono stati rivolti verso la praticità e usabilità degli strumenti da utilizzare.
Alla fine sono stati scelti i seguenti strumenti di comunicazione in maniera democratica:
\begin{itemize}
	\item \gl{Telegram}, per le comunicazioni generali e organizzative;
	\item \gl{Slack}, per le comunicazioni specifiche ad attività di sviluppo e verifica;
	\item \gl{TexStudio}, come \gl{IDE} per la redazione dei documenti in \LaTeX;
	\item \gl{StarUML}, come tool per i diagrammi \gl{UML};
	\item \gl{Redmine}, come piattaforma di project management.
\end{itemize}
Si è infine deciso che eventuali altri strumenti saranno scelti con successive riunioni interne.
Per l'organizzazione del gruppo si sono concordate alcune regole per venire incontro alle esigenze di alcuni membri del \gl{team}. 

\subsection*{Conclusioni}
Questo primo incontro è stato sufficientemente produttivo e ha permesso di capire all'intero \gl{team} le modalità di comunicazione e gestione strumenti.
\subsection*{Problemi irrisolti}
Non vi sono stati problemi da risolvere per questa riunione.

\vspace{2cm}
Il responsabile \RESPONSABILE{} approva il verbale.

\end{document}
